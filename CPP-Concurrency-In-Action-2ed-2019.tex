
\documentclass[11pt,a4paper,UTF8]{book}

\usepackage[outputdir=build]{minted}
\usepackage[T1]{fontenc}
\usepackage[utf8]{inputenc}
\usepackage{authblk}

\usepackage{fontspec}                  %引入字体设置宏包
\setmainfont{Times New Roman}             %设置英文正文字体
% Courier New
% Book Antique
\setsansfont{Arial}                    %英文无衬线字体
\setmonofont{Courier New}              %英文等宽字体

\usepackage{ctex} %导入中文包
%\usepackage{ulem}
\usepackage{tocvsec2}
\usepackage{verbatim}

\usepackage{multirow}
\usepackage{lscape}
\usepackage{adjustbox}

\usepackage{pifont}
\usepackage{xeCJK}
% \setCJKmainfont[BoldFont=Source Han Sans CN, ItalicFont=Kaiti SC]{Source Han Serif SC}
\setmainfont{Source Han Serif SC}
\setCJKsansfont{Source Han Sans CN}
% \setCJKmainfont[BoldFont=STHeiti, ItalicFont=STKaiti]{STSong}
% \setCJKsansfont{STHeiti}
% \setCJKmonofont{STFangsong}
\setCJKmonofont{Source Han Mono SC}
\punctstyle{hangmobanjiao} %行末半角式:所有标点占一个汉字宽度,行首行末对齐;

\usepackage{tabularx}
\usepackage{longtable}
\usepackage{booktabs}
\usepackage{multirow}
\usepackage{bbding}
\usepackage{float}
\usepackage{xspace}
\usepackage[none]{hyphenat}

\usepackage{graphicx}
\usepackage{subfigure}
\usepackage{pifont}

\usepackage{hyperref}  %制作pdf的目录
\usepackage{subfiles} %使用多文件方式进行

\usepackage{geometry} %设置页边距的包
\geometry{left=2.5cm,right=2cm,top=2.54cm,bottom=2.54cm} %设置书籍的页边距

\usepackage{url}
\hypersetup{hidelinks, %去红框
	colorlinks=true,
	allcolors=black,
	pdfstartview=Fit,
	breaklinks=true
}

% 调整itemlist中的行间距
\usepackage{enumitem}
\setenumerate[1]{itemsep=0pt,partopsep=0pt,parsep=\parskip,topsep=5pt}
\setitemize[1]{itemsep=0pt,partopsep=0pt,parsep=\parskip,topsep=5pt}
\setdescription{itemsep=0pt,partopsep=0pt,parsep=\parskip,topsep=5pt}

% 超链接样式设置
\usepackage{hyperref}
\hypersetup{
	colorlinks=true,
	linkcolor=blue,
	filecolor=blue,
	urlcolor=blue,
	citecolor=cyan,
}

\usepackage{indentfirst}

\usepackage{listings}
\usepackage[usenames,dvipsnames,svgnames, x11names,xcdraw,table]{xcolor}
\usepackage{pagecolor}

% 设置护眼绿色
\definecolor{mygreen}{RGB}{199, 237, 204}
\pagecolor{mygreen}


\usepackage[most]{tcolorbox}
\tcbuselibrary{breakable} % 引入 breakable 库
\tcbuselibrary{skins} % 引入 skins 库

%定义CMake
\lstdefinelanguage{CMake}
{morekeywords={
		cmake\_minimum\_required,
		project,
		add\_executable,
		add\_library,
		target\_link\_libraries,
		cmake\_parse\_arguments,
		cmake\_language,
		set, unset,
		option,
		string,
		list,
		math,
		message,
		if, elseif, else, endif,
		mark\_as\_advanced,
		foreach, endforeach,
		while, endwhile,
		add\_subdirectory, include, return, include\_gurad,
		function, endfunction,
		macro, endmacro,
		find\_package,
		cmake\_push\_check\_state,
		cmake\_pop\_check\_state,
		cmake\_reset\_check\_state,
		add\_test,
		set\_tests\_properties,
		check\_c\_source\_runs,
		check\_cxx\_source\_runs,
		check\_fortran\_source\_runs,
		check\_source\_runs,
		check\_compiler\_flag,
		check\_c\_compiler\_flag,
		check\_cxx\_compiler\_flag,
		check\_fortran\_compiler\_flag,
		check\_symbol\_exists,
		check\_cxx\_symbol\_exists,
		check\_linker\_flag,
		cmake\_policy,
		set\_property,
		get\_property,
		define\_property,
		get\_cmake\_property,
		set\_cmake\_property,
		set\_target\_properties,
		get\_target\_property,
		set\_directory\_properties,
		get\_directory\_property,
		set\_source\_files\_properties,
		get\_source\_file\_property,
		set\_tests\_properties,
		get\_tests\_property,
		get\_test\_property,
		cmake\_print\_properties,
		cmake\_print\_variables,
		variable\_watch,
		include\_guard,
		target\_link\_options,
		target\_compile\_definitions,
		target\_compile\_options,
		include\_directories,
		add\_definitions,
		remove\_definitions,
		add\_compile\_definitions,
		add\_compile\_options,
		link\_libraries,
		link\_directories,
		add\_link\_options,
		target\_include\_directories,
		target\_compile\_features,
		add\_custom\_command,
		add\_custom\_target,
		execute\_process,
		cmake\_path,
		get\_filename\_component,
		file,
		configure\_file,
		generate\_export\_header,
		export,
		find\_file,
		find\_library,
		find\_package,
		find\_program,
		pkg\_check\_modules,
		pkg\_search\_module,
		pkg\_get\_variable,
		add\_test,
		enable\_testing,
		set\_tests\_properties,
		site\_name,
		ctest\_empty\_binary\_directory,
		ctest\_start,
		ctest\_configure,
		ctest\_submit,
		ctest\_build,
		ctest\_memcheck,
		ctest\_upload,
		ctest\_test,
		gtest\_add\_tests,
		gtest\_discover\_tests,
		install,
		write\_basic\_package\_version\_file,
		configure\_package\_config\_file,
		cpack\_add\_component,
		cpack\_add\_install\_type,
		cpack\_add\_component\_group,
		ExternalProject\_Add,
		ExternalProject\_Add\_StepDependencies,
		ExternalProject\_Get\_Property,
		ExternalProject\_Add\_Step,
		FetchContent\_Declare,
		FetchContent\_GetProperties,
		FetchContent\_Populate,
		source\_group,
		target\_precompile\_headers,
		qt5\_wrap\_cpp,
		qt5\_wrap\_ui,
		qt5\_add\_resources,
		qt5\_add\_big\_resources,
		qt5\_add\_binary\_resources,
		qt5\_add\_translation,
		qt5\_create\_translation,
		compile\_definitions,
		add\_llvm\_component\_library,
		add\_llvm\_tool,
		llvm\_multisource,
		llvm\_test\_data,
		doxygen\_add\_docs,
		cmake\_dependent\_option,
		target\_sources,
		conan\_cmake\_autodetect,
		conan\_cmake\_configure,
		conan\_cmake\_install,
		doxygen\_add\_docs,
		check\_source\_compiles,
		check\_language,
		enable\_language,
		add\_dependencies,
		find\_path,
		find\_package\_handle\_standard\_args,
	}, %定义关键字
	sensitive=false, %是否大小写敏感
	morecomment=[l]{\#},
	morestring=[b]",
	morestring=[d]',
}

\lstdefinestyle{styleCMake}{
	language=CMake,
	backgroundcolor=\color{blue!3!white},
	basicstyle=\tt,
	breakatwhitespace = false,
	breaklines = true,
	captionpos = b,
	commentstyle = \color{mygray}\bfseries,
	extendedchars =false,
	frame=shadowbox,
	tabsize=2,
	framerule=0.5pt,
	keepspaces=true,
	keywordstyle=\color{blue}\bfseries, % keyword style
	otherkeywords={string},
	rulecolor=\color{black},
	showspaces=false,
	showstringspaces=false,
	showtabs=false,
	stepnumber=1,
	stringstyle=\color{purple},        % string literal style
}

\lstdefinestyle{stylePython}{
	language        =   Python, % 语言选Python
	backgroundcolor=\color{blue!3!white},
	basicstyle      =   \zihao{-5}\ttfamily,
	numberstyle     =   \zihao{-5}\ttfamily,
	keywordstyle    =   \color{blue},
	keywordstyle    =   [2] \color{teal},
	stringstyle     =   \color{magenta},
	commentstyle    =   \color{red}\ttfamily,
	frame = shadowbox,
	breaklines      =   true,   % 自动换行,建议不要写太长的行
	columns         =   fixed,  % 如果不加这一句,字间距就不固定,很丑,必须加
	basewidth       =   0.5em,
	%basicstyle          =   \sffamily,          % 基本代码风格
	%keywordstyle        =   \bfseries,          % 关键字风格
	%commentstyle        =   \rmfamily\itshape,  % 注释的风格,斜体
	%stringstyle         =   \ttfamily,  % 字符串风格
	flexiblecolumns,                % 别问为什么,加上这个
	%numbers             =   left,   % 行号的位置在左边
	showspaces          =   false,  % 是否显示空格,显示了有点乱,所以不现实了
	numberstyle         =   \zihao{-5}\ttfamily,    % 行号的样式,小五号,tt等宽字体
	showstringspaces    =   false,
	captionpos          =   t,      % 这段代码的名字所呈现的位置,t指的是top上面
	frame               =   lrtb,   % 显示边框
	tabsize=2,
}

\tcbset{
	breakable,
	commandshell/.style={
		listing only,
		colback=black!75!white,
		colupper=white,
		lowerbox=ignored,
		listing options={
			language={bash},
			breaklines=true,
			basicstyle=\ttfamily,
			columns = fixed,
			flexiblecolumns
		}
}}

\usepackage{tikz}

% URL 正确换行
% https://liam.page/2017/05/17/help-the-url-command-from-hyperref-to-break-at-line-wrapping-point/
\makeatletter
\def\UrlAlphabet{%
	\do\a\do\b\do\c\do\d\do\e\do\f\do\g\do\h\do\i\do\j%
	\do\k\do\l\do\m\do\n\do\o\do\p\do\q\do\r\do\s\do\t%
	\do\u\do\v\do\w\do\x\do\y\do\z\do\A\do\B\do\C\do\D%
	\do\E\do\F\do\G\do\H\do\I\do\J\do\K\do\L\do\M\do\N%
	\do\O\do\P\do\Q\do\R\do\S\do\T\do\U\do\V\do\W\do\X%
	\do\Y\do\Z}
\def\UrlDigits{\do\1\do\2\do\3\do\4\do\5\do\6\do\7\do\8\do\9\do\0}
\g@addto@macro{\UrlBreaks}{\UrlOrds}
\g@addto@macro{\UrlBreaks}{\UrlAlphabet}
\g@addto@macro{\UrlBreaks}{\UrlDigits}
\makeatother

% enable subsubsubsection
% from https://tex.stackexchange.com/练习题/274212/correct-hierarchy-levels-of-pdf-bookmarks-for-custom-section-subsubsubsection
\usepackage[depth=3]{bookmark}
\setcounter{secnumdepth}{3}
\setcounter{tocdepth}{4}
\hypersetup{bookmarksdepth=4}

\makeatletter

\newcommand{\toclevel@subsubsubsection}{4}
\newcounter{subsubsubsection}[subsubsection]

\renewcommand{\thesubsubsubsection}{\thesubsubsection.\arabic{subsubsubsection}}

\newcommand{\subsubsubsection}{\@startsection{subsubsubsection}{4}{\z@}%
	{-3.25ex\@plus -1ex \@minus -.2ex}%
	{1.5ex \@plus .2ex}%
	{\normalfont\normalsize\bf\bfseries}}

\newcommand*{\l@subsubsubsection}{\@dottedtocline{4}{11em}{5em}}

\newcommand{\subsubsubsectionmark}[1]{}
\makeatother

\ExplSyntaxOn

% Setup enumerate, itemize and description
\setenumerate  { nosep }
\setitemize    { nosep }
\setdescription{ nosep }

% Setup minted
\setminted { obeytabs, tabsize=2, breaklines=true, fontsize=\footnotesize, frame=single }

% Def \filename
\NewDocumentCommand { \filename } { m }
{ \noindent \textit { #1 } \vspace*{ -1ex } \nopagebreak[4] }

% Def \mySamllsection
\NewDocumentCommand { \mySamllsection } { m }
{ \noindent \hspace*{\fill} \\ \textbf { #1 } \vspace*{ -1ex } \nopagebreak[4] \\ }

% Def \inlcpp
\NewDocumentCommand { \inlcpp }   { m }
{ \mintinline { cpp } { #1 } }

% Def cpp environment
\NewDocumentEnvironment { cpp } { }
{ \VerbatimEnvironment
	\begin { minted } [ linenos=true ] { cpp } }
{ \end   { minted } }

% Def shell environment
\NewDocumentEnvironment { shell } { }
{ \VerbatimEnvironment
	\begin { minted } { text } }
{ \end   { minted } }

\NewDocumentEnvironment { notice } { }
{ \begin { tcolorbox } [ colback = green!5!white, colframe=green!75!black ] }
{ \end   { tcolorbox } }

\NewDocumentCommand { \mySubsubsection } { mm }
{
\subsubsection*{\zihao{3} {#1} \hspace{0.2cm}{#2}}
\addcontentsline{toc}{subsubsection}{{#1}\hspace{0.2cm}{#2}}
}

\NewDocumentCommand { \mySubsection } { mmm }
{
\subsection*{\zihao{3}{#1}\hspace{0.2cm}{#2}}
\addcontentsline{toc}{subsection}{{#1}\hspace{0.2cm}{#2}}
\subfile{{#3}}
}

\NewDocumentCommand { \mySection } { mmm }
{
\section*{\zihao{2}{#1}\hspace{0.5cm}{#2}}
\addcontentsline{toc}{section}{{#1}\hspace{0.5cm}{#2}}
\subfile{{#3}}
}

% Latex如何在文本模式批量处理下划线
% https://zhuanlan.zhihu.com/p/615108006

\ExplSyntaxOff

\begin{document}
	\begin{sloppypar} %latex中一行文字出现溢出问题的解决方法
		%\maketitle

		\begin{center}
			\thispagestyle{empty}
			%\includegraphics[width=\textwidth,height=\textheight,keepaspectratio]{cover.jpg}
			\begin{tikzpicture}[remember picture, overlay, inner sep=0pt]
				\node at (current page.center)
				{\includegraphics[width=\paperwidth, keepaspectratio=false]{cover.jpg}};
			\end{tikzpicture}
			\newpage
			\thispagestyle{empty}
			\huge
			\textbf{C++20 - The Complete Guide}
			\\[9pt]
			\normalsize
			作者: \href{http://www.josuttis.com/welcomee.html}{Nicolai M. Josuttis}
			\\[8pt]
			\normalsize
			译者:\href{https://github.com/xiaoweiChen/CXX20-The-Complete-Guide}{陈晓伟}
			\\[8pt]
			\normalsize
			版本:\href{http://leanpub.com/cpp20}{2022-10-30}
		\end{center}

		\newpage

		\pagestyle{empty}
		\tableofcontents
		\newpage

		\setsecnumdepth{section}

		\mySection{}{C++ Concurrency In Action}{content/README.tex}
		\newpage

		\mySection{}{第一版的赞许}{content/Praise_for_the_first_edition.tex}
		\newpage

		\mySection{}{前言}{content/preface.tex}
		\newpage

        \mySection{}{感谢}{content/acknowledgements-chinese.tex}
        \newpage

		\mySection{}{关于本书}{content/chapter00/0.tex}
		\mySubsection{}{路线图}{content/chapter00/1.tex}
		\mySubsection{}{适读人群}{content/chapter00/2.tex}
		\mySubsection{}{代码公约和下载}{content/chapter00/3.tex}
		\mySubsection{}{作者在线}{content/chapter00/4.tex}
		\newpage

		\mySection{}{关于作者}{content/about_the_author-chinese.tex}
		\newpage

		\mySection{}{关于封面}{content/about_cover_illustration-chinese.tex}
		\newpage

		\mySection{第1章}{你好,并发世界}{content/chapter01/0.tex}
		\mySubsection{1.1.}{何谓并发}{content/chapter01/1.tex}
		\mySubsection{1.2.}{为什么使用并发}{content/chapter01/2.tex}
		\mySubsection{1.3.}{并发和多线程}{content/chapter01/3.tex}
		\mySubsection{1.4.}{开始入门}{content/chapter01/4.tex}
		\mySubsection{1.5.}{本章总结}{content/chapter01/5.tex}
		\newpage

		\mySection{第2章}{线程管理}{content/chapter02/0.tex}
		\mySubsection{2.1.}{线程的基本操作}{content/chapter02/1.tex}
		\mySubsection{2.2.}{传递参数}{content/chapter02/2.tex}
		\mySubsection{2.3.}{转移所有权}{content/chapter02/3.tex}
		\mySubsection{2.4.}{确定线程数量}{content/chapter02/4.tex}
		\mySubsection{2.5.}{线程标识}{content/chapter02/5.tex}
		\mySubsection{2.6.}{本章总结}{content/chapter02/6.tex}
		\newpage

		\mySection{第3章}{共享数据}{content/chapter03/0.tex}
		\mySubsection{3.1.}{共享数据的问题}{content/chapter03/1.tex}
		\mySubsection{3.2.}{使用互斥量}{content/chapter03/2.tex}
		\mySubsection{3.3.}{保护共享数据的方式}{content/chapter03/3.tex}
		\mySubsection{3.4.}{本章总结}{content/chapter03/4.tex}
		\newpage

		\mySection{第4章}{同步操作}{content/chapter04/0.tex}
		\mySubsection{4.1.}{等待事件或条件}{content/chapter04/1.tex}
		\mySubsection{4.2.}{使用future}{content/chapter04/2.tex}
		\mySubsection{4.3.}{限时等待}{content/chapter04/3.tex}
		\mySubsection{4.4.}{简化代码}{content/chapter04/4.tex}
		\mySubsection{4.5.}{本章总结}{content/chapter04/5.tex}
		\newpage

		\mySection{第5章}{内存模型和原子操作}{content/chapter05/0.tex}
		\mySubsection{5.1.}{内存模型}{content/chapter05/1.tex}
		\mySubsection{5.2.}{原子操作和原子类型}{content/chapter05/2.tex}
		\mySubsection{5.3.}{同步操作和强制排序}{content/chapter05/3.tex}
		\mySubsection{5.4.}{本章总结}{content/chapter05/4.tex}
		\newpage

		\mySection{第6章}{设计基于锁的并发数据结构}{content/chapter06/0.tex}
		\mySubsection{6.1.}{并发设计的意义}{content/chapter06/1.tex}
		\mySubsection{6.2.}{基于锁的并发数据结构}{content/chapter06/2.tex}
		\mySubsection{6.3.}{设计更加复杂的数据结构}{content/chapter06/3.tex}
		\mySubsection{6.4.}{本章总结}{content/chapter06/4.tex}
		\newpage

		\mySection{第7章}{设计无锁的并发数据结构}{content/chapter07/0.tex}
		\mySubsection{7.1.}{定义和意义}{content/chapter07/1.tex}
		\mySubsection{7.2.}{无锁数据结构的例子}{content/chapter07/2.tex}
		\mySubsection{7.3.}{设计无锁数据结构的指导建议}{content/chapter07/3.tex}
		\mySubsection{7.4.}{本章总结}{content/chapter07/4.tex}
		\newpage

		\mySection{第8章}{并发设计}{content/chapter08/0.tex}
		\mySubsection{8.1.}{线程间划分工作}{content/chapter08/1.tex}
		\mySubsection{8.2.}{并发代码的性能}{content/chapter08/2.tex}
		\mySubsection{8.3.}{为多线程性能设计数据结构}{content/chapter08/3.tex}
		\mySubsection{8.4.}{设计并发代码的注意事项}{content/chapter08/4.tex}
		\mySubsection{8.5.}{在实践中设计并发代码}{content/chapter08/5.tex}
		\mySubsection{8.5.}{本章总结}{content/chapter08/6.tex}
		\newpage

		\mySection{第9章}{高级线程管理}{content/chapter09/0.tex}
		\mySubsection{9.1.}{线程池}{content/chapter09/1.tex}
		\mySubsection{9.2.}{中断线程}{content/chapter09/2.tex}
		\mySubsection{9.3.}{本章总结}{content/chapter09/3.tex}
		\newpage

		\mySection{第10章}{并行算法}{content/chapter10/0.tex}
		\mySubsection{10.1.}{并行化标准库算法}{content/chapter10/1.tex}
		\mySubsection{10.2.}{执行策略}{content/chapter10/2.tex}
		\mySubsection{10.3.}{C++标准库中的并行算法}{content/chapter10/3.tex}
		\mySubsection{10.4.}{本章总结}{content/chapter10/4.tex}
		\newpage

		\mySection{第11章}{多线程程序的测试和调试}{content/chapter11/0.tex}
		\mySubsection{11.1.}{与并发相关的错误类型}{content/chapter11/1.tex}
		\mySubsection{11.2.}{定位并发Bug的技巧}{content/chapter11/2.tex}
		\mySubsection{11.3.}{本章总结}{content/chapter11/3.tex}
		\newpage

		\mySection{附录A}{对C++11特性的简要介绍}{content/appendix_A/0.tex}
		\mySubsection{A.1.}{右值引用}{content/appendix_A/1.tex}
		\mySubsection{A.2.}{删除函数}{content/appendix_A/2.tex}
		\mySubsection{A.3.}{右值引用}{content/appendix_A/3.tex}
		\mySubsection{A.4.}{常量表达式函数}{content/appendix_A/4.tex}
		\mySubsection{A.5.}{Lambda函数}{content/appendix_A/5.tex}
		\mySubsection{A.6.}{变参模板}{content/appendix_A/6.tex}
		\mySubsection{A.7.}{自动推导变量类型}{content/appendix_A/7.tex}
		\mySubsection{A.8.}{线程本地变量}{content/appendix_A/8.tex}
		\mySubsection{A.9.}{模板类参数的推导}{content/appendix_A/9.tex}
		\mySubsection{A.10.}{本章总结}{content/appendix_A/10.tex}
		\newpage

		\mySection{附录B}{并发库的简单比较}{content/appendix_B/0.tex}
		\newpage

		\mySection{附录C}{消息传递框架与完整的ATM示例}{content/appendix_C/0.tex}
		\newpage

		\mySection{附录D}{C++线程库参考}{content/appendix_D/0.tex}
		\newpage
	\end{sloppypar}
\end{document}
