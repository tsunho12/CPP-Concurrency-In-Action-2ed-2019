% # A.2 删除函数

有时让类去做拷贝是没有意义的。\texttt{std::mutex}就是一个例子——拷贝一个互斥量,意义何在?\texttt{std::unique\_lock<>}是另一个例子——一个实例只能拥有一个锁。如果要复制,拷贝的那个实例也能获取相同的锁,这样\texttt{std::unique\_lock<>}就没有存在的意义了。实例中转移所有权(A.1.2节)是有意义的,其并不是使用的拷贝。

为了避免拷贝操作,会将拷贝构造函数和拷贝赋值操作符声明为私有成员,并且不进行实现。如果对实例进行拷贝,将会引起编译错误。如果有其他成员函数或友元函数想要拷贝一个实例,将会引起链接错误(因为缺少实现):

\begin{cpp}
class no_copies
{
public:
  no_copies(){}
private:
  no_copies(no_copies const&);  // 无实现
  no_copies& operator=(no_copies const&);  // 无实现
};

no_copies a;
no_copies b(a);  // 编译错误
\end{cpp}

C++11中,委员会意识到这种情况。因此,委员会提供了更多的通用机制:可以通过添加\texttt{= delete}将一个函数声明为删除函数。

no\_copise类就可以写为:

\begin{cpp}
class no_copies
{
public:
  no_copies(){}
  no_copies(no_copies const&) = delete;
  no_copies& operator=(no_copies const&) = delete;
};
\end{cpp}

这样的描述要比之前的代码更加清晰。也允许编译器提供更多的错误信息描述,当成员函数想要执行拷贝操作的时候,可将链接错误转移到编译时。

拷贝构造和拷贝赋值操作删除后,需要显式写一个移动构造函数和移动赋值操作符,与\texttt{std::thread}和\texttt{std::unique\_lock<>}一样,类是只移动的。

下面代码中的例子,就展示了一个只移动的类。

代码A.2 只移动类

\begin{cpp}
class move_only
{
  std::unique_ptr<my_class> data;
public:
  move_only(const move_only&) = delete;
  move_only(move_only&& other):
    data(std::move(other.data))
  {}
  move_only& operator=(const move_only&) = delete;
  move_only& operator=(move_only&& other)
  {
    data=std::move(other.data);
    return *this;
  }
};

move_only m1;
move_only m2(m1);  // 错误,拷贝构造声明为“已删除”
move_only m3(std::move(m1));  // OK,找到移动构造函数
\end{cpp}

只移动对象可以作为函数的参数进行传递,并且从函数中返回,不过当想要移动左值,通常需要显式的使用\texttt{std::move()}或\texttt{static\_cast<T\&\&>}。

可以为任意函数添加\texttt{= delete}说明符,添加后就说明这些函数不能使用。当然,还可以用于很多的地方。删除函数可以以正常的方式参与重载解析,并且如果使用,就会引起编译错误,这种方式可以用来删除特定的重载。比如,当函数以short作为参数,为了避免扩展为int类型,可以写出重载函数(以int为参数)的声明,然后添加删除说明符:

\begin{cpp}
void foo(short);
void foo(int) = delete;
\end{cpp}

现在,任何向foo函数传递int类型参数都会产生一个编译错误,不过调用者可以显式的将其他类型转化为short:

\begin{cpp}
foo(42);  // 错误,int重载声明已经删除
foo((short)42);  // OK
\end{cpp}

