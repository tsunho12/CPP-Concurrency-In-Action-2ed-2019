% # A.3 默认函数

删除函数的函数可以不进行实现,默认函数就则不同:编译器会创建函数实现,通常都是“默认”实现。当然,这些函数可以直接使用(它们都会自动生成):默认构造函数,析构函数,拷贝构造函数,移动构造函数,拷贝赋值操作符和移动赋值操作符。

为什么要这样做呢?这里列出一些原因:
\begin{itemize}
\item 改变函数的可访问性——编译器生成的默认函数通常都是声明为public(如果想让其为protected或private成员,必须自己实现)。将其声明为默认,可以让编译器来帮助你实现函数和改变访问级别。

\item 作为文档——编译器生成版本已经足够使用,那么显式声明就利于其他人阅读这段代码,会让代码结构看起来很清晰。

\item 没有单独实现的时候,编译器自动生成函数——通常默认构造函数来做这件事,如果用户没有定义构造函数,编译器将会生成一个。当需要自定一个拷贝构造函数时(假设),如果将其声明为默认,也可以获得编译器为你实现的拷贝构造函数。

\item 编译器生成虚析构函数。

\item 声明一个特殊版本的拷贝构造函数。比如:参数类型是非const引用,而不是const引用。

\item 利用编译生成函数的特殊性质(如果提供了对应的函数,将不会自动生成对应函数——会在后面具体讲解)。
\end{itemize}

就像删除函数是在函数后面添加\texttt{= delete}一样,默认函数需要在函数后面添加\texttt{= default},例如:

\begin{cpp}
class Y
{
private:
  Y() = default;  // 改变访问级别
public:
  Y(Y&) = default;  // 以非const引用作为参数
  T& operator=(const Y&) = default; // 作为文档的形式,声明为默认函数
protected:
  virtual ~Y() = default;  // 改变访问级别,以及添加虚函数标签
};
\end{cpp}

编译器生成函数都有独特的特性,这是用户定义版本所不具备的。最大的区别就是编译器生成的函数都很简单。

列出了几点重要的特性:

\begin{itemize}
\item 对象具有简单的拷贝构造函数,拷贝赋值操作符和析构函数,都能通过memcpy或memmove进行拷贝。

\item 字面类型用于constexpr函数(可见A.4节),必须有简单的构造,拷贝构造和析构函数。

\item 类的默认构造、拷贝、拷贝赋值操作符和析构函数,也可以用在一个已有构造和析构函数(用户定义)的联合体内。

\item 类的简单拷贝赋值操作符可以使用\texttt{std::atomic<>}类型模板(见5.2.6节),为某种类型的值提供原子操作。
\end{itemize}

仅添加\texttt{=default}不会让函数变得简单——如果类还支持其他相关标准的函数,那这个函数就是简单的——不过,用户显式的实现就不会让这些函数变简单。

第二个区别,编译器生成函数和用户提供的函数等价,也就是类中无用户提供的构造函数可以看作为一个aggregate,并且可以通过聚合初始化函数进行初始化:

\begin{cpp}
struct aggregate
{
  aggregate() = default;
  aggregate(aggregate const&) = default;
  int a;
  double b;
};
aggregate x={42,3.141};
\end{cpp}

例子中,x.a被42初始化,x.b被3.141初始化。

第三个区别,编译器生成的函数只适用于构造函数;换句话说,只适用于符合某些标准的默认构造函数。

\begin{cpp}
struct X
{
  int a;
};
\end{cpp}

如果创建了一个X的实例(未初始化),其中int(a)将会被默认初始化。如果对象有静态存储过程,那么a将会被初始化为0;另外,当a没赋值的时候,其不定值可能会触发未定义行为:

\begin{cpp}
X x1;  // x1.a的值不明确
\end{cpp}

另外,当使用显示调用构造函数的方式对X进行初始化,a就会被初始化为0:

\begin{cpp}
X x2 = X();  // x2.a == 0
\end{cpp}

这种奇怪的属性会扩展到基础类和成员函数中。当类的默认构造函数是由编译器提供,并且一些数据成员和基类都是有编译器提供默认构造函数时,还有基类的数据成员和该类中的数据成员都是内置类型的时候,其值要不就是不确定的,要不就是初始化为0(与默认构造函数是否能被显式调用有关)。

虽然这条规则令人困惑,并且容易造成错误,不过也很有用。当你编写构造函数的时候,就不会用到这个特性。数据成员,通常都可以被初始化(指定了一个值或调用了显式构造函数),或不会被初始化(因为不需要):

\begin{cpp}
X::X():a(){}  // a == 0
X::X():a(42){}  // a == 42
X::X(){}  // 1
\end{cpp}

第三个例子中①,省略了对a的初始化,X中a就是一个未被初始化的非静态实例,初始化的X实例都会有静态存储过程。

通常的情况下,如果写了其他构造函数,编译器就不会生成默认构造函数。所以,想要自己写一个的时候,就意味着你放弃了这种奇怪的初始化特性。不过,将构造函数显式声明成默认,就能强制编译器为你生成一个默认构造函数,并且刚才说的那种特性会保留:

\begin{cpp}
X::X() = default;  // 应用默认初始化规则
\end{cpp}

这种特性用于原子变量(见5.2节),默认构造函数显式为默认。初始值通常都没有定义,除非具有(a)一个静态存储的过程(静态初始化为0),(b)显式调用默认构造函数,将成员初始化为0,(c)指定一个特殊的值。注意,这种情况下的原子变量,为允许静态初始化过程,构造函数会通过一个声明为constexpr(见A.4节)的值为原子变量进行初始化。