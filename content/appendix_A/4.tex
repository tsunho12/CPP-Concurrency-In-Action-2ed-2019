% # A.4 常量表达式函数

整型字面值,例如42,就是常量表达式。所以,简单的数学表达式,例如,23x2-4。可以使用其来初始化const整型变量,然后将const整型变量作为新表达的一部分:

\begin{cpp}
const int i=23;
const int two_i=i*2;
const int four=4;
const int forty_two=two_i-four;
\end{cpp}

使用常量表达式创建变量也可用在其他常量表达式中,有些事只能用常量表达式去做:

\begin{itemize}
\item 指定数组长度:
\end{itemize}

\begin{cpp}
int bounds=99;
int array[bounds];  // 错误,bounds不是一个常量表达式
const int bounds2=99;
int array2[bounds2];  // 正确,bounds2是一个常量表达式
\end{cpp}

\begin{itemize}
\item 指定非类型模板参数的值:
\end{itemize}

\begin{cpp}
template<unsigned size>
struct test
{};
test<bounds> ia;  // 错误,bounds不是一个常量表达式
test<bounds2> ia2;  // 正确,bounds2是一个常量表达式
\end{cpp}

\begin{itemize}
\item 对类中static const整型成员变量进行初始化:
\end{itemize}

\begin{cpp}
class X
{
  static const int the_answer=forty_two;
};
\end{cpp}

\begin{itemize}
\item 对内置类型进行初始化或可用于静态初始化集合:
\end{itemize}

\begin{cpp}
struct my_aggregate
{
  int a;
  int b;
};
static my_aggregate ma1={forty_two,123};  // 静态初始化
int dummy=257;
static my_aggregate ma2={dummy,dummy};  // 动态初始化
\end{cpp}

\begin{itemize}
\item 静态初始化可以避免初始化顺序和条件变量的问题。
\end{itemize}

这些都不是新添加的——可以在1998版本的C++标准中找到对应上面实例的条款。不过,C++11标准中常量表达式进行了扩展,并添加了新的关键字——\texttt{constexpr}。C++14和C++17对constexpr做了进一步的扩展:其完整性超出了附录的介绍范围。

\texttt{constexpr}会对功能进行修改,当参数和函数返回类型符合要求,并且实现很简单,那么这样的函数就能够被声明为\texttt{constexpr},这样函数可以当做常数表达式来使用:

\begin{cpp}
constexpr int square(int x)
{
  return x*x;
}
int array[square(5)];
\end{cpp}

这个例子中,array有25个元素,因为square函数的声明为\texttt{constexpr}。当然,这种方式可以当做常数表达式来使用,不意味着什么情况下都是能够自动转换为常数表达式:

\begin{cpp}
int dummy=4;
int array[square(dummy)];  // 错误,dummy不是常数表达式
\end{cpp}

dummy不是常数表达式,所以square(dummy)也不是——就是一个普通函数调用——所以其不能用来指定array的长度。

\mySubsubsection{A.4.1}{常量表达式和自定义类型}

目前为止的例子都是以内置int型展开的。不过,在新C++标准库中,对于满足字面类型要求的任何类型,都可以用常量表达式来表示。

要想划分到字面类型中,需要满足一下几点:

\begin{itemize}
\item 一般的拷贝构造函数。
\item 一般的析构函数。
\item 所有成员变量都是非静态的,且基类需要是一般类型。
\item 必须具有一个一般的默认构造函数,或一个constexpr构造函数。
\end{itemize}

后面会了解一下constexpr构造函数。

现在,先将注意力集中在默认构造函数上,就像下面代码中的CX类一样。

代码A.3(一般)默认构造函数的类

\begin{cpp}
class CX
{
private:
  int a;
  int b;
public:
  CX() = default;  // 1
  CX(int a_, int b_):  // 2
    a(a_),b(b_)
  {}
  int get_a() const
  {
    return a;
  }
  int get_b() const
  {
    return b;
  }
  int foo() const
  {
    return a+b;
  }
};
\end{cpp}

注意,这里显式的声明了默认构造函数①(见A.3节),为了保存用户定义的构造函数②。因此,这种类型符合字面类型的要求,可以将其用在常量表达式中。

可以提供一个constexpr函数来创建一个实例,例如:

\begin{cpp}
constexpr CX create_cx()
{
  return CX();
}
\end{cpp}

也可以创建一个简单的constexpr函数来拷贝参数:

\begin{cpp}
constexpr CX clone(CX val)
{
  return val;
}
\end{cpp}

不过,constexpr函数只有其他constexpr函数可以进行调用。在C++14中,这个限制被解除,并且只要不修改任何非局部范围的对象,就可以在constexpr函数中完成所有的操作。在C++11中,CX类中声明成员函数和构造函数为constexpr:

\begin{cpp}
class CX
{
private:
  int a;
  int b;
public:
  CX() = default;
  constexpr CX(int a_, int b_):
    a(a_),b(b_)
  {}
  constexpr int get_a() const  // 1
  {
    return a;
  }
  constexpr int get_b()  // 2
  {
    return b;
  }
  constexpr int foo()
  {
    return a+b;
  }
};
\end{cpp}

C++11中,const对于get\_a()①来说就是多余的,因为在使用constexpr时就为const了,所以const描述符在这里会被忽略②。C++14中,这会发生变化(由于constexpr 函数的扩展能力),因此get\_b()无需再为隐式的const。

这就允许更多复杂的constexpr函数存在:

\begin{cpp}
constexpr CX make_cx(int a)
{
  return CX(a,1);
}
constexpr CX half_double(CX old)
{
  return CX(old.get_a()/2,old.get_b()*2);
}
constexpr int foo_squared(CX val)
{
  return square(val.foo());
}
int array[foo_squared(half_double(make_cx(10)))];  // 49个元素
\end{cpp}

函数都很有趣,如果想要计算数组的长度或一个整型常量,就需要使用这种方式。最大的好处是常量表达式和constexpr函数会涉及到用户定义类型的对象,可以使用这些函数对这些对象进行初始化。因为常量表达式的初始化过程是静态初始化,所以能避免条件竞争和初始化顺序的问题:

\begin{cpp}
CX si=half_double(CX(42,19));  // 静态初始化
\end{cpp}

当构造函数被声明为constexpr,且构造函数参数是常量表达式时,初始化过程就是常数初始化(可能作为静态初始化的一部分)。随着并发的发展,C++11标准中有一个重要的改变:允许用户定义构造函数进行静态初始化,就可以在初始化的时候避免条件竞争,因为静态过程能保证初始化过程在代码运行前进行。

特别是关于\texttt{std::mutex}(见3.2.1节)或\texttt{std::atomic<>}(见5.2.6节),当想要使用一个全局实例来同步其他变量的访问时,同步访问就能避免条件竞争的发生。构造函数中,互斥量不可能产生条件竞争,因此对于\texttt{std::mutex}的默认构造函数应该被声明为constexpr,为了保证互斥量初始化过程是一个静态初始化过程的一部分。

\mySubsubsection{A.4.2}{常量表达式对象}

目前,已经了解了constexpr在函数上的应用。constexpr也可以用在对象上,主要是用来做判断的。验证对象是否是使用常量表达式,constexpr构造函数或组合常量表达式进行初始化。

且这个对象需要声明为const:

\begin{cpp}
constexpr int i=45;  // ok
constexpr std::string s("hello");  // 错误,std::string不是字面类型

int foo();
constexpr int j=foo();  // 错误,foo()没有声明为constexpr
\end{cpp}

\mySubsubsection{A.4.3}{常量表达式函数的要求}

将一个函数声明为constexpr,也是有几点要求,当不满足这些要求,constexpr声明将会报编译错误。C++11中,对constexpr函数的要求如下:

\begin{itemize}
\item 所有参数都必须是字面类型。
\item 返回类型必须是字面类型。
\item 函数体内必须有一个return。
\item return的表达式需要满足常量表达式的要求。
\item 构造返回值/表达式的任何构造函数或转换操作,都需要是constexpr。
\end{itemize}

看起来很简单,要在内联函数中使用到常量表达式,返回的还是个常量表达式,还不能对任何东西进行改动。

constexpr函数就是无害的纯函数。

C++14中,要求减少了。尽管保留了无副作用纯功能概念,但允许包含更多:

\begin{itemize}
    \item 允许使用多个return语句。
    \item 可以修改在函数中创建的对象。
    \item 允许使用循环、条件和switch语句。
\end{itemize}

constexpr类成员函数,需要追加几点要求:

\begin{itemize}
\item constexpr成员函数不能是虚函数。
\item 对应类必须有字面类的成员。
\end{itemize}

constexpr构造函数的规则也有些不同:

\begin{itemize}
\item 构造函数体必须为空。
\item 每一个基类必须可初始化。
\item 每个非静态数据成员都需要初始化。
\item 初始化列表的任何表达式,必须是常量表达式。
\item 构造函数可选择要进行初始化的数据成员,并且基类必须有constexpr构造函数。
\item 任何用于构建数据成员的构造函数和转换操作,以及和初始化表达式相关的基类必须为constexpr。
\end{itemize}

这些条件同样适用于成员函数,除非函数没有返回值,也就没有return语句。另外,构造函数对初始化列表中的所有基类和数据成员进行初始化,一般的拷贝构造函数会隐式的声明为constexpr。

\mySubsubsection{A.4.4}{常量表达式和模板}

将constexpr应用于函数模板,或一个类模板的成员函数。根据参数,如果模板的返回类型不是字面类,编译器会忽略其常量表达式的声明。当模板参数类型合适,且为一般inline函数,就可以将类型写成constexpr类型的函数模板。

\begin{cpp}
template<typename T>
constexpr T sum(T a,T b)
{
  return a+b;
}
constexpr int i=sum(3,42);  // ok,sum<int>是constexpr
std::string s=
  sum(std::string("hello"),
      std::string(" world"));  // 也行,不过sum<std::string>就不是constexpr了
\end{cpp}

函数需要满足所有constexpr函数所需的条件。不能用多个constexpr来声明一个函数,因为其是一个模板,所以也会带来一些编译错误。

