% # A.5 Lambda函数

Lambda函数在C++11中的加入很是令人兴奋,因为Lambda函数能够大大简化代码复杂度(语法糖:利于理解具体的功能),避免实现调用对象。C++11的Lambda函数语法允许在需要使用的时候进行定义。能为等待函数,例如\texttt{std::condition\_variable}(如同4.1.1节中的例子)提供很好谓词函数,其语义可以用来快速的表示可访问的变量,而非使用类中函数来对成员变量进行捕获。

Lambda表达式就一个自给自足的函数,不需要传入函数仅依赖全局变量和函数,甚至都可以不用返回一个值。这样的Lambda表达式的一系列语义都需要封闭在括号中,还要以方括号作为前缀:

\begin{cpp}
[]{  // Lambda表达式以[]开始
  do_stuff();
  do_more_stuff();
}();  // 表达式结束,可以直接调用
\end{cpp}

例子中,Lambda表达式通过后面的括号调用,不过这种方式不常用。一方面,如果想要直接调用,可以在写完对应的语句后就对函数进行调用。对于函数模板,传递一个参数进去时很常见的事情,甚至可以将可调用对象作为其参数传入。可调用对象通常也需要一些参数,或返回一个值,亦或两者都有。如果想给Lambda函数传递参数,可以参考下面的Lambda函数,其使用起来就像是一个普通函数。例如,下面代码是将vector中的元素使用\texttt{std::cout}进行打印:

\begin{cpp}
std::vector<int> data=make_data();
std::for_each(data.begin(),data.end(),[](int i){std::cout<<i<<"\n";});
\end{cpp}

返回值也是很简单的,当Lambda函数体包括一个return语句,返回值的类型就作为Lambda表达式的返回类型。例如,使用一个简单的Lambda函数来等待\texttt{std::condition\_variable}(见4.1.1节)中的标志被设置。

代码A.4 Lambda函数推导返回类型

\begin{cpp}
std::condition_variable cond;
bool data_ready;
std::mutex m;
void wait_for_data()
{
  std::unique_lock<std::mutex> lk(m);
  cond.wait(lk,[]{return data_ready;});  // 1
}
\end{cpp}

Lambda的返回值传递给cond.wait()①,函数就能推断出data\_ready的类型是bool。当条件变量从等待中苏醒后,上锁阶段会调用Lambda函数,并且当data\_ready为true时,返回到wait()中。

当Lambda函数体中有多个return语句,就需要显式的指定返回类型。只有一个返回语句的时候,也可以这样做,不过这样可能会让你的Lambda函数体看起来更复杂。返回类型可以使用跟在参数列表后面的箭头(->)进行设置。如果Lambda函数没有任何参数,还需要包含(空)的参数列表,这样做是为了能显式的对返回类型进行指定。对条件变量的预测可以写成下面这种方式:

\begin{cpp}
cond.wait(lk,[]()->bool{return data_ready;});
\end{cpp}

还可以对Lambda函数进行扩展,比如:加上log信息的打印,或做更加复杂的操作:

\begin{cpp}
cond.wait(lk,[]()->bool{
  if(data_ready)
  {
    std::cout<<"Data ready"<<std::endl;
    return true;
  }
  else
  {
    std::cout<<"Data not ready, resuming wait"<<std::endl;
    return false;
  }
});
\end{cpp}

虽然简单的Lambda函数很强大,能简化代码,不过其真正的强大的地方在于对本地变量的捕获。

\mySubsubsection{A.5.1}{引用本地变量的Lambda函数}

Lambda函数使用空的\texttt{[]}(Lambda introducer)就不能引用当前范围内的本地变量;其只能使用全局变量,或将其他值以参数的形式进行传递。当想要访问一个本地变量,需要对其进行捕获。最简单的方式就是将范围内的所有本地变量都进行捕获,使用\texttt{[=]}就可以完成这样的功能。函数被创建的时候,就能对本地变量的副本进行访问了。

实践一下:

\begin{cpp}
std::function<int(int)> make_offseter(int offset)
{
  return [=](int j){return offset+j;};
}
\end{cpp}

当调用make\_offseter时,就会通过\texttt{std::function<>}函数包装返回一个新的Lambda函数体。

这个带有返回的函数添加了对参数的偏移功能。例如:

\begin{cpp}
int main()
{
  std::function<int(int)> offset_42=make_offseter(42);
  std::function<int(int)> offset_123=make_offseter(123);
  std::cout<<offset_42(12)<<","<<offset_123(12)<<std::endl;
  std::cout<<offset_42(12)<<","<<offset_123(12)<<std::endl;
}
\end{cpp}

屏幕上将打印出54,135两次,因为第一次从make\_offseter中返回,都是对参数加42。第二次调用后,make\_offseter会对参数加上123。所以,会打印两次相同的值。

这种本地变量捕获的方式相当安全,所有的东西都进行了拷贝,所以可以通过Lambda函数对表达式的值进行返回,并且可在原始函数之外的地方对其进行调用。这也不是唯一的选择,也可以选择通过引用的方式捕获本地变量。在本地变量被销毁的时候,Lambda函数会出现未定义的行为。

下面的例子,就介绍一下怎么使用\texttt{[\&]}对所有本地变量进行引用:

\begin{cpp}
int main()
{
  int offset=42;  // 1
  std::function<int(int)> offset_a=[&](int j){return offset+j;};  // 2
  offset=123;  // 3
  std::function<int(int)> offset_b=[&](int j){return offset+j;};  // 4
  std::cout<<offset_a(12)<<","<<offset_b(12)<<std::endl;  // 5
  offset=99;  // 6
  std::cout<<offset_a(12)<<","<<offset_b(12)<<std::endl;  // 7
}
\end{cpp}

之前的例子中,使用\texttt{[=]}来对要偏移的变量进行拷贝,offset\_a函数就是个使用\texttt{[\&]}捕获offset的引用的例子②。所以,offset初始化成42也没什么关系①;offset\_a(12)的例子通常会依赖与当前offset的值。在③上,offset的值会变为123,offset\_b④函数将会使用到这个值,同样第二个函数也是使用引用的方式。

现在,第一行打印信息⑤,offset为123,所以输出为135,135。不过,第二行打印信息⑦就有所不同,offset变成99⑥,所以输出为111,111。offset\_a和offset\_b都对当前值进行了加12的操作。

这些选项不会让你感觉到特别困惑,你可以选择以引用或拷贝的方式对变量进行捕获,并且还可以通过调整中括号中的表达式,来对特定的变量进行显式捕获。如果想要拷贝所有变量,可以使用\texttt{[=]},通过参考中括号中的符号,对变量进行捕获。下面的例子将会打印出1239,因为i是拷贝进Lambda函数中的,而j和k是通过引用的方式进行捕获的:

\begin{cpp}
int main()
{
  int i=1234,j=5678,k=9;
  std::function<int()> f=[=,&j,&k]{return i+j+k;};
  i=1;
  j=2;
  k=3;
  std::cout<<f()<<std::endl;
}
\end{cpp}

或者,也可以通过默认引用方式对一些变量做引用,而对一些特别的变量进行拷贝。这种情况下,就要使用\texttt{[\&]}与拷贝符号相结合的方式对列表中的变量进行拷贝捕获。下面的例子将打印出5688,因为i通过引用捕获,但j和k通过拷贝捕获:

\begin{cpp}
int main()
{
  int i=1234,j=5678,k=9;
  std::function<int()> f=[&,j,k]{return i+j+k;};
  i=1;
  j=2;
  k=3;
  std::cout<<f()<<std::endl;
}
\end{cpp}

如果只想捕获某些变量,可以忽略=或\&,仅使用变量名进行捕获就行。加上\&前缀,是将对应变量以引用的方式进行捕获,而非拷贝的方式。下面的例子将打印出5682,因为i和k是通过引用的范式获取的,而j是通过拷贝的方式:

\begin{cpp}
int main()
{
  int i=1234,j=5678,k=9;
  std::function<int()> f=[&i,j,&k]{return i+j+k;};
  i=1;
  j=2;
  k=3;
  std::cout<<f()<<std::endl;
}
\end{cpp}

最后一种方式为了确保预期的变量能捕获。当在捕获列表中引用任何不存在的变量都会引起编译错误。当选择这种方式,就要小心类成员的访问方式,确定类中是否包含一个Lambda函数的成员变量。类成员变量不能直接捕获,如果想通过Lambda方式访问类中的成员,需要在捕获列表中添加this指针。下面的例子中,Lambda捕获this后,就能访问到some\_data类中的成员:

\begin{cpp}
struct X
{
  int some_data;
  void foo(std::vector<int>& vec)
  {
    std::for_each(vec.begin(),vec.end(),
         [this](int& i){i+=some_data;});
  }
};
\end{cpp}

并发的上下文中,Lambda是很有用的,其可以作为谓词放在\texttt{std::condition\_variable::wait()}(见4.1.1节)和\texttt{std::packaged\_task<>}(见4.2.1节)中,或是用在线程池中,对小任务进行打包。也可以线程函数的方式\texttt{std::thread}的构造函数(见2.1.1),以及作为一个并行算法实现,在parallel\_for\_each()(见8.5.1节)中使用。

C++14后,Lambda表达式可以是真正通用Lamdba了,参数类型被声明为auto而不是指定类型。这种情况下,函数调用运算也是一个模板。当调用Lambda时,参数的类型可从提供的参数中推导出来,例如:

\begin{cpp}
auto f=[](auto x){ std::cout<<"x="<<x<<std::endl;};
f(42); // x is of type int; outputs "x=42"
f("hello"); // x is of type const char*; outputs "x=hello"
\end{cpp}

C++14还添加了广义捕获的概念,因此可以捕获表达式的结果,而不是对局部变量的直接拷贝或引用。最常见的方法是通过移动只移动的类型来捕获类型,而不是通过引用来捕获,例如:

\begin{cpp}
std::future<int> spawn_async_task(){
  std::promise<int> p;
  auto f=p.get_future();
  std::thread t([p=std::move(p)](){ p.set_value(find_the_answer());});
  t.detach();
  return f;
}
\end{cpp}

这里,promise通过p=std::move(p)捕获移到Lambda中,因此可以安全地分离线程,从而不用担心对局部变量的悬空引用。构建Lambda之后,p处于转移过来的状态,这就是为什么需要提前获得future的原因。