% # A.6 变参模板

变参模板:就是可以使用不定数量的参数进行特化的模板。就像你接触到的变参函数一样,printf就接受可变参数。现在,就可以给你的模板指定不定数量的参数了。变参模板在整个\texttt{C++}线程库中都有使用,例如:\texttt{std::thread}的构造函数就是一个变参类模板。从使用者的角度看,仅知道模板可以接受无限个参数就够了,不过当要写一个模板或对其工作原理很感兴趣就需要了解一些细节。

和变参函数一样,变参部分可以在参数列表章使用省略号\texttt{...}代表,变参模板需要在参数列表中使用省略号:

\begin{cpp}
template<typename ... ParameterPack>
class my_template
{};
\end{cpp}

即使主模板不是变参模板,模板进行部分特化的类中,也可以使用可变参数模板。例如,\texttt{std::packaged\_task<>}(见4.2.1节)的主模板就是一个简单的模板,这个简单的模板只有一个参数:

\begin{cpp}
template<typename FunctionType>
class packaged_task;
\end{cpp}

不过,并不是所有地方都这样定义。对于部分特化模板来说像是一个“占位符”:

\begin{cpp}
template<typename ReturnType,typename ... Args>
class packaged_task<ReturnType(Args...)>;
\end{cpp}

部分特化的类就包含实际定义的类。在第4章,可以写一个\texttt{std::packaged\_task<int(std::string,double)>}来声明一个以\texttt{std::string}和double作为参数的任务,当执行这个任务后结果会由\texttt{std::future<int>}进行保存。

声明展示了两个变参模板的附加特性。第一个比较简单:普通模板参数(例如ReturnType)和可变模板参数(Args)可以同时声明。第二个特性,展示了\texttt{Args...}特化类的模板参数列表中如何使用,为了展示实例化模板中的Args的组成类型。实际上,因为是部分特化,所以其作为一种模式进行匹配。在列表中出现的类型(被Args捕获)都会进行实例化。参数包(parameter pack)调用可变参数Args,并且使用\texttt{Args...}作为包的扩展。

和可变参函数一样,变参部分可能什么都没有,也可能有很多类型项。例如,\texttt{std::packaged\_task<my\_class()>}中ReturnType参数就是my\_class,并且Args参数包是空的,不过\texttt{std::packaged\_task<void(int,double,my\_class\&,std::string*)>}中,ReturnType为void,并且Args列表中的类型就有:int, double, my\_class\&和std::string*。

\mySubsubsection{A.6.1}{扩展参数包}

变参模板主要依赖扩展功能,因为不能限制有更多的类型添加到模板参数中。首先,列表中的参数类型使用到的时候,可以使用包扩展,比如:需要给其他模板提供类型参数。

\begin{cpp}
template<typename ... Params>
struct dummy
{
  std::tuple<Params...> data;
};
\end{cpp}

成员变量data是一个\texttt{std::tuple<>}实例,包含所有指定类型,所以dummy<int, double, char>的成员变量就为\texttt{std::tuple<int, double, char>}。可以将包扩展和普通类型相结合:

\begin{cpp}
template<typename ... Params>
struct dummy2
{
  std::tuple<std::string,Params...> data;
};
\end{cpp}

这次,元组中添加了额外的(第一个)成员类型\texttt{std::string}。其优雅之处在于,可以通过包扩展的方式创建一种模式,这种模式会在之后将每个元素拷贝到扩展之中,可以使用\texttt{...}来表示扩展模式的结束。例如,创建使用参数包来创建元组中所有的元素,不如在元组中创建指针,或使用\texttt{std::unique\_ptr<>}指针,指向对应元素:

\begin{cpp}
template<typename ... Params>
struct dummy3
{
  std::tuple<Params* ...> pointers;
  std::tuple<std::unique_ptr<Params> ...> unique_pointers;
};
\end{cpp}

类型表达式会比较复杂,提供的参数包是在类型表达式中产生,并且表达式中使用\texttt{...}作为扩展。当参数包已经扩展 ,包中的每一项都会代替对应的类型表达式,在结果列表中产生相应的数据项。因此,当参数包Params包含int,int,char类型,那么\texttt{std::tuple<std::pair<std::unique\_ptr<Params>,double> ... >}将扩展为\texttt{std::tuple<std::pair<std::unique\_ptr<int>,double>},\texttt{std::pair<std::unique\_ptr<int>,double>},\texttt{std::pair<std::unique\_ptr<char>, double> >}。如果包扩展当做模板参数列表使用时,模板就不需要变长的参数了。如果不需要了,参数包就要对模板参数的要求进行准确的匹配:

\begin{cpp}
template<typename ... Types>
struct dummy4
{
  std::pair<Types...> data;
};
dummy4<int,char> a;  // 1 ok,为std::pair<int, char>
dummy4<int> b;  // 2 错误,无第二个类型
dummy4<int,int,int> c;  // 3 错误,类型太多
\end{cpp}

可以使用包扩展的方式,对函数的参数进行声明:

\begin{cpp}
template<typename ... Args>
void foo(Args ... args);
\end{cpp}

这将会创建一个新参数包args,其是一组函数参数,而非一组类型,并且这里\texttt{...}也能像之前一样进行扩展。例如,可以在\texttt{std::thread}的构造函数中使用,使用右值引用的方式获取函数所有的参数(见A.1节):

\begin{cpp}
template<typename CallableType,typename ... Args>
thread::thread(CallableType&& func,Args&& ... args);
\end{cpp}

函数参数包也可以用来调用其他函数,将制定包扩展成参数列表,匹配调用的函数。如同类型扩展一样,也可以使用某种模式对参数列表进行扩展。例如,使用\texttt{std::forward()}以右值引用的方式来保存提供给函数的参数:

\begin{cpp}
template<typename ... ArgTypes>
void bar(ArgTypes&& ... args)
{
  foo(std::forward<ArgTypes>(args)...);
}
\end{cpp}

注意一下这个例子,包扩展包括对类型包ArgTypes和函数参数包args的扩展,并且省略了其余的表达式。当这样调用bar函数:

\begin{cpp}
int i;
bar(i,3.141,std::string("hello "));
\end{cpp}

将会扩展为

\begin{cpp}
template<>
void bar<int&,double,std::string>(
         int& args_1,
         double&& args_2,
         std::string&& args_3)
{
  foo(std::forward<int&>(args_1),
      std::forward<double>(args_2),
      std::forward<std::string>(args_3));
}
\end{cpp}

这样就将第一个参数以左值引用的形式,正确的传递给了foo函数,其他两个函数都是以右值引用的方式传入的。

最后一件事,参数包中使用\texttt{sizeof...}操作可以获取类型参数类型的大小,\texttt{sizeof...(p)}就是p参数包中所包含元素的个数。不管是类型参数包或函数参数包,结果都是一样的。这可能是唯一一次在使用参数包的时候,没有加省略号;这里的省略号是作为\texttt{sizeof...}操作的一部分,所以不算是用到省略号。

下面的函数会返回参数的数量:

\begin{cpp}
template<typename ... Args>
unsigned count_args(Args ... args)
{
  return sizeof... (Args);
}
\end{cpp}

就像普通的sizeof操作一样,\texttt{sizeof...}的结果为常量表达式,所以其可以用来指定定义数组长度,等等。