% # A.7 自动推导变量类型

C++是静态语言:所有变量的类型,都会在编译时被准确指定。所以,作为开发者需要为每个变量指定对应的类型。有些时候就需要使用一些繁琐类型定义,比如:

\begin{cpp}
std::map<std::string,std::unique_ptr<some_data>> m;
std::map<std::string,std::unique_ptr<some_data>>::iterator
      iter=m.find("my key");
\end{cpp}

常规的解决办法是使用typedef来缩短类型名的长度。这种方式在C++11中仍然可行,不过这里要介绍一种新的解决办法:如果一个变量需要通过一个已初始化的变量类型来为其做声明,那么就可以直接使用`auto`关键字。这样,编译器就会通过已初始化的变量,去自动推断变量的类型。

\begin{cpp}
auto iter=m.find("my key");
\end{cpp}

当然,`auto`还有很多种用法:可以使用它来声明const、指针或引用变量。这里使用`auto`对相关类型进行了声明:

\begin{cpp}
auto i=42; // int
auto& j=i; // int&
auto const k=i; // int const
auto* const p=&i; // int * const
\end{cpp}

变量类型的推导规则是建立一些语言规则基础上:函数模板参数。其声明形式如下:

\begin{cpp}
some-type-expression-involving-auto var=some-expression;
\end{cpp}

var变量的类型与声明函数模板的参数的类型相同。要想替换`auto`,需要使用完整的类型参数:

\begin{cpp}
template<typename T>
void f(type-expression var);
f(some-expression);
\end{cpp}

在使用`auto`的时候,数组类型将衰变为指针,引用将会被删除(除非将类型进行显式为引用),比如:

\begin{cpp}
int some_array[45];
auto p=some_array; // int*
int& r=*p;
auto x=r; // int
auto& y=r; // int&
\end{cpp}

这样能大大简化变量的声明过程,特别是在类型标识符特别长,或不清楚具体类型的时候(例如,调用函数模板,等到的目标值类型就是不确定的)。