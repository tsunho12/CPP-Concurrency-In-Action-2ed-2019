% # A.8 线程本地变量

线程本地变量允许程序中的每个线程都有一个独立的实例拷贝,可以使用\texttt{thread\_local}关键字来对这样的变量进行声明。命名空间内的变量,静态成员变量,以及本地变量都可以声明成线程本地变量,为了在线程运行前对这些数据进行存储操作:

\begin{cpp}
thread_local int x;  // 命名空间内的线程本地变量

class X
{
  static thread_local std::string s;  // 线程本地的静态成员变量
};
static thread_local std::string X::s;  // 这里需要添加X::s

void foo()
{
  thread_local std::vector<int> v;  // 一般线程本地变量
}
\end{cpp}

由命名空间或静态数据成员构成的线程本地变量,需要在线程单元对其进行使用**前**进行构建。有些实现中,会将对线程本地变量的初始化过程放在线程中去做,还有一些可能会在其他时间点做初始化。在一些有依赖的组合中,会根据具体情况来进行决定。将没有构造好的线程本地变量传递给线程单元使用,不能保证它们会在线程中进行构造。这样就可以动态加载带有线程本地变量的模块——变量首先需要在一个给定的线程中进行构造,之后其他线程就可以通过动态加载模块对线程本地变量进行引用。

函数中声明的线程本地变量,需要使用一个给定线程进行初始化(通过第一波控制流将这些声明传递给指定线程)。如果指定线程调用没有调用函数,那么这个函数中声明的线程本地变量就不会构造。本地静态变量也是同样的情况,除非其单独的应用于每一个线程。

静态变量与线程本地变量会共享一些属性——它们可以做进一步的初始化(比如,动态初始化)。如果在构造线程本地变量时抛出异常,\texttt{std::terminate()}就会将程序终止。

析构函数会在构造线程本地变量的那个线程返回时调用,析构顺序是构造的逆序。当初始化顺序没有指定时,确定析构函数和这些变量是否有相互依存关系就尤为重要了。当线程本地变量的析构函数抛出异常时,\texttt{std::terminate()}会被调用,将程序终止。

当线程调用\texttt{std::exit()}或从main()函数返回(等价于调用\texttt{std::exit()}作为main()的“返回值”)时,线程本地变量也会为了这个线程进行销毁。应用退出时还有线程在运行,对于这些线程来说,线程本地变量的析构函数就没有被调用。

虽然,线程本地变量在不同线程上有不同的地址,不过还是可以获取指向这些变量的一般指针。指针会在线程中,通过获取地址的方式,引用相应的对象。当引用被销毁的对象时,会出现未定义行为,所以在向其他线程传递线程本地变量指针时,就需要保证指向对象所在的线程结束后,不对相应的指针进行解引用。