% # 附录B 并发库的简单比较

虽然,C++11才开始正式支持并发,不过高级编程语言都支持并发和多线程已经不是什么新鲜事了。例如,Java在第一个发布版本中就支持多线程编程,在某些平台上也提供符合POSIX C标准的多线程接口,还有\href{www.erlang.org}{Erlang}支持消息的同步传递(有点类似于MPI)。当然还有使用C++类的库,比如Boost,其将底层多线程接口进行包装,适用于任何给定的平台(不论是使用POSIX C的接口,或其他接口),其对支持的平台会提供可移植接口。

这些库或者编程语言,已经写了很多多线程应用,并且在使用这些库写多线程代码的经验,可以借鉴到C++中,本附录就对Java,POSIX C,使用Boost线程库的C++,以及C++11中的多线程工具进行简单的比较,当然也会交叉引用本书的相关章节。

\begin{landscape}
\begin{table}[h]
\centering
\begin{adjustbox}{max width=\textwidth}
\begin{tabular}{|c|c|}
\hline
\textbf{特性} & \textbf{内容} \\
\hline
\multirow{8}{*}{章节引用} & 启动线程:第2章 \\
 & 互斥量:第3章 \\
 & 监视/等待谓词:第4章 \\
 & 原子操作和并发感知内存模型:第5章 \\
 & 线程安全容器:第6章和第7章 \\
 & Futures(期望):第4章 \\
 & 线程池:第9章 \\
 & 线程中断:第9章 \\
\hline
\multirow{8}{*}{C++11} & 启动线程:std::thread和其成员函数 \\
 & 互斥量:std::mutex类和其成员函数,std::lock\_guard<>模板,std::unique\_lock<>模板 \\
 & 监视/等待谓词:std::condition\_variable,std::condition\_variable\_any类和其成员函数 \\
 & 原子操作和并发感知内存模型:std::atomic\_xxx类型,std::atomic<>类模板,std::atomic\_thread\_fence()函数 \\
 & 线程安全容器:N/A \\
 & Futures(期望):std::future<>,std::shared\_future<>,std::atomic\_future<>类模板 \\
 & 线程池:N/A \\
 & 线程中断:N/A \\
\hline
\multirow{7}{*}{Boost线程库} & 启动线程:boost::thread类和成员函数 \\
 & 互斥量:boost::mutex类和其成员函数,boost::lock\_guard<>类模板,boost::unique\_lock<>类模板 \\
 & 监视/等待谓词:boost::condition\_variable类和其成员函数,boost::condition\_variable\_any类和其成员函数 \\
 & 原子操作和并发感知内存模型:N/A \\
 & 线程安全容器:N/A \\
 & Futures(期望):boost::unique\_future<>类模板,boost::shared\_future<>类模板 \\
 & 线程池:N/A \\
 & 线程中断:boost::thread类的interrupt()成员函数 \\
\hline
\multirow{8}{*}{POSIX C} & 启动线程:pthread\_t类型相关的API函数,pthread\_create(),pthread\_detach(),pthread\_join() \\
 & 互斥量:pthread\_mutex\_t类型相关的API函数,pthread\_mutex\_lock(),pthread\_mutex\_unlock() \\
 & 监视/等待谓词:pthread\_cond\_t类型相关的API函数,pthread\_cond\_wait(),pthread\_cond\_timed\_wait() \\
 & 原子操作和并发感知内存模型:N/A \\
 & 线程安全容器:N/A \\
 & Futures(期望):N/A \\
 & 线程池:N/A \\
 & 线程中断:pthread\_cancel() \\
\hline
\multirow{8}{*}{Java} & 启动线程:java.lang.thread类 \\
 & 互斥量:synchronized块 \\
 & 监视/等待谓词:java.lang.Object类的wait()和notify()函数,用在内部synchronized块中 \\
 & 原子操作和并发感知内存模型:java.util.concurrent.atomic包中的volatile类型变量 \\
 & 线程安全容器:java.util.concurrent包中的容器 \\
 & Futures(期望):与java.util.concurrent.future接口相关的类 \\
 & 线程池:java.util.concurrent.ThreadPoolExecutor类 \\
 & 线程中断:java.lang.Thread类的interrupt()函数 \\
\hline
\end{tabular}
\end{adjustbox}
% \caption{线程相关特性比较}
\end{table}
\end{landscape}

