% # D.2 <condition\_variable>头文件

<condition\_variable>头文件提供了条件变量的定义。其作为基本同步机制,允许被阻塞的线程在某些条件达成或超时时,解除阻塞继续执行。

%### 头文件内容

\begin{cpp}
namespace std
{
  enum class cv_status { timeout, no_timeout };

  class condition_variable;
  class condition_variable_any;
}
\end{cpp}

\mySubsubsection{D.2.1} {std::condition\_variable类}

\texttt{std::condition\_variable}允许阻塞一个线程,直到条件达成。

\texttt{std::condition\_variable}实例不支持CopyAssignable(拷贝赋值), CopyConstructible(拷贝构造), MoveAssignable(移动赋值)和 MoveConstructible(移动构造)。

%### 类型定义

\begin{cpp}
class condition_variable
{
public:
  condition_variable();
  ~condition_variable();

  condition_variable(condition_variable const& ) = delete;
  condition_variable& operator=(condition_variable const& ) = delete;

  void notify_one() noexcept;
  void notify_all() noexcept;

  void wait(std::unique_lock<std::mutex>& lock);

  template <typename Predicate>
  void wait(std::unique_lock<std::mutex>& lock,Predicate pred);

  template <typename Clock, typename Duration>
  cv_status wait_until(
       std::unique_lock<std::mutex>& lock,
       const std::chrono::time_point<Clock, Duration>& absolute_time);

  template <typename Clock, typename Duration, typename Predicate>
  bool wait_until(
       std::unique_lock<std::mutex>& lock,
       const std::chrono::time_point<Clock, Duration>& absolute_time,
       Predicate pred);

  template <typename Rep, typename Period>
  cv_status wait_for(
       std::unique_lock<std::mutex>& lock,
       const std::chrono::duration<Rep, Period>& relative_time);

  template <typename Rep, typename Period, typename Predicate>
  bool wait_for(
       std::unique_lock<std::mutex>& lock,
       const std::chrono::duration<Rep, Period>& relative_time,
       Predicate pred);
};

void notify_all_at_thread_exit(condition_variable&,unique_lock<mutex>);
\end{cpp}

% ### std::condition\_variable 默认构造函数

构造一个\texttt{std::condition\_variable}对象。

\textbf{声明}

\begin{cpp}
condition_variable();
\end{cpp}

\textbf{效果}
构造一个新的\texttt{std::condition\_variable}实例。

\textbf{抛出}
当条件变量无法够早的时候,将会抛出一个\texttt{std::system\_error}异常。

%### std::condition\_variable 析构函数

销毁一个\texttt{std::condition\_variable}对象。

\textbf{声明}

\begin{cpp}
~condition_variable();
\end{cpp}

\textbf{先决条件}
之前没有使用*this总的wait(),wait\_for()或wait\_until()阻塞过线程。

\textbf{效果}
销毁*this。

\textbf{抛出}
无

%### std::condition\_variable::notify\_one 成员函数

唤醒一个等待当前\texttt{std::condition\_variable}实例的线程。

\textbf{声明}

\begin{cpp}
void notify_one() noexcept;
\end{cpp}

\textbf{效果}
唤醒一个等待*this的线程。如果没有线程在等待,那么调用没有任何效果。

\textbf{抛出}
当效果没有达成,就会抛出\texttt{std::system\_error}异常。

\textbf{同步}
\texttt{std::condition\_variable}实例中的notify\_one(),notify\_all(),wait(),wait\_for()和wait\_until()都是序列化函数(串行调用)。调用notify\_one()或notify\_all()只能唤醒正在等待中的线程。

%### std::condition\_variable::notify\_all 成员函数

唤醒所有等待当前\texttt{std::condition\_variable}实例的线程。

\textbf{声明}

\begin{cpp}
void notify_all() noexcept;
\end{cpp}

\textbf{效果}
唤醒所有等待*this的线程。如果没有线程在等待,那么调用没有任何效果。

\textbf{抛出}
当效果没有达成,就会抛出\texttt{std::system\_error}异常

\textbf{同步}
\texttt{std::condition\_variable}实例中的notify\_one(),notify\_all(),wait(),wait\_for()和wait\_until()都是序列化函数(串行调用)。调用notify\_one()或notify\_all()只能唤醒正在等待中的线程。

%### std::condition\_variable::wait 成员函数

通过\texttt{std::condition\_variable}的notify\_one()、notify\_all()或伪唤醒结束等待。

\textbf{等待}

\begin{cpp}
void wait(std::unique_lock<std::mutex>& lock);
\end{cpp}

\textbf{先决条件}
当线程调用wait()即可获得锁的所有权,lock.owns\_lock()必须为true。

\textbf{效果}
自动解锁lock对象,对于线程等待线程,当其他线程调用notify\_one()或notify\_all()时被唤醒,亦或该线程处于伪唤醒状态。在wait()返回前,lock对象将会再次上锁。

\textbf{抛出}
当效果没有达成的时候,将会抛出\texttt{std::system\_error}异常。当lock对象在调用wait()阶段被解锁,那么当wait()退出的时候lock会再次上锁,即使函数是通过异常的方式退出。

\textbf{效果}:伪唤醒意味着一个线程调用wait()后,在没有其他线程调用notify\_one()或notify\_all()时,还处以苏醒状态。因此,建议对wait()进行重载,在可能的情况下使用一个谓词。否则,建议wait()使用循环检查与条件变量相关的谓词。

\textbf{同步}
\texttt{std::condition\_variable}实例中的notify\_one(),notify\_all(),wait(),wait\_for()和wait\_until()都是序列化函数(串行调用)。调用notify\_one()或notify\_all()只能唤醒正在等待中的线程。

%### std::condition\_variable::wait 需要一个谓词的成员函数重载

等待\texttt{std::condition\_variable}上的notify\_one()或notify\_all()被调用,或谓词为true的情况,来唤醒线程。

\textbf{声明}

\begin{cpp}
template<typename Predicate>
void wait(std::unique_lock<std::mutex>& lock,Predicate pred);
\end{cpp}

\textbf{先决条件}
pred()谓词必须是合法的,并且需要返回一个值,这个值可以和bool互相转化。当线程调用wait()即可获得锁的所有权,lock.owns\_lock()必须为true。

\textbf{效果}
正如

\begin{cpp}
while(!pred())
{
  wait(lock);
}
\end{cpp}

\textbf{抛出}
pred中可以抛出任意异常,或者当效果没有达到的时候,抛出\texttt{std::system\_error}异常。

\textbf{效果}:潜在的伪唤醒意味着不会指定pred调用的次数。通过lock进行上锁,pred经常会被互斥量引用所调用,并且函数必须返回(只能返回)一个值,在\texttt{(bool)pred()}评估后,返回true。

\textbf{同步}
\texttt{std::condition\_variable}实例中的notify\_one(),notify\_all(),wait(),wait\_for()和wait\_until()都是序列化函数(串行调用)。调用notify\_one()或notify\_all()只能唤醒正在等待中的线程。

%### std::condition\_variable::wait\_for 成员函数

\texttt{std::condition\_variable}在调用notify\_one()、调用notify\_all()、超时或线程伪唤醒时,结束等待。

\textbf{声明}

\begin{cpp}
template<typename Rep,typename Period>
cv_status wait_for(
    std::unique_lock<std::mutex>& lock,
    std::chrono::duration<Rep,Period> const& relative_time);
\end{cpp}

\textbf{先决条件}
当线程调用wait()即可获得锁的所有权,lock.owns\_lock()必须为true。

\textbf{效果}
当其他线程调用notify\_one()或notify\_all()函数时,或超出了relative\_time的时间,亦或是线程被伪唤醒,则将lock对象自动解锁,并将阻塞线程唤醒。当wait\_for()调用返回前,lock对象会再次上锁。

\textbf{返回}
线程被notify\_one()、notify\_all()或伪唤醒唤醒时,会返回\texttt{std::cv\_status::no\_timeout};反之,则返回\texttt{std::cv\_status::timeout}。

\textbf{抛出}
当效果没有达成的时候,会抛出\texttt{std::system\_error}异常。当lock对象在调用wait\_for()函数前解锁,那么lock对象会在wait\_for()退出前再次上锁,即使函数是以异常的方式退出。

\textbf{效果}:伪唤醒意味着,一个线程在调用wait\_for()的时候,即使没有其他线程调用notify\_one()和notify\_all()函数,也处于苏醒状态。因此,这里建议重载wait\_for()函数,重载函数可以使用谓词。要不,则建议wait\_for()使用循环的方式对与谓词相关的条件变量进行检查。在这样做的时候还需要小心,以确保超时部分依旧有效;wait\_until()可能适合更多的情况。这样的话,线程阻塞的时间就要比指定的时间长了。在有这样可能性的地方,流逝的时间是由稳定时钟决定。

\textbf{同步}
\texttt{std::condition\_variable}实例中的notify\_one(),notify\_all(),wait(),wait\_for()和wait\_until()都是序列化函数(串行调用)。调用notify\_one()或notify\_all()只能唤醒正在等待中的线程。

%### std::condition\_variable::wait\_for 需要一个谓词的成员函数重载

\texttt{std::condition\_variable}在调用notify\_one()、调用notify\_all()、超时或线程伪唤醒时,结束等待。

\textbf{声明}

\begin{cpp}
template<typename Rep,typename Period,typename Predicate>
bool wait_for(
    std::unique_lock<std::mutex>& lock,
    std::chrono::duration<Rep,Period> const& relative_time,
    Predicate pred);
\end{cpp}

\textbf{先决条件}
pred()谓词必须是合法的,并且需要返回一个值,这个值可以和bool互相转化。当线程调用wait()即可获得锁的所有权,lock.owns\_lock()必须为true。

\textbf{效果}
等价于

\begin{cpp}
internal_clock::time_point end=internal_clock::now()+relative_time;
while(!pred())
{
  std::chrono::duration<Rep,Period> remaining_time=
      end-internal_clock::now();
  if(wait_for(lock,remaining_time)==std::cv_status::timeout)
      return pred();
}
return true;
\end{cpp}

\textbf{返回}
当pred()为true,则返回true;当超过relative\_time并且pred()返回false时,返回false。

\textbf{效果}:潜在的伪唤醒意味着不会指定pred调用的次数。通过lock进行上锁,pred经常会被互斥量引用所调用,并且函数必须返回(只能返回)一个值,在\texttt{(bool)pred()}评估后返回true,或在指定时间relative\_time内完成。线程阻塞的时间就要比指定的时间长了。在有这样可能性的地方,流逝的时间是由稳定时钟决定。

\textbf{抛出}
当效果没有达成时,会抛出\texttt{std::system\_error}异常或者由pred抛出任意异常。

\textbf{同步}
\texttt{std::condition\_variable}实例中的notify\_one(),notify\_all(),wait(),wait\_for()和wait\_until()都是序列化函数(串行调用)。调用notify\_one()或notify\_all()只能唤醒正在等待中的线程。

%### std::condition\_variable::wait\_until 成员函数

\texttt{std::condition\_variable}在调用notify\_one()、调用notify\_all()、指定时间内达成条件或线程伪唤醒时,结束等待。

\textbf{声明}

\begin{cpp}
template<typename Clock,typename Duration>
cv_status wait_until(
    std::unique_lock<std::mutex>& lock,
    std::chrono::time_point<Clock,Duration> const& absolute_time);
\end{cpp}

\textbf{先决条件}
当线程调用wait()即可获得锁的所有权,lock.owns\_lock()必须为true。

\textbf{效果}
当其他线程调用notify\_one()或notify\_all()函数,或Clock::now()返回一个大于或等于absolute\_time的时间,亦或线程伪唤醒,lock都将自动解锁,并且唤醒阻塞的线程。在wait\_until()返回之前lock对象会再次上锁。

\textbf{返回}
线程被notify\_one()、notify\_all()或伪唤醒唤醒时,会返回\texttt{std::cv\_status::no\_timeout};反之,则返回\texttt{std::cv\_status::timeout}。

\textbf{抛出}
当效果没有达成的时候,会抛出\texttt{std::system\_error}异常。当lock对象在调用wait\_for()函数前解锁,那么lock对象会在wait\_for()退出前再次上锁,即使函数是以异常的方式退出。

\textbf{效果}:伪唤醒意味着一个线程调用wait()后,在没有其他线程调用notify\_one()或notify\_all()时,还处以苏醒状态。因此,这里建议重载wait\_until()函数,重载函数可以使用谓词。要不,则建议wait\_until()使用循环的方式对与谓词相关的条件变量进行检查。这里不保证线程会被阻塞多长时间,只有当函数返回false后(Clock::now()的返回值大于或等于absolute\_time),线程才能解除阻塞。

\textbf{同步}
\texttt{std::condition\_variable}实例中的notify\_one(),notify\_all(),wait(),wait\_for()和wait\_until()都是序列化函数(串行调用)。调用notify\_one()或notify\_all()只能唤醒正在等待中的线程。

%### std::condition\_variable::wait\_until 需要一个谓词的成员函数重载

\texttt{std::condition\_variable}在调用notify\_one()、调用notify\_all()、谓词返回true或指定时间内达到条件,结束等待。

\textbf{声明}

\begin{cpp}
template<typename Clock,typename Duration,typename Predicate>
bool wait_until(
    std::unique_lock<std::mutex>& lock,
    std::chrono::time_point<Clock,Duration> const& absolute_time,
    Predicate pred);
\end{cpp}

\textbf{先决条件}
pred()必须是合法的,并且其返回值能转换为bool值。当线程调用wait()即可获得锁的所有权,lock.owns\_lock()必须为true。

\textbf{效果}
等价于

\begin{cpp}
while(!pred())
{
  if(wait_until(lock,absolute_time)==std::cv_status::timeout)
    return pred();
}
return true;
\end{cpp}

\textbf{返回}
当调用pred()返回true时,返回true;当Clock::now()的时间大于或等于指定的时间absolute\_time,并且pred()返回false时,返回false。

\textbf{效果}:潜在的伪唤醒意味着不会指定pred调用的次数。通过lock进行上锁,pred经常会被互斥量引用所调用,并且函数必须返回(只能返回)一个值,在\texttt{(bool)pred()}评估后返回true,或Clock::now()返回的时间大于或等于absolute\_time。这里不保证调用线程将被阻塞的时长,只有当函数返回false后(Clock::now()返回一个等于或大于absolute\_time的值),线程接触阻塞。

\textbf{抛出}
当效果没有达成时,会抛出\texttt{std::system\_error}异常或者由pred抛出任意异常。

\textbf{同步}
\texttt{std::condition\_variable}实例中的notify\_one(),notify\_all(),wait(),wait\_for()和wait\_until()都是序列化函数(串行调用)。调用notify\_one()或notify\_all()只能唤醒正在等待中的线程。

%### std::notify\_all\_at\_thread\_exit 非成员函数

当当前调用函数的线程退出时,等待\texttt{std::condition\_variable}的所有线程将会被唤醒。

\textbf{声明}

\begin{cpp}
void notify_all_at_thread_exit(
  condition_variable& cv,unique_lock<mutex> lk);
\end{cpp}

\textbf{先决条件}
当线程调用wait()即可获得锁的所有权,lk.owns\_lock()必须为true。lk.mutex()需要返回的值要与并发等待线程相关的任意cv中锁对象提供的wait(),wait\_for()或wait\_until()相同。

\textbf{效果}
将lk的所有权转移到内部存储中,并且当有线程退出时,安排被提醒的cv类。这里的提醒等价于

\begin{cpp}
lk.unlock();
cv.notify_all();
\end{cpp}

\textbf{抛出}
当效果没有达成时,抛出\texttt{std::system\_error}异常。

\textbf{效果}:在线程退出前,掌握着锁的所有权,所以这里要避免死锁发生。这里建议调用该函数的线程应该尽快退出,并且在该线程可以执行一些阻塞的操作。用户必须保证等地线程不会错误的将唤醒线程当做已退出的线程,特别是伪唤醒。可以通过等待线程上的谓词测试来实现这一功能,在互斥量保护的情况下,只有谓词返回true时线程才能被唤醒,并且在调用notify\_all\_at\_thread\_exit(std::condition\_variable\_any类中函数)前是不会释放锁。

\mySubsubsection{D.2.2}{std::condition\_variable\_any类}

\texttt{std::condition\_variable\_any}类允许线程等待某一条件为true的时候继续运行。不过\texttt{std::condition\_variable}只能和\texttt{std::unique\_lock<std::mutex>}一起使用,\texttt{std::condition\_variable\_any}可以和任意可上锁(Lockable)类型一起使用。

\texttt{std::condition\_variable\_any}实例不能进行拷贝赋值(CopyAssignable)、拷贝构造(CopyConstructible)、移动赋值(MoveAssignable)或移动构造(MoveConstructible)。

%### 类型定义

\begin{cpp}
class condition_variable_any
{
public:
  condition_variable_any();
  ~condition_variable_any();

  condition_variable_any(
      condition_variable_any const& ) = delete;
  condition_variable_any& operator=(
      condition_variable_any const& ) = delete;

  void notify_one() noexcept;
  void notify_all() noexcept;

  template<typename Lockable>
  void wait(Lockable& lock);

  template <typename Lockable, typename Predicate>
  void wait(Lockable& lock, Predicate pred);

  template <typename Lockable, typename Clock,typename Duration>
  std::cv_status wait_until(
      Lockable& lock,
      const std::chrono::time_point<Clock, Duration>& absolute_time);

  template <
      typename Lockable, typename Clock,
      typename Duration, typename Predicate>
  bool wait_until(
      Lockable& lock,
      const std::chrono::time_point<Clock, Duration>& absolute_time,
      Predicate pred);

  template <typename Lockable, typename Rep, typename Period>
  std::cv_status wait_for(
      Lockable& lock,
      const std::chrono::duration<Rep, Period>& relative_time);

  template <
      typename Lockable, typename Rep,
      typename Period, typename Predicate>
  bool wait_for(
      Lockable& lock,
      const std::chrono::duration<Rep, Period>& relative_time,
      Predicate pred);
};
\end{cpp}

%### std::condition_variable_any 默认构造函数

构造一个\texttt{std::condition\_variable\_any}对象。

\textbf{声明}

\begin{cpp}
condition_variable_any();
\end{cpp}

\textbf{效果}
构造一个新的\texttt{std::condition\_variable\_any}实例。

\textbf{抛出}
当条件变量构造成功,将抛出\texttt{std::system\_error}异常。

%### std::condition\_variable\_any 析构函数

销毁\texttt{std::condition\_variable\_any}对象。

\textbf{声明}

\begin{cpp}
~condition_variable_any();
\end{cpp}

\textbf{先决条件}
之前没有使用*this总的wait(),wait\_for()或wait\_until()阻塞过线程。

\textbf{效果}
销毁*this。

\textbf{抛出}
无

%### std::condition\_variable\_any::notify\_one 成员函数

\texttt{std::condition\_variable\_any}唤醒一个等待该条件变量的线程。

\textbf{声明}

\begin{cpp}
void notify_all() noexcept;
\end{cpp}

\textbf{效果}
唤醒一个等待*this的线程。如果没有线程在等待,那么调用没有任何效果

\textbf{抛出}
当效果没有达成,就会抛出std::system\_error异常。

\textbf{同步}
\texttt{std::condition\_variable}实例中的notify\_one(),notify\_all(),wait(),wait\_for()和wait\_until()都是序列化函数(串行调用)。调用notify\_one()或notify\_all()只能唤醒正在等待中的线程。

%### std::condition\_variable\_any::notify\_all 成员函数

唤醒所有等待当前\texttt{std::condition\_variable\_any}实例的线程。

\textbf{声明}

\begin{cpp}
void notify_all() noexcept;
\end{cpp}

\textbf{效果}
唤醒所有等待*this的线程。如果没有线程在等待,那么调用没有任何效果

\textbf{抛出}
当效果没有达成,就会抛出std::system\_error异常。

\textbf{同步}
\texttt{std::condition\_variable}实例中的notify\_one(),notify\_all(),wait(),wait\_for()和wait\_until()都是序列化函数(串行调用)。调用notify\_one()或notify\_all()只能唤醒正在等待中的线程。

%### std::condition\_variable\_any::wait 成员函数

通过\texttt{std::condition\_variable\_any}的notify\_one()、notify\_all()或伪唤醒结束等待。

\textbf{声明}

\begin{cpp}
template<typename Lockable>
void wait(Lockable& lock);
\end{cpp}

\textbf{先决条件}
Lockable类型需要能够上锁,lock对象拥有一个锁。

\textbf{效果}
自动解锁lock对象,对于线程等待线程,当其他线程调用notify\_one()或notify\_all()时被唤醒,亦或该线程处于伪唤醒状态。在wait()返回前,lock对象将会再次上锁。

\textbf{抛出}
当效果没有达成的时候,将会抛出\texttt{std::system\_error}异常。当lock对象在调用wait()阶段被解锁,那么当wait()退出的时候lock会再次上锁,即使函数是通过异常的方式退出。

\textbf{效果}:伪唤醒意味着一个线程调用wait()后,在没有其他线程调用notify\_one()或notify\_all()时,还处以苏醒状态。因此,建议对wait()进行重载,在可能的情况下使用一个谓词。否则,建议wait()使用循环检查与条件变量相关的谓词。

\textbf{同步}
std::condition\_variable\_any实例中的notify\_one(),notify\_all(),wait(),wait\_for()和wait\_until()都是序列化函数(串行调用)。调用notify\_one()或notify\_all()只能唤醒正在等待中的线程。

%### std::condition\_variable\_any::wait 需要一个谓词的成员函数重载

等待\texttt{std::condition\_variable\_any}上的notify\_one()或notify\_all()被调用,或谓词为true的情况,来唤醒线程。

\textbf{声明}

\begin{cpp}
template<typename Lockable,typename Predicate>
void wait(Lockable& lock,Predicate pred);
\end{cpp}

\textbf{先决条件}
pred()谓词必须是合法的,并且需要返回一个值,这个值可以和bool互相转化。当线程调用wait()即可获得锁的所有权,lock.owns\_lock()必须为true。

\textbf{效果}
正如

\begin{cpp}
while(!pred())
{
wait(lock);
}
\end{cpp}

\textbf{抛出}
pred中可以抛出任意异常,或者当效果没有达到的时候,抛出\texttt{std::system\_error}异常。

\textbf{效果}:潜在的伪唤醒意味着不会指定pred调用的次数。通过lock进行上锁,pred经常会被互斥量引用所调用,并且函数必须返回(只能返回)一个值,在\texttt{(bool)pred()}评估后,返回true。

\textbf{同步}
\texttt{std::condition\_variable\_any}实例中的notify\_one(),notify\_all(),wait(),wait\_for()和wait\_until()都是序列化函数(串行调用)。调用notify\_one()或notify\_all()只能唤醒正在等待中的线程。

%### std::condition\_variable\_any::wait\_for 成员函数

\texttt{std::condition\_variable\_any}在调用notify\_one()、调用notify\_all()、超时或线程伪唤醒时,结束等待。

\textbf{声明}

\begin{cpp}
template<typename Lockable,typename Rep,typename Period>
std::cv_status wait_for(
    Lockable& lock,
    std::chrono::duration<Rep,Period> const& relative_time);
\end{cpp}

\textbf{先决条件}
当线程调用wait()即可获得锁的所有权,lock.owns\_lock()必须为true。

\textbf{效果}
当其他线程调用notify\_one()或notify\_all()函数时,或超出了relative\_time的时间,亦或是线程被伪唤醒,则将lock对象自动解锁,并将阻塞线程唤醒。当wait\_for()调用返回前,lock对象会再次上锁。

\textbf{返回}
线程被notify\_one()、notify\_all()或伪唤醒唤醒时,会返回\texttt{std::cv\_status::no\_timeout};反之,则返回std::cv\_status::timeout。

\textbf{抛出}
当效果没有达成的时候,会抛出\texttt{std::system\_error}异常。当lock对象在调用wait\_for()函数前解锁,那么lock对象会在wait\_for()退出前再次上锁,即使函数是以异常的方式退出。

\textbf{效果}:伪唤醒意味着,一个线程在调用wait\_for()的时候,即使没有其他线程调用notify\_one()和notify\_all()函数,也处于苏醒状态。因此,这里建议重载wait\_for()函数,重载函数可以使用谓词。要不,则建议wait\_for()使用循环的方式对与谓词相关的条件变量进行检查。在这样做的时候还需要小心,以确保超时部分依旧有效;wait\_until()可能适合更多的情况。这样的话,线程阻塞的时间就要比指定的时间长了。在有这样可能性的地方,流逝的时间是由稳定时钟决定。

\textbf{同步}
\texttt{std::condition\_variable\_any}实例中的notify\_one(),notify\_all(),wait(),wait\_for()和wait\_until()都是序列化函数(串行调用)。调用notify\_one()或notify\_all()只能唤醒正在等待中的线程。

%### std::condition\_variable\_any::wait\_for 需要一个谓词的成员函数重载

\texttt{std::condition\_variable\_any}在调用notify\_one()、调用notify\_all()、超时或线程伪唤醒时,结束等待。

\textbf{声明}

\begin{cpp}
template<typename Lockable,typename Rep,
    typename Period, typename Predicate>
bool wait_for(
    Lockable& lock,
    std::chrono::duration<Rep,Period> const& relative_time,
    Predicate pred);
  \end{cpp}

\textbf{先决条件}
pred()谓词必须是合法的,并且需要返回一个值,这个值可以和bool互相转化。当线程调用wait()即可获得锁的所有权,lock.owns\_lock()必须为true。

\textbf{效果}
正如

\begin{cpp}
internal_clock::time_point end=internal_clock::now()+relative_time;
while(!pred())
{
  std::chrono::duration<Rep,Period> remaining_time=
      end-internal_clock::now();
  if(wait_for(lock,remaining_time)==std::cv_status::timeout)
      return pred();
}
return true;
\end{cpp}

\textbf{返回}
当pred()为true,则返回true;当超过relative\_time并且pred()返回false时,返回false。

\textbf{效果}:
潜在的伪唤醒意味着不会指定pred调用的次数。通过lock进行上锁,pred经常会被互斥量引用所调用,并且函数必须返回(只能返回)一个值,在(bool)pred()评估后返回true,或在指定时间relative\_time内完成。线程阻塞的时间就要比指定的时间长了。在有这样可能性的地方,流逝的时间是由稳定时钟决定。

\textbf{抛出}
当效果没有达成时,会抛出\texttt{std::system\_error}异常或者由pred抛出任意异常。

\textbf{同步}
\texttt{std::condition\_variable\_any}实例中的notify\_one(),notify\_all(),wait(),wait\_for()和wait\_until()都是序列化函数(串行调用)。调用notify\_one()或notify\_all()只能唤醒正在等待中的线程。

%### std::condition\_variable\_any::wait\_until 成员函数

\texttt{std::condition\_variable\_any}在调用notify\_one()、调用notify\_all()、指定时间内达成条件或线程伪唤醒时,结束等待

\textbf{声明}

\begin{cpp}
template<typename Lockable,typename Clock,typename Duration>
std::cv_status wait_until(
    Lockable& lock,
    std::chrono::time_point<Clock,Duration> const& absolute_time);
\end{cpp}

\textbf{先决条件}
Lockable类型需要能够上锁,lock对象拥有一个锁。

\textbf{效果}
当其他线程调用notify\_one()或notify\_all()函数,或Clock::now()返回一个大于或等于absolute\_time的时间,亦或线程伪唤醒,lock都将自动解锁,并且唤醒阻塞的线程。在wait\_until()返回之前lock对象会再次上锁。

\textbf{返回}
线程被notify\_one()、notify\_all()或伪唤醒唤醒时,会返回std::cv\_status::no\_timeout;反之,则返回\texttt{std::cv\_status::timeout}。

\textbf{抛出}
当效果没有达成的时候,会抛出\texttt{std::system\_error}异常。当lock对象在调用wait\_for()函数前解锁,那么lock对象会在wait\_for()退出前再次上锁,即使函数是以异常的方式退出。

\textbf{效果}:伪唤醒意味着一个线程调用wait()后,在没有其他线程调用notify\_one()或notify\_all()时,还处以苏醒状态。因此,这里建议重载wait\_until()函数,重载函数可以使用谓词。要不,则建议wait\_until()使用循环的方式对与谓词相关的条件变量进行检查。这里不保证线程会被阻塞多长时间,只有当函数返回false后(Clock::now()的返回值大于或等于absolute\_time),线程才能解除阻塞。

\textbf{同步}
\texttt{std::condition\_variable\_any}实例中的notify\_one(),notify\_all(),wait(),wait\_for()和wait\_until()都是序列化函数(串行调用)。调用notify\_one()或notify\_all()只能唤醒正在等待中的线程。

%### std::condition\_variable\_any::wait\_unti 需要一个谓词的成员函数重载

\texttt{std::condition\_variable\_any}在调用notify\_one()、调用notify\_all()、谓词返回true或指定时间内达到条件,结束等待。

\textbf{声明}

\begin{cpp}
template<typename Lockable,typename Clock,
    typename Duration, typename Predicate>
bool wait_until(
    Lockable& lock,
    std::chrono::time_point<Clock,Duration> const& absolute_time,
    Predicate pred);
\end{cpp}

\textbf{先决条件}
pred()必须是合法的,并且其返回值能转换为bool值。当线程调用wait()即可获得锁的所有权,lock.owns\_lock()必须为true。

\textbf{效果}
等价于

\begin{cpp}
while(!pred())
{
  if(wait_until(lock,absolute_time)==std::cv_status::timeout)
    return pred();
}
return true;
\end{cpp}

\textbf{返回}
当调用pred()返回true时,返回true;当Clock::now()的时间大于或等于指定的时间absolute\_time,并且pred()返回false时,返回false。

\textbf{效果}:潜在的伪唤醒意味着不会指定pred调用的次数。通过lock进行上锁,pred经常会被互斥量引用所调用,并且函数必须返回(只能返回)一个值,在(bool)pred()评估后返回true,或Clock::now()返回的时间大于或等于absolute\_time。这里不保证调用线程将被阻塞的时长,只有当函数返回false后(Clock::now()返回一个等于或大于absolute\_time的值),线程接触阻塞。

\textbf{抛出}
当效果没有达成时,会抛出\texttt{std::system\_error}异常或者由pred抛出任意异常。

\textbf{同步}
\texttt{std::condition\_variable\_any}实例中的notify\_one(),notify\_all(),wait(),wait\_for()和wait\_until()都是序列化函数(串行调用)。调用notify\_one()或notify\_all()只能唤醒正在等待中的线程。