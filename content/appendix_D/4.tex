% # D.4 <future>头文件

\texttt{<future>}头文件提供处理异步结果(在其他线程上执行额结果)的工具。

\textbf{头文件内容}

\begin{cpp}
namespace std
{
  enum class future_status {
      ready, timeout, deferred };

  enum class future_errc
  {
    broken_promise,
    future_already_retrieved,
    promise_already_satisfied,
    no_state
  };

  class future_error;

  const error_category& future_category();

  error_code make_error_code(future_errc e);
  error_condition make_error_condition(future_errc e);

  template<typename ResultType>
  class future;

  template<typename ResultType>
  class shared_future;

  template<typename ResultType>
  class promise;

  template<typename FunctionSignature>
  class packaged_task; // no definition provided

  template<typename ResultType,typename ... Args>
  class packaged_task<ResultType (Args...)>;

  enum class launch {
    async, deferred
  };

  template<typename FunctionType,typename ... Args>
  future<result_of<FunctionType(Args...)>::type>
  async(FunctionType&& func,Args&& ... args);

  template<typename FunctionType,typename ... Args>
  future<result_of<FunctionType(Args...)>::type>
  async(std::launch policy,FunctionType&& func,Args&& ... args);
}
\end{cpp}

\mySubsubsection{D.4.1}{std::future类型模板}

\texttt{std::future}类型模板是为了等待其他线程上的异步结果。其和\texttt{std::promise},\texttt{std::packaged\_task}类型模板,还有\texttt{std::async}函数模板,都是为异步结果准备的工具。只有\texttt{std::future}实例可以在任意时间引用异步结果。

\texttt{std::future}实例是MoveConstructible(移动构造)和MoveAssignable(移动赋值),不过不能CopyConstructible(拷贝构造)和CopyAssignable(拷贝赋值)。

\textbf{类型声明}

\begin{cpp}
template<typename ResultType>
class future
{
public:
  future() noexcept;
  future(future&&) noexcept;
  future& operator=(future&&) noexcept;
  ~future();

  future(future const&) = delete;
  future& operator=(future const&) = delete;

  shared_future<ResultType> share();

  bool valid() const noexcept;

  see description get();

  void wait();

  template<typename Rep,typename Period>
  future_status wait_for(
      std::chrono::duration<Rep,Period> const& relative_time);

  template<typename Clock,typename Duration>
  future_status wait_until(
      std::chrono::time_point<Clock,Duration> const& absolute_time);
};
\end{cpp}

% ### std::future 默认构造函数

不使用异步结果构造一个\texttt{std::future}对象。

\textbf{声明}

\begin{cpp}
future() noexcept;
\end{cpp}

\textbf{效果}
构造一个新的\texttt{std::future}实例。

\textbf{后置条件}
valid()返回false。

\textbf{抛出}
无

% ### std::future 移动构造函数

使用另外一个对象,构造一个\texttt{std::future}对象,将相关异步结果的所有权转移给新\texttt{std::future}对象。

\textbf{声明}

\begin{cpp}
future(future&& other) noexcept;
\end{cpp}

\textbf{效果}
使用已有对象构造一个新的\texttt{std::future}对象。

\textbf{后置条件}
已有对象中的异步结果,将于新的对象相关联。然后,解除已有对象和异步之间的关系。\texttt{this->valid()}返回的结果与之前已有对象\texttt{other.valid()}返回的结果相同。在调用该构造函数后,\texttt{other.valid()}将返回false。

\textbf{抛出}
无

% ### std::future 移动赋值操作

将已有\texttt{std::future}对象中异步结果的所有权,转移到另一对象当中。

\textbf{声明}

\begin{cpp}
future(future&& other) noexcept;
\end{cpp}

\textbf{效果}
在两个\texttt{std::future}实例中转移异步结果的状态。

\textbf{后置条件}
当执行完赋值操作后,\texttt{*this.other}就与异步结果没有关系了。异步状态(如果有的话)在释放后与\texttt{*this}相关,并且在最后一次引用后,销毁该状态。\texttt{this->valid()}返回的结果与之前已有对象\texttt{other.valid()}返回的结果相同。在调用该构造函数后,\texttt{other.valid()}将返回false。

\textbf{抛出}
无

% ### std::future 析构函数

销毁一个\texttt{std::future}对象。

\textbf{声明}

\begin{cpp}
~future();
\end{cpp}

\textbf{效果}
销毁\texttt{*this}。如果这是最后一次引用与\texttt{*this}相关的异步结果,之后就会将该异步结果销毁。

\textbf{抛出}
无

% ### std::future::share 成员函数

构造一个新\texttt{std::shared\_future}实例,并且将\texttt{*this}异步结果的所有权转移到新的\texttt{std::shared\_future}实例中。

\textbf{声明}

\begin{cpp}
shared_future<ResultType> share();
\end{cpp}

\textbf{效果}
如同 shared\_future<ResultType>(std::move(*this))。

\textbf{后置条件}
当调用share()成员函数,与\texttt{*this}相关的异步结果将与新构造的\texttt{std::shared\_future}实例相关。\texttt{this->valid()}将返回false。

\textbf{抛出}
无

% ### std::future::valid 成员函数

检查\texttt{std::future}实例是否与一个异步结果相关联。

\textbf{声明}

\begin{cpp}
bool valid() const noexcept;
\end{cpp}

\textbf{返回}
当与异步结果相关时,返回true,否则返回false。

\textbf{抛出}
无

% ### std::future::wait 成员函数

如果与\texttt{*this}相关的状态包含延迟函数,将调用该函数。否则,会等待\texttt{std::future}实例中的异步结果准备就绪。

\textbf{声明}

\begin{cpp}
void wait();
\end{cpp}

\textbf{先决条件}
\texttt{this->valid()}将会返回true。

\textbf{效果}
当相关状态包含延迟函数,调用延迟函数,并保存返回的结果,或将抛出的异常保存成为异步结果。否则,会阻塞到\texttt{*this}准备就绪。

\textbf{抛出}
无

% ### std::future::wait\_for 成员函数

等待\texttt{std::future}实例上相关异步结果准备就绪,或超过某个给定的时间。

\textbf{声明}

\begin{cpp}
template<typename Rep,typename Period>
future_status wait_for(
    std::chrono::duration<Rep,Period> const& relative_time);
\end{cpp}

\textbf{先决条件}
\texttt{this->valid()}将会返回true。

\textbf{效果}
如果与\texttt{*this}相关的异步结果包含一个\texttt{std::async}调用的延迟函数(还未执行),那么就不阻塞立即返回。否则将阻塞实例,直到与\texttt{*this}相关异步结果准备就绪,或超过给定的relative\_time时长。

\textbf{返回}
当与\texttt{*this}相关的异步结果包含一个\texttt{std::async}调用的延迟函数(还未执行),返回\texttt{std::future\_status::deferred};当与\texttt{*this}相关的异步结果准备就绪,返回\texttt{std::future\_status::ready};当给定时间超过relative\_time时,返回\texttt{std::future\_status::timeout}。

\textbf{NOTE}:线程阻塞的时间可能超多给定的时长。时长尽可能由一个稳定的时钟决定。

\textbf{抛出}
无

% ### std::future::wait\_until 成员函数

等待\texttt{std::future}实例上相关异步结果准备就绪,或超过某个给定的时间。

\textbf{声明}

\begin{cpp}
template<typename Clock,typename Duration>
future_status wait_until(
  std::chrono::time_point<Clock,Duration> const& absolute_time);
\end{cpp}

\textbf{先决条件}
this->valid()将返回true。

\textbf{效果}
如果与\texttt{*this}相关的异步结果包含一个\texttt{std::async}调用的延迟函数(还未执行),那么就不阻塞立即返回。否则将阻塞实例,直到与\texttt{*this}相关异步结果准备就绪,或\texttt{Clock::now()}返回的时间大于等于absolute\_time。

\textbf{返回}
当与\texttt{*this}相关的异步结果包含一个\texttt{std::async}调用的延迟函数(还未执行),返回\texttt{std::future\_status::deferred};当与\texttt{*this}相关的异步结果准备就绪,返回\texttt{std::future\_status::ready};\texttt{Clock::now()}返回的时间大于等于absolute\_time,返回\texttt{std::future\_status::timeout}。

\textbf{NOTE}:这里不保证调用线程会被阻塞多久,只有函数返回\texttt{std::future\_status::timeout},然后\texttt{Clock::now()}返回的时间大于等于absolute\_time的时候,线程才会解除阻塞。

\textbf{抛出}
无

% ### std::future::get 成员函数

当相关状态包含一个\texttt{std::async}调用的延迟函数,调用该延迟函数,并返回结果;否则,等待与\texttt{std::future}实例相关的异步结果准备就绪,之后返回存储的值或异常。

\textbf{声明}

\begin{cpp}
void future<void>::get();
R& future<R&>::get();
R future<R>::get();
\end{cpp}

\textbf{先决条件}
this->valid()将返回true。

\textbf{效果}
如果*this相关状态包含一个延期函数,那么调用这个函数并返回结果,或将抛出的异常进行传播。

否则,线程就要被阻塞,直到与*this相关的异步结果就绪。当结果存储了一个异常,那么就就会将存储异常抛出。否则,将会返回存储值。

\textbf{返回}
当相关状态包含一个延期函数,那么这个延期函数的结果将被返回。否则,当ResultType为void时,就会按照常规调用返回。如果ResultType是R\&(R类型的引用),存储的引用值将会被返回。否则,存储的值将会返回。

\textbf{抛出}
异常由延期函数,或存储在异步结果中的异常(如果有的话)抛出。

\textbf{后置条件}

\begin{cpp}
this->valid()==false
\end{cpp}

\mySubsubsection{D.4.2}{std::shared\_future类型模板}

\texttt{std::shared\_future}类型模板是为了等待其他线程上的异步结果。其和\texttt{std::promise},\texttt{std::packaged\_task}类型模板,还有\texttt{std::async}函数模板,都是为异步结果准备的工具。多个\texttt{std::shared\_future}实例可以引用同一个异步结果。

\texttt{std::shared\_future}实例是CopyConstructible(拷贝构造)和CopyAssignable(拷贝赋值)。你也可以同ResultType的\texttt{std::future}类型对象,移动构造一个\texttt{std::shared\_future}类型对象。

访问给定\texttt{std::shared\_future}实例是非同步的。因此,当有多个线程访问同一个\texttt{std::shared\_future}实例,且无任何外围同步操作时,这样的访问是不安全的。不过访问关联状态时是同步的,所以多个线程访问多个独立的\texttt{std::shared\_future}实例,且没有外围同步操作的时候,是安全的。

\textbf{类型定义}

\begin{cpp}
template<typename ResultType>
class shared_future
{
public:
  shared_future() noexcept;
  shared_future(future<ResultType>&&) noexcept;

  shared_future(shared_future&&) noexcept;
  shared_future(shared_future const&);
  shared_future& operator=(shared_future const&);
  shared_future& operator=(shared_future&&) noexcept;
  ~shared_future();

  bool valid() const noexcept;

  see description get() const;

  void wait() const;

  template<typename Rep,typename Period>
  future_status wait_for(
     std::chrono::duration<Rep,Period> const& relative_time) const;

  template<typename Clock,typename Duration>
  future_status wait_until(
     std::chrono::time_point<Clock,Duration> const& absolute_time)
    const;
};
\end{cpp}

% ### std::shared\_future 默认构造函数

不使用关联异步结果,构造一个\texttt{std::shared\_future}对象。

\textbf{声明}

\begin{cpp}
shared_future() noexcept;
\end{cpp}

\textbf{效果}
构造一个新的\texttt{std::shared\_future}实例。

\textbf{后置条件}
当新实例构建完成后,调用valid()将返回false。

\textbf{抛出}
无

% ### std::shared\_future 移动构造函数

以一个已创建\texttt{std::shared\_future}对象为准,构造\texttt{std::shared\_future}实例,并将使用\texttt{std::shared\_future}对象关联的异步结果的所有权转移到新的实例中。

\textbf{声明}

\begin{cpp}
shared_future(shared_future&& other) noexcept;
\end{cpp}

\textbf{效果}
构造一个新\texttt{std::shared\_future}实例。

\textbf{后置条件}
将other对象中关联异步结果的所有权转移到新对象中,这样other对象就没有与之相关联的异步结果了。

\textbf{抛出}
无

% ### std::shared\_future 移动对应std::future对象的构造函数

以一个已创建\texttt{std::future}对象为准,构造\texttt{std::shared\_future}实例,并将使用\texttt{std::shared\_future}对象关联的异步结果的所有权转移到新的实例中。

\textbf{声明}

\begin{cpp}
shared_future(std::future<ResultType>&& other) noexcept;
\end{cpp}

\textbf{效果}
构造一个\texttt{std::shared\_future}对象。

\textbf{后置条件}
将other对象中关联异步结果的所有权转移到新对象中,这样other对象就没有与之相关联的异步结果了。

\textbf{抛出}
无

% ### std::shared\_future 拷贝构造函数

以一个已创建\texttt{std::future}对象为准,构造\texttt{std::shared\_future}实例,并将使用\texttt{std::shared\_future}对象关联的异步结果(如果有的话)拷贝到新创建对象当中,两个对象共享该异步结果。

\textbf{声明}

\begin{cpp}
shared_future(shared_future const& other);
\end{cpp}

\textbf{效果}
构造一个\texttt{std::shared\_future}对象。

\textbf{后置条件}
将other对象中关联异步结果拷贝到新对象中,与other共享关联的异步结果。

\textbf{抛出}
无

% ### std::shared\_future 析构函数

销毁一个\texttt{std::shared\_future}对象。

\textbf{声明}

\begin{cpp}
~shared_future();
\end{cpp}

\textbf{效果}
将\texttt{*this}销毁。如果\texttt{*this}关联的异步结果与\texttt{std::promise}或\texttt{std::packaged\_task}不再有关联,那么该函数将会切断\texttt{std::shared\_future}实例与异步结果的联系,并销毁异步结果。

\textbf{抛出}
无

% ### std::shared\_future::valid 成员函数

检查\texttt{std::shared\_future}实例是否与一个异步结果相关联。

\textbf{声明}

\begin{cpp}
bool valid() const noexcept;
\end{cpp}

\textbf{返回}
当与异步结果相关时,返回true,否则返回false。

\textbf{抛出}
无

% ### std::shared\_future::wait 成员函数

当*this关联状态包含一个延期函数,那么调用这个函数。否则,等待直到与\texttt{std::shared\_future}实例相关的异步结果就绪为止。

\textbf{声明}

\begin{cpp}
void wait() const;
\end{cpp}

\textbf{先决条件}
this->valid()将返回true。

\textbf{效果}
当有多个线程调用\texttt{std::shared\_future}实例上的get()和wait()时,实例会序列化的共享同一关联状态。如果关联状态包括一个延期函数,那么第一个调用get()或wait()时就会调用延期函数,并且存储返回值,或将抛出异常以异步结果的方式保存下来。

\textbf{抛出}
无

% ### std::shared\_future::wait\_for 成员函数

等待\texttt{std::shared\_future}实例上相关异步结果准备就绪,或超过某个给定的时间。

\textbf{声明}

\begin{cpp}
template<typename Rep,typename Period>
future_status wait_for(
    std::chrono::duration<Rep,Period> const& relative_time) const;
\end{cpp}

\textbf{先决条件}
\texttt{this->valid()}将会返回true。

\textbf{效果}
如果与\texttt{*this}相关的异步结果包含一个\texttt{std::async}调用的延期函数(还未执行),那么就不阻塞立即返回。否则将阻塞实例,直到与\texttt{*this}相关异步结果准备就绪,或超过给定的relative\_time时长。

\textbf{返回}
当与\texttt{*this}相关的异步结果包含一个\texttt{std::async}调用的延迟函数(还未执行),返回\texttt{std::future\_status::deferred};当与\texttt{*this}相关的异步结果准备就绪,返回\texttt{std::future\_status::ready};当给定时间超过relative\_time时,返回\texttt{std::future\_status::timeout}。

\textbf{NOTE}:线程阻塞的时间可能超多给定的时长。时长尽可能由一个稳定的时钟决定。

\textbf{抛出}
无

% ### std::shared\_future::wait\_until 成员函数

等待\texttt{std::future}实例上相关异步结果准备就绪,或超过某个给定的时间。

\textbf{声明}

\begin{cpp}
template<typename Clock,typename Duration>
future_status wait_until(
  std::chrono::time_point<Clock,Duration> const& absolute_time) const;
\end{cpp}

\textbf{先决条件}
this->valid()将返回true。

\textbf{效果}
如果与\texttt{*this}相关的异步结果包含一个\texttt{std::async}调用的延迟函数(还未执行),那么就不阻塞立即返回。否则将阻塞实例,直到与\texttt{*this}相关异步结果准备就绪,或\texttt{Clock::now()}返回的时间大于等于absolute\_time。

\textbf{返回}
当与\texttt{*this}相关的异步结果包含一个\texttt{std::async}调用的延迟函数(还未执行),返回\texttt{std::future\_status::deferred};当与\texttt{*this}相关的异步结果准备就绪,返回\texttt{std::future\_status::ready};\texttt{Clock::now()}返回的时间大于等于absolute\_time,返回\texttt{std::future\_status::timeout}。

\textbf{NOTE}:这里不保证调用线程会被阻塞多久,只有函数返回\texttt{std::future\_status::timeout},然后\texttt{Clock::now()}返回的时间大于等于absolute\_time的时候,线程才会解除阻塞。

\textbf{抛出}
无

% ### std::shared\_future::get 成员函数

当相关状态包含一个\texttt{std::async}调用的延迟函数,调用该延迟函数,并返回结果;否则,等待与\texttt{std::shared\_future}实例相关的异步结果准备就绪,之后返回存储的值或异常。

\textbf{声明}

\begin{cpp}
void shared_future<void>::get() const;
R& shared_future<R&>::get() const;
R const& shared_future<R>::get() const;
\end{cpp}

\textbf{先决条件}
this->valid()将返回true。

\textbf{效果}
当有多个线程调用\texttt{std::shared\_future}实例上的get()和wait()时,实例会序列化的共享同一关联状态。如果关联状态包括一个延期函数,那么第一个调用get()或wait()时就会调用延期函数,并且存储返回值,或将抛出异常以异步结果的方式保存下来。

阻塞会知道*this关联的异步结果就绪后解除。当异步结果存储了一个一行,那么就会抛出这个异常。否则,返回存储的值。

\textbf{返回}
当ResultType为void时,就会按照常规调用返回。如果ResultType是R\&(R类型的引用),存储的引用值将会被返回。否则,返回存储值的const引用。

\textbf{抛出}
抛出存储的异常(如果有的话)。

\mySubsubsection{D.4.3}{std::packaged\_task类型模板}

\texttt{std::packaged\_task}类型模板可打包一个函数或其他可调用对象,所以当函数通过\texttt{std::packaged\_task}实例被调用时,结果将会作为异步结果。这个结果可以通过检索\texttt{std::future}实例来查找。

\texttt{std::packaged\_task}实例是可以MoveConstructible(移动构造)和MoveAssignable(移动赋值),不过不能CopyConstructible(拷贝构造)和CopyAssignable(拷贝赋值)。

\textbf{类型定义}

\begin{cpp}
template<typename FunctionType>
class packaged_task; // undefined

template<typename ResultType,typename... ArgTypes>
class packaged_task<ResultType(ArgTypes...)>
{
public:
  packaged_task() noexcept;
  packaged_task(packaged_task&&) noexcept;
  ~packaged_task();

  packaged_task& operator=(packaged_task&&) noexcept;

  packaged_task(packaged_task const&) = delete;
  packaged_task& operator=(packaged_task const&) = delete;

  void swap(packaged_task&) noexcept;

  template<typename Callable>
  explicit packaged_task(Callable&& func);

  template<typename Callable,typename Allocator>
  packaged_task(std::allocator_arg_t, const Allocator&,Callable&&);

  bool valid() const noexcept;
  std::future<ResultType> get_future();
  void operator()(ArgTypes...);
  void make_ready_at_thread_exit(ArgTypes...);
  void reset();
};
\end{cpp}

% ### std::packaged\_task 默认构造函数

构造一个\texttt{std::packaged\_task}对象。

\textbf{声明}

\begin{cpp}
packaged_task() noexcept;
\end{cpp}

\textbf{效果}
不使用关联任务或异步结果来构造一个\texttt{std::packaged\_task}对象。

\textbf{抛出}
无

% ### std::packaged\_task 通过可调用对象构造

使用关联任务和异步结果,构造一个\texttt{std::packaged\_task}对象。

\textbf{声明}

\begin{cpp}
template<typename Callable>
packaged_task(Callable&& func);
\end{cpp}

\textbf{先决条件}
表达式\texttt{func(args...)}必须是合法的,并且在\texttt{args...}中的args-i参数,必须是\texttt{ArgTypes...}中ArgTypes-i类型的一个值。且返回值必须可转换为ResultType。

\textbf{效果}
使用ResultType类型的关联异步结果,构造一个\texttt{std::packaged\_task}对象,异步结果是未就绪的,并且Callable类型相关的任务是对func的一个拷贝。

\textbf{抛出}
当构造函数无法为异步结果分配出内存时,会抛出\texttt{std::bad\_alloc}类型的异常。其他异常会在使用Callable类型的拷贝或移动构造过程中抛出。

% ### std::packaged\_task 通过有分配器的可调用对象构造

使用关联任务和异步结果,构造一个\texttt{std::packaged\_task}对象。使用以提供的分配器为关联任务和异步结果分配内存。

\textbf{声明}

\begin{cpp}
template<typename Allocator,typename Callable>
packaged_task(
    std::allocator_arg_t, Allocator const& alloc,Callable&& func);
\end{cpp}

\textbf{先决条件}
表达式\texttt{func(args...)}必须是合法的,并且在\texttt{args...}中的args-i参数,必须是\texttt{ArgTypes...}中ArgTypes-i类型的一个值。且返回值必须可转换为ResultType。

\textbf{效果}
使用ResultType类型的关联异步结果,构造一个\texttt{std::packaged\_task}对象,异步结果是未就绪的,并且Callable类型相关的任务是对func的一个拷贝。异步结果和任务的内存通过内存分配器alloc进行分配,或进行拷贝。

\textbf{抛出}
当构造函数无法为异步结果分配出内存时,会抛出\texttt{std::bad\_alloc}类型的异常。其他异常会在使用Callable类型的拷贝或移动构造过程中抛出。

% ### std::packaged\_task 移动构造函数

通过一个\texttt{std::packaged\_task}对象构建另一个,将与已存在的\texttt{std::packaged\_task}相关的异步结果和任务的所有权转移到新构建的对象当中。

\textbf{声明}

\begin{cpp}
packaged_task(packaged_task&& other) noexcept;
\end{cpp}

\textbf{效果}
构建一个新的\texttt{std::packaged\_task}实例。

\textbf{后置条件}
通过other构建新的\texttt{std::packaged\_task}对象。在新对象构建完成后,other与其之前相关联的异步结果就没有任何关系了。

\textbf{抛出}
无

% ### std::packaged\_task 移动赋值操作

将一个\texttt{std::packaged\_task}对象相关的异步结果的所有权转移到另外一个。

\textbf{声明}

\begin{cpp}
packaged_task& operator=(packaged_task&& other) noexcept;
\end{cpp}

\textbf{效果}
将other相关异步结果和任务的所有权转移到\texttt{*this}中,并且切断异步结果和任务与other对象的关联,如同\texttt{std::packaged\_task(other).swap(*this)}。

\textbf{后置条件}
与other相关的异步结果与任务移动转移,使*this.other无关联的异步结果。

\textbf{返回}

\begin{cpp}
*this
\end{cpp}

\textbf{抛出}
无

% ### std::packaged\_task::swap 成员函数

将两个\texttt{std::packaged\_task}对象所关联的异步结果的所有权进行交换。

\textbf{声明}

\begin{cpp}
void swap(packaged_task& other) noexcept;
\end{cpp}

\textbf{效果}
将other和*this关联的异步结果与任务进行交换。

\textbf{后置条件}
将与other关联的异步结果和任务,通过调用swap的方式,与*this相交换。

\textbf{抛出}
无

% ### std::packaged\_task 析构函数

销毁一个\texttt{std::packaged\_task}对象。

\textbf{声明}

\begin{cpp}
~packaged_task();
\end{cpp}

\textbf{效果}
将\texttt{*this}销毁。如果\texttt{*this}有关联的异步结果,并且结果不是一个已存储的任务或异常,那么异步结果状态将会变为就绪,伴随就绪的是一个\texttt{std::future\_error}异常和错误码\texttt{std::future\_errc::broken\_promise}。

\textbf{抛出}
无

% ### std::packaged\_task::get\_future 成员函数

在*this相关异步结果中,检索一个\texttt{std::future}实例。

\textbf{声明}

\begin{cpp}
std::future<ResultType> get_future();
\end{cpp}

\textbf{先决条件}
*this具有关联异步结果。

\textbf{返回}
一个与*this关联异构结果相关的一个\texttt{std::future}实例。

\textbf{抛出}
如果一个\texttt{std::future}已经通过get\_future()获取了异步结果,在抛出\texttt{std::future\_error}异常时,错误码是\texttt{std::future\_errc::future\_already\_retrieved}

% ### std::packaged\_task::reset 成员函数

将一个\texttt{std::packaged\_task}对实例与一个新的异步结果相关联。

\textbf{声明}

\begin{cpp}
void reset();
\end{cpp}

\textbf{先决条件}
*this具有关联的异步任务。

\textbf{效果}
如同\texttt{*this=packaged\_task(std::move(f))},f是*this中已存储的关联任务。

\textbf{抛出}
如果内存不足以分配给新的异构结果,那么将会抛出\texttt{std::bad\_alloc}类异常。

% ### std::packaged\_task::valid 成员函数

检查*this中是都具有关联任务和异步结果。

\textbf{声明}

\begin{cpp}
bool valid() const noexcept;
\end{cpp}

\textbf{返回}
当*this具有相关任务和异步结构,返回true;否则,返回false。

\textbf{抛出}
无

% ### std::packaged\_task::operator() 函数调用操作

调用一个\texttt{std::packaged\_task}实例中的相关任务,并且存储返回值,或将异常存储到异常结果当中。

\textbf{声明}

\begin{cpp}
void operator()(ArgTypes... args);
\end{cpp}

\textbf{先决条件}
*this具有相关任务。

\textbf{效果}
像\texttt{INVOKE(func,args...)}那要调用相关的函数func。如果返回征程,那么将会存储到*this相关的异步结果中。当返回结果是一个异常,将这个异常存储到*this相关的异步结果中。

\textbf{后置条件}
*this相关联的异步结果状态为就绪,并且存储了一个值或异常。所有阻塞线程,在等待到异步结果的时候被解除阻塞。

\textbf{抛出}
当异步结果已经存储了一个值或异常,那么将抛出一个\texttt{std::future\_error}异常,错误码为\texttt{std::future\_errc::promise\_already\_satisfied}。

\textbf{同步}
\texttt{std::future<ResultType>::get()}或\texttt{std::shared\_future<ResultType>::get()}的成功调用,代表同步操作的成功,函数将会检索异步结果中的值或异常。

% ### std::packaged\_task::make\_ready\_at\_thread\_exit 成员函数

调用一个\texttt{std::packaged\_task}实例中的相关任务,并且存储返回值,或将异常存储到异常结果当中,直到线程退出时,将相关异步结果的状态置为就绪。

\textbf{声明}

\begin{cpp}
void make_ready_at_thread_exit(ArgTypes... args);
\end{cpp}

\textbf{先决条件}
*this具有相关任务。

\textbf{效果}
像\texttt{INVOKE(func,args...)}那要调用相关的函数func。如果返回征程,那么将会存储到\texttt{*this}相关的异步结果中。当返回结果是一个异常,将这个异常存储到\texttt{*this}相关的异步结果中。当当前线程退出的时候,可调配相关异步状态为就绪。

\textbf{后置条件}
*this的异步结果中已经存储了一个值或一个异常,不过在当前线程退出的时候,这个结果都是非就绪的。当当前线程退出时,阻塞等待异步结果的线程将会被解除阻塞。

\textbf{抛出}
当异步结果已经存储了一个值或异常,那么将抛出一个\texttt{std::future\_error}异常,错误码为\texttt{std::future\_errc::promise\_already\_satisfied}。当无关联异步状态时,抛出\texttt{std::future\_error}异常,错误码为\texttt{std::future\_errc::no\_state}。

\textbf{同步}
\texttt{std::future<ResultType>::get()}或\texttt{std::shared\_future<ResultType>::get()}在线程上的成功调用,代表同步操作的成功,函数将会检索异步结果中的值或异常。

\mySubsubsection{D.4.4}{std::promise类型模板}

\texttt{std::promise}类型模板提供设置异步结果的方法,这样其他线程就可以通过\texttt{std::future}实例来索引该结果。

ResultType模板参数,该类型可以存储异步结果。

\texttt{std::promise}实例中的异步结果与某个\texttt{std::future}实例相关联,并且可以通过调用get\_future()成员函数来获取这个\texttt{std::future}实例。ResultType类型的异步结果,可以通过set\_value()成员函数对存储值进行设置,或者使用set\_exception()将对应异常设置进异步结果中。

\texttt{std::promise}实例是可以MoveConstructible(移动构造)和MoveAssignable(移动赋值),但是不能CopyConstructible(拷贝构造)和CopyAssignable(拷贝赋值)。

\textbf{类型定义}

\begin{cpp}
template<typename ResultType>
class promise
{
public:
  promise();
  promise(promise&&) noexcept;
  ~promise();
  promise& operator=(promise&&) noexcept;

  template<typename Allocator>
  promise(std::allocator_arg_t, Allocator const&);

  promise(promise const&) = delete;
  promise& operator=(promise const&) = delete;

  void swap(promise& ) noexcept;

  std::future<ResultType> get_future();

  void set_value(see description);
  void set_exception(std::exception_ptr p);
};
\end{cpp}

% ### std::promise 默认构造函数

构造一个\texttt{std::promise}对象。

\textbf{声明}

\begin{cpp}
promise();
\end{cpp}

\textbf{效果}
使用ResultType类型的相关异步结果来构造\texttt{std::promise}实例,不过异步结果并未就绪。

\textbf{抛出}
当没有足够内存为异步结果进行分配,那么将抛出\texttt{std::bad\_alloc}型异常。

% ### std::promise 带分配器的构造函数

构造一个\texttt{std::promise}对象,使用提供的分配器来为相关异步结果分配内存。

\textbf{声明}

\begin{cpp}
template<typename Allocator>
promise(std::allocator_arg_t, Allocator const& alloc);
\end{cpp}

\textbf{效果}
使用ResultType类型的相关异步结果来构造\texttt{std::promise}实例,不过异步结果并未就绪。异步结果的内存由alloc分配器来分配。

\textbf{抛出}
当分配器为异步结果分配内存时,如有抛出异常,就为该函数抛出的异常。

% ### std::promise 移动构造函数

通过另一个已存在对象,构造一个\texttt{std::promise}对象。将已存在对象中的相关异步结果的所有权转移到新创建的\texttt{std::promise}对象当中。

\textbf{声明}

\begin{cpp}
promise(promise&& other) noexcept;
\end{cpp}

\textbf{效果}
构造一个\texttt{std::promise}实例。

\textbf{后置条件}
当使用other来构造一个新的实例,那么other中相关异构结果的所有权将转移到新创建的对象上。之后,other将无关联异步结果。

\textbf{抛出}
无

% ### std::promise 移动赋值操作符

在两个\texttt{std::promise}实例中转移异步结果的所有权。

\textbf{声明}

\begin{cpp}
promise& operator=(promise&& other) noexcept;
\end{cpp}

\textbf{效果}
在other和\texttt{*this}之间进行异步结果所有权的转移。当\texttt{*this}已经有关联的异步结果,那么该异步结果的状态将会为就绪态,且伴随一个\texttt{std::future\_error}类型异常,错误码为\texttt{std::future\_errc::broken\_promise}。

\textbf{后置条件}
将other中关联的异步结果转移到*this当中。other中将无关联异步结果。

\textbf{返回}

\begin{cpp}
*this
\end{cpp}

\textbf{抛出}
无

% ### std::promise::swap 成员函数

将两个\texttt{std::promise}实例中的关联异步结果进行交换。

\textbf{声明}

\begin{cpp}
void swap(promise& other);
\end{cpp}

\textbf{效果}
交换other和*this当中的关联异步结果。

\textbf{后置条件}
当swap使用other时,other中的异步结果就会与*this中关联异步结果相交换。二者返回来亦然。

\textbf{抛出}
无

% ### std::promise 析构函数

销毁\texttt{std::promise}对象。

\textbf{声明}

\begin{cpp}
~promise();
\end{cpp}

\textbf{效果}
销毁\texttt{*this}。当\texttt{*this}具有关联的异步结果,并且结果中没有存储值或异常,那么结果将会置为就绪,伴随一个\texttt{std::future\_error}异常,错误码为\texttt{std::future\_errc::broken\_promise}。

\textbf{抛出}
无

% ### std::promise::get\_future 成员函数

通过*this关联的异步结果,检索出所要的\texttt{std::future}实例。

\textbf{声明}

\begin{cpp}
std::future<ResultType> get_future();
\end{cpp}

\textbf{先决条件}
*this具有关联异步结果。

\textbf{返回}
与*this关联异步结果关联的\texttt{std::future}实例。

\textbf{抛出}
当\texttt{std::future}已经通过get\_future()获取过了,将会抛出一个\texttt{std::future\_error}类型异常,伴随的错误码为\texttt{std::future\_errc::future\_already\_retrieved}。

% ### std::promise::set\_value 成员函数

存储一个值到与*this关联的异步结果中。

\textbf{声明}

\begin{cpp}
void promise<void>::set_value();
void promise<R&>::set_value(R& r);
void promise<R>::set_value(R const& r);
void promise<R>::set_value(R&& r);
\end{cpp}

\textbf{先决条件}
*this具有关联异步结果。

\textbf{效果}
当ResultType不是void型,就存储r到*this相关的异步结果当中。

\textbf{后置条件}
*this相关的异步结果的状态为就绪,且将值存入。任意等待异步结果的阻塞线程将解除阻塞。

\textbf{抛出}
当异步结果已经存有一个值或一个异常,那么将抛出\texttt{std::future\_error}型异常,伴随错误码为\texttt{std::future\_errc::promise\_already\_satisfied}。r的拷贝构造或移动构造抛出的异常,即为本函数抛出的异常。

\textbf{同步}
并发调用set\_value()和set\_exception()的线程将被序列化。要想成功的调用set\_exception(),需要在之前调用\texttt{std::future<Result-Type>::get()}或\texttt{std::shared\_future<ResultType>::get()},这两个函数将会查找已存储的异常。

% ### std::promise::set\_value\_at\_thread\_exit 成员函数

存储一个值到与*this关联的异步结果中,到线程退出时,异步结果的状态会被设置为就绪。

\textbf{声明}

\begin{cpp}
void promise<void>::set_value_at_thread_exit();
void promise<R&>::set_value_at_thread_exit(R& r);
void promise<R>::set_value_at_thread_exit(R const& r);
void promise<R>::set_value_at_thread_exit(R&& r);
\end{cpp}

\textbf{先决条件}
*this具有关联异步结果。

\textbf{效果}
当ResultType不是void型,就存储r到*this相关的异步结果当中。标记异步结果为“已存储值”。当前线程退出时,会安排相关异步结果的状态为就绪。

\textbf{后置条件}
将值存入*this相关的异步结果,且直到当前线程退出时,异步结果状态被置为就绪。任何等待异步结果的阻塞线程将解除阻塞。

\textbf{抛出}
当异步结果已经存有一个值或一个异常,那么将抛出\texttt{std::future\_error}型异常,伴随错误码为\texttt{std::future\_errc::promise\_already\_satisfied}。r的拷贝构造或移动构造抛出的异常,即为本函数抛出的异常。

\textbf{同步}
并发调用set\_value(), set\_value\_at\_thread\_exit(), set\_exception()和set\_exception\_at\_thread\_exit()的线程将被序列化。要想成功的调用set\_exception(),需要在之前调用\texttt{std::future<Result-Type>::get()}或\texttt{std::shared\_future<ResultType>::get()},这两个函数将会查找已存储的异常。

% ### std::promise::set\_exception 成员函数

存储一个异常到与*this关联的异步结果中。

\textbf{声明}

\begin{cpp}
void set_exception(std::exception_ptr e);
\end{cpp}

\textbf{先决条件}
*this具有关联异步结果。(bool)e为true。

\textbf{效果}
将e存储到*this相关的异步结果中。

\textbf{后置条件}
在存储异常后,*this相关的异步结果的状态将置为继续。任何等待异步结果的阻塞线程将解除阻塞。

\textbf{抛出}
当异步结果已经存有一个值或一个异常,那么将抛出\texttt{std::future\_error}型异常,伴随错误码为\texttt{std::future\_errc::promise\_already\_satisfied}。

\textbf{同步}
并发调用set\_value()和set\_exception()的线程将被序列化。要想成功的调用set\_exception(),需要在之前调用\texttt{std::future<Result-Type>::get()}或\texttt{std::shared\_future<ResultType>::get()},这两个函数将会查找已存储的异常。

% ### std::promise::set\_exception\_at\_thread\_exit 成员函数

存储一个异常到与*this关联的异步结果中,知道当前线程退出,异步结果被置为就绪。

\textbf{声明}

\begin{cpp}
void set_exception_at_thread_exit(std::exception_ptr e);
\end{cpp}

\textbf{先决条件}
*this具有关联异步结果。(bool)e为true。

\textbf{效果}
将e存储到*this相关的异步结果中。标记异步结果为“已存储值”。当前线程退出时,会安排相关异步结果的状态为就绪。

\textbf{后置条件}
将值存入*this相关的异步结果,且直到当前线程退出时,异步结果状态被置为就绪。任何等待异步结果的阻塞线程将解除阻塞。

\textbf{抛出}
当异步结果已经存有一个值或一个异常,那么将抛出\texttt{std::future\_error}型异常,伴随错误码为\texttt{std::future\_errc::promise\_already\_satisfied}。

\textbf{同步}
并发调用set\_value(), set\_value\_at\_thread\_exit(), set\_exception()和set\_exception\_at\_thread\_exit()的线程将被序列化。要想成功的调用set\_exception(),需要在之前调用\texttt{std::future<Result-Type>::get()}或\texttt{std::shared\_future<ResultType>::get()},这两个函数将会查找已存储的异常。

\mySubsubsection{D.4.5}{std::async函数模板}

\texttt{std::async}能够简单的使用可用的硬件并行来运行自身包含的异步任务。当调用\texttt{std::async}返回一个包含任务结果的\texttt{std::future}对象。根据投放策略,任务在其所在线程上是异步运行的,当有线程调用了这个future对象的wait()和get()成员函数,则该任务会同步运行。

\textbf{声明}

\begin{cpp}
enum class launch
{
  async,deferred
};

template<typename Callable,typename ... Args>
future<result_of<Callable(Args...)>::type>
async(Callable&& func,Args&& ... args);

template<typename Callable,typename ... Args>
future<result_of<Callable(Args...)>::type>
async(launch policy,Callable&& func,Args&& ... args);
\end{cpp}

\textbf{先决条件}
表达式\texttt{INVOKE(func,args)}能都为func提供合法的值和args。Callable和Args的所有成员都可MoveConstructible(可移动构造)。

\textbf{效果}
在内部存储中拷贝构造\texttt{func}和\texttt{arg...}(分别使用fff和xyz...进行表示)。

当policy是\texttt{std::launch::async},运行\texttt{INVOKE(fff,xyz...)}在所在线程上。当这个线程完成时,返回的\texttt{std::future}状态将会为就绪态,并且之后会返回对应的值或异常(由调用函数抛出)。析构函数会等待返回的\texttt{std::future}相关异步状态为就绪时,才解除阻塞。

当policy是\texttt{std::launch::deferred},fff和xyx...都会作为延期函数调用,存储在返回的\texttt{std::future}。首次调用future的wait()或get()成员函数,将会共享相关状态,之后执行的\texttt{INVOKE(fff,xyz...)}与调用wait()或get()函数的线程同步执行。

执行\texttt{INVOKE(fff,xyz...)}后,在调用\texttt{std::future}的成员函数get()时,就会有值返回或有异常抛出。

当policy是\texttt{std::launch::async | std::launch::deferred}或是policy参数被省略,其行为如同已指定的\texttt{std::launch::async}或\texttt{std::launch::deferred}。具体实现将会通过逐渐递增的方式(call-by-call basis)最大化利用可用的硬件并行,并避免超限分配的问题。

在所有的情况下,\texttt{std::async}调用都会直接返回。

\textbf{同步}
完成函数调用的先行条件是,需要通过调用\texttt{std::future}和\texttt{std::shared\_future}实例的wait(),get(),wait\_for()或wait\_until(),返回的对象与\texttt{std::async}返回的\texttt{std::future}对象关联的状态相同才算成功。就\texttt{std::launch::async}这个policy来说,在完成线程上的函数前,也需要先行对上面的函数调用后,成功的返回才行。

\textbf{抛出}
当内部存储无法分配所需的空间,将抛出\texttt{std::bad\_alloc}类型异常;否则,当效果没有达到,或任何异常在构造fff和xyz...发生时,抛出\texttt{std::future\_error}异常。
