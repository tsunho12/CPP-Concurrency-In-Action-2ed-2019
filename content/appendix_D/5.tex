% # D.5 <mutex>头文件

\texttt{<mutex>}头文件提供互斥工具:互斥类型,锁类型和函数,还有确保操作只执行一次的机制。

\textbf{头文件内容}

\begin{cpp}
namespace std
{
  class mutex;
  class recursive_mutex;
  class timed_mutex;
  class recursive_timed_mutex;

  struct adopt_lock_t;
  struct defer_lock_t;
  struct try_to_lock_t;

  constexpr adopt_lock_t adopt_lock{};
  constexpr defer_lock_t defer_lock{};
  constexpr try_to_lock_t try_to_lock{};

  template<typename LockableType>
  class lock_guard;

  template<typename LockableType>
  class unique_lock;

  template<typename LockableType1,typename... LockableType2>
  void lock(LockableType1& m1,LockableType2& m2...);

  template<typename LockableType1,typename... LockableType2>
  int try_lock(LockableType1& m1,LockableType2& m2...);

  struct once_flag;

  template<typename Callable,typename... Args>
  void call_once(once_flag& flag,Callable func,Args args...);
}
\end{cpp}

\mySubsubsection{D.5.1}{std::mutex类}

\texttt{std::mutex}类型为线程提供基本的互斥和同步工具,这些工具可以用来保护共享数据。互斥量可以用来保护数据,互斥量上锁必须要调用lok()或try\_lock()。当有一个线程获取已经获取了锁,那么其他线程想要在获取锁的时候,会在尝试或取锁的时候失败(调用try\_lock())或阻塞(调用lock()),具体酌情而定。当线程完成对共享数据的访问,之后就必须调用unlock()对锁进行释放,并且允许其他线程来访问这个共享数据。

\texttt{std::mutex}符合Lockable的需求。

\textbf{类型定义}

\begin{cpp}
class mutex
{
public:
  mutex(mutex const&)=delete;
  mutex& operator=(mutex const&)=delete;

  constexpr mutex() noexcept;
  ~mutex();

  void lock();
  void unlock();
  bool try_lock();
};
\end{cpp}

% ###std::mutex 默认构造函数

构造一个\texttt{std::mutex}对象。

\textbf{声明}

\begin{cpp}
constexpr mutex() noexcept;
\end{cpp}

\textbf{效果}
构造一个\texttt{std::mutex}实例。

\textbf{后置条件}
新构造的\texttt{std::mutex}对象是未锁的。

\textbf{抛出}
无

% ###std::mutex 析构函数

销毁一个\texttt{std::mutex}对象。

\textbf{声明}

\begin{cpp}
~mutex();
\end{cpp}

\textbf{先决条件}
*this必须是未锁的。

\textbf{效果}
销毁*this。

\textbf{抛出}
无

% ###std::mutex::lock 成员函数

为当前线程获取\texttt{std::mutex}上的锁。

\textbf{声明}

\begin{cpp}
void lock();
\end{cpp}

\textbf{先决条件}
*this上必须没有持有一个锁。

\textbf{效果}
阻塞当前线程,知道*this获取锁。

\textbf{后置条件}
*this被调用线程锁住。

\textbf{抛出}
当有错误产生,抛出\texttt{std::system\_error}类型异常。

% ###std::mutex::try\_lock 成员函数

尝试为当前线程获取\texttt{std::mutex}上的锁。

\textbf{声明}

\begin{cpp}
bool try_lock();
\end{cpp}

\textbf{先决条件}
*this上必须没有持有一个锁。

\textbf{效果}
尝试为当前线程*this获取上的锁,失败时当前线程不会被阻塞。

\textbf{返回}
当调用线程获取锁时,返回true。

\textbf{后置条件}
当*this被调用线程锁住,则返回true。

\textbf{抛出}
无

\textbf{NOTE} 该函数在获取锁时,可能失败(并返回false),即使没有其他线程持有*this上的锁。

% ###std::mutex::unlock 成员函数

释放当前线程获取的\texttt{std::mutex}锁。

\textbf{声明}

\begin{cpp}
void unlock();
\end{cpp}

\textbf{先决条件}
*this上必须持有一个锁。

\textbf{效果}<be>
释放当前线程获取到\texttt{*this}上的锁。任意等待获取\texttt{*this}上的线程,会在该函数调用后解除阻塞。

\textbf{后置条件}
调用线程不在拥有*this上的锁。

\textbf{抛出}
无

\mySubsubsection{D.5.2}{std::recursive\_mutex类}

\texttt{std::recursive\_mutex}类型为线程提供基本的互斥和同步工具,可以用来保护共享数据。互斥量可以用来保护数据,互斥量上锁必须要调用lok()或try\_lock()。当有一个线程获取已经获取了锁,那么其他线程想要在获取锁的时候,会在尝试或取锁的时候失败(调用try\_lock())或阻塞(调用lock()),具体酌情而定。当线程完成对共享数据的访问,之后就必须调用unlock()对锁进行释放,并且允许其他线程来访问这个共享数据。

这个互斥量是可递归的,所以一个线程获取\texttt{std::recursive\_mutex}后可以在之后继续使用lock()或try\_lock()来增加锁的计数。只有当线程调用unlock释放锁,其他线程才可能用lock()或try\_lock()获取锁。

\texttt{std::recursive\_mutex}符合Lockable的需求。

\textbf{类型定义}

\begin{cpp}
class recursive_mutex
{
public:
  recursive_mutex(recursive_mutex const&)=delete;
  recursive_mutex& operator=(recursive_mutex const&)=delete;

  recursive_mutex() noexcept;
 ~recursive_mutex();

  void lock();
  void unlock();
  bool try_lock() noexcept;
};
\end{cpp}

% ###std::recursive\_mutex 默认构造函数

构造一个\texttt{std::recursive\_mutex}对象。

\textbf{声明}

\begin{cpp}
recursive_mutex() noexcept;
\end{cpp}

\textbf{效果}
构造一个\texttt{std::recursive\_mutex}实例。

\textbf{后置条件}
新构造的\texttt{std::recursive\_mutex}对象是未锁的。

\textbf{抛出}
当无法创建一个新的\texttt{std::recursive\_mutex}时,抛出\texttt{std::system\_error}异常。

% ###std::recursive\_mutex 析构函数

销毁一个\texttt{std::recursive\_mutex}对象。

\textbf{声明}

\begin{cpp}
~recursive_mutex();
\end{cpp}

\textbf{先决条件}
*this必须是未锁的。

\textbf{效果}
销毁*this。

\textbf{抛出}
无

% ###std::recursive\_mutex::lock 成员函数

为当前线程获取\texttt{std::recursive\_mutex}上的锁。

\textbf{声明}

\begin{cpp}
void lock();
\end{cpp}

\textbf{效果}
阻塞线程,直到获取*this上的锁。

\textbf{先决条件}
调用线程锁住*this上的锁。当调用已经持有一个*this的锁时,锁的计数会增加1。

\textbf{抛出}
当有错误产生,将抛出\texttt{std::system\_error}异常。

% ###std::recursive\_mutex::try\_lock 成员函数

尝试为当前线程获取\texttt{std::recursive\_mutex}上的锁。

\textbf{声明}

\begin{cpp}
bool try_lock() noexcept;
\end{cpp}

\textbf{效果}
尝试为当前线程*this获取上的锁,失败时当前线程不会被阻塞。

\textbf{返回}
当调用线程获取锁时,返回true;否则,返回false。

\textbf{后置条件}
当*this被调用线程锁住,则返回true。

\textbf{抛出}
无

\textbf{NOTE} 该函数在获取锁时,当函数返回true时,\texttt{*this}上对锁的计数会加一。如果当前线程还未获取\texttt{*this}上的锁,那么该函数在获取锁时,可能失败(并返回false),即使没有其他线程持有\texttt{*this}上的锁。

% ###std::recursive\_mutex::unlock 成员函数

释放当前线程获取的\texttt{std::recursive\_mutex}锁。

\textbf{声明}

\begin{cpp}
void unlock();
\end{cpp}

\textbf{先决条件}
*this上必须持有一个锁。

\textbf{效果}<be>
释放当前线程获取到\texttt{*this}上的锁。如果这是\texttt{*this}在当前线程上最后一个锁,那么任意等待获取\texttt{*this}上的线程,会在该函数调用后解除其中一个线程的阻塞。

\textbf{后置条件}
\texttt{*this}上锁的计数会在该函数调用后减一。

\textbf{抛出}
无

\mySubsubsection{D.5.3}{std::timed\_mutex类}

\texttt{std::timed\_mutex}类型在\texttt{std::mutex}基本互斥和同步工具的基础上,让锁支持超时。互斥量可以用来保护数据,互斥量上锁必须要调用lok(),try\_lock\_for(),或try\_lock\_until()。当有一个线程获取已经获取了锁,那么其他线程想要在获取锁的时候,会在尝试或取锁的时候失败(调用try\_lock())或阻塞(调用lock()),或直到想要获取锁可以获取,亦或想要获取的锁超时(调用try\_lock\_for()或try\_lock\_until())。在线程调用unlock()对锁进行释放,其他线程才能获取这个锁被获取(不管是调用的哪个函数)。

\texttt{std::timed\_mutex}符合TimedLockable的需求。

\textbf{类型定义}

\begin{cpp}
class timed_mutex
{
public:
  timed_mutex(timed_mutex const&)=delete;
  timed_mutex& operator=(timed_mutex const&)=delete;

  timed_mutex();
  ~timed_mutex();

  void lock();
  void unlock();
  bool try_lock();

  template<typename Rep,typename Period>
  bool try_lock_for(
      std::chrono::duration<Rep,Period> const& relative_time);

  template<typename Clock,typename Duration>
  bool try_lock_until(
      std::chrono::time_point<Clock,Duration> const& absolute_time);
};
\end{cpp}

% ###std::timed\_mutex 默认构造函数

构造一个\texttt{std::timed\_mutex}对象。

\textbf{声明}

\begin{cpp}
timed_mutex();
\end{cpp}

\textbf{效果}
构造一个\texttt{std::timed\_mutex}实例。

\textbf{后置条件}
新构造一个未上锁的\texttt{std::timed\_mutex}对象。

\textbf{抛出}
当无法创建出新的\texttt{std::timed\_mutex}实例时,抛出\texttt{std::system\_error}类型异常。

% ###std::timed\_mutex 析构函数

销毁一个\texttt{std::timed\_mutex}对象。

\textbf{声明}

\begin{cpp}
~timed_mutex();
\end{cpp}

\textbf{先决条件}
*this必须没有上锁。

\textbf{效果}
销毁*this。

\textbf{抛出}
无

% ###std::timed\_mutex::lock 成员函数

为当前线程获取\texttt{std::timed\_mutex}上的锁。

\textbf{声明}

\begin{cpp}
void lock();
\end{cpp}

\textbf{先决条件}
调用线程不能已经持有*this上的锁。

\textbf{效果}
阻塞当前线程,直到获取到*this上的锁。

\textbf{后置条件}
*this被调用线程锁住。

\textbf{抛出}
当有错误产生,抛出\texttt{std::system\_error}类型异常。

% ###std::timed\_mutex::try\_lock 成员函数

尝试获取为当前线程获取\texttt{std::timed\_mutex}上的锁。

\textbf{声明}

\begin{cpp}
bool try_lock();
\end{cpp}

\textbf{先决条件}
调用线程不能已经持有*this上的锁。

\textbf{效果}
尝试获取*this上的锁,当获取失败时,不阻塞调用线程。

\textbf{返回}
当锁被调用线程获取,返回true;反之,返回false。

\textbf{后置条件}
当函数返回为true,*this则被当前线程锁住。

\textbf{抛出}
无

\textbf{NOTE} 即使没有线程已获取*this上的锁,函数还是有可能获取不到锁(并返回false)。

% ###std::timed\_mutex::try\_lock\_for 成员函数

尝试获取为当前线程获取\texttt{std::timed\_mutex}上的锁。

\textbf{声明}

\begin{cpp}
template<typename Rep,typename Period>
bool try_lock_for(
    std::chrono::duration<Rep,Period> const& relative_time);
\end{cpp}

\textbf{先决条件}
调用线程不能已经持有*this上的锁。

\textbf{效果}
在指定的relative\_time时间内,尝试获取*this上的锁。当relative\_time.count()为0或负数,将会立即返回,就像调用try\_lock()一样。否则,将会阻塞,直到获取锁或超过给定的relative\_time的时间。

\textbf{返回}
当锁被调用线程获取,返回true;反之,返回false。

\textbf{后置条件}
当函数返回为true,*this则被当前线程锁住。

\textbf{抛出}
无

\textbf{NOTE} 即使没有线程已获取*this上的锁,函数还是有可能获取不到锁(并返回false)。线程阻塞的时长可能会长于给定的时间。逝去的时间可能是由一个稳定时钟所决定。

% ###std::timed\_mutex::try\_lock\_until 成员函数

尝试获取为当前线程获取\texttt{std::timed\_mutex}上的锁。

\textbf{声明}

\begin{cpp}
template<typename Clock,typename Duration>
bool try_lock_until(
    std::chrono::time_point<Clock,Duration> const& absolute_time);
\end{cpp}

\textbf{先决条件}
调用线程不能已经持有*this上的锁。

\textbf{效果}
在指定的absolute\_time时间内,尝试获取*this上的锁。当`absolute\_time<=Clock::now()`时,将会立即返回,就像调用try\_lock()一样。否则,将会阻塞,直到获取锁或Clock::now()返回的时间等于或超过给定的absolute\_time的时间。

\textbf{返回}
当锁被调用线程获取,返回true;反之,返回false。

\textbf{后置条件}
当函数返回为true,*this则被当前线程锁住。

\textbf{抛出}
无

\textbf{NOTE} 即使没有线程已获取*this上的锁,函数还是有可能获取不到锁(并返回false)。这里不保证调用函数要阻塞多久,只有在函数返回false后,在Clock::now()返回的时间大于或等于absolute\_time时,线程才会接触阻塞。

% ###std::timed\_mutex::unlock 成员函数

将当前线程持有\texttt{std::timed\_mutex}对象上的锁进行释放。

\textbf{声明}

\begin{cpp}
void unlock();
\end{cpp}

\textbf{先决条件}
调用线程已经持有*this上的锁。

\textbf{效果}
当前线程释放\texttt{*this}上的锁。任一阻塞等待获取\texttt{*this}上的线程,将被解除阻塞。

\textbf{后置条件}
*this未被调用线程上锁。

\textbf{抛出}
无

\mySubsubsection{D.5.4}{std::recursive\_timed\_mutex类}

\texttt{std::recursive\_timed\_mutex}类型在\texttt{std::recursive\_mutex}提供的互斥和同步工具的基础上,让锁支持超时。互斥量可以用来保护数据,互斥量上锁必须要调用lok(),try\_lock\_for(),或try\_lock\_until()。当有一个线程获取已经获取了锁,那么其他线程想要在获取锁的时候,会在尝试或取锁的时候失败(调用try\_lock())或阻塞(调用lock()),或直到想要获取锁可以获取,亦或想要获取的锁超时(调用try\_lock\_for()或try\_lock\_until())。在线程调用unlock()对锁进行释放,其他线程才能获取这个锁被获取(不管是调用的哪个函数)。

该互斥量是可递归的,所以获取\texttt{std::recursive\_timed\_mutex}锁的线程,可以多次的对该实例上的锁获取。所有的锁将会在调用相关unlock()操作后,可由其他线程获取该实例上的锁。

\texttt{std::recursive\_timed\_mutex}符合TimedLockable的需求。

\textbf{类型定义}

\begin{cpp}
class recursive_timed_mutex
{
public:
  recursive_timed_mutex(recursive_timed_mutex const&)=delete;
  recursive_timed_mutex& operator=(recursive_timed_mutex const&)=delete;

  recursive_timed_mutex();
  ~recursive_timed_mutex();

  void lock();
  void unlock();
  bool try_lock() noexcept;

  template<typename Rep,typename Period>
  bool try_lock_for(
      std::chrono::duration<Rep,Period> const& relative_time);

  template<typename Clock,typename Duration>
  bool try_lock_until(
      std::chrono::time_point<Clock,Duration> const& absolute_time);
};
\end{cpp}

% ###std::recursive\_timed\_mutex 默认构造函数

构造一个\texttt{std::recursive\_timed\_mutex}对象。

\textbf{声明}

\begin{cpp}
recursive_timed_mutex();
\end{cpp}

\textbf{效果}
构造一个\texttt{std::recursive\_timed\_mutex}实例。

\textbf{后置条件}
新构造的\texttt{std::recursive\_timed\_mutex}实例是没有上锁的。

\textbf{抛出}
当无法创建一个\texttt{std::recursive\_timed\_mutex}实例时,抛出\texttt{std::system\_error}类异常。

% ###std::recursive\_timed\_mutex 析构函数

析构一个\texttt{std::recursive\_timed\_mutex}对象。

\textbf{声明}

\begin{cpp}
~recursive_timed_mutex();
\end{cpp}

\textbf{先决条件}
*this不能上锁。

\textbf{效果}
销毁*this。

\textbf{抛出}
无

% ###std::recursive\_timed\_mutex::lock 成员函数

为当前线程获取\texttt{std::recursive\_timed\_mutex}对象上的锁。

\textbf{声明}

\begin{cpp}
void lock();
\end{cpp}

\textbf{先决条件}
*this上的锁不能被线程调用。

\textbf{效果}
阻塞当前线程,直到获取*this上的锁。

\textbf{后置条件}
\texttt{*this}被调用线程锁住。当调用线程已经获取\texttt{*this}上的锁,那么锁的计数会再增加1。

\textbf{抛出}
当错误出现时,抛出\texttt{std::system\_error}类型异常。

% ###std::recursive\_timed\_mutex::try\_lock 成员函数

尝试为当前线程获取\texttt{std::recursive\_timed\_mutex}对象上的锁。

\textbf{声明}

\begin{cpp}
bool try_lock() noexcept;
\end{cpp}

\textbf{效果}
尝试获取*this上的锁,当获取失败时,直接不阻塞线程。

\textbf{返回}
当调用线程获取了锁,返回true,否则返回false。

\textbf{后置条件}
当函数返回true,\texttt{*this}会被调用线程锁住。

\textbf{抛出}
无

\textbf{NOTE} 该函数在获取锁时,当函数返回true时,\texttt{*this}上对锁的计数会加一。如果当前线程还未获取\texttt{*this}上的锁,那么该函数在获取锁时,可能失败(并返回false),即使没有其他线程持有\texttt{*this}上的锁。

% ###std::recursive\_timed\_mutex::try\_lock\_for 成员函数

尝试为当前线程获取\texttt{std::recursive\_timed\_mutex}对象上的锁。

\textbf{声明}

\begin{cpp}
template<typename Rep,typename Period>
bool try_lock_for(
    std::chrono::duration<Rep,Period> const& relative_time);
\end{cpp}

\textbf{效果}
在指定时间relative\_time内,尝试为调用线程获取*this上的锁。当relative\_time.count()为0或负数时,将会立即返回,就像调用try\_lock()一样。否则,调用会阻塞,直到获取相应的锁,或超出了relative\_time时限时,调用线程解除阻塞。

\textbf{返回}
当调用线程获取了锁,返回true,否则返回false。

\textbf{后置条件}
当函数返回true,\texttt{*this}会被调用线程锁住。

\textbf{抛出}
无

\textbf{NOTE} 该函数在获取锁时,当函数返回true时,\texttt{*this}上对锁的计数会加一。如果当前线程还未获取\texttt{*this}上的锁,那么该函数在获取锁时,可能失败(并返回false),即使没有其他线程持有\texttt{*this}上的锁。等待时间可能要比指定的时间长很多。逝去的时间可能由一个稳定时钟来计算。

% ###std::recursive\_timed\_mutex::try\_lock\_until 成员函数

尝试为当前线程获取\texttt{std::recursive\_timed\_mutex}对象上的锁。

\textbf{声明}

\begin{cpp}
template<typename Clock,typename Duration>
bool try_lock_until(
    std::chrono::time_point<Clock,Duration> const& absolute_time);
\end{cpp}

\textbf{效果}
在指定时间absolute\_time内,尝试为调用线程获取*this上的锁。当absolute\_time<=Clock::now()时,将会立即返回,就像调用try\_lock()一样。否则,调用会阻塞,直到获取相应的锁,或Clock::now()返回的时间大于或等于absolute\_time时,调用线程解除阻塞。

\textbf{返回}
当调用线程获取了锁,返回true,否则返回false。

\textbf{后置条件}
当函数返回true,\texttt{*this}会被调用线程锁住。

\textbf{抛出}
无

\textbf{NOTE} 该函数在获取锁时,当函数返回true时,\texttt{*this}上对锁的计数会加一。如果当前线程还未获取\texttt{*this}上的锁,那么该函数在获取锁时,可能失败(并返回false),即使没有其他线程持有\texttt{*this}上的锁。这里阻塞的时间并不确定,只有当函数返回false,然后Clock::now()返回的时间大于或等于absolute\_time时,调用线程将会解除阻塞。

% ###std::recursive\_timed\_mutex::unlock 成员函数

释放当前线程获取到的\texttt{std::recursive\_timed\_mutex}上的锁。

\textbf{声明}

\begin{cpp}
void unlock();
\end{cpp}

\textbf{效果}
当前线程释放\texttt{*this}上的锁。当\texttt{*this}上最后一个锁被释放后,任何等待获取\texttt{*this}上的锁将会解除阻塞,不过只能解除其中一个线程的阻塞。

\textbf{后置条件}
调用线程*this上锁的计数减一。

\textbf{抛出}
无

\mySubsubsection{D.5.5}{std::lock\_guard类型模板}

\texttt{std::lock\_guard}类型模板为基础锁包装所有权。所要上锁的互斥量类型,由模板参数Mutex来决定,并且必须符合Lockable的需求。指定的互斥量在构造函数中上锁,在析构函数中解锁。这就为互斥量锁部分代码提供了一个简单的方式;当程序运行完成时,阻塞解除,互斥量解锁(无论是执行到最后,还是通过控制流语句break或return,亦或是抛出异常)。

\texttt{std::lock\_guard}是不可MoveConstructible(移动构造), CopyConstructible(拷贝构造)和CopyAssignable(拷贝赋值)。

\textbf{类型定义}

\begin{cpp}
template <class Mutex>
class lock_guard
{
public:
  typedef Mutex mutex_type;

  explicit lock_guard(mutex_type& m);
  lock_guard(mutex_type& m, adopt_lock_t);
  ~lock_guard();

  lock_guard(lock_guard const& ) = delete;
  lock_guard& operator=(lock_guard const& ) = delete;
};
\end{cpp}

% ###std::lock\_guard 自动上锁的构造函数

使用互斥量构造一个\texttt{std::lock\_guard}实例。

\textbf{声明}

\begin{cpp}
explicit lock_guard(mutex_type& m);
\end{cpp}

\textbf{效果}
通过引用提供的互斥量,构造一个新的\texttt{std::lock\_guard}实例,并调用m.lock()。

\textbf{抛出}
m.lock()抛出的任何异常。

\textbf{后置条件}
*this拥有m上的锁。

% ###std::lock\_guard 获取锁的构造函数

使用已提供互斥量上的锁,构造一个\texttt{std::lock\_guard}实例。

\textbf{声明}

\begin{cpp}
lock_guard(mutex_type& m,std::adopt_lock_t);
\end{cpp}

\textbf{先决条件}
调用线程必须拥有m上的锁。

\textbf{效果}
调用线程通过引用提供的互斥量,以及获取m上锁的所有权,来构造一个新的\texttt{std::lock\_guard}实例。

\textbf{抛出}
无

\textbf{后置条件}
*this拥有m上的锁。

% ###std::lock\_guard 析构函数

销毁一个\texttt{std::lock\_guard}实例,并且解锁相关互斥量。

\textbf{声明}

\begin{cpp}
~lock_guard();
\end{cpp}

\textbf{效果}
当*this被创建后,调用m.unlock()。

\textbf{抛出}
无

\mySubsubsection{D.5.6}{std::unique\_lock类型模板}

\texttt{std::unique\_lock}类型模板相较\texttt{std::loc\_guard}提供了更通用的所有权包装器。上锁的互斥量可由模板参数Mutex提供,这个类型必须满足BasicLockable的需求。虽然,通常情况下,制定的互斥量会在构造的时候上锁,析构的时候解锁,但是附加的构造函数和成员函数提供灵活的功能。互斥量上锁,意味着对操作同一段代码的线程进行阻塞;当互斥量解锁,就意味着阻塞解除(不论是裕兴到最后,还是使用控制语句break和return,亦或是抛出异常)。\texttt{std::condition\_variable}的邓丹函数是需要\texttt{std::unique\_lock<std::mutex>}实例的,并且所有\texttt{std::unique\_lock}实例都适用于\texttt{std::conditin\_variable\_any}等待函数的Lockable参数。

当提供的Mutex类型符合Lockable的需求,那么\texttt{std::unique\_lock<Mutex>}也是符合Lockable的需求。此外,如果提供的Mutex类型符合TimedLockable的需求,那么\texttt{std::unique\_lock<Mutex>}也符合TimedLockable的需求。

\texttt{std::unique\_lock}实例是MoveConstructible(移动构造)和MoveAssignable(移动赋值),但是不能CopyConstructible(拷贝构造)和CopyAssignable(拷贝赋值)。

\textbf{类型定义}

\begin{cpp}
template <class Mutex>
class unique_lock
{
public:
  typedef Mutex mutex_type;

  unique_lock() noexcept;
  explicit unique_lock(mutex_type& m);
  unique_lock(mutex_type& m, adopt_lock_t);
  unique_lock(mutex_type& m, defer_lock_t) noexcept;
  unique_lock(mutex_type& m, try_to_lock_t);

  template<typename Clock,typename Duration>
  unique_lock(
      mutex_type& m,
      std::chrono::time_point<Clock,Duration> const& absolute_time);

  template<typename Rep,typename Period>
      unique_lock(
      mutex_type& m,
      std::chrono::duration<Rep,Period> const& relative_time);

  ~unique_lock();

  unique_lock(unique_lock const& ) = delete;
  unique_lock& operator=(unique_lock const& ) = delete;

  unique_lock(unique_lock&& );
  unique_lock& operator=(unique_lock&& );

  void swap(unique_lock& other) noexcept;

  void lock();
  bool try_lock();
  template<typename Rep, typename Period>
  bool try_lock_for(
      std::chrono::duration<Rep,Period> const& relative_time);
  template<typename Clock, typename Duration>
  bool try_lock_until(
      std::chrono::time_point<Clock,Duration> const& absolute_time);
  void unlock();

  explicit operator bool() const noexcept;
  bool owns_lock() const noexcept;
  Mutex* mutex() const noexcept;
  Mutex* release() noexcept;
};
\end{cpp}

% ###std::unique\_lock 默认构造函数

不使用相关互斥量,构造一个\texttt{std::unique\_lock}实例。

\textbf{声明}

\begin{cpp}
unique_lock() noexcept;
\end{cpp}

\textbf{效果}
构造一个\texttt{std::unique\_lock}实例,这个新构造的实例没有相关互斥量。

\textbf{后置条件}
this->mutex()==NULL, this->owns\_lock()==false.

% ###std::unique\_lock 自动上锁的构造函数

使用相关互斥量,构造一个\texttt{std::unique\_lock}实例。

\textbf{声明}

\begin{cpp}
explicit unique_lock(mutex_type& m);
\end{cpp}

\textbf{效果}
通过提供的互斥量,构造一个\texttt{std::unique\_lock}实例,且调用m.lock()。

\textbf{抛出}
m.lock()抛出的任何异常。

\textbf{后置条件}
this->owns\_lock()==true, this->mutex()==\&m.

% ###std::unique\_lock 获取锁的构造函数

使用相关互斥量和持有的锁,构造一个\texttt{std::unique\_lock}实例。

\textbf{声明}

\begin{cpp}
unique_lock(mutex_type& m,std::adopt_lock_t);
\end{cpp}

\textbf{先决条件}
调用线程必须持有m上的锁。

\textbf{效果}
通过提供的互斥量和已经拥有m上的锁,构造一个\texttt{std::unique\_lock}实例。

\textbf{抛出}
无

\textbf{后置条件}
this->owns\_lock()==true, this->mutex()==\&m.

% ###std::unique\_lock 递延锁的构造函数

使用相关互斥量和非持有的锁,构造一个\texttt{std::unique\_lock}实例。

\textbf{声明}

\begin{cpp}
unique_lock(mutex_type& m,std::defer_lock_t) noexcept;
\end{cpp}

\textbf{效果}
构造的\texttt{std::unique\_lock}实例引用了提供的互斥量。

\textbf{抛出}
无

\textbf{后置条件}
this->owns\_lock()==false, this->mutex()==\&m.

% ###std::unique\_lock 尝试获取锁的构造函数

使用提供的互斥量,并尝试从互斥量上获取锁,从而构造一个\texttt{std::unique\_lock}实例。

\textbf{声明}

\begin{cpp}
unique_lock(mutex_type& m,std::try_to_lock_t);
\end{cpp}

\textbf{先决条件}
使\texttt{std::unique\_lock}实例化的Mutex类型,必须符合Loackable的需求。

\textbf{效果}
构造的\texttt{std::unique\_lock}实例引用了提供的互斥量,且调用m.try\_lock()。

\textbf{抛出}
无

\textbf{后置条件}
this->owns\_lock()将返回m.try\_lock()的结果,且this->mutex()==\&m。

% ###std::unique\_lock 在给定时长内尝试获取锁的构造函数

使用提供的互斥量,并尝试从互斥量上获取锁,从而构造一个\texttt{std::unique\_lock}实例。

\textbf{声明}

\begin{cpp}
template<typename Rep,typename Period>
unique_lock(
    mutex_type& m,
    std::chrono::duration<Rep,Period> const& relative_time);
\end{cpp}

\textbf{先决条件}
使\texttt{std::unique\_lock}实例化的Mutex类型,必须符合TimedLockable的需求。

\textbf{效果}
构造的\texttt{std::unique\_lock}实例引用了提供的互斥量,且调用m.try\_lock\_for(relative\_time)。

\textbf{抛出}
无

\textbf{后置条件}
this->owns\_lock()将返回m.try\_lock\_for()的结果,且this->mutex()==\&m。

% ###std::unique\_lock 在给定时间点内尝试获取锁的构造函数

使用提供的互斥量,并尝试从互斥量上获取锁,从而构造一个\texttt{std::unique\_lock}实例。

\textbf{声明}

\begin{cpp}
template<typename Clock,typename Duration>
unique_lock(
    mutex_type& m,
    std::chrono::time_point<Clock,Duration> const& absolute_time);
\end{cpp}

\textbf{先决条件}
使\texttt{std::unique\_lock}实例化的Mutex类型,必须符合TimedLockable的需求。

\textbf{效果}
构造的\texttt{std::unique\_lock}实例引用了提供的互斥量,且调用m.try\_lock\_until(absolute\_time)。

\textbf{抛出}
无

\textbf{后置条件}
this->owns\_lock()将返回m.try\_lock\_until()的结果,且this->mutex()==\&m。

% ###std::unique\_lock 移动构造函数

将一个已经构造\texttt{std::unique\_lock}实例的所有权,转移到新的\texttt{std::unique\_lock}实例上去。

\textbf{声明}

\begin{cpp}
unique_lock(unique_lock&& other) noexcept;
\end{cpp}

\textbf{先决条件}
使\texttt{std::unique\_lock}实例化的Mutex类型,必须符合TimedLockable的需求。

\textbf{效果}
构造的\texttt{std::unique\_lock}实例。当other在函数调用的时候拥有互斥量上的锁,那么该锁的所有权将被转移到新构建的\texttt{std::unique\_lock}对象当中去。

\textbf{后置条件}
对于新构建的\texttt{std::unique\_lock}对象x,x.mutex等价与在构造函数调用前的other.mutex(),并且x.owns\_lock()等价于函数调用前的other.owns\_lock()。在调用函数后,other.mutex()==NULL,other.owns\_lock()=false。

\textbf{抛出}
无

\textbf{NOTE} \texttt{std::unique\_lock}对象是不可CopyConstructible(拷贝构造),所以这里没有拷贝构造函数,只有移动构造函数。

% ###std::unique\_lock 移动赋值操作

将一个已经构造\texttt{std::unique\_lock}实例的所有权,转移到新的\texttt{std::unique\_lock}实例上去。

\textbf{声明}

\begin{cpp}
unique_lock& operator=(unique_lock&& other) noexcept;
\end{cpp}

\textbf{效果}
当this->owns\_lock()返回true时,调用this->unlock()。如果other拥有mutex上的锁,那么这个所将归*this所有。

\textbf{后置条件}
this->mutex()等于在为进行赋值前的other.mutex(),并且this->owns\_lock()的值与进行赋值操作前的other.owns\_lock()相等。other.mutex()==NULL, other.owns\_lock()==false。

\textbf{抛出}
无

\textbf{NOTE} \texttt{std::unique\_lock}对象是不可CopyAssignable(拷贝赋值),所以这里没有拷贝赋值函数,只有移动赋值函数。

% ###std::unique\_lock 析构函数

销毁一个\texttt{std::unique\_lock}实例,如果该实例拥有锁,那么会将相关互斥量进行解锁。

\textbf{声明}

\begin{cpp}
~unique_lock();
\end{cpp}

\textbf{效果}
当this->owns\_lock()返回true时,调用this->mutex()->unlock()。

\textbf{抛出}
无

% ###std::unique\_lock::swap 成员函数

交换\texttt{std::unique\_lock}实例中相关的所有权。

\textbf{声明}

\begin{cpp}
void swap(unique_lock& other) noexcept;
\end{cpp}

\textbf{效果}
如果other在调用该函数前拥有互斥量上的锁,那么这个锁将归\texttt{*this}所有。如果\texttt{*this}在调用哎函数前拥有互斥量上的锁,那么这个锁将归other所有。

\textbf{抛出}
无

% ###std::unique\_lock 上非成员函数swap

交换\texttt{std::unique\_lock}实例中相关的所有权。

\textbf{声明}

\begin{cpp}
void swap(unique_lock& lhs,unique_lock& rhs) noexcept;
\end{cpp}

\textbf{效果}
lhs.swap(rhs)

\textbf{抛出}
无

% ###std::unique\_lock::lock 成员函数

获取与*this相关互斥量上的锁。

\textbf{声明}

\begin{cpp}
void lock();
\end{cpp}

\textbf{先决条件}
this->mutex()!=NULL, this->owns\_lock()==false.

\textbf{效果}
调用this->mutex()->lock()。

\textbf{抛出}
抛出任何this->mutex()->lock()所抛出的异常。当this->mutex()==NULL,抛出\texttt{std::sytem\_error}类型异常,错误码为\texttt{std::errc::operation\_not\_permitted}。当this->owns\_lock()==true时,抛出\texttt{std::system\_error},错误码为\texttt{std::errc::resource\_deadlock\_would\_occur}。

\textbf{后置条件}
this->owns\_lock()==true。

% ###std::unique\_lock::try\_lock 成员函数

尝试获取与*this相关互斥量上的锁。

\textbf{声明}

\begin{cpp}
bool try_lock();
\end{cpp}

\textbf{先决条件}
\texttt{std::unique\_lock}实例化说是用的Mutex类型,必须满足Lockable需求。this->mutex()!=NULL, this->owns\_lock()==false。

\textbf{效果}
调用this->mutex()->try\_lock()。

\textbf{抛出}
抛出任何this->mutex()->try\_lock()所抛出的异常。当this->mutex()==NULL,抛出\texttt{std::sytem\_error}类型异常,错误码为\texttt{std::errc::operation\_not\_permitted}。当this->owns\_lock()==true时,抛出\texttt{std::system\_error},错误码为\texttt{std::errc::resource\_deadlock\_would\_occur}。

\textbf{后置条件}
当函数返回true时,this->ows\_lock()==true,否则this->owns\_lock()==false。

% ###std::unique\_lock::unlock 成员函数

释放与*this相关互斥量上的锁。

\textbf{声明}

\begin{cpp}
void unlock();
\end{cpp}

\textbf{先决条件}
this->mutex()!=NULL, this->owns\_lock()==true。

\textbf{抛出}
抛出任何this->mutex()->unlock()所抛出的异常。当this->owns\_lock()==false时,抛出\texttt{std::system\_error},错误码为\texttt{std::errc::operation\_not\_permitted}。

\textbf{后置条件}
this->owns\_lock()==false。

% ###std::unique\_lock::try\_lock\_for 成员函数

在指定时间内尝试获取与*this相关互斥量上的锁。

\textbf{声明}

\begin{cpp}
template<typename Rep, typename Period>
bool try_lock_for(
    std::chrono::duration<Rep,Period> const& relative_time);
\end{cpp}

\textbf{先决条件}
\texttt{std::unique\_lock}实例化说是用的Mutex类型,必须满足TimedLockable需求。this->mutex()!=NULL, this->owns\_lock()==false。

\textbf{效果}
调用this->mutex()->try\_lock\_for(relative\_time)。

\textbf{返回}
当this->mutex()->try\_lock\_for()返回true,返回true,否则返回false。

\textbf{抛出}
抛出任何this->mutex()->try\_lock\_for()所抛出的异常。当this->mutex()==NULL,抛出\texttt{std::sytem\_error}类型异常,错误码为\texttt{std::errc::operation\_not\_permitted}。当this->owns\_lock()==true时,抛出\texttt{std::system\_error},错误码为\texttt{std::errc::resource\_deadlock\_would\_occur}。

\textbf{后置条件}
当函数返回true时,this->ows\_lock()==true,否则this->owns\_lock()==false。

% ###std::unique\_lock::try\_lock\_until 成员函数

在指定时间点尝试获取与*this相关互斥量上的锁。

\textbf{声明}

\begin{cpp}
template<typename Clock, typename Duration>
bool try_lock_until(
    std::chrono::time_point<Clock,Duration> const& absolute_time);
\end{cpp}

\textbf{先决条件}
\texttt{std::unique\_lock}实例化说是用的Mutex类型,必须满足TimedLockable需求。this->mutex()!=NULL, this->owns\_lock()==false。

\textbf{效果}
调用this->mutex()->try\_lock\_until(absolute\_time)。

\textbf{返回}
当this->mutex()->try\_lock\_for()返回true,返回true,否则返回false。

\textbf{抛出}
抛出任何this->mutex()->try\_lock\_for()所抛出的异常。当this->mutex()==NULL,抛出\texttt{std::sytem\_error}类型异常,错误码为\texttt{std::errc::operation\_not\_permitted}。当this->owns\_lock()==true时,抛出\texttt{std::system\_error},错误码为\texttt{std::errc::resource\_deadlock\_would\_occur}。

\textbf{后置条件}
当函数返回true时,this->ows\_lock()==true,否则this->owns\_lock()==false。

% ###std::unique\_lock::operator bool成员函数

检查*this是否拥有一个互斥量上的锁。

\textbf{声明}

\begin{cpp}
explicit operator bool() const noexcept;
\end{cpp}

\textbf{返回}
this->owns\_lock()

\textbf{抛出}
无

\textbf{NOTE} 这是一个explicit转换操作,所以当这样的操作在上下文中只能被隐式的调用,所返回的结果需要被当做一个布尔量进行使用,而非仅仅作为整型数0或1。

% ###std::unique\_lock::owns\_lock 成员函数

检查*this是否拥有一个互斥量上的锁。

\textbf{声明}

\begin{cpp}
bool owns_lock() const noexcept;
\end{cpp}

\textbf{返回}
当*this持有一个互斥量的锁,返回true;否则,返回false。

\textbf{抛出}
无

% ###std::unique\_lock::mutex 成员函数

当*this具有相关互斥量时,返回这个互斥量

\textbf{声明}

\begin{cpp}
mutex_type* mutex() const noexcept;
\end{cpp}

\textbf{返回}
当*this有相关互斥量,则返回该互斥量;否则,返回NULL。

\textbf{抛出}
无

% ###std::unique\_lock::release 成员函数

当*this具有相关互斥量时,返回这个互斥量,并将这个互斥量进行释放。

\textbf{声明}

\begin{cpp}
mutex_type* release() noexcept;
\end{cpp}

\textbf{效果}
将*this与相关的互斥量之间的关系解除,同时解除所有持有锁的所有权。

\textbf{返回}
返回与*this相关的互斥量指针,如果没有相关的互斥量,则返回NULL。

\textbf{后置条件}
this->mutex()==NULL, this->owns\_lock()==false。

\textbf{抛出}
无

\textbf{NOTE} 如果this->owns\_lock()在调用该函数前返回true,那么调用者则有责任里解除互斥量上的锁。

\mySubsubsection{D.5.7}{std::lock函数模板}

\texttt{std::lock}函数模板提供同时锁住多个互斥量的功能,且不会有因改变锁的一致性而导致的死锁。

\textbf{声明}

\begin{cpp}
template<typename LockableType1,typename... LockableType2>
void lock(LockableType1& m1,LockableType2& m2...);
\end{cpp}

\textbf{先决条件}
提供的可锁对象LockableType1, LockableType2...,需要满足Lockable的需求。

\textbf{效果}
使用未指定顺序调用lock(),try\_lock()获取每个可锁对象(m1, m2...)上的锁,还有unlock()成员来避免这个类型陷入死锁。

\textbf{后置条件}
当前线程拥有提供的所有可锁对象上的锁。

\textbf{抛出}
任何lock(), try\_lock()和unlock()抛出的异常。

\textbf{NOTE} 如果一个异常由\texttt{std::lock}所传播开来,当可锁对象上有锁被lock()或try\_lock()获取,那么unlock()会使用在这些可锁对象上。

\mySubsubsection{D.5.8}{std::try\_lock函数模板}

\texttt{std::try\_lock}函数模板允许尝试获取一组可锁对象上的锁,所以要不全部获取,要不一个都不获取。

\textbf{声明}

\begin{cpp}
template<typename LockableType1,typename... LockableType2>
int try_lock(LockableType1& m1,LockableType2& m2...);
\end{cpp}

\textbf{先决条件}
提供的可锁对象LockableType1, LockableType2...,需要满足Lockable的需求。

\textbf{效果}
使用try\_lock()尝试从提供的可锁对象m1,m2...上逐个获取锁。当锁在之前获取过,但被当前线程使用unlock()对相关可锁对象进行了释放后,try\_lock()会返回false或抛出一个异常。

\textbf{返回}
当所有锁都已获取(每个互斥量调用try\_lock()返回true),则返回-1,否则返回以0为基数的数字,其值为调用try\_lock()返回false的个数。

\textbf{后置条件}
当函数返回-1,当前线程获取从每个可锁对象上都获取一个锁。否则,通过该调用获取的任何锁都将被释放。

\textbf{抛出}
try\_lock()抛出的任何异常。

\textbf{NOTE} 如果一个异常由\texttt{std::try\_lock}所传播开来,则通过try\_lock()获取锁对象,将会调用unlock()解除对锁的持有。

\mySubsubsection{D.5.9}{std::once\_flag类}

\texttt{std::once\_flag}和\texttt{std::call\_once}一起使用,为了保证某特定函数只执行一次(即使有多个线程在并发的调用该函数)。

\texttt{std::once\_flag}实例是不能CopyConstructible(拷贝构造),CopyAssignable(拷贝赋值),MoveConstructible(移动构造),以及MoveAssignable(移动赋值)。

\textbf{类型定义}

\begin{cpp}
struct once_flag
{
  constexpr once_flag() noexcept;

  once_flag(once_flag const& ) = delete;
  once_flag& operator=(once_flag const& ) = delete;
};
\end{cpp}

% ###std::once\_flag 默认构造函数

\texttt{std::once\_flag}默认构造函数创建了一个新的\texttt{std::once\_flag}实例(并包含一个状态,这个状态表示相关函数没有被调用)。

\textbf{声明}

\begin{cpp}
constexpr once_flag() noexcept;
\end{cpp}

\textbf{效果}
\texttt{std::once\_flag}默认构造函数创建了一个新的\texttt{std::once\_flag}实例(并包含一个状态,这个状态表示相关函数没有被调用)。因为这是一个constexpr构造函数,在构造的静态初始部分,实例是静态存储的,这样就避免了条件竞争和初始化顺序的问题。

\mySubsubsection{D.5.10}{std::call\_once函数模板}

\texttt{std::call\_once}和\texttt{std::once\_flag}一起使用,为了保证某特定函数只执行一次(即使有多个线程在并发的调用该函数)。

\textbf{声明}

\begin{cpp}
template<typename Callable,typename... Args>
void call_once(std::once_flag& flag,Callable func,Args args...);
\end{cpp}

\textbf{先决条件}
表达式`INVOKE(func,args)`提供的func和args必须是合法的。Callable和每个Args的成员都是可MoveConstructible(移动构造)。

\textbf{效果}
在同一个\texttt{std::once\_flag}对象上调用\texttt{std::call\_once}是串行的。如果之前没有在同一个\texttt{std::once\_flag}对象上调用过\texttt{std::call\_once},参数func(或副本)被调用,就像INVOKE(func, args),并且只有可调用的func不抛出任何异常时,调用\texttt{std::call\_once}才是有效的。当有异常抛出,异常会被调用函数进行传播。如果之前在\texttt{std::once\_flag}上的\texttt{std::call\_once}是有效的,那么再次调用\texttt{std::call\_once}将不会在调用func。

\textbf{同步}
在\texttt{std::once\_flag}上完成对\texttt{std::call\_once}的调用的先决条件是,后续所有对\texttt{std::call\_once}调用都在同一\texttt{std::once\_flag}对象。

\textbf{抛出}
当效果没有达到,或任何异常由调用func而传播,则抛出\texttt{std::system\_error}。