% # D.6 <ratio>头文件

\texttt{<ratio>}头文件提供在编译时进行的计算。

\textbf{头文件内容}

\begin{cpp}
namespace std
{
  template<intmax_t N,intmax_t D=1>
  class ratio;

  // ratio arithmetic
  template <class R1, class R2>
  using ratio_add = see description;

  template <class R1, class R2>
  using ratio_subtract = see description;

  template <class R1, class R2>
  using ratio_multiply = see description;

  template <class R1, class R2>
  using ratio_divide = see description;

  // ratio comparison
  template <class R1, class R2>
  struct ratio_equal;

  template <class R1, class R2>
  struct ratio_not_equal;

  template <class R1, class R2>
  struct ratio_less;

  template <class R1, class R2>
  struct ratio_less_equal;

  template <class R1, class R2>
  struct ratio_greater;

  template <class R1, class R2>
  struct ratio_greater_equal;

  typedef ratio<1, 1000000000000000000> atto;
  typedef ratio<1, 1000000000000000> femto;
  typedef ratio<1, 1000000000000> pico;
  typedef ratio<1, 1000000000> nano;
  typedef ratio<1, 1000000> micro;
  typedef ratio<1, 1000> milli;
  typedef ratio<1, 100> centi;
  typedef ratio<1, 10> deci;
  typedef ratio<10, 1> deca;
  typedef ratio<100, 1> hecto;
  typedef ratio<1000, 1> kilo;
  typedef ratio<1000000, 1> mega;
  typedef ratio<1000000000, 1> giga;
  typedef ratio<1000000000000, 1> tera;
  typedef ratio<1000000000000000, 1> peta;
  typedef ratio<1000000000000000000, 1> exa;
}
\end{cpp}

\mySubsubsection{D.6.1}{std::ratio类型模板}

\texttt{std::ratio}类型模板提供了一种对在编译时进行计算的机制,通过调用合理的数,例如:半(\texttt{std::ratio<1,2>}),2/3(std::ratio<2, 3>)或15/43(std::ratio<15, 43>)。其使用在C++标准库内部,用于初始化\texttt{std::chrono::duration}类型模板。

\textbf{类型定义}

\begin{cpp}
template <intmax_t N, intmax_t D = 1>
class ratio
{
public:
  typedef ratio<num, den> type;
  static constexpr intmax_t num= see below;
  static constexpr intmax_t den= see below;
};
\end{cpp}

\textbf{要求}
D不能为0。

\textbf{同步}
num和den分别为分子和分母,构造分数N/D。den总是正数。当N和D的符号相同,那么num为正数;否则num为负数。

\textbf{例子}

\begin{cpp}
ratio<4,6>::num == 2
ratio<4,6>::den == 3
ratio<4,-6>::num == -2
ratio<4,-6>::den == 3
\end{cpp}

\mySubsubsection{D.6.2}{std::ratio\_add模板别名}

\texttt{std::ratio\_add}模板别名提供了两个\texttt{std::ratio}在编译时相加的机制(使用有理计算)。

\textbf{定义}

\begin{cpp}
template <class R1, class R2>
using ratio_add = std::ratio<see below>;
\end{cpp}

\textbf{先决条件}
R1和R2必须使用\texttt{std::ratio}进行初始化。

\textbf{效果}
ratio\_add<R1, R2>被定义为一个别名,如果两数可以计算,且无溢出,该类型可以表示两个\texttt{std::ratio}对象R1和R2的和。如果计算出来的结果溢出了,那么程序里面就有问题了。在算术溢出的情况下,\texttt{std::ratio\_add<R1, R2>}应该应该与\texttt{std::ratio<R1::num * R2::den + R2::num * R1::den, R1::den * R2::den>}相同。

\textbf{例子}

\begin{cpp}
std::ratio_add<std::ratio<1,3>, std::ratio<2,5> >::num == 11
std::ratio_add<std::ratio<1,3>, std::ratio<2,5> >::den == 15

std::ratio_add<std::ratio<1,3>, std::ratio<7,6> >::num == 3
std::ratio_add<std::ratio<1,3>, std::ratio<7,6> >::den == 2
\end{cpp}

\mySubsubsection{D.6.3}{std::ratio\_subtract模板别名}

\texttt{std::ratio\_subtract}模板别名提供两个\texttt{std::ratio}数在编译时进行相减(使用有理计算)。

\textbf{定义}

\begin{cpp}
template <class R1, class R2>
using ratio_subtract = std::ratio<see below>;
\end{cpp}

\textbf{先决条件}
R1和R2必须使用\texttt{std::ratio}进行初始化。

\textbf{效果}
ratio\_add<R1, R2>被定义为一个别名,如果两数可以计算,且无溢出,该类型可以表示两个\texttt{std::ratio}对象R1和R2的和。如果计算出来的结果溢出了,那么程序里面就有问题了。在算术溢出的情况下,\texttt{std::ratio\_subtract<R1, R2>}应该应该与\texttt{std::ratio<R1::num * R2::den - R2::num * R1::den, R1::den * R2::den>}相同。

\textbf{例子}

\begin{cpp}
std::ratio_subtract<std::ratio<1,3>, std::ratio<1,5> >::num == 2
std::ratio_subtract<std::ratio<1,3>, std::ratio<1,5> >::den == 15

std::ratio_subtract<std::ratio<1,3>, std::ratio<7,6> >::num == -5
std::ratio_subtract<std::ratio<1,3>, std::ratio<7,6> >::den == 6
\end{cpp}

\mySubsubsection{D.6.4}{std::ratio\_multiply模板别名}

\texttt{std::ratio\_multiply}模板别名提供两个\texttt{std::ratio}数在编译时进行相乘(使用有理计算)。

\textbf{定义}

\begin{cpp}
template <class R1, class R2>
using ratio_multiply = std::ratio<see below>;
\end{cpp}

\textbf{先决条件}
R1和R2必须使用\texttt{std::ratio}进行初始化。

\textbf{效果}
ratio\_add<R1, R2>被定义为一个别名,如果两数可以计算,且无溢出,该类型可以表示两个\texttt{std::ratio}对象R1和R2的和。如果计算出来的结果溢出了,那么程序里面就有问题了。在算术溢出的情况下,\texttt{std::ratio\_multiply<R1, R2>}应该应该与\texttt{std::ratio<R1::num * R2::num, R1::den * R2::den>}相同。

\textbf{例子}

\begin{cpp}
std::ratio_multiply<std::ratio<1,3>, std::ratio<2,5> >::num == 2
std::ratio_multiply<std::ratio<1,3>, std::ratio<2,5> >::den == 15

std::ratio_multiply<std::ratio<1,3>, std::ratio<15,7> >::num == 5
std::ratio_multiply<std::ratio<1,3>, std::ratio<15,7> >::den == 7
\end{cpp}

\mySubsubsection{D.6.5}{std::ratio\_divide模板别名}

\texttt{std::ratio\_divide}模板别名提供两个\texttt{std::ratio}数在编译时进行相除(使用有理计算)。

\textbf{定义}

\begin{cpp}
template <class R1, class R2>
using ratio_multiply = std::ratio<see below>;
\end{cpp}

\textbf{先决条件}
R1和R2必须使用\texttt{std::ratio}进行初始化。

\textbf{效果}
ratio\_add<R1, R2>被定义为一个别名,如果两数可以计算,且无溢出,该类型可以表示两个\texttt{std::ratio}对象R1和R2的和。如果计算出来的结果溢出了,那么程序里面就有问题了。在算术溢出的情况下,\texttt{std::ratio\_multiply<R1, R2>}应该应该与\texttt{std::ratio<R1::num * R2::num * R2::den, R1::den * R2::den>}相同。

\textbf{例子}

\begin{cpp}
std::ratio_divide<std::ratio<1,3>, std::ratio<2,5> >::num == 5
std::ratio_divide<std::ratio<1,3>, std::ratio<2,5> >::den == 6

std::ratio_divide<std::ratio<1,3>, std::ratio<15,7> >::num == 7
std::ratio_divide<std::ratio<1,3>, std::ratio<15,7> >::den == 45
\end{cpp}

\mySubsubsection{D.6.6}{std::ratio\_equal类型模板}

\texttt{std::ratio\_equal}类型模板提供在编译时比较两个\texttt{std::ratio}数(使用有理计算)。

\textbf{类型定义}

\begin{cpp}
template <class R1, class R2>
class ratio_equal:
  public std::integral_constant<
    bool,(R1::num == R2::num) && (R1::den == R2::den)>
{};
\end{cpp}

\textbf{先决条件}
R1和R2必须使用\texttt{std::ratio}进行初始化。

\textbf{例子}

\begin{cpp}
std::ratio_equal<std::ratio<1,3>, std::ratio<2,6> >::value == true
std::ratio_equal<std::ratio<1,3>, std::ratio<1,6> >::value == false
std::ratio_equal<std::ratio<1,3>, std::ratio<2,3> >::value == false
std::ratio_equal<std::ratio<1,3>, std::ratio<1,3> >::value == true
\end{cpp}

\mySubsubsection{D.6.7}{std::ratio\_not\_equal类型模板}

\texttt{std::ratio\_not\_equal}类型模板提供在编译时比较两个\texttt{std::ratio}数(使用有理计算)。

\textbf{类型定义}

\begin{cpp}
template <class R1, class R2>
class ratio_not_equal:
  public std::integral_constant<bool,!ratio_equal<R1,R2>::value>
{};
\end{cpp}

\textbf{先决条件}
R1和R2必须使用\texttt{std::ratio}进行初始化。

\textbf{例子}

\begin{cpp}
std::ratio_not_equal<std::ratio<1,3>, std::ratio<2,6> >::value == false
std::ratio_not_equal<std::ratio<1,3>, std::ratio<1,6> >::value == true
std::ratio_not_equal<std::ratio<1,3>, std::ratio<2,3> >::value == true
std::ratio_not_equal<std::ratio<1,3>, std::ratio<1,3> >::value == false
\end{cpp}

\mySubsubsection{D.6.8}{std::ratio\_less类型模板}

\texttt{std::ratio\_less}类型模板提供在编译时比较两个\texttt{std::ratio}数(使用有理计算)。

\textbf{类型定义}

\begin{cpp}
template <class R1, class R2>
class ratio_less:
  public std::integral_constant<bool,see below>
{};
\end{cpp}

\textbf{先决条件}
R1和R2必须使用\texttt{std::ratio}进行初始化。

\textbf{效果}
std::ratio\_less<R1,R2>可通过\texttt{std::integral\_constant<bool, value >}导出,这里value为\texttt{(R1::num*R2::den) < (R2::num*R1::den)}。如果有可能,需要实现使用一种机制来避免计算结果已出。当溢出发生,那么程序中就肯定有错误。

\textbf{例子}

\begin{cpp}
std::ratio_less<std::ratio<1,3>, std::ratio<2,6> >::value == false
std::ratio_less<std::ratio<1,6>, std::ratio<1,3> >::value == true
std::ratio_less<
  std::ratio<999999999,1000000000>,
  std::ratio<1000000001,1000000000> >::value == true
std::ratio_less<
  std::ratio<1000000001,1000000000>,
  std::ratio<999999999,1000000000> >::value == false
\end{cpp}

\mySubsubsection{D.6.9}{std::ratio\_greater类型模板}

\texttt{std::ratio\_greater}类型模板提供在编译时比较两个\texttt{std::ratio}数(使用有理计算)。

\textbf{类型定义}

\begin{cpp}
template <class R1, class R2>
class ratio_greater:
  public std::integral_constant<bool,ratio_less<R2,R1>::value>
{};
\end{cpp}

\textbf{先决条件}
R1和R2必须使用\texttt{std::ratio}进行初始化。

\mySubsubsection{D.6.10}{std::ratio\_less\_equal类型模板}

\texttt{std::ratio\_less\_equal}类型模板提供在编译时比较两个\texttt{std::ratio}数(使用有理计算)。

\textbf{类型定义}

\begin{cpp}
template <class R1, class R2>
class ratio_less_equal:
  public std::integral_constant<bool,!ratio_less<R2,R1>::value>
{};
\end{cpp}

\textbf{先决条件}
R1和R2必须使用\texttt{std::ratio}进行初始化。

\mySubsubsection{D.6.11}{std::ratio\_greater\_equal类型模板}

\texttt{std::ratio\_greater\_equal}类型模板提供在编译时比较两个\texttt{std::ratio}数(使用有理计算)。

\textbf{类型定义}

\begin{cpp}
template <class R1, class R2>
class ratio_greater_equal:
  public std::integral_constant<bool,!ratio_less<R1,R2>::value>
{};
\end{cpp}

\textbf{先决条件}
R1和R2必须使用\texttt{std::ratio}进行初始化。