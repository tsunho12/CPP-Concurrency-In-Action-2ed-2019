% # 附录D C++线程库参考

% # D.1 <chrono>头文件

<chrono>头文件作为\texttt{time\_point}的提供者,具有代表时间点的类,duration类和时钟类。每个时钟都有一个\texttt{is\_steady}静态数据成员,这个成员用来表示该时钟是否是一个\textit{稳定的}时钟(以匀速计时的时钟,且不可调节)。\texttt{std::chrono::steady\_clock}是唯一个能保证稳定的时钟类。

头文件正文

\begin{cpp}
namespace std
{
  namespace chrono
  {
    template<typename Rep,typename Period = ratio<1>>
    class duration;
    template<
        typename Clock,
        typename Duration = typename Clock::duration>
    class time_point;
    class system_clock;
    class steady_clock;
    typedef unspecified-clock-type high_resolution_clock;
  }
}
\end{cpp}

% ## D.1.1 std::chrono::duration类型模板

\texttt{std::chrono::duration}类模板可以用来表示时间。模板参数\texttt{Rep}和\texttt{Period}是用来存储持续时间的数据类型,\texttt{std::ratio}实例代表了时间的长度(几分之一秒),其表示了在两次“时钟滴答”后的时间(时钟周期)。因此,\texttt{std::chrono::duration<int, std::milli>}即为,时间以毫秒数的形式存储到int类型中,而\texttt{std::chrono::duration<short, std::ratio<1,50>>}则是记录1/50秒的个数,并将个数存入short类型的变量中,还有\texttt{std::chrono::duration <long long, std::ratio<60,1>>}则是将分钟数存储到long long类型的变量中。

%### 类的定义

\begin{cpp}
template <class Rep, class Period=ratio<1> >
class duration
{
public:
  typedef Rep rep;
  typedef Period period;

  constexpr duration() = default;
  ~duration() = default;

  duration(const duration&) = default;
  duration& operator=(const duration&) = default;

  template <class Rep2>
  constexpr explicit duration(const Rep2& r);

  template <class Rep2, class Period2>
  constexpr duration(const duration<Rep2, Period2>& d);

  constexpr rep count() const;
  constexpr duration operator+() const;
  constexpr duration operator-() const;

  duration& operator++();
  duration operator++(int);
  duration& operator--();
  duration operator--(int);

  duration& operator+=(const duration& d);
  duration& operator-=(const duration& d);
  duration& operator*=(const rep& rhs);
  duration& operator/=(const rep& rhs);

  duration& operator%=(const rep& rhs);
  duration& operator%=(const duration& rhs);

  static constexpr duration zero();
  static constexpr duration min();
  static constexpr duration max();
};

template <class Rep1, class Period1, class Rep2, class Period2>
constexpr bool operator==(
    const duration<Rep1, Period1>& lhs,
    const duration<Rep2, Period2>& rhs);

template <class Rep1, class Period1, class Rep2, class Period2>
    constexpr bool operator!=(
    const duration<Rep1, Period1>& lhs,
    const duration<Rep2, Period2>& rhs);

template <class Rep1, class Period1, class Rep2, class Period2>
    constexpr bool operator<(
    const duration<Rep1, Period1>& lhs,
    const duration<Rep2, Period2>& rhs);

template <class Rep1, class Period1, class Rep2, class Period2>
    constexpr bool operator<=(
    const duration<Rep1, Period1>& lhs,
    const duration<Rep2, Period2>& rhs);

template <class Rep1, class Period1, class Rep2, class Period2>
    constexpr bool operator>(
    const duration<Rep1, Period1>& lhs,
    const duration<Rep2, Period2>& rhs);

template <class Rep1, class Period1, class Rep2, class Period2>
    constexpr bool operator>=(
    const duration<Rep1, Period1>& lhs,
    const duration<Rep2, Period2>& rhs);

template <class ToDuration, class Rep, class Period>
    constexpr ToDuration duration_cast(const duration<Rep, Period>& d);
\end{cpp}

\textbf{要求}

\texttt{Rep}必须是内置数值类型,或是自定义的类数值类型。

\texttt{Period}必须是\texttt{std::ratio<>}实例。

%### std::chrono::duration::Rep 类型

用来记录\texttt{dration}中时钟周期的数量。

\textbf{声明}

\begin{cpp}
typedef Rep rep;
\end{cpp}

\texttt{rep}类型用来记录\texttt{duration}对象内部的表示。

%### std::chrono::duration::Period 类型

这个类型必须是一个\texttt{std::ratio}特化实例,用来表示在继续时间中,1s所要记录的次数。例如,当\texttt{period}是\texttt{std::ratio<1, 50>},\texttt{duration}变量的count()就会在N秒钟返回50N。

\textbf{声明}

\begin{cpp}
typedef Period period;
\end{cpp}

%### std::chrono::duration 默认构造函数

使用默认值构造\texttt{std::chrono::duration}实例

\textbf{声明}

\begin{cpp}
constexpr duration() = default;
\end{cpp}

\textbf{效果}
\texttt{duration}内部值(例如\texttt{rep}类型的值)都已初始化。

%### std::chrono::duration 需要计数值的转换构造函数

通过给定的数值来构造\texttt{std::chrono::duration}实例。

\textbf{声明}

\begin{cpp}
template <class Rep2>;
constexpr explicit duration(const Rep2& r);
\end{cpp}

\textbf{效果}

\texttt{duration}对象的内部值会使用\texttt{static\_cast<rep>(r)}进行初始化。

\textbf{结果}
当Rep2隐式转换为Rep,Rep是浮点类型或Rep2不是浮点类型,这个构造函数才能使用。

\textbf{后验条件}

\begin{cpp}
this->count()==static_cast<rep>(r)
\end{cpp}

%### std::chrono::duration 需要另一个std::chrono::duration值的转化构造函数

通过另一个\texttt{std::chrono::duration}类实例中的计数值来构造一个\texttt{std::chrono::duration}类实例。

\textbf{声明}

\begin{cpp}
template <class Rep2, class Period>
constexpr duration(const duration<Rep2,Period2>& d);
\end{cpp}

\textbf{结果}

duration对象的内部值通过\texttt{duration\_cast<duration<Rep,Period>>(d).count()}初始化。

\textbf{要求}

当Rep是一个浮点类或Rep2不是浮点类,且Period2是Period数的倍数(比如,ratio\_divide&lt;Period2,Period&gt;::den==1)时,才能调用该重载。当一个较小的数据转换为一个较大的数据时,使用该构造函数就能避免数位截断和精度损失。

\textbf{后验条件}

\texttt{this->count() == dutation\_cast&lt;duration<Rep, Period>>(d).count()}

}例子}

\begin{cpp}
duration<int, ratio<1, 1000>> ms(5);  // 5毫秒
duration<int, ratio<1, 1>> s(ms);  // 错误:不能将ms当做s进行存储
duration<double, ratio<1,1>> s2(ms);  // 合法:s2.count() == 0.005
duration<int, ration<1, 1000000>> us<ms>;  // 合法:us.count() == 5000
\end{cpp}

%### std::chrono::duration::count 成员函数

查询持续时长。

\textbf{声明}

\begin{cpp}
constexpr rep count() const;
\end{cpp}

\textbf{返回}

返回duration的内部值,其值类型和rep一样。

%### std::chrono::duration::operator+ 加法操作符

这是一个空操作:只会返回*this的副本。

\textbf{声明}

\begin{cpp}
constexpr duration operator+() const;
\end{cpp}

\textbf{返回}
\texttt{*this}

%### std::chrono::duration::operator- 减法操作符

返回将内部值只为负数的*this副本。

\texttt{声明}

\begin{cpp}
constexpr duration operator-() const;
\end{cpp}

\textbf{返回}
\texttt{duration(--this->count());}

%### std::chrono::duration::operator++ 前置自加操作符

增加内部计数值。

\textbf{声明}

\begin{cpp}
duration& operator++();
\end{cpp}

\textbf{结果}

\begin{cpp}
++this->internal_count;
\end{cpp}

\textbf{返回}
\texttt{*this}

%### std::chrono::duration::operator++ 后置自加操作符

自加内部计数值,并且返回还没有增加前的*this。

\textbf{声明}

\begin{cpp}
duration operator++(int);
\end{cpp}

\textbf{结果}

\begin{cpp}
duration temp(*this);
++(*this);
return temp;
\end{cpp}

%### std::chrono::duration::operator-- 前置自减操作符

自减内部计数值

\textbf{声明}

\begin{cpp}
duration& operator--();
\end{cpp}

\textbf{结果}

\begin{cpp}
--this->internal_count;
\end{cpp}

\textbf{返回}
\texttt{*this}

%### std::chrono::duration::operator-- 前置自减操作符

自减内部计数值,并且返回还没有减少前的*this。

\textbf{声明}

\begin{cpp}
duration operator--(int);
\end{cpp}

\textbf{结果}

\begin{cpp}
duration temp(*this);
--(*this);
return temp;
\end{cpp}

%### std::chrono::duration::operator+= 复合赋值操作符

将其他duration对象中的内部值增加到现有duration对象当中。

\textbf{声明}

\begin{cpp}
duration& operator+=(duration const& other);
\end{cpp}

\textbf{结果}

\begin{cpp}
internal_count+=other.count();
\end{cpp}

\textbf{返回}
\texttt{*this}

%### std::chrono::duration::operator-= 复合赋值操作符

现有duration对象减去其他duration对象中的内部值。

\textbf{声明}

\begin{cpp}
duration& operator-=(duration const& other);
\end{cpp}

\textbf{结果}

\begin{cpp}
internal_count-=other.count();
\end{cpp}

\textbf{返回}
\texttt{*this}

%### std::chrono::duration::operator*= 复合赋值操作符

内部值乘以一个给定的值。

\textbf{声明}

\begin{cpp}
duration& operator*=(rep const& rhs);
\end{cpp}

\textbf{结果}

\begin{cpp}
internal_count*=rhs;
\end{cpp}

\textbf{返回}
\texttt{*this}

%### std::chrono::duration::operator/= 复合赋值操作符

内部值除以一个给定的值。

\textbf{声明}

\begin{cpp}
duration& operator/=(rep const& rhs);
\end{cpp}

\textbf{结果}

\begin{cpp}
internal_count/=rhs;
\end{cpp}

\textbf{返回}
\texttt{*this}

%### std::chrono::duration::operator%= 复合赋值操作符

内部值对一个给定的值求余。

\textbf{声明}

\begin{cpp}
duration& operator%=(rep const& rhs);
\end{cpp}

\textbf{结果}

\begin{cpp}
internal_count%=rhs;
\end{cpp}

\textbf{返回}
\texttt{*this}

%### std::chrono::duration::operator%= 复合赋值操作符(重载)

内部值对另一个duration类的内部值求余。

\textbf{声明}

\begin{cpp}
duration& operator%=(duration const& rhs);
\end{cpp}

\textbf{结果}

\begin{cpp}
internal_count%=rhs.count();
\end{cpp}

\textbf{返回}
\texttt{*this}

%### std::chrono::duration::zero 静态成员函数

返回一个内部值为0的duration对象。

\textbf{声明}

\begin{cpp}
constexpr duration zero();
\end{cpp}

\textbf{返回}

\begin{cpp}
duration(duration_values<rep>::zero());
\end{cpp}

%### std::chrono::duration::min 静态成员函数

返回duration类实例化后能表示的最小值。

\textbf{声明}

\begin{cpp}
constexpr duration min();
\end{cpp}

\textbf{返回}

\begin{cpp}
duration(duration_values<rep>::min());
\end{cpp}

%### std::chrono::duration::max 静态成员函数

返回duration类实例化后能表示的最大值。

\textbf{声明}

\begin{cpp}
constexpr duration max();
\end{cpp}

\textbf{返回}

\begin{cpp}
duration(duration_values<rep>::max());
\end{cpp}

%### std::chrono::duration 等于比较操作符

比较两个duration对象是否相等。

\textbf{声明}

\begin{cpp}
template <class Rep1, class Period1, class Rep2, class Period2>
constexpr bool operator==(
const duration<Rep1, Period1>& lhs,
const duration<Rep2, Period2>& rhs);
\end{cpp}

\textbf{要求}

\texttt{lhs}和\texttt{rhs}两种类型可以互相进行隐式转换。当两种类型无法进行隐式转换,或是可以互相转换的两个不同类型的duration类,则表达式不合理。

\textbf{结果}

当}CommonDuration}和\texttt{std::common\_type< duration< Rep1, Period1>, duration< Rep2, Period2>>::type}同类,那么\texttt{lhs==rhs}就会返回}CommonDuration(lhs).count()==CommonDuration(rhs).count()}。

%### std::chrono::duration 不等于比较操作符

比较两个duration对象是否不相等。

\textbf{声明}

\begin{cpp}
template <class Rep1, class Period1, class Rep2, class Period2>
constexpr bool operator!=(
   const duration<Rep1, Period1>& lhs,
   const duration<Rep2, Period2>& rhs);
\end{cpp}

\textbf{要求}

\texttt{lhs}和\texttt{rhs}两种类型可以互相进行隐式转换。当两种类型无法进行隐式转换,或是可以互相转换的两个不同类型的duration类,则表达式不合理。

\textbf{返回}
\texttt{!(lhs==rhs)}

%### std::chrono::duration 小于比较操作符

比较两个duration对象是否小于。

\textbf{声明}

\begin{cpp}
template <class Rep1, class Period1, class Rep2, class Period2>
constexpr bool operator<(
   const duration<Rep1, Period1>& lhs,
   const duration<Rep2, Period2>& rhs);
\end{cpp}

\textbf{要求}

\texttt{lhs}和\texttt{rhs}两种类型可以互相进行隐式转换。当两种类型无法进行隐式转换,或是可以互相转换的两个不同类型的duration类,则表达式不合理。

\textbf{结果}

当}CommonDuration}和\texttt{std::common\_type< duration< Rep1, Period1>, duration< Rep2, Period2>>::type}同类,那么\texttt{lhs&lt;rhs}就会返回}CommonDuration(lhs).count()&lt;CommonDuration(rhs).count()}。

%### std::chrono::duration 大于比较操作符

比较两个duration对象是否大于。

\textbf{声明}

\begin{cpp}
template <class Rep1, class Period1, class Rep2, class Period2>
constexpr bool operator>(
   const duration<Rep1, Period1>& lhs,
   const duration<Rep2, Period2>& rhs);
\end{cpp}

\textbf{要求}

\texttt{lhs}和\texttt{rhs}两种类型可以互相进行隐式转换。当两种类型无法进行隐式转换,或是可以互相转换的两个不同类型的duration类,则表达式不合理。

\textbf{返回}
\texttt{rhs<lhs}

%### std::chrono::duration 小于等于比较操作符

比较两个duration对象是否小于等于。

\textbf{声明}

\begin{cpp}
template <class Rep1, class Period1, class Rep2, class Period2>
constexpr bool operator<=(
   const duration<Rep1, Period1>& lhs,
   const duration<Rep2, Period2>& rhs);
\end{cpp}

\textbf{要求}

\texttt{lhs}和\texttt{rhs}两种类型可以互相进行隐式转换。当两种类型无法进行隐式转换,或是可以互相转换的两个不同类型的duration类,则表达式不合理。

\textbf{返回}
\texttt{!(rhs<lhs)}

%### std::chrono::duration 大于等于比较操作符

比较两个duration对象是否大于等于。

\textbf{声明}

\begin{cpp}
template <class Rep1, class Period1, class Rep2, class Period2>
constexpr bool operator>=(
   const duration<Rep1, Period1>& lhs,
   const duration<Rep2, Period2>& rhs);
\end{cpp}

\textbf{要求}

\texttt{lhs}和\texttt{rhs}两种类型可以互相进行隐式转换。当两种类型无法进行隐式转换,或是可以互相转换的两个不同类型的duration类,则表达式不合理。

\textbf{返回}
\texttt{!(lhs<rhs)}

%### std::chrono::duration\_cast 非成员函数

显示将一个\texttt{std::chrono::duration}对象转化为另一个\texttt{std::chrono::duration}实例。

\textbf{声明}

\begin{cpp}
template <class ToDuration, class Rep, class Period>
constexpr ToDuration duration_cast(const duration<Rep, Period>& d);
\end{cpp}

\textbf{要求}

ToDuration必须是\texttt{std::chrono::duration}的实例。

\textbf{返回}

duration类d转换为指定类型ToDuration。这种方式可以在不同尺寸和表示类型的转换中尽可能减少精度损失。

## D.1.2 std::chrono::time\_point类型模板

\texttt{std::chrono::time\_point}类型模板通过(特别的)时钟来表示某个时间点。这个时钟代表的是从epoch(1970-01-01 00:00:00 UTC,作为UNIX系列系统的特定时间戳)到现在的时间。模板参数Clock代表使用的使用(不同的使用必定有自己独特的类型),而Duration模板参数使用来测量从epoch到现在的时间,并且这个参数的类型必须是\texttt{std::chrono::duration}类型。Duration默认存储Clock上的测量值。

%### 类型定义

\begin{cpp}
template <class Clock,class Duration = typename Clock::duration>
class time_point
{
public:
  typedef Clock clock;
  typedef Duration duration;
  typedef typename duration::rep rep;
  typedef typename duration::period period;

  time_point();
  explicit time_point(const duration& d);

  template <class Duration2>
  time_point(const time_point<clock, Duration2>& t);

  duration time_since_epoch() const;

  time_point& operator+=(const duration& d);
  time_point& operator-=(const duration& d);

  static constexpr time_point min();
  static constexpr time_point max();
};
\end{cpp}

%### std::chrono::time\_point 默认构造函数

构造time\_point代表着,使用相关的Clock,记录从epoch到现在的时间;其内部计时使用Duration::zero()进行初始化。

\textbf{声明}

\begin{cpp}
time_point();
\end{cpp}

\textbf{后验条件}

对于使用默认构造函数构造出的time\_point对象tp,\texttt{tp.time\_since\_epoch() == tp::duration::zero()}。

%### std::chrono::time\_point 需要时间长度的构造函数

构造time\_point代表着,使用相关的Clock,记录从epoch到现在的时间。

\textbf{声明}

\begin{cpp}
explicit time_point(const duration& d);
\end{cpp}

\textbf{后验条件}

当有一个time\_point对象tp,是通过duration d构造出来的(tp(d)),那么\texttt{tp.time\_since\_epoch() == d}。

%### std::chrono::time\_point 转换构造函数

构造time\_point代表着,使用相关的Clock,记录从epoch到现在的时间。

\textbf{声明}

\begin{cpp}
template <class Duration2>
time_point(const time_point<clock, Duration2>& t);
\end{cpp}

\textbf{要求}

Duration2必须呢个隐式转换为Duration。

\textbf{效果}

当}time\_point(t.time\_since\_epoch())}存在,从t.time\_since\_epoch()中获取的返回值,可以隐式转换成Duration类型的对象,并且这个值可以存储在一个新的time\_point对象中。

(扩展阅读:[as-if准则](http://stackoverflow.com/questions/15718262/what-exactly-is-the-as-if-rule))

%### std::chrono::time\_point::time\_since\_epoch 成员函数

返回当前time\_point从epoch到现在的具体时长。

\textbf{声明}

\begin{cpp}
duration time_since_epoch() const;
\end{cpp}

\textbf{返回}

duration的值存储在*this中。

%### std::chrono::time\_point::operator+= 复合赋值函数

将指定的duration的值与原存储在指定的time\_point对象中的duration相加,并将加后值存储在*this对象中。

\textbf{声明}

\begin{cpp}
time_point& operator+=(const duration& d);
\end{cpp}

\textbf{效果}

将d的值和duration对象的值相加,存储在*this中,就如同this-&gt;internal\_duration += d;

\textbf{返回}
\texttt{*this}

%### std::chrono::time\_point::operator-= 复合赋值函数

将指定的duration的值与原存储在指定的time\_point对象中的duration相减,并将加后值存储在*this对象中。

\textbf{声明}

\begin{cpp}
time_point& operator-=(const duration& d);
\end{cpp}

\textbf{效果}

将d的值和duration对象的值相减,存储在*this中,就如同this-&gt;internal\_duration -= d;

\textbf{返回}
\texttt{*this}

%### std::chrono::time\_point::min 静态成员函数

获取time\_point对象可能表示的最小值。

\textbf{声明}

\begin{cpp}
static constexpr time_point min();
\end{cpp}

\textbf{返回}

\begin{cpp}
time_point(time_point::duration::min()) (see 11.1.1.15)
\end{cpp}

%### std::chrono::time\_point::max 静态成员函数

获取time\_point对象可能表示的最大值。

\textbf{声明}

\begin{cpp}
static constexpr time_point max();
\end{cpp}

\textbf{返回}

\begin{cpp}
time_point(time_point::duration::max()) (see 11.1.1.16)
\end{cpp}

##D.1.3 std::chrono::system\_clock类

\texttt{std::chrono::system\_clock}类提供给了从系统实时时钟上获取当前时间功能。可以调用\texttt{std::chrono::system\_clock::now()}来获取当前的时间。\texttt{std::chrono::system\_clock::time\_point}也可以通过\texttt{std::chrono::system\_clock::to\_time\_t()}和\texttt{std::chrono::system\_clock::to\_time\_point()}函数返回值转换成time\_t类型。系统时钟不稳定,所以\texttt{std::chrono::system\_clock::now()}获取到的时间可能会早于之前的一次调用(比如,时钟被手动调整过或与外部时钟进行了同步)。

%###类型定义

\begin{cpp}
class system_clock
{
public:
  typedef unspecified-integral-type rep;
  typedef std::ratio<unspecified,unspecified> period;
  typedef std::chrono::duration<rep,period> duration;
  typedef std::chrono::time_point<system_clock> time_point;
  static const bool is_steady=unspecified;

  static time_point now() noexcept;

  static time_t to_time_t(const time_point& t) noexcept;
  static time_point from_time_t(time_t t) noexcept;
};
\end{cpp}

%### std::chrono::system\_clock::rep 类型定义

将时间周期数记录在一个duration值中

\textbf{声明}

\begin{cpp}
typedef unspecified-integral-type rep;
\end{cpp}

%### std::chrono::system\_clock::period 类型定义

类型为\texttt{std::ratio}类型模板,通过在两个不同的duration或time\_point间特化最小秒数(或将1秒分为好几份)。period指定了时钟的精度,而非时钟频率。

\textbf{声明}

\begin{cpp}
typedef std::ratio<unspecified,unspecified> period;
\end{cpp}

%### std::chrono::system\_clock::duration 类型定义

类型为\texttt{std::ratio}类型模板,通过系统实时时钟获取两个时间点之间的时长。

\textbf{声明}

\begin{cpp}
typedef std::chrono::duration<
   std::chrono::system_clock::rep,
   std::chrono::system_clock::period> duration;
\end{cpp}

%### std::chrono::system\_clock::time\_point 类型定义

类型为\texttt{std::ratio}类型模板,通过系统实时时钟获取当前时间点的时间。

\textbf{声明}

\begin{cpp}
typedef std::chrono::time_point&lt;std::chrono::system_clock&gt; time_point;
\end{cpp}

%### std::chrono::system\_clock::now 静态成员函数

从系统实时时钟上获取当前的外部设备显示的时间。

\textbf{声明}

\begin{cpp}
time_point now() noexcept;
\end{cpp}

\textbf{返回}

time\_point类型变量来代表当前系统实时时钟的时间。

\textbf{抛出}

当错误发生,\texttt{std::system\_error}异常将会抛出。

%### std::chrono::system\_clock:to\_time\_t 静态成员函数

将time\_point类型值转化为time\_t。

\textbf{声明}

\begin{cpp}
time_t to_time_t(time_point const& t) noexcept;
\end{cpp}

\textbf{返回}

通过对t进行舍入或截断精度,将其转化为一个time\_t类型的值。

\textbf{抛出}

当错误发生,\texttt{std::system\_error}异常将会抛出。

%### std::chrono::system\_clock::from\_time\_t 静态成员函数

\textbf{声明}

\begin{cpp}
time_point from_time_t(time_t const& t) noexcept;
\end{cpp}

\textbf{返回}

time\_point中的值与t中的值一样。

\textbf{抛出}

当错误发生,\texttt{std::system\_error}异常将会抛出。

## D.1.4 std::chrono::steady\_clock类

\texttt{std::chrono::steady\_clock}能访问系统稳定时钟。可以通过调用\texttt{std::chrono::steady\_clock::now()}获取当前的时间。设备上显示的时间,与使用\texttt{std::chrono::steady\_clock::now()}获取的时间没有固定的关系。稳定时钟是无法回调的,所以在\texttt{std::chrono::steady\_clock::now()}两次调用后,第二次调用获取的时间必定等于或大于第一次获得的时间。

%### 类型定义

\begin{cpp}
class steady_clock
{
public:
  typedef unspecified-integral-type rep;
  typedef std::ratio<
      unspecified,unspecified> period;
  typedef std::chrono::duration<rep,period> duration;
  typedef std::chrono::time_point<steady_clock>
      time_point;
  static const bool is_steady=true;

  static time_point now() noexcept;
};
\end{cpp}

%### std::chrono::steady\_clock::rep 类型定义

定义一个整型,用来保存duration的值。

\textbf{声明}

\begin{cpp}
typedef unspecified-integral-type rep;
\end{cpp}

%### std::chrono::steady\_clock::period 类型定义

类型为\texttt{std::ratio}类型模板,通过在两个不同的duration或time\_point间特化最小秒数(或将1秒分为好几份)。period指定了时钟的精度,而非时钟频率。

\textbf{声明}

\begin{cpp}
typedef std::ratio<unspecified,unspecified> period;
\end{cpp}

%### std::chrono::steady\_clock::duration 类型定义

类型为\texttt{std::ratio}类型模板,通过系统实时时钟获取两个时间点之间的时长。

\textbf{声明}

\begin{cpp}
typedef std::chrono::duration<
   std::chrono::system_clock::rep,
   std::chrono::system_clock::period> duration;
\end{cpp}

%### std::chrono::steady\_clock::time\_point 类型定义

\texttt{std::chrono::time\_point}类型实例,可以存储从系统稳定时钟返回的时间点。

\textbf{声明}

\begin{cpp}
typedef std::chrono::time_point<std::chrono::steady_clock> time_point;
\end{cpp}

%### std::chrono::steady\_clock::now 静态成员函数

从系统稳定时钟获取当前时间。

\textbf{声明}

\begin{cpp}
time_point now() noexcept;
\end{cpp}

\textbf{返回}

time\_point表示当前系统稳定时钟的时间。

\textbf{抛出}
当遇到错误,会抛出\texttt{std::system\_error}异常。

}同步}
当先行调用过一次\texttt{std::chrono::steady\_clock::now()},那么下一次time\_point获取的值,一定大于等于第一次获取的值。

## D.1.5 std::chrono::high\_resolution\_clock类定义

\texttt{std::chrono::high\_resolution\_clock}类能访问系统高精度时钟。和所有其他时钟一样,通过调用\texttt{std::chrono::high\_resolution\_clock::now()}来获取当前时间。\texttt{std::chrono::high\_resolution\_clock}可能是\texttt{std::chrono::system\_clock}类或\texttt{std::chrono::steady\_clock}类的别名,也可能就是独立的一个类。

通过\texttt{std::chrono::high\_resolution\_clock}具有所有标准库支持时钟中最高的精度,这就意味着使用
\texttt{std::chrono::high\_resolution\_clock::now()}要花掉一些时间。所以,当你再调用\texttt{std::chrono::high\_resolution\_clock::now()}的时候,需要注意函数本身的时间开销。

%### 类型定义

\begin{cpp}
class high_resolution_clock
{
public:
  typedef unspecified-integral-type rep;
  typedef std::ratio<
      unspecified,unspecified> period;
  typedef std::chrono::duration<rep,period> duration;
  typedef std::chrono::time_point<
      unspecified> time_point;
  static const bool is_steady=unspecified;

  static time_point now() noexcept;
};
\end{cpp}