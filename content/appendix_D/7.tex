% # D.7 <thread>头文件

\texttt{<thread>}头文件提供了管理和辨别线程的工具,并且提供函数,可让当前线程休眠。

\textbf{头文件内容}

\begin{cpp}
namespace std
{
  class thread;

  namespace this_thread
  {
    thread::id get_id() noexcept;

    void yield() noexcept;

    template<typename Rep,typename Period>
    void sleep_for(
        std::chrono::duration<Rep,Period> sleep_duration);

    template<typename Clock,typename Duration>
    void sleep_until(
        std::chrono::time_point<Clock,Duration> wake_time);
  }
}
\end{cpp}

\mySubsubsection{D.7.1}{std::thread类}

\texttt{std::thread}用来管理线程的执行。其提供让新的线程执行或执行,也提供对线程的识别,以及提供其他函数用于管理线程的执行。

\begin{cpp}
class thread
{
public:
  // Types
  class id;
  typedef implementation-defined native_handle_type; // optional

  // Construction and Destruction
  thread() noexcept;
  ~thread();

  template<typename Callable,typename Args...>
  explicit thread(Callable&& func,Args&&... args);

  // Copying and Moving
  thread(thread const& other) = delete;
  thread(thread&& other) noexcept;

  thread& operator=(thread const& other) = delete;
  thread& operator=(thread&& other) noexcept;

  void swap(thread& other) noexcept;

  void join();
  void detach();
  bool joinable() const noexcept;

  id get_id() const noexcept;
  native_handle_type native_handle();
  static unsigned hardware_concurrency() noexcept;
};

void swap(thread& lhs,thread& rhs);
\end{cpp}

% % ###std::thread::id 类

可以通过\texttt{std::thread::id}实例对执行线程进行识别。

\textbf{类型定义}

\begin{cpp}
class thread::id
{
public:
  id() noexcept;
};

bool operator==(thread::id x, thread::id y) noexcept;
bool operator!=(thread::id x, thread::id y) noexcept;
bool operator<(thread::id x, thread::id y) noexcept;
bool operator<=(thread::id x, thread::id y) noexcept;
bool operator>(thread::id x, thread::id y) noexcept;
bool operator>=(thread::id x, thread::id y) noexcept;

template<typename charT, typename traits>
basic_ostream<charT, traits>&
operator<< (basic_ostream<charT, traits>&& out, thread::id id);
\end{cpp}

\textbf{NOTEs}
\texttt{std::thread::id}的值可以识别不同的执行,每个\texttt{std::thread::id}默认构造出来的值都不一样,不同值代表不同的执行线程。

\texttt{std::thread::id}的值是不可预测的,在同一程序中的不同线程的id也不同。

\texttt{std::thread::id}是可以CopyConstructible(拷贝构造)和CopyAssignable(拷贝赋值),所以对于\texttt{std::thread::id}的拷贝和赋值是没有限制的。

% % #% ###std::thread::id 默认构造函数

构造一个\texttt{std::thread::id}对象,其不能表示任何执行线程。

\textbf{声明}

\begin{cpp}
id() noexcept;
\end{cpp}

\textbf{效果}
构造一个\texttt{std::thread::id}实例,不能表示任何一个线程值。

\textbf{抛出}
无

\textbf{NOTE} 所有默认构造的\texttt{std::thread::id}实例存储的同一个值。

% #% ###std::thread::id 相等比较操作

比较两个\texttt{std::thread::id}的值,看是两个执行线程是否相等。

\textbf{声明}

\begin{cpp}
bool operator==(std::thread::id lhs,std::thread::id rhs) noexcept;
\end{cpp}

\textbf{返回}
当lhs和rhs表示同一个执行线程或两者不代表没有任何线程,则返回true。当lsh和rhs表示不同执行线程或其中一个代表一个执行线程,另一个不代表任何线程,则返回false。

\textbf{抛出}
无

% #% ###std::thread::id 不相等比较操作

比较两个\texttt{std::thread::id}的值,看是两个执行线程是否相等。

\textbf{声明}

\begin{cpp}
bool operator!=(std::thread::id lhs,std::thread::id rhs) noexcept;
\end{cpp}

\textbf{返回}
\texttt{!(lhs==rhs)}

\textbf{抛出}
无

% #% ###std::thread::id 小于比较操作

比较两个\texttt{std::thread::id}的值,看是两个执行线程哪个先执行。

\textbf{声明}

\begin{cpp}
bool operator<(std::thread::id lhs,std::thread::id rhs) noexcept;
\end{cpp}

\textbf{返回}
当lhs比rhs的线程ID靠前,则返回true。当lhs!=rhs,且\texttt{lhs<rhs}或\texttt{rhs<lhs}返回true,其他情况则返回false。当lhs==rhs,在\texttt{lhs<rhs}和\texttt{rhs<lhs}时返回false。

\textbf{抛出}
无

\textbf{NOTE} 当默认构造的\texttt{std::thread::id}实例,在不代表任何线程的时候,其值小于任何一个代表执行线程的实例。当两个实例相等,那么两个对象代表两个执行线程。任何一组不同的\texttt{std::thread::id}的值,是由同一序列构造,这与程序执行的顺序相同。同一个可执行程序可能有不同的执行顺序。

% #% ###std::thread::id 小于等于比较操作

比较两个\texttt{std::thread::id}的值,看是两个执行线程的ID值是否相等,或其中一个先行。

\textbf{声明}

\begin{cpp}
bool operator<(std::thread::id lhs,std::thread::id rhs) noexcept;
\end{cpp}

\textbf{返回}
\texttt{!(rhs<lhs)}

\textbf{抛出}
无

% #% ###std::thread::id 大于比较操作

比较两个\texttt{std::thread::id}的值,看是两个执行线程的是后行的。

\textbf{声明}

\begin{cpp}
bool operator>(std::thread::id lhs,std::thread::id rhs) noexcept;
\end{cpp}

\textbf{返回}
\texttt{rhs<lhs}

\textbf{抛出}
无

% #% ###std::thread::id 大于等于比较操作

比较两个\texttt{std::thread::id}的值,看是两个执行线程的ID值是否相等,或其中一个后行。

\textbf{声明}

\begin{cpp}
bool operator>=(std::thread::id lhs,std::thread::id rhs) noexcept;
\end{cpp}

\textbf{返回}
\texttt{!(lhs<rhs)}

\textbf{抛出}
无

% #% ###std::thread::id 插入流操作

将\texttt{std::thread::id}的值通过给指定流写入字符串。

\textbf{声明}

\begin{cpp}
template<typename charT, typename traits>
basic_ostream<charT, traits>&
operator<< (basic_ostream<charT, traits>&& out, thread::id id);
\end{cpp}

\textbf{效果}
将\texttt{std::thread::id}的值通过给指定流插入字符串。

\textbf{返回}
无

\textbf{NOTE} 字符串的格式并未给定。\texttt{std::thread::id}实例具有相同的表达式时,是相同的;当实例表达式不同,则代表不同的线程。

% ###std::thread::native\_handler 成员函数

\texttt{native\_handle\_type}是由另一类型定义而来,这个类型会随着指定平台的API而变化。

\textbf{声明}

\begin{cpp}
typedef implementation-defined native_handle_type;
\end{cpp}

\textbf{NOTE} 这个类型定义是可选的。如果提供,实现将使用原生平台指定的API,并提供合适的类型作为实现。

% ###std::thread 默认构造函数

返回一个\texttt{native\_handle\_type}类型的值,这个值可以可以表示*this相关的执行线程。

\textbf{声明}

\begin{cpp}
native_handle_type native_handle();
\end{cpp}

\textbf{NOTE} 这个函数是可选的。如果提供,会使用原生平台指定的API,并返回合适的值。

% ###std::thread 构造函数

构造一个无相关线程的\texttt{std::thread}对象。

\textbf{声明}

\begin{cpp}
thread() noexcept;
\end{cpp}

\textbf{效果}
构造一个无相关线程的\texttt{std::thread}实例。

\textbf{后置条件}
对于一个新构造的\texttt{std::thread}对象x,x.get\_id() == id()。

\textbf{抛出}
无

% ###std::thread 移动构造函数

将已存在\texttt{std::thread}对象的所有权,转移到新创建的对象中。

\textbf{声明}

\begin{cpp}
thread(thread&& other) noexcept;
\end{cpp}

\textbf{效果}
构造一个\texttt{std::thread}实例。与other相关的执行线程的所有权,将转移到新创建的\texttt{std::thread}对象上。否则,新创建的\texttt{std::thread}对象将无任何相关执行线程。

\textbf{后置条件}
对于一个新构建的\texttt{std::thread}对象x来说,x.get\_id()等价于未转移所有权时的other.get\_id()。get\_id()==id()。

\textbf{抛出}
无

\textbf{NOTE} \texttt{std::thread}对象是不可CopyConstructible(拷贝构造),所以该类没有拷贝构造函数,只有移动构造函数。

% ###std::thread 析构函数

销毁\texttt{std::thread}对象。

\textbf{声明}

\begin{cpp}
~thread();
\end{cpp}

\textbf{效果}
销毁\texttt{*this}。当\texttt{*this}与执行线程相关(this->joinable()将返回true),调用\texttt{std::terminate()}来终止程序。

\textbf{抛出}
无

% ###std::thread 移动赋值操作

将一个\texttt{std::thread}的所有权,转移到另一个\texttt{std::thread}对象上。

\textbf{声明}

\begin{cpp}
thread& operator=(thread&& other) noexcept;
\end{cpp}

\textbf{效果}
在调用该函数前,this->joinable返回true,则调用\texttt{std::terminate()}来终止程序。当other在执行赋值前,具有相关的执行线程,那么执行线程现在就与\texttt{*this}相关联。否则,\texttt{*this}无相关执行线程。

\textbf{后置条件}
this->get\_id()的值等于调用该函数前的other.get\_id()。oter.get\_id()==id()。

\textbf{抛出}
无

\textbf{NOTE} \texttt{std::thread}对象是不可CopyAssignable(拷贝赋值),所以该类没有拷贝赋值函数,只有移动赋值函数。

% ###std::thread::swap 成员函数

将两个\texttt{std::thread}对象的所有权进行交换。

\textbf{声明}

\begin{cpp}
void swap(thread& other) noexcept;
\end{cpp}

\textbf{效果}
当other在执行赋值前,具有相关的执行线程,那么执行线程现在就与\texttt{*this}相关联。否则,\texttt{*this}无相关执行线程。对于\texttt{*this}也是一样。

\textbf{后置条件}
this->get\_id()的值等于调用该函数前的other.get\_id()。other.get\_id()的值等于没有调用函数前this->get\_id()的值。

\textbf{抛出}
无

% ###std::thread的非成员函数swap

将两个\texttt{std::thread}对象的所有权进行交换。

\textbf{声明}

\begin{cpp}
void swap(thread& lhs,thread& rhs) noexcept;
\end{cpp}

\textbf{效果}
lhs.swap(rhs)

\textbf{抛出}
无

% ###std::thread::joinable 成员函数

查询*this是否具有相关执行线程。

\textbf{声明}

\begin{cpp}
bool joinable() const noexcept;
\end{cpp}

\textbf{返回}
如果*this具有相关执行线程,则返回true;否则,返回false。

\textbf{抛出}
无

% ###std::thread::join 成员函数

等待*this相关的执行线程结束。

\textbf{声明}

\begin{cpp}
void join();
\end{cpp}

\textbf{先决条件}
this->joinable()返回true。

\textbf{效果}
阻塞当前线程,直到与*this相关的执行线程执行结束。

\textbf{后置条件}
this->get\_id()==id()。与*this先关的执行线程将在该函数调用后结束。

\textbf{同步}
想要在*this上成功的调用该函数,则需要依赖有joinable()的返回。

\textbf{抛出}
当效果没有达到或this->joinable()返回false,则抛出\texttt{std::system\_error}异常。

% ###std::thread::detach 成员函数

将*this上的相关线程进行分离。

\textbf{声明}

\begin{cpp}
void detach();
\end{cpp}

\textbf{先决条件}
this->joinable()返回true。

\textbf{效果}
将*this上的相关线程进行分离。

\textbf{后置条件}
this->get\_id()==id(), this->joinable()==false

与*this相关的执行线程在调用该函数后就会分离,并且不在会与当前\texttt{std::thread}对象再相关。

\textbf{抛出}
当效果没有达到或this->joinable()返回false,则抛出\texttt{std::system\_error}异常。

% ###std::thread::get\_id 成员函数

返回\texttt{std::thread::id}的值来表示*this上相关执行线程。

\textbf{声明}

\begin{cpp}
thread::id get_id() const noexcept;
\end{cpp}

\textbf{返回}
当*this具有相关执行线程,将返回\texttt{std::thread::id}作为识别当前函数的依据。否则,返回默认构造的\texttt{std::thread::id}。

\textbf{抛出}
无

% ###std::thread::hardware\_concurrency 静态成员函数

返回硬件上可以并发线程的数量。

\textbf{声明}

\begin{cpp}
unsigned hardware_concurrency() noexcept;
\end{cpp}

\textbf{返回}
硬件上可以并发线程的数量。这个值可能是系统处理器的数量。当信息不用或只有定义,则该函数返回0。

\textbf{抛出}
无

\mySubsubsection{D.7.2}{this\_thread命名空间}

这里介绍一下\texttt{std::this\_thread}命名空间内提供的函数操作。

% ###this\_thread::get\_id 非成员函数

返回\texttt{std::thread::id}用来识别当前执行线程。

\textbf{声明}

\begin{cpp}
thread::id get_id() noexcept;
\end{cpp}

\textbf{返回}
可通过\texttt{std:thread::id}来识别当前线程。

\textbf{抛出}
无

% ###this\_thread::yield 非成员函数

该函数用于通知库,调用线程不需要立即运行。一般使用小循环来避免消耗过多CPU时间。

\textbf{声明}

\begin{cpp}
void yield() noexcept;
\end{cpp}

\textbf{效果}
使用标准库的实现来安排线程的一些事情。

\textbf{抛出}
无

% ###this\_thread::sleep\_for 非成员函数

在指定的指定时长内,暂停执行当前线程。

\textbf{声明}

\begin{cpp}
template<typename Rep,typename Period>
void sleep_for(std::chrono::duration<Rep,Period> const& relative_time);
\end{cpp}

\textbf{效果}
在超出relative\_time的时长内,阻塞当前线程。

\textbf{NOTE} 线程可能阻塞的时间要长于指定时长。如果可能,逝去的时间由将会由一个稳定时钟决定。

\textbf{抛出}
无

% ###this\_thread::sleep\_until 非成员函数

暂停指定当前线程,直到到了指定的时间点。

\textbf{声明}

\begin{cpp}
template<typename Clock,typename Duration>
void sleep_until(
    std::chrono::time_point<Clock,Duration> const& absolute_time);
\end{cpp}

\textbf{效果}
在到达absolute\_time的时间点前,阻塞当前线程,这个时间点由指定的Clock决定。

\textbf{NOTE} 这里不保证会阻塞多长时间,只有Clock::now()返回的时间等于或大于absolute\_time时,阻塞的线程才能被解除阻塞。

\textbf{抛出}
无