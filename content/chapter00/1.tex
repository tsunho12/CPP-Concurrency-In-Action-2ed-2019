前4章,介绍了标准库中的各种工具,展示使用方法。

第5章,涵盖了内存模型和原子操作,包括原子操作如何对执行顺序进行限制(这章标志着介绍部分的结束)。

第6、7章,开始讨论高级主题,如何使用基本工具去构建复杂的数据结构——第6章是基于锁的数据结构,第7章是无锁数据结构。

第8章,针对设计多线程代码给了一些指导意见,覆盖了性能问题和并行算法。

第9章,线程管理——线程池,工作队列和中断操作。

第10章,介绍C++17中标准库算法对并行性的支持。

第11章,测试和调试——Bug类型,定位Bug的技巧,以及如何进行测试等等。

附录,包括新标准中语言特性的简要描述,主要是与多线程相关的特性,以及在第4章中提到的消息传递库的实现细节和C++17线程库的完整的参考。