主要内容

\begin{itemize}
    \item 定义并发和多线程
    \item 使用并发和多线程
    \item C++的并发史
    \item 简单的C++多线程
\end{itemize}

距初始C++标准(1998年)发布后的13年后,标准委员对C++进行了重大的修改。新标准(也称C++11或C++0x)在2011年发布,一系列的修改让C++编程更加简单和高效。同时,委员会也确立了标准更新模式——每三年发布一个新标准。从模式确立至今,委员会已经发布了两个标准:2014年的C++14标准和2017的C++17标准,以及若干个C++技术规范标准扩展。

其中最重要的特性就是对多线程的支持。C++标准第一次包含了多线程,并在标准库中提供了多线程组件,这让使用C++编写与平台无关的多线程程序成为可能,也为可移植性提供了强有力的保证。与此同时,开发者们为提高应用的性能,对并发的关注也是与日俱增,特别在多线程方面。在C++11的基础上,C++14、C++17标准,以及一些技术规范标准,都在为C++的多线程和并发添砖加瓦。

本书会使用C++11多线程来编写并发程序,并介绍相关的语言特性和工具。本章以“为什么要使用并发”作为起始点,会对“什么情况下不使用并发”进行阐述,并且对C++的并发方式进行总结。最后,以一个简单的并发实例结束这一章。后面的章节中,会有更多的例子,以便大家对线程库进行更加深入的了解。