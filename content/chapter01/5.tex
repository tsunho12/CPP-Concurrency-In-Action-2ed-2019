% # 1.5 本章总结

本章中,提及了并发与多线程的含义,以及在应用中为什么使用(或不使用)并发。还提及了多线程在C++中的发展历程,从1998标准中完全缺乏支持,经历了各种平台相关的扩展,再到C++11/C++14/C++17标准和并发技术规范对多线程的支持。芯片制造商选择了以多核芯的形式,使得更多任务可以同时执行的方式来增加处理能力,而不是增加单个核心的执行速度。在这个趋势下,C++多线程来的正是时候,它使得开发者们可以利用CPU带来的更加强大的硬件并发。

1.4节中例子,展示C++标准库中的类和函数有多么的简单。C++中使用多线程并不复杂,复杂的是如何设计代码以实现预期的行为。

尝试了1.4节的示例后,可以了解更多实质性的内容。

第2章中,我们将了解用于管理线程的类和函数。
