% # 2.1 线程的基本操作

每个程序至少有一个执行main()函数的线程,其他线程与主线程同时运行。如main()函数执行完会退出一样,线程执行完函数也会退出。为线程创建\texttt{std::thread}对象后,需要等待这个线程结束。那么,就先来启动线程。

\mySubsubsection{2.1.1}{启动线程}

第1章中,线程在\texttt{std::thread}对象创建时启动,通常使用的是无参数无返回的函数。这种函数在执行完毕,线程也就结束了。一些情况下,任务函数对象需要通过某种通讯机制进行参数的传递,或者执行一系列独立操作,通过通讯机制传递信号让线程停止。先放下这些特殊情况不谈,简单来说,使用C++线程库启动线程,就是构造\texttt{std::thread}对象:

\begin{cpp}
void do_some_work();
std::thread my_thread(do_some_work);
\end{cpp}

这里需要包含\texttt{<thread>}头文件,\texttt{std::thread}可以通过有函数操作符类型的实例进行构造:

\begin{cpp}
class background_task
{
public:
  void operator()() const
  {
    do_something();
    do_something_else();
  }
};

background_task f;
std::thread my_thread(f);
\end{cpp}

代码中,提供的函数对象会复制到新线程的存储空间中,函数对象的执行和调用都在线程的内存空间中进行。

有件事需要注意,当把函数对象传入到线程构造函数中时,需要避免\href{http://en.wikipedia.org/wiki/Most_vexing_parse}{"最令人头痛的语法解析"} (C++’s most vexing parse, \href{http://qiezhuifeng.diandian.com/post/2012-08-27/40038339477}{中文简介})。如果你传递了一个临时变量,而不是一个命名的变量。C++编译器会将其解析为函数声明,而不是类型对象的定义。

\begin{cpp}
std::thread my_thread(background_task());
\end{cpp}

这相当于声明了一个名为my\_thread的函数,这个函数带有一个参数(函数指针指向没有参数并返回background\_task对象的函数),返回一个\texttt{std::thread}对象的函数。

使用在前面命名函数对象的方式,或使用多组括号\symbol{"2460},或使用统一的初始化语法\symbol{"2461},都可以避免这个问题。

如下所示:

\begin{cpp}
std::thread my_thread((background_task()));  // 1
std::thread my_thread{background_task()};    // 2
\end{cpp}

Lambda表达式也能避免这个问题。Lambda表达式是C++11的一个新特性,允许使用一个可以捕获局部变量的局部函数(可以避免传递参数,参见2.2节)。想要详细了解Lambda表达式,可以阅读附录A的A.5节。之前的例子可以改写为Lambda表达式的方式:

\begin{cpp}
std::thread my_thread([]{
  do_something();
  do_something_else();
});
\end{cpp}

线程启动后是要等待线程结束,还是让其自主运行。当\texttt{std::thread}对象销毁之前还没有做出决定,程序就会终止(\texttt{std::thread}的析构函数会调用\texttt{std::terminate()})。因此,即便是有异常存在,也需要确保线程能够正确\textit{汇入}(joined)或\textit{分离}(detached)。

如果不等待线程汇入 ,就必须保证线程结束之前,访问数据的有效性。这不是一个新问题——单线程代码中,对象销毁之后再去访问,会产生未定义行为——不过,线程的生命周期增加了这个问题发生的几率。

这种情况很可能发生在线程还没结束,函数已经退出的时候,这时线程函数还持有函数局部变量的指针或引用。

代码2.1  函数已经返回,线程依旧访问局部变量

\begin{cpp}
struct func
{
  int& i;
  func(int& i_) : i(i_) {}
  void operator() ()
  {
    for (unsigned j=0 ; j<1000000 ; ++j)
    {
      do_something(i);           // 1 潜在访问隐患:空引用
    }
  }
};

void oops()
{
  int some_local_state=0;
  func my_func(some_local_state);
  std::thread my_thread(my_func);
  my_thread.detach();          // 2 不等待线程结束
}                              // 3 新线程可能还在运行
\end{cpp}

代码中,已经决定不等待线程(使用了detach()\symbol{"2461}),所以当oops()函数执行完成时\symbol{"2462},线程中的函数可能还在运行。如果线程还在运行,就会去调用do\_something(i)\symbol{"2460},这时就会访问已经销毁的变量。如同一个单线程程序——允许在函数完成后继续持有局部变量的指针或引用。当然,这种情况发生时,错误并不明显,会使多线程更容易出错。运行顺序参考表2.1。

表2.1 分离线程在局部变量销毁后,仍对该变量进行访问

\begin{table}[htbp]
  \begin{tabular}{|l|l|}
  \hline
  主线程                            & 新线程  \\ \hline
  使用some\_local\_state构造my\_func &      \\ \hline
  开启新线程my\_thread                &      \\ \hline
                                 & 启动   \\ \hline
  将my\_thread分离 & \begin{tabular}[c]{@{}l@{}}将执行func::operator();可能会在do\_something中\\ 调用some\_local\_state的引用\end{tabular}         \\ \hline
  销毁some\_local\_state           & 持续运行 \\ \hline
  退出oops函数      & \begin{tabular}[c]{@{}l@{}}持续执行func::operator();可能会在do\_something中\\ 调用some\_local\_state的引用导致未定义行为\end{tabular} \\ \hline
  \end{tabular}
\end{table}

这种情况的常规处理方法:将数据复制到线程中。如果使用一个可调用的对象作为线程函数,这个对象就会复制到线程中,而后原始对象会立即销毁。如代码2.1所示,但对于对象中包含的指针和引用还需谨慎。使用访问局部变量的函数去创建线程是一个糟糕的主意。

此外,可以通过join()函数来确保线程在主函数完成前结束。

\mySubsubsection{2.1.2}{等待线程完成}

如需等待线程,需要使用join()。将代码2.1中的\texttt{my\_thread.detach()}替换为\texttt{my\_thread.join()},就可以确保局部变量在线程完成后才销毁。因为主线程并没有做什么事,使用独立的线程去执行函数变得意义不大。但在实际中,原始线程要么有自己的工作要做,要么会启动多个子线程来做一些有用的工作,并等待这些线程结束。

当你需要对等待中的线程有更灵活的控制时,比如:看一下某个线程是否结束,或者只等待一段时间(超过时间就判定为超时)。想要做到这些,需要使用其他机制来完成,比如条件变量和future。调用join(),还可以清理了线程相关的内存,这样\texttt{std::thread}对象将不再与已经完成的线程有任何关联。这意味着,只能对一个线程使用一次join(),一旦使用过join(),\texttt{std::thread}对象就不能再次汇入了。当对其使用joinable()时,将返回false。

\mySubsubsection{2.1.3}{特殊情况下的等待}

如前所述,需要对一个未销毁的\texttt{std::thread}对象使用join()或detach()。如果想要分离线程,可以在线程启动后,直接使用detach()进行分离。如果等待线程,则需要细心挑选使用join()的位置。当在线程运行后产生的异常,会在join()调用之前抛出,这样就会跳过join()。

避免应用被抛出的异常所终止。通常,在无异常的情况下使用join()时,需要在异常处理过程中调用join(),从而避免生命周期的问题。

代码2.2 等待线程完成

\begin{cpp}
struct func; // 定义在代码2.1中
void f()
{
  int some_local_state=0;
  func my_func(some_local_state);
  std::thread t(my_func);
  try
  {
    do_something_in_current_thread();
  }
  catch(...)
  {
    t.join();  // 1
    throw;
  }
  t.join();  // 2
}
\end{cpp}

代码2.2中使用了\texttt{try/catch}块确保线程退出后函数才结束。当函数正常退出后,会执行到\symbol{"2461}处。当执行过程中抛出异常,程序会执行到\symbol{"2460}处。如果线程在函数之前结束——就要查看是否因为线程函数使用了局部变量的引用——而后再确定一下程序可能会退出的途径,无论正常与否,有一个简单的机制,可以解决这个问题。

一种方式是使用“资源获取即初始化方式”(RAII,Resource Acquisition Is Initialization),提供一个类,在析构函数中使用join()。如同下面代码。

代码2.3 使用RAII等待线程完成

\begin{cpp}
class thread_guard
{
  std::thread& t;
public:
  explicit thread_guard(std::thread& t_):
    t(t_)
  {}
  ~thread_guard()
  {
    if(t.joinable()) // 1
    {
      t.join();      // 2
    }
  }
  thread_guard(thread_guard const&)=delete;   // 3
  thread_guard& operator=(thread_guard const&)=delete;
};

struct func; // 定义在代码2.1中

void f()
{
  int some_local_state=0;
  func my_func(some_local_state);
  std::thread t(my_func);
  thread_guard g(t);
  do_something_in_current_thread();
}    // 4
\end{cpp}

线程执行到\symbol{"2463}处时,局部对象就要被逆序销毁了。因此,thread\_guard对象g是第一个被销毁的,这时线程在析构函数中被加入\symbol{"2461}到原始线程中。即使do\_something\_in\_current\_thread抛出一个异常,这个销毁依旧会发生。

在thread\_guard析构函数的测试中,首先判断线程是否可汇入\symbol{"2460}。如果可汇入,会调用join()\symbol{"2461}进行汇入。

拷贝构造函数和拷贝赋值操作标记为\texttt{=delete}\symbol{"2462},是为了不让编译器自动生成。直接对对象进行拷贝或赋值是很危险的,因为这可能会弄丢已汇入的线程。通过删除声明,任何尝试给thread\_guard对象赋值的操作都会引发一个编译错误。想要了解删除函数的更多知识,请参阅附录A的A.2节。

如果不想等待线程结束,可以分离线程,从而避免异常。不过,这就打破了线程与\texttt{std::thread}对象的联系,即使线程仍然在后台运行着,分离操作也能确保在\texttt{std::thread}对象销毁时不调用\texttt{std::terminate()}。

\mySubsubsection{2.1.4}{后台运行线程}

使用detach()会让线程在后台运行,这就意味着与主线程不能直接交互。如果线程分离,就不可能有\texttt{std::thread}对象能引用它,分离线程的确在后台运行,所以分离的线程不能汇入。不过C++运行库保证,当线程退出时,相关资源的能够正确回收。

分离线程通常称为\textit{守护线程}(daemon threads)。UNIX中守护线程,是指没有任何显式的接口,并在后台运行的线程,这种线程的特点就是长时间运行。线程的生命周期可能会从应用的起始到结束,可能会在后台监视文件系统,还有可能对缓存进行清理,亦或对数据结构进行优化。另外,分离线程只能确定线程什么时候结束,\textit{发后即忘}(fire and forget)的任务使用到就是分离线程。

如2.1.2节所示,调用\texttt{std::thread}成员函数detach()来分离一个线程。之后,相应的\texttt{std::thread}对象就与实际执行的线程无关了,并且这个线程也无法汇入:

\begin{cpp}
std::thread t(do_background_work);
t.detach();
assert(!t.joinable());
\end{cpp}

为了从\texttt{std::thread}对象中分离线程,不能对没有执行线程的\texttt{std::thread}对象使用detach(),并且要用同样的方式进行检查——当\texttt{std::thread}对象使用t.joinable()返回的是true,就可以使用t.detach()。

试想如何能让一个文字处理应用同时编辑多个文档。无论是用户界面,还是在内部应用内部进行,都有很多的解决方法。虽然,这些窗口看起来是完全独立的,每个窗口都有自己独立的菜单选项,但他们却运行在同一个应用实例中。一种内部处理方式是,让每个文档处理窗口拥有自己的线程。每个线程运行同样的的代码,并隔离不同窗口处理的数据。如此这般,打开一个文档就要启动一个新线程。因为是对独立文档进行操作,所以没有必要等待其他线程完成,这里就可以让文档处理窗口运行在分离线程上。

代码2.4 使用分离线程处理文档

\begin{cpp}
void edit_document(std::string const& filename)
{
  open_document_and_display_gui(filename);
  while(!done_editing())
  {
    user_command cmd=get_user_input();
    if(cmd.type==open_new_document)
    {
      std::string const new_name=get_filename_from_user();
      std::thread t(edit_document,new_name);  // 1
      t.detach();  // 2
    }
    else
    {
       process_user_input(cmd);
    }
  }
}
\end{cpp}


如果用户选择打开一个新文档,需要启动一个新线程去打开新文档\symbol{"2461},并分离线程\symbol{"2461}。与当前线程做出的操作一样,新线程只不过是打开另一个文件而已。所以,edit\_document函数可以复用, 并通过传参的形式打开新的文件。

这个例子也展示了传参启动线程的方法:不仅可以向\texttt{std::thread}构造函数\symbol{"2460}传递函数名,还可以传递函数所需的参数(实参)。当然,也有其他方法可以完成这项功能,比如:使用带有数据的成员函数,代替需要传参的普通函数。

