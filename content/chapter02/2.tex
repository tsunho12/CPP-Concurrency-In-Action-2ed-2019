% # 2.2 传递参数

如代码2.4所示,向可调用对象或函数传递参数很简单,只需要将这些参数作为 \texttt{std::thread}构造函数的附加参数即可。需要注意的是,这些参数会拷贝至新线程的内存空间中(同临时变量一样)。即使函数中的参数是引用的形式,拷贝操作也会执行。来看一个例子:

\begin{cpp}
void f(int i, std::string const& s);
std::thread t(f, 3, "hello");
\end{cpp}

代码创建了一个调用f(3, "hello")的线程。注意,函数f需要一个\texttt{std::string}对象作为第二个参数,但这里使用的是字符串的字面值,也就是\texttt{char const *}类型,线程的上下文完成字面值向\texttt{std::string}的转化。需要特别注意,指向动态变量的指针作为参数的情况,代码如下:

\begin{cpp}
void f(int i,std::string const& s);
void oops(int some_param)
{
  char buffer[1024]; // 1
  sprintf(buffer, "\%i",some_param);
  std::thread t(f,3,buffer); // 2
  t.detach();
}
\end{cpp}

buffer\symbol{"2460}是一个指针变量,指向局部变量,然后此局部变量通过buffer传递到新线程中\symbol{"2461}。此时,函数\texttt{oops}可能会在buffer转换成\texttt{std::string}之前结束,从而导致未定义的行为。因为,无法保证隐式转换的操作和\texttt{std::thread}构造函数的拷贝操作的顺序,有可能\texttt{std::thread}的构造函数拷贝的是转换前的变量(buffer指针)。解决方案就是在传递到\texttt{std::thread}构造函数之前,就将字面值转化为\texttt{std::string}:

\begin{cpp}
void f(int i,std::string const& s);
void not_oops(int some_param)
{
  char buffer[1024];
  sprintf(buffer,"\%i",some_param);
  std::thread t(f,3,std::string(buffer));  // 使用std::string,避免悬空指针
  t.detach();
}
\end{cpp}

相反的情形(期望传递一个非常量引用,但复制了整个对象)倒是不会出现,因为会出现编译错误。比如,尝试使用线程更新引用传递的数据结构:

\begin{cpp}
void update_data_for_widget(widget_id w,widget_data& data); // 1
void oops_again(widget_id w)
{
  widget_data data;
  std::thread t(update_data_for_widget,w,data); // 2
  display_status();
  t.join();
  process_widget_data(data);
}
\end{cpp}

虽然update\_data\_for\_widget\symbol{"2460}的第二个参数期待传入一个引用,但\texttt{std::thread}的构造函数\symbol{"2461}并不知晓,构造函数无视函数参数类型,盲目地拷贝已提供的变量。不过,内部代码会将拷贝的参数以右值的方式进行传递,这是为了那些只支持移动的类型,而后会尝试以右值为实参调用update\_data\_for\_widget。但因为函数期望的是一个非常量引用作为参数(而非右值),所以会在编译时出错。对于熟悉\texttt{std::bind}的开发者来说,问题的解决办法很简单:可以使用\texttt{std::ref}将参数转换成引用的形式。因此可将线程的调用改为以下形式:

\begin{cpp}
std::thread t(update_data_for_widget,w,std::ref(data));
\end{cpp}

这样update\_data\_for\_widget就会收到data的引用,而非data的拷贝副本,这样代码就能顺利的通过编译了。

如果熟悉\texttt{std::bind},就应该不会对以上述传参的语法感到陌生,因为\texttt{std::thread}构造函数和\texttt{std::bind}的操作在标准库中以相同的机制进行定义。比如,你也可以传递一个成员函数指针作为线程函数,并提供一个合适的对象指针作为第一个参数:

\begin{cpp}
class X
{
public:
  void do_lengthy_work();
};
X my_x;
std::thread t(&X::do_lengthy_work, &my_x); // 1
\end{cpp}

这段代码中,新线程将会调用my\_x.do\_lengthy\_work(),其中my\_x的地址\symbol{"2460}作为对象指针提供给函数。也可以为成员函数提供参数:\texttt{std::thread}构造函数的第三个参数就是成员函数的第一个参数,以此类推(代码如下,译者自加)。

\begin{cpp}
class X
{
public:
  void do_lengthy_work(int);
};
X my_x;
int num(0);
std::thread t(&X::do_lengthy_work, &my_x, num);
\end{cpp}

另一种有趣的情形是,提供的参数仅支持\textit{移动}(move),不能\textit{拷贝}。“移动”是指原始对象中的数据所有权转移给另一对象,从而这些数据就不再在原始对象中保存(译者:比较像在文本编辑的剪切操作)。\texttt{std::unique\_ptr}就是这样一种类型(译者:C++11中的智能指针),这种类型为动态分配的对象提供内存自动管理机制(译者:类似垃圾回收机制)。同一时间内,只允许一个\texttt{std::unique\_ptr}实例指向一个对象,并且当这个实例销毁时,指向的对象也将被删除。\textit{移动构造函数}(move constructor)和\textit{移动赋值操作符}(move assignment operator)允许一个对象的所有权在多个\texttt{std::unique\_ptr}实例中传递(有关“移动”的更多内容,请参考附录A的A.1.1节)。使用“移动”转移对象所有权后,就会留下一个空指针。使用移动操作可以将对象转换成函数可接受的实参类型,或满足函数返回值类型要求。当原对象是临时变量时,则自动进行移动操作,但当原对象是一个命名变量,转移的时候就需要使用\texttt{std::move()}进行显示移动。下面的代码展示了\texttt{std::move}的用法,展示了\texttt{std::move}是如何转移动态对象的所有权到线程中去的:

\begin{cpp}
void process_big_object(std::unique_ptr<big_object>);

std::unique_ptr<big_object> p(new big_object);
p->prepare_data(42);
std::thread t(process_big_object,std::move(p));
\end{cpp}

通过在\texttt{std::thread}构造函数中执行\texttt{std::move(p)},big\_object 对象的所有权首先被转移到新创建线程的的内部存储中,之后再传递给process\_big\_object函数。

C++标准线程库中和\texttt{std::unique\_ptr}在所属权上相似的类有好几种,\texttt{std::thread}为其中之一。虽然,\texttt{std::thread}不像\texttt{std::unique\_ptr}能占有动态对象的所有权,但是它能占有其他资源:每个实例都负责管理一个线程。线程的所有权可以在多个\texttt{std::thread}实例中转移,这依赖于\texttt{std::thread}实例的\textit{可移动}且\textit{不可复制}性。不可复制性表示在某一时间点,一个\texttt{std::thread}实例只能关联一个执行线程。可移动性使得开发者可以自己决定,哪个实例拥有线程实际执行的所有权。