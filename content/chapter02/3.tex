% # 2.3 转移所有权

假设通过新线程返回的所有权去调用一个需要后台启动线程的函数,并需要在函数中转移线程的所有权。这些操作都要等待线程结束才能进行,并且需要线程的所有权能够进行转移。

这就是将移动操作引入\texttt{std::thread}的原因,C++标准库中有很多\textit{资源占有}(resource-owning)类型,比如\texttt{std::ifstream},\texttt{std::unique\_ptr}还有\texttt{std::thread}都是可移动,但不可复制。这说明执行线程的所有权可以在\texttt{std::thread}实例中移动,下面将展示一个例子。例子中,创建了两个执行线程,并在\texttt{std::thread}实例之间(t1,t2和t3)转移所有权:

\begin{cpp}
void some_function();
void some_other_function();
std::thread t1(some_function);            // 1
std::thread t2=std::move(t1);            // 2
t1=std::thread(some_other_function);    // 3
std::thread t3;                            // 4
t3=std::move(t2);                        // 5
t1=std::move(t3);                        // 6 赋值操作将使程序崩溃
\end{cpp}

首先,新线程与t1相关联\symbol{"2460}。当显式使用\texttt{std::move()}创建t2后\symbol{"2461},t1的所有权就转移给了t2。之后,t1和执行线程已经没有关联了,执行some\_function的函数线程与t2关联。

然后,临时\texttt{std::thread}对象相关的线程启动了\symbol{"2462}。为什么不显式调用\texttt{std::move()}转移所有权呢?因为,所有者是一个临时对象——移动操作将会隐式的调用。

t3使用默认构造方式创建\symbol{"2463},没有与任何线程进行关联。调用\texttt{std::move()}将t2关联线程的所有权转移到t3中\symbol{"2464}。因为t2是一个命名对象,需要显式的调用\texttt{std::move()}。移动操作\symbol{"2465}完成后,t1与执行some\_other\_function的线程相关联,t2与任何线程都无关联,t3与执行some\_function的线程相关联。

最后一个移动操作,将some\_function线程的所有权转移\symbol{"2465}给t1。不过,t1已经有了一个关联的线程(执行some\_other\_function的线程),所以这里系统直接调用\texttt{std::terminate()}终止程序继续运行。这样做(不抛出异常,\texttt{std::terminate()}是\href{https://en.cppreference.com/w/cpp/language/noexcept}{noexcept}函数)是为了保证与\texttt{std::thread}的析构函数的行为一致。2.1.1节中,需要在线程对象析构前,显式的等待线程完成,或者分离它,进行赋值时也需要满足这些条件(说明:不能通过赋新值给\texttt{std::thread}对象的方式来"丢弃"一个线程)。

\texttt{std::thread}支持移动,线程的所有权可以在函数外进行转移,就如下面程序一样。

代码2.5 函数返回\texttt{std::thread}对象

\begin{cpp}
std::thread f()
{
  void some_function();
  return std::thread(some_function);
}

std::thread g()
{
  void some_other_function(int);
  std::thread t(some_other_function,42);
  return t;
}
\end{cpp}

当所有权可以在函数内部传递,就允许\texttt{std::thread}实例作为参数进行传递,代码如下:

\begin{cpp}
void f(std::thread t);
void g()
{
  void some_function();
  f(std::thread(some_function));
  std::thread t(some_function);
  f(std::move(t));
}
\end{cpp}

\texttt{std::thread}支持移动可以创建thread\_guard类的实例(定义见代码2.3),并且拥有线程所有权。当引用thread\_guard对象所持有的线程时,移动操作就可以避免很多不必要的麻烦。当某个对象转移了线程的所有权,就不能对线程进行汇入或分离。为了确保线程在程序退出前完成,定义了scoped\_thread类。现在,我们来看一下这个类型:

代码2.6 scoped\_thread的用法

\begin{cpp}
class scoped_thread
{
  std::thread t;
public:
  explicit scoped_thread(std::thread t_): // 1
    t(std::move(t_))
  {
    if(!t.joinable())  // 2
      throw std::logic_error("No thread");
  }
  ~scoped_thread()
  {
    t.join(); // 3
  }
  scoped_thread(scoped_thread const&)=delete;
  scoped_thread& operator=(scoped_thread const&)=delete;
};

struct func; // 定义在代码2.1中

void f()
{
  int some_local_state;
  scoped_thread t(std::thread(func(some_local_state)));    // 4
  do_something_in_current_thread();
} // 5
\end{cpp}

与代码2.3相似,不过新线程会直接传递到scoped\_thread中\symbol{"2463},而非创建一个独立变量。当主线程到达f()末尾时\symbol{"2464},scoped\_thread对象就会销毁,然后在析构函数中完成汇入\symbol{"2462}。代码2.3中的thread\_guard类,需要在析构中检查线程是否“可汇入”。这里把检查放在了构造函数中\symbol{"2461},并且当线程不可汇入时抛出异常。

C++17标准给出一个建议,就是添加一个joining\_thread的类型,这个类型与\texttt{std::thread}类似,不同是的添加了析构函数,就类似于scoped\_thread。委员会成员们对此并没有达成统一共识,所以这个类没有添加入C++17标准中(C++20仍旧对这种方式进行探讨,不过名称为\texttt{std::jthread}),这个类实现起来也不是很困难。

代码2.7 joining\_thread类的实现

\begin{cpp}
class joining_thread
{
  std::thread t;
public:
  joining_thread() noexcept=default;
  template<typename Callable,typename ... Args>
  explicit joining_thread(Callable&& func,Args&& ... args):
    t(std::forward<Callable>(func),std::forward<Args>(args)...)
  {}
  explicit joining_thread(std::thread t_) noexcept:
    t(std::move(t_))
  {}
  joining_thread(joining_thread&& other) noexcept:
    t(std::move(other.t))
  {}
  joining_thread& operator=(joining_thread&& other) noexcept
  {
    if(joinable()){
      join();
    }
    t = std::move(other.t);
    return *this;
  }
  joining_thread& operator=(std::thread other) noexcept
  {
    if(joinable())
      join();
    t=std::move(other);
    return *this;
  }
  ~joining_thread() noexcept
  {
    if(joinable())
    join();
  }
  void swap(joining_thread& other) noexcept
  {
    t.swap(other.t);
  }
  std::thread::id get_id() const noexcept{
    return t.get_id();
  }
  bool joinable() const noexcept
  {
    return t.joinable();
  }
  void join()
  {
    t.join();
  }
  void detach()
  {
    t.detach();
  }
  std::thread& as_thread() noexcept
  {
    return t;
  }
  const std::thread& as_thread() const noexcept
  {
    return t;
  }
};
\end{cpp}

\texttt{std::thread}中对移动语义的支持,也适用于使用\texttt{std::thread}的\textbf{移动敏感}(move-aware)容器(比如,\texttt{std::vector<>})。了解这些后,就可以量产了一些线程,并且等待它们结束,代码如下所示。

代码2.8 量产线程,等待它们结束

\begin{cpp}
void do_work(unsigned id);

void f()
{
  std::vector<std::thread> threads;
  for (unsigned i = 0; i < 20; ++i)
  {
    threads.emplace_back(do_work,i); // 产生线程
  }
  for (auto& entry : threads) // 对每个线程调用 join()
    entry.join();
}
\end{cpp}

我们有时需要线程去分割一个算法的工作总量,所以在算法结束的之前,所有的线程必须结束。代码2.8中线程所做的工作都是独立的,并且结果仅会受到共享数据的影响。如果f()有返回值,这个返回值就依赖于线程得到的结果。写入返回值之前,程序会检查使用共享数据的线程是否终止。结果在不同线程中转移的方案,会在第4章中再次讨论。

将\texttt{std::thread}放入\texttt{std::vector}是向线程自动化管理迈出的第一步:并非为这些线程创建独立的变量,而是把它们当做一个组。创建一组线程(数量在运行时确定),而非代码2.8那样创建固定数量的线程。

