% # 2.5 线程标识

线程标识为\texttt{std::thread::id}类型,可以通过两种方式进行检索。第一种,可以通过调用\texttt{std::thread}对象的成员函数\texttt{get\_id()}来直接获取。如果\texttt{std::thread}对象没有与任何执行线程相关联,\texttt{get\_id()}将返回\texttt{std::thread::type}默认构造值,这个值表示“无线程”。第二种,当前线程中调用\texttt{std::this\_thread::get\_id()}(这个函数定义在\texttt{<thread>}头文件中)也可以获得线程标识。

\texttt{std::thread::id}对象可以自由的拷贝和对比,因为标识符可以复用。如果两个对象的\texttt{std::thread::id}相等,那就是同一个线程,或者都“无线程”。如果不等,那么就代表了两个不同线程,或者一个有线程,另一没有线程。

C++线程库不会限制你去检查线程标识是否一样,\texttt{std::thread::id}类型对象提供了相当丰富的对比操作。比如,为不同的值进行排序。这意味着开发者可以将其当做为容器的键值做排序,或做其他比较。按默认顺序比较不同的\texttt{std::thread::id}:当\texttt{a<b},\texttt{b<c}时,得\texttt{a<c},等等。标准库也提供\texttt{std::hash<std::thread::id>}容器,\texttt{std::thread::id}也可以作为无序容器的键值。

\texttt{std::thread::id}实例常用作检测线程是否需要进行一些操作。比如:当用线程来分割一项工作(如代码2.9),主线程可能要做一些与其他线程不同的工作,启动其他线程前,可以通过\texttt{std::this\_thread::get\_id()}得到自己的线程ID。每个线程都要检查一下,其拥有的线程ID是否与初始线程的ID相同。

\begin{cpp}
std::thread::id master_thread;
void some_core_part_of_algorithm()
{
  if(std::this_thread::get_id()==master_thread)
  {
    do_master_thread_work();
  }
  do_common_work();
}
\end{cpp}

另外,当前线程的\texttt{std::thread::id}将存储到数据结构中。之后这个结构体对当前线程的ID与存储的线程ID做对比,来决定操作是“允许”,还是“需要”(permitted/required)。

同样,作为线程和本地存储不适配的替代方案,线程ID在容器中可作为键值。例如,容器可以存储其掌控下每个线程的信息,或在多个线程中互传信息。

\texttt{std::thread::id}可以作为线程的通用标识符,当标识符只与语义相关(比如,数组的索引)时,就需要这个方案了。也可以使用输出流(\texttt{std::cout})来记录一个\texttt{std::thread::id}对象的值。

\begin{cpp}
std::cout<<std::this_thread::get_id();
\end{cpp}

具体的输出结果是严格依赖于具体实现的,C++标准的要求就是保证ID相同的线程必须有相同的输出。