% # 3.4 本章总结

本章讨论了当线程间的共享数据发生恶性条件竞争时,将会带来多么严重的灾难。还讨论了如何使用\texttt{std::mutex}和如何避免这些问题。虽然C++标准库提供了一些工具来避免这些问题,但互斥量并不是灵丹妙药,也还有自己的问题(比如:死锁)。还见识了一些用于避免死锁的技术,之后了解了锁的所有权转移,以及围绕如何选取适当粒度锁产生的问题。最后,在具体情况下讨论了其他数据保护的方案,例如:\texttt{std::call\_once()}和\texttt{std::shared\_mutex}。

还有一个方面没有涉及,那就是等待其他线程作为输入的情况。我们的线程安全栈,仅是在栈为空时,抛出一个异常,所以当一个线程要等待其他线程向栈压入一个值时(这是线程安全栈的主要用途之一),它需要多次尝试去弹出一个值,当捕获抛出的异常时,再次进行尝试。这种消耗资源的检查,没有任何意义。并且,不断的检查会影响系统中其他线程的运行,这反而会妨碍程序的运行。我们需要一些方法让一个线程等待其他线程完成任务,但在等待过程中不占用CPU。第4章中,会去建立一些保护共享数据的工具,还会介绍一些线程同步的操作机制。第6章中会展示,如何构建更大型的可复用的数据类型。