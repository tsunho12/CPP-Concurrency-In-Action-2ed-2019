% # 第4章 同步操作

本章主要内容

\begin{itemize}
    \item 带有future的等待
    \item 在限定时间内等待
    \item 使用同步操作简化代码
\end{itemize}


上一章中,我们了解了线程间保护共享数据的方法。当然,我们不仅想要保护数据,还想对单独的线程进行同步。例如,在第一个线程完成前,等待另一个线程执行完成。通常,线程会等待特定事件发生,或者等待某一条件达成。这可能需要定期检查“任务完成”标识,或将类似的东西放到共享数据中。像这种情况就需要在线程中进行同步,C++标准库提供了一些工具可用于同步,形式上表现为\textit{条件变量}(condition variables)和future。并发技术规范中,为future添加了非常多的操作,并可与新工具\textit{锁存器}(latches)(轻量级锁资源)和\textit{栅栏}(barriers)一起使用。

本章将讨论如何使用条件变量等待事件,介绍future,锁存器和栅栏,以及如何简化同步操作。