% # 4.1 等待事件或条件

假设你正在一辆在夜间运行的火车上,在夜间如何在正确的站点下车呢?有一种方法是整晚都要醒着,每停一站都能知道,这样就不会错过你要到达的站点,但会很疲倦。另外,可以看一下时间表,估计一下火车到达目的地的时间,然后在一个稍早的时间点上设置闹铃,然后安心的睡会。这个方法听起来也很不错,也没有错过你要下车的站点,但是当火车晚点时,就要被过早的叫醒了。当然,闹钟的电池也可能会没电了,并导致你睡过站。理想的方式是,无论是早或晚,只要当火车到站的时候,有人或其他东西能把你叫醒就好了。

这和线程有什么关系呢?当一个线程等待另一个线程完成时,可以持续的检查共享数据标志(用于做保护工作的互斥量),直到另一线程完成工作时对这个标识进行重置。不过,这种方式会消耗线程的执行时间检查标识,并且当互斥量上锁后,其他线程就没有办法获取锁,就会持续等待。因为对等待线程资源的限制,并且在任务完成时阻碍对标识的设置。类似于保持清醒状态和列车驾驶员聊了一晚上:驾驶员不得不缓慢驾驶,因为你分散了他的注意力,所以火车需要更长的时间,才能到站。同样,等待的线程会等待更长的时间,也会消耗更多的系统资源。

另外,在等待线程在检查间隙,使用\texttt{std::this\_thread::sleep\_for()}进行周期性的间歇(详见4.3节):

\begin{cpp}
bool flag;
std::mutex m;

void wait_for_flag()
{
  std::unique_lock<std::mutex> lk(m);
  while(!flag)
  {
    lk.unlock();  // 1 解锁互斥量
    std::this_thread::sleep_for(std::chrono::milliseconds(100));  // 2 休眠100ms
    lk.lock();   // 3 再锁互斥量
  }
}
\end{cpp}

循环中,休眠前\symbol{"2461}函数对互斥量进行解锁\symbol{"2460},并且在休眠结束后再对互斥量上锁,所以另外的线程就有机会获取锁并设置标识。

这个实现就进步很多,当线程休眠时没有浪费执行时间,但很难确定正确的休眠时间。太短的休眠和没有一样,都会浪费执行时间。太长的休眠时间,可能会让任务等待时间过久。休眠时间过长比较少见,这会影响到程序的行为,在高节奏的游戏中,就意味着丢帧或错过了一个时间片。

第三个选择(也是优先选择的),使用C++标准库提供的工具去等待事件的发生。通过另一线程触发等待事件的机制是最基本的唤醒方式(例如:流水线上存在额外的任务时),这种机制就称为“条件变量”。从概念上来说,条件变量会与多个事件或其他条件相关,并且一个或多个线程会等待条件的达成。当某些线程被终止时,为了唤醒等待线程(允许等待线程继续执行),终止线程将会向等待着的线程广播“条件达成”的信息。

\mySubsubsection{4.1.1}{等待条件达成}

C++标准库对条件变量有两套实现:\texttt{std::condition\_variable}和\texttt{std::condition\_variable\_any},这两个实现都包含在\texttt{<condition\_variable>}头文件的声明中。两者都需要与互斥量一起才能工作(互斥量是为了同步),前者仅能与\texttt{std::mutex}一起工作,而后者可以和合适的互斥量一起工作,从而加上了\texttt{\_any}的后缀。因为\texttt{std::condition\_variable\_any}更加通用,不过在性能和系统资源的使用方面会有更多的开销,所以通常会将\texttt{std::condition\_variable}作为首选类型。当对灵活性有要求时,才会考虑\texttt{std::condition\_variable\_any}。

所以,使用\texttt{std::condition\_variable}去处理之前提到的情况——当有数据需要处理时,如何唤醒休眠中的线程?以下代码展示了使用条件变量唤醒线程的方式。

代码4.1 使用\texttt{std::condition\_variable}处理数据等待

\begin{cpp}
std::mutex mut;
std::queue<data_chunk> data_queue;  // 1
std::condition_variable data_cond;

void data_preparation_thread()
{
  while(more_data_to_prepare())
  {
    data_chunk const data=prepare_data();
    std::lock_guard<std::mutex> lk(mut);
    data_queue.push(data);  // 2
    data_cond.notify_one();  // 3
  }
}

void data_processing_thread()
{
  while(true)
  {
    std::unique_lock<std::mutex> lk(mut);  // 4
    data_cond.wait(
         lk,[]{return !data_queue.empty();});  // 5
    data_chunk data=data_queue.front();
    data_queue.pop();
    lk.unlock();  // 6
    process(data);
    if(is_last_chunk(data))
      break;
  }
}
\end{cpp}

首先,队列中中有两个线程,两个线程之间会对数据进行传递\symbol{"2460}。数据准备好时,使用\texttt{std::lock\_guard}锁定队列,将准备好的数据压入队列\symbol{"2461}之后,线程会对队列中的数据上锁,并调用\texttt{std::condition\_variable}的notify\_one()成员函数,对等待的线程(如果有等待线程)进行通知\symbol{"2462}。

另外的一个线程正在处理数据,线程首先对互斥量上锁(这里使用\texttt{std::unique\_lock}要比\texttt{std::lock\_guard}\symbol{"2463}更加合适)。之后会调用\texttt{std::condition\_variable}的成员函数wait(),传递一个锁和一个Lambda表达式(作为等待的条件\symbol{"2464})。Lambda函数是C++11添加的新特性,可以让一个匿名函数作为其他表达式的一部分,并且非常合适作为标准函数的谓词。例子中,简单的Lambda函数\texttt{[]{return !data\_queue.empty();}}会去检查data\_queue是否为空,当data\_queue不为空,就说明数据已经准备好了。附录A的A.5节有Lambda函数更多的信息。

wait()会去检查这些条件(通过Lambda函数),当条件满足(Lambda函数返回true)时返回。如果条件不满足(Lambda函数返回false),wait()将解锁互斥量,并且将线程(处理数据的线程)置于阻塞或等待状态。当准备数据的线程调用notify\_one()通知条件变量时,处理数据的线程从睡眠中苏醒,重新获取互斥锁,并且再次进行条件检查。在条件满足的情况下,从wait()返回并继续持有锁。当条件不满足时,线程将对互斥量解锁,并重新等待。这就是为什么用\texttt{std::unique\_lock}而不使用\texttt{std::lock\_guard}的原因——等待中的线程必须在等待期间解锁互斥量,并对互斥量再次上锁,而\texttt{std::lock\_guard}没有这么灵活。如果互斥量在线程休眠期间保持锁住状态,准备数据的线程将无法锁住互斥量,也无法添加数据到队列中。同样,等待线程也永远不会知道条件何时满足。

代码4.1使用了简单的Lambda函数用于等待\symbol{"2464}(用于检查队列何时不为空),不过任意的函数和可调用对象都可以传入wait()。当写好函数做为检查条件时,不一定非要放在一个Lambda表达式中,也可以直接将这个函数传入wait()。调用wait()的过程中,在互斥量锁定时,可能会去检查条件变量若干次,当提供测试条件的函数返回true就会立即返回。当等待线程重新获取互斥量并检查条件变量时,并非直接响应另一个线程的通知,就是所谓的\textit{伪唤醒}(spurious wakeup)。因为任何伪唤醒的数量和频率都是不确定的,所以不建议使用有副作用的函数做条件检查。

本质上,\texttt{std::condition\_variable::wait}是“忙碌-等待”的优化。下面用简单的循环实现了一个“忙碌-等待”:

\begin{cpp}
template<typename Predicate>
void minimal_wait(std::unique_lock<std::mutex>& lk, Predicate pred){
  while(!pred()){
    lk.unlock();
    lk.lock();
  }
}
\end{cpp}

为wait()准备一个最小化实现,只需要notify\_one()或notify\_all()。

\texttt{std::unique\_lock}的灵活性,不仅适用于对wait()的调用,还可以用于待处理的数据⑥。处理数据可能是耗时的操作,并且长时间持有锁是个糟糕的主意。

使用队列在多个线程中转移数据(如代码4.1)很常见。做得好的话,同步操作可以在队列内部完成,这样同步问题和条件竞争出现的概率也会降低。鉴于这些好处,需要从代码4.1中提取出一个通用线程安全的队列。

\mySubsubsection{4.1.2}{构建线程安全队列}

设计通用队列时,就要花时间想想,哪些操作需要添加到队列实现中去,就如之前在3.2.3节看到的线程安全的栈。可以看一下C++标准库提供的实现,找找灵感。\texttt{std::queue<>}容器的接口展示如下:

代码4.2 \texttt{std::queue}接口

\begin{cpp}
template <class T, class Container = std::deque<T> >
class queue {
public:
  explicit queue(const Container&);
  explicit queue(Container&& = Container());
  template <class Alloc> explicit queue(const Alloc&);
  template <class Alloc> queue(const Container&, const Alloc&);
  template <class Alloc> queue(Container&&, const Alloc&);
  template <class Alloc> queue(queue&&, const Alloc&);

  void swap(queue& q);

  bool empty() const;
  size_type size() const;

  T& front();
  const T& front() const;
  T& back();
  const T& back() const;

  void push(const T& x);
  void push(T&& x);
  void pop();
  template <class... Args> void emplace(Args&&... args);
};
\end{cpp}

忽略构造、赋值以及交换操作,剩下了三组操作:

1. 对整个队列的状态进行查询(empty()和size())
2. 查询在队列中的各个元素(front()和back())
3. 修改队列的操作(push(), pop()和emplace())

和3.2.3中的栈一样,也会遇到接口上的条件竞争。因此,需要将front()和pop()合并成一个函数调用,就像之前在栈实现时合并top()和pop()一样。与代码4.1不同的是,当队列在多个线程中传递数据时,接收线程通常需要等待数据的压入。这里提供pop()函数的两个变种:try\_pop()和wait\_and\_pop()。

try\_pop() ,尝试从队列中弹出数据,即使没有值可检索,也会直接返回。

wait\_and\_pop(),将会等待有值可检索的时候才返回。

当使用之前栈的方式来实现队列,接口可能会是下面这样:

代码4.3 线程安全队列的接口

\begin{cpp}
#include <memory> // 为了使用std::shared_ptr

template<typename T>
class threadsafe_queue
{
public:
  threadsafe_queue();
  threadsafe_queue(const threadsafe_queue&);
  threadsafe_queue& operator=(
      const threadsafe_queue&) = delete;  // 不允许简单的赋值

  void push(T new_value);

  bool try_pop(T& value);  // 1
  std::shared_ptr<T> try_pop();  // 2

  void wait_and_pop(T& value);
  std::shared_ptr<T> wait_and_pop();

  bool empty() const;
};
\end{cpp}

就像之前的栈,裁剪了很多构造函数,并禁止简单赋值。需要提供两个版本的try\_pop()和wait\_for\_pop()。第一个重载的try\_pop()\symbol{"2460}在引用变量中存储着检索值,可以用来返回队列中值的状态。当检索到一个变量时,将返回true,否则返回false(详见A.2节)。第二个重载\symbol{"2461}就不行了,因为它是用来直接返回检索值的,当没有值可检索时,这个函数返回NULL。

那么问题来了,如何将以上这些和代码4.1相关联呢?从之前的代码中提取push()和wait\_for\_pop(),如以下代码所示。

代码4.4 从代码4.1中提取push()和wait\_for\_pop()

\begin{cpp}
#include <queue>
#include <mutex>
#include <condition_variable>

template<typename T>
class threadsafe_queue
{
private:
  std::mutex mut;
  std::queue<T> data_queue;
  std::condition_variable data_cond;
public:
  void push(T new_value)
  {
    std::lock_guard<std::mutex> lk(mut);
    data_queue.push(new_value);
    data_cond.notify_one();
  }

  void wait_and_pop(T& value)
  {
    std::unique_lock<std::mutex> lk(mut);
    data_cond.wait(lk,[this]{return !data_queue.empty();});
    value=data_queue.front();
    data_queue.pop();
  }
};
threadsafe_queue<data_chunk> data_queue;  // 1

void data_preparation_thread()
{
  while(more_data_to_prepare())
  {
    data_chunk const data=prepare_data();
    data_queue.push(data);  // 2
  }
}

void data_processing_thread()
{
  while(true)
  {
    data_chunk data;
    data_queue.wait_and_pop(data);  // 3
    process(data);
    if(is_last_chunk(data))
      break;
  }
}
\end{cpp}

线程队列中有互斥量和条件变量,所以独立的变量就不需要了\symbol{"2460},并且push()不需要外部同步\symbol{"2461}。当然,wait\_for\_pop()还要兼顾条件变量的等待\symbol{"2462}。

另一个wait\_for\_pop()的重载写起来就很琐碎,剩下的函数就像从代码3.5实现的栈中粘过来一样。

代码4.5 使用条件变量的线程安全队列(完整版)

\begin{cpp}
#include <queue>
#include <memory>
#include <mutex>
#include <condition_variable>

template<typename T>
class threadsafe_queue
{
private:
  mutable std::mutex mut;  // 1 互斥量必须是可变的
  std::queue<T> data_queue;
  std::condition_variable data_cond;
public:
  threadsafe_queue()
  {}
  threadsafe_queue(threadsafe_queue const& other)
  {
    std::lock_guard<std::mutex> lk(other.mut);
    data_queue=other.data_queue;
  }

  void push(T new_value)
  {
    std::lock_guard<std::mutex> lk(mut);
    data_queue.push(new_value);
    data_cond.notify_one();
  }

  void wait_and_pop(T& value)
  {
    std::unique_lock<std::mutex> lk(mut);
    data_cond.wait(lk,[this]{return !data_queue.empty();});
    value=data_queue.front();
    data_queue.pop();
  }

  std::shared_ptr<T> wait_and_pop()
  {
    std::unique_lock<std::mutex> lk(mut);
    data_cond.wait(lk,[this]{return !data_queue.empty();});
    std::shared_ptr<T> res(std::make_shared<T>(data_queue.front()));
    data_queue.pop();
    return res;
  }

  bool try_pop(T& value)
  {
    std::lock_guard<std::mutex> lk(mut);
    if(data_queue.empty())
      return false;
    value=data_queue.front();
    data_queue.pop();
    return true;
  }

  std::shared_ptr<T> try_pop()
  {
    std::lock_guard<std::mutex> lk(mut);
    if(data_queue.empty())
      return std::shared_ptr<T>();
    std::shared_ptr<T> res(std::make_shared<T>(data_queue.front()));
    data_queue.pop();
    return res;
  }

  bool empty() const
  {
    std::lock_guard<std::mutex> lk(mut);
    return data_queue.empty();
  }
};
\end{cpp}

empty()是一个const成员函数,并且传入拷贝构造函数的other形参是一个const引用。因为其他线程可能有非const引用对象,并调用变种成员函数,所以这里有必要对互斥量上锁。又因为锁住互斥量是个可变操作,所以互斥量成员必须为mutable\symbol{"2460}才能在empty()和拷贝构造函数中进行上锁。

条件变量在多个线程等待同一个事件时也很有用。当线程用来分解工作负载,并且只有一个线程可以对通知做出反应时,与代码4.1中结构完全相同。当数据准备完成时,调用notify\_one()将会唤醒一个正在wait()的线程,检查条件和wait()函数的返回状态(因为仅是向data\_queue添加了一个数据项)。 这里不保证线程一定会被通知到,即使只有一个等待线程收到通知,其他处理线程也有可能因为在处理数据,而忽略了这个通知。

另一种可能是,很多线程等待同一事件。对于通知,都需要做出回应。这会发生在共享数据初始化的时候,当处理线程使用同一数据时,就要等待数据被初始化,或等待共享数据的更新,比如:\textit{周期性初始化}(periodic reinitialization)。这些情况下,线程准备好数据时,就会通过条件变量调用notify\_all(),而非调用notify\_one()。顾名思义,这就是全部线程在都去执行wait()(检查他们等待的条件是否满足)的原因。

当条件为true时,等待线程只等待一次,就不会再等待条件变量了,所以尤其是在等待一组可用的数据块时,一个条件变量并非同步操作最好的选择。

接下来就来了解一下future,对于条件变量的补足。