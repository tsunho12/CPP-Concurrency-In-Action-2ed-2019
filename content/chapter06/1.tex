% # 6.1 并发设计的意义

设计并发数据结构是为了让多线程并发访问,并且线程可对数据结构做相同或不同的操作。多线程环境下,无数据丢失和损毁,所有的数据需要维持原样,且无条件竞争的数据结构,称之为“线程安全”的数据结构。通常情况下,多个线程对数据结构进行并发操作是安全的,但不同操作需要单线程独立访问数据结构。当线程执行不同的操作时,对同一数据结构的并发操作是安全的,而多线程执行同样的操作时,可能会出现问题。

实际的设计意义并不止上面提到的那样,而是要为线程提供并发访问数据结构的机会。本质上,在互斥量的保护下同一时间内只有一个线程可以获取锁。互斥量为了保护数据,会显式阻止线程对数据结构的并发访问。

\textit{串行化}(serialzation)则是线程轮流访问数据,对数据进行串行访问。因此,需要对数据结构仔细斟酌,确保能进行真正的并发。虽然,有些数据结构比其他结构的并发访问范围更大,但思路都是一样的:减少保护区域,减少序列化操作,提升并发访问的能力。

进行数据结构的设计之前,快速浏览一下并发设计的指导指南。

\mySubsubsection{6.1.1}{并发数据结构设计的指南}

设计并发数据结构时,需要考量两方面:一是确保访问安全,二是真正并发访问。第3章已经对如何保证数据结构是线程安全的做过简单的描述:

\begin{itemize}
    \item 确保无线程能够看到“不变量”变化时的状态。
    \item 小心会引起条件竞争的接口,提供完整操作的函数,而非操作步骤。
    \item 注意数据结构的行为是否会产生异常,从而确保“不变量”的状态。
    \item 将死锁的概率降到最低。限制锁的范围,避免嵌套锁的存在。
\end{itemize}

还需要考虑数据结构对于使用者有什么限制,当线程通过特殊的函数对数据结构进行访问时,其他的线程还有哪些函数能安全调用?

这是一个很重要的问题,普通的构造函数和析构函数需要独立访问数据结构,所以用户使用时,就不能在构造函数完成前或析构函数完成后对数据结构进行访问。当数据结构支持赋值操作swap()或拷贝构造时,作为数据结构的设计者,即使线程操纵数据结构中有大量的函数,也需要保证这些操作在并发下是安全的(或确保这些操作能够独立访问),以保证并发访问时不会出错。

第二个方面是确保真正的并发访问,这里没有更多的指导意见。不过,作为一个数据结构的设计者,需要考虑以下问题:

\begin{itemize}
    \item 操作在锁的范围中进行,是否允许在锁外执行?
    \item 数据结构中不同的互斥量能否保护不同的区域?
    \item 所有操作都需要同级互斥量的保护吗?
    \item 能否对数据结构进行简单的修改,增加并发访问的概率?
\end{itemize}

这些问题都源于一个指导思想:如何让序列化访问最小化,让真实并发最大化?允许线程并发读取的数据结构并不少见,但修改必须是单线程的,这种结构类似于\texttt{std::shared\_mutex}。同样,这种数据结构也很常见——支持多线程的不同操作时,也能串行执行相同的操作。

最简单的线程安全结构通常会对数据使用互斥量或锁。虽然,这么做还有问题,不过这样做相对简单,并且能保证只有一个线程在同一时间对数据结构进行独立访问。为了更轻松的设计线程安全的数据结构,接下来了解一下基于锁的数据结构。

