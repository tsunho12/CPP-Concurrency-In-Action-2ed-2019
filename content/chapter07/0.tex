% # 第7章 设计无锁的并发数据结构

本章主要内容

\begin{itemize}
    \item 设计无锁的并发数据结构
    \item 无锁结构中的内存管理
    \item 设计无锁结构的简单指导
\end{itemize}

上一章了解了在设计并发数据结构时会遇到的问题,根据指导意见确定设计的安全性。对一些通用数据结构进行检查,并了解使用互斥锁对共享数据进行保护的例子。第一组例子就是使用单个互斥量来保护整个数据结构,但之后的例子使用多个锁来保护数据结构不同的部分,并且允许进行更高级别的并发访问。

互斥量是一个强大的工具,可以保证在多线程情况下安全的访问数据结构,且不会有条件竞争或破坏不变量的情况存在。使用互斥量的原因很简单:就是让互斥量来保护数据。不过,这并不会如所想的那样,可以回看一下第3章,回顾一下死锁形成的原因,再回顾一下基于锁的队列和查询表的例子,看一下细粒度锁是如何影响并发的。如果能写出一个无锁安全的并发数据结构,那就能避免这些问题。

本章中还会使用原子操作(第5章介绍)的内存序,来构建无锁数据结构。这样的数据结构不是那么容易正确实现的,所以要格外小心,并且让其问题会很难复现。我们将从无锁数据结构的定义开始,通过几个例子来了解使用无锁数据结构的意义,最后给出通用的指导意见。