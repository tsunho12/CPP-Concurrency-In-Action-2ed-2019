% # 7.1 定义和意义

使用互斥量,条件变量,以及future可以用来同步算法和数据结构。调用库函数将会挂起执行线程,直到其他线程完成某个特定动作。库函数将调用阻塞操作来对线程进行阻塞,在阻塞解除前线程无法继续自己的任务。通常,操作系统会完全挂起一个阻塞线程(并将其时间片交给其他线程),直到解阻塞。“解阻塞”的方式很多,比如互斥锁解锁,通知条件变量达成,或让“future状态”就绪。

不使用阻塞库的数据结构和算法称为“无阻塞结构”。不过,无阻塞的数据结构并非都是无锁的,那么就来见识一下各种各样的无阻塞的数据结构吧!

\mySubsubsection{7.1.1}{非阻塞数据结构}

第5章中,使用\texttt{std::atomic\_flag}实现了一个简单的自旋锁。

代码7.1 使用\texttt{std::atomic\_flag}实现了一个简单的自旋锁

\begin{cpp}
class spinlock_mutex
{
  std::atomic_flag flag;
public:
  spinlock_mutex():
    flag(ATOMIC_FLAG_INIT)
  {}
  void lock()
  {
    while(flag.test_and_set(std::memory_order_acquire));
  }
  void unlock()
  {
    flag.clear(std::memory_order_release);
  }
};
\end{cpp}

这段代码没有调用任何阻塞函数,lock()只是让循环持续调用test\_and\_set(),并返回false。这就是为什么取名为“自旋锁”的原因——代码“自旋”于循环当中。所以没有阻塞调用,任意代码使用互斥量来保护共享数据都是非阻塞的。不过,自旋锁并不是无锁结构。这里用了一个锁,并且一次能锁住一个线程。让我们来看一下无锁结构的具体定义,这将有助于你判断哪些类型的数据结构是无锁的。这些类型有:

\begin{itemize}
    \item 无阻碍——如果其他线程都暂停了,任何给定的线程都将在一定时间内完成操作。
    \item 无锁——如果多个线程对一个数据结构进行操作,经过一定时间后,其中一个线程将完成其操作。
    \item 无等待——即使有其他线程也在对该数据结构进行操作,每个线程都将在一定的时间内完成操作。
\end{itemize}

大多数情况下无阻塞算法用的很少——其他线程都暂停的情况太少见了,因此这种方式用于描述一个失败的无锁实现更为合适。从无锁结构开始,来了解这些特性到都涉及到了哪些数据结构。

\mySubsubsection{7.1.2}{无锁数据结构}

无锁结构意味着线程可以并发的访问数据结构,线程不能做相同的操作。一个无锁队列可能允许一个线程压入数据,另一个线程弹出数据,当有两个线程同时添加元素时,将破坏这个数据结构。不仅如此,当调度器中途挂起其中一个访问线程时,其他线程必须能够继续完成自己的工作,而无需等待挂起线程。

具有“比较/交换”操作的数据结构,通常有一个循环。使用“比较/交换”操作的原因:当有其他线程同时对指定的数据进行修改时,代码将尝试恢复数据。当其他线程挂起时,“比较/交换”操作执行成功,这样的代码就是无锁的。当执行失败时,就需要一个自旋锁,且这个结构就是“无阻塞-有锁”的结构。

无锁算法中的循环会让一些线程处于“饥饿”状态。如有线程在“错误”时间执行,那么第一个线程将会不停的尝试所要完成的操作(其他程序继续执行)。“无锁-无等待”数据结构的出现,就为了避免这种问题。

\mySubsubsection{7.1.3}{无等待数据结构}

无等待数据结构:首先是无锁数据结构,并且每个线程都能在有限的时间内完成操作,暂且不管其他线程是如何工作的。由于可能会和其他线程的行为冲突,从而算法会进行了若干次尝试,因此无法做到无等待。本章的大多数例子都有一种特性——对compare\_exchange\_weak或compare\_exchange\_strong操作进行循环,并且循环次数没有上限。操作系统对线程进行进行管理,有些线程的循环次数非常多,有些线程的循环次数就非常少。因此,这些操作不是无等待的。

正确实现一个无等待结构十分困难,要保证每个线程都能在有限的步骤内完成操作,必须确保每次执行的操作都是一次性的,并且当前线程中的操作不会影响其他线程的操作,这就会让所使用到的操作变的相当复杂。

考虑到实现无锁或无等待的数据结构非常困难,索性就实现一个数据结构吧。不过,需要保证收益要大于成本,那么先来找一下成本和收益的平衡点吧!

\mySubsubsection{7.1.4}{无锁数据结构的利与弊}

使用无锁结构的主要原因:最大化并发。使用基于锁的容器,会让线程阻塞或等待,并且互斥锁削弱了结构的并发性。无锁数据结构中,某些线程可以逐步执行。无等待数据结构中,每一个线程都可以独自向前运行,这种理想的方式实现起来很难。结构太简单,反而不容易实现。

使用无锁数据结构的第二个原因就是鲁棒性。当一个线程在持有锁时被终止,那么数据结构将会永久性的破坏。不过,当线程在无锁数据结构上执行操作,在执行到一半终止时,数据结构上的数据没有丢失(除了线程本身的数据),其他线程依旧可以正常执行。

另一方面,当不能限制访问数据结构的线程数量时,就需要注意不变量的状态,或选择替代品来保持不变量的状态。同时,还需要注意操作的顺序。为了避免未定义行为,及相关的数据竞争,必须使用原子操作对修改操作进行限制。不过,仅使用原子操作是不够的,需要确定其他线程看到的修改,是否遵循正确的顺序。

因为,没有任何锁(有可能存在活锁),死锁问题不会困扰无锁数据结构。活锁的产生是两个线程同时尝试修改数据结构,但每个线程所做的修改操作都会让另一个线程重启,所以两个线程就会陷入循环,多次的尝试完成自己的操作。试想有两个人要过独木桥,当两个人从两头向中间走的时候,他们会在中间碰到,然后需要再走回出发的地方,再次尝试过独木桥。要打破僵局,除非有人先到独木桥的另一端(或是商量好了,或是走的快,或纯粹是运气),要不这个循环将一直重复下去。不过活锁的存在时间并不久,因为其依赖于线程调度。所以只是对性能有所消耗,而不是一个长期问题,但这个问题仍需要关注。根据定义,因其操作执行步骤有上限,无等待的代码不会受活锁所困扰。换句话说,无等待的算法要比等待算法的复杂度高,即使没有其他线程访问数据结构,也可能需要更多步骤来完成相应操作。

“无锁-无等待”代码的缺点:虽然提高了并发访问的能力,减少了单个线程的等待时间,但是其可能会将整体性能拉低。首先,原子操作的无锁代码要慢于无原子操作的代码,原子操作就相当于无锁数据结构中的锁。不仅如此,硬件必须通过同一个原子变量对线程间的数据进行同步。第8章将看到与“乒乓缓存”相关的原子变量(多个线程访问同时进行访问),将会形成一个明显的性能瓶颈。提交代码之前,无论是基于锁的数据结构,还是无锁的数据结构,对性能的检查都很重要(最坏的等待时间,平均等待时间,整体执行时间或者其他指标)。
