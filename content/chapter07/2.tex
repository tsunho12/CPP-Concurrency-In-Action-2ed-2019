% # 7.2 无锁数据结构的例子

为了展示在设计无锁数据结构中使用到的技术,我们将了解一些无锁实现的简单数据结构。不仅要在每个例子中描述一个数据结构的实现,还将使用这些例子的某些特别之处来阐述对于无锁数据结构的设计。

无锁结构依赖于原子操作和内存序,以确保多线程以正确的顺序访问数据结构,原子操作默认使用memory\_order\_seq\_cst内存序。不过,后面的例子中会降低内存序的要求。虽然例子中没有直接使用锁,但需要注意\texttt{std::atomic\_flag}。一些平台上无锁结构的实现(实际上在C++标准库中实现)使用了内部锁。另一些平台上,基于锁的简单数据结构可能会更加合适,还有很多平台的实现细节不明确。选择一种实现前,需要明确需求,并且配置各种选项以满足需求。

回到数据结构上,从最简单的数据结构开始——栈。

\mySubsubsection{7.2.1}{实现一个无锁的线程安全栈}

栈的要求很简单:查询顺序是添加顺序的逆序——先入后出(LIFO)。所以,确保值能安全的入栈就十分重要,因为可能在入栈后,会马上被其他线程索引,同时确保只有一个线程能索引到指定数据。最简单的栈就是链表,head指针指向第一个节点(可能是下一个被索引到的节点),并且每个节点依次指向下一个节点。

这样的情况下,添加一个节点很简单:

1. 创建新节点node。
2. 让node->next指向head->next。
3. head->next指向node。

单线程中这种方式没有问题,不过使用多线程对栈进行修改时,这几步就不够用了。有两个线程同时添加节点时,第2步和第3步会产生条件竞争:一个线程可能在修改head值时,另一个线程正在执行第2步,并且在第3步中对head进行更新。就会使之前那个线程的结果丢弃,亦或是造成更加糟糕的后果。了解了如何解决这个条件竞争前,还要注意当head更新并指向了新节点时,另一个线程就能读取到这个节点,因此head设置在指向新节点前。因为在这之后就不能对节点进行修改了,所以新节点准备就绪就很重要。

那如何应对条件竞争呢?答案就是:第3步的时候使用原子“比较/交换”操作,来保证步骤2对head进行读取时,不会对head进行修改,有修改时可以循环“比较/交换”操作。下面的代码中,就不用锁来实现线程安全的push()函数。

代码7.2 不用锁实现push()

\begin{cpp}
template<typename T>
class lock_free_stack
{
private:
  struct node
  {
    T data;
    node* next;

    node(T const& data_):  // 1
     data(data_)
    {}
  };

  std::atomic<node*> head;
public:
  void push(T const& data)
  {
    node* const new_node=new node(data); // 2
    new_node->next=head.load();  // 3
    while(!head.compare_exchange_weak(new_node->next,new_node));  // 4
  }
};
\end{cpp}

上面代码能匹配之前的三个步骤:创建一个新节点②,设置新节点的next指针指向当前head③,并设置head指针指向新节点④。node结构用其自身的构造函数来进行数据填充①,必须保证节点在构造完成后能随时弹出。之后需要使用compare\_exchange\_weak()来保证在被存储到new\_node->next的head指针和之前的一样③。代码的亮点是使用“比较/交换”操作:返回false时,因为比较失败(例如,head被其他线程锁修改),会使用head中的内容更新new\_node->next(第一个参数)的内容。因为编译器会重新加载head指针,所以循环中不需要每次都重新加载head指针。同样,因为循环可能直接就失败了,所以使用compare\_exchange\_weak要好于使用compare\_exchange\_strong(详见第5章)。

所以,暂时不需要pop()操作,可以先快速检查一下push()的实现是否有违指导意见。这里能抛出异常的点就在构造新node时①,其会自行处理且链表中的内容,所以是安全的。构建数据时将其作为node的一部分进行存储,并且使用compare\_exchange\_weak()来更新head指针,所以没有恶性条件竞争。“比较/交换”成功时,节点已经准备就绪且随时可以提取。因为没有锁,所以不存在死锁,这里的push()函数实现的很成功。

已经有往栈中添加数据的方法了,现在需要删除数据的方法。也很简单,其步骤如下:

1. 获取head。
2. 读取head->next指向的结点node。
3. 设置head->next指向node->next。
4. 通过node返回携带的数据data。
5. 删除node节点。

多线程环境下,就不那么简单了。有两个线程要从栈中移除数据时,两个线程可能在步骤1中读取到同一个head(值相同)。其中一个线程处理到步骤5时,另一个线程还在处理步骤2,这个还在处理步骤2的线程将会解引用一个悬空指针。这是书写无锁代码的问题之一,所以现在只能跳过步骤5,让节点泄露。

另一个问题:两个线程读取到同一个head值时,将它们返回给同一个节点。这就违反了栈结构的意图,所以需要避免这样的情况发生。可以像在push()函数中解决条件竞争那样来解决这个问题:使用“比较/交换”操作更新head。当“比较/交换”操作失败时,不是一个新节点已被推入,就是其他线程已经弹出了节点。无论是哪种情况,都得返回步骤1(“比较/交换”操作将会重新读取head)。

当“比较/交换”成功,就可以确定当前线程是弹出指定节点的唯一线程,之后就可以放心的执行步骤4了。这里先看一下pop()的雏形:

\begin{cpp}
template<typename T>
class lock_free_stack
{
public:
  void pop(T& result)
  {
    node* old_head=head.load();
    while(!head.compare_exchange_weak(old_head,old_head->next));
    result=old_head->data;
  }
};
\end{cpp}

这段代码很优雅,但有两个节点泄露的问题。首先,这段代码在空链表时不工作:当head指针是空指针时,要访问next指针时,将引起未定义行为。很容易通过对nullptr的检查进行修复(在while循环中),要不对空栈抛出一个异常,要不返回一个bool值来表明成功与否。

第二个问题就是异常安全。第3章中介绍栈结构时,了解了在返回值时会出现异常安全问题:有异常抛出时,复制的值将丢失。这种情况下,传入引用是一种可以接受的解决方案,这样就能保证当有异常抛出时,栈上的数据不会丢失。不幸的是, 当线程返回节点时才能安全地进行拷贝,这时这个节点已经从队列中删除了。因此,通过引用获取返回值的方式不可取。若想要安全的返回,必须使用第3章中的其他方法:返回指向数据值的(智能)指针。

返回的是智能指针时,返回nullptr表明没有值可返回,但是要求在堆上对智能指针进行内存分配。将分配过程做为pop()的一部分时(也没有更好的选择了),堆分配内存时可能会抛出一个异常。与此相反,在push()操作中对内存进行分配——无论怎样,都需要对node进行内存分配。返回一个\texttt{std::shared\_ptr<>}不会抛出异常,所以在pop()中进行内存分配是安全的。

代码7.3 带有节点泄露的无锁栈

\begin{cpp}
template<typename T>
class lock_free_stack
{
private:
  struct node
  {
    std::shared_ptr<T> data;  // 1 指针获取数据
    node* next;

    node(T const& data_):
      data(std::make_shared<T>(data_))  // 2 让std::shared_ptr指向新分配出来的T
    {}
  };

  std::atomic<node*> head;
public:
  void push(T const& data)
  {
    node* const new_node=new node(data);
    new_node->next=head.load();
    while(!head.compare_exchange_weak(new_node->next,new_node));
  }
  std::shared_ptr<T> pop()
  {
    node* old_head=head.load();
    while(old_head && // 3 在解引用前检查old_head是否为空指针
      !head.compare_exchange_weak(old_head,old_head->next));
    return old_head ? old_head->data : std::shared_ptr<T>();  // 4
  }
};
\end{cpp}

智能指针指向当前数据①,必须在堆上为数据分配内存(node结构体中)②。而后,compare\_exchage\_weak()循环中③,需要在old\_head指针前,检查指针是否为空。如果存在相关节点,将会返回相关节点的值。不存在时,将返回一个空指针④。注意,结构是无锁的,不是无等待的,因为在push()和pop()函数中都有while循环,当compare\_exchange\_weak()失败的时候,循环将会持续下去。

如果一个垃圾收集机制(就像在管理语言中,比如C\#语言或Java语言),那到这就结束了,节点一旦它不再被任何线程访问,将被收集和循环利用。然而,并不是所有C++编译器都有垃圾回收机制。

\mySubsubsection{7.2.2}{终止内存泄露:使用无锁数据结构管理内存}

第一次了解pop()时,为了避免条件竞争(当有线程删除一个节点的同时,其他线程还持有指向该节点的指针,并且要解引用)选择了带有内存泄露的节点。因为内存泄露是不可接受的,所以现在来解决这个问题!

基本问题在于,当释放一个节点时,需要确认其他线程没有持有这个节点。只有一个线程调用pop()时,可以放心的释放。当节点添加入栈后,push()就不会与节点有任何的关系了,所以调用pop()函数的线程只和已加入节点有关,并且能够安全的将节点删除。

另一方面,栈处理多线程对pop()的调用时,就知道节点什么时候删除,这需要一个专用的垃圾回收器。听起来相当棘手,不过也没多难:检查pop()可访问哪些节点。不需担心push()中的节点,因为这些节点要推到栈上以后才能访问,而多线程只能通过pop()访问同一节点。

没有线程调用pop()时,就可以删除栈上的节点。添加节点到“可删除”列表中时,就能从结点中提取数据了。没有线程通过pop()访问节点时,可以安全的删除这些节点了。

怎么知道没有线程调用pop()?计数即可。当计数器数值增加时,就有节点推入。当减少时,就有节点删除。从“可删除”列表中删除节点就很安全,直到计数器的值为0为止。计数器必须是原子的,这样才能在多线程的情况下正确的进行计数。下面的代码中,展示了修改后的pop()函数,有些功能实现将在代码7.5中给出。

代码7.4 没有线程使用pop()访问节点时,就对节点进行回收

\begin{cpp}
template<typename T>
class lock_free_stack
{
private:
  std::atomic<unsigned> threads_in_pop;  // 1 原子变量
  void try_reclaim(node* old_head);
public:
  std::shared_ptr<T> pop()
  {
    ++threads_in_pop;  // 2 在做事之前,计数值加1
    node* old_head=head.load();
    while(old_head &&
      !head.compare_exchange_weak(old_head,old_head->next));
    std::shared_ptr<T> res;
    if(old_head)
    {
      res.swap(old_head->data);  // 3 回收删除的节点
    }
    try_reclaim(old_head);  // 4 从节点中直接提取数据,而非拷贝指针
    return res;
  }
};
\end{cpp}

threads\_in\_pop①原子变量用来记录有多少线程试图弹出栈中的元素。调用pop()②函数时,计数器加1。调用try\_reclaim()时,计数器减1。节点调用函数时,说明节点已删除④。因为暂时不需要将节点删除,可以通过swap()函数来删除节点上的数据③(而非只是拷贝指针),当不再需要这些数据时,数据会自动删除,而不是持续存在(因为还有对未删除节点的引用)。接下来看一下try\_reclaim()是如何实现的。

代码7.5 使用引用计数的回收机制

\begin{cpp}
template<typename T>
class lock_free_stack
{
private:
  std::atomic<node*> to_be_deleted;

  static void delete_nodes(node* nodes)
  {
    while(nodes)
    {
      node* next=nodes->next;
      delete nodes;
      nodes=next;
    }
  }
  void try_reclaim(node* old_head)
  {
    if(threads_in_pop==1)  // 1
    {
      node* nodes_to_delete=to_be_deleted.exchange(nullptr);  // 2 声明“可删除”列表
      if(!--threads_in_pop)  // 3 是否只有一个线程调用pop()?
      {
        delete_nodes(nodes_to_delete);  // 4
      }
      else if(nodes_to_delete)  // 5
      {
         chain_pending_nodes(nodes_to_delete);  // 6
      }
      delete old_head;  // 7
    }
    else
    {
      chain_pending_node(old_head);  // 8
      --threads_in_pop;
    }
  }
  void chain_pending_nodes(node* nodes)
  {
    node* last=nodes;
    while(node* const next=last->next)  // 9 让next指针指向链表的末尾
    {
      last=next;
    }
    chain_pending_nodes(nodes,last);
  }

  void chain_pending_nodes(node* first,node* last)
  {
    last->next=to_be_deleted;  // 10
    while(!to_be_deleted.compare_exchange_weak(  // 11 用循环来保证last->next的正确性
      last->next,first));
    }
    void chain_pending_node(node* n)
    {
      chain_pending_nodes(n,n);  // 12
    }
};
\end{cpp}

回收节点时①,threads\_in\_pop是1,当前线程对pop()进行访问时,就可以安全的将节点删除⑦(将等待节点删除也是安全的)。当数值不是1时,删除任何节点都不安全,所以需要向等待列表中继续添加节点⑧。

假设某一时刻,threads\_in\_pop值为1。就可以尝试回收等待列表,如果不回收,节点就会继续等待,直到整个栈被销毁。要做到回收,首先要通过原子exchange操作声明②删除列表,并将计数器减1③。如果之后计数的值为0,意味着没有其他线程访问等待节点链表。不必为出现新的等待节点而烦恼,因为它们会安全的回收。而后,可以使用delete\_nodes对链表进行迭代,并将其删除④。

计数值在减后不为0时,回收节点就不安全。如果存在⑤,就需要将其挂在等待删除链表后⑥,这种情况会发生在多个线程同时访问数据结构的时候。一些线程在第一次测试threads\_in\_pop①和对“回收”链表的声明②操作间调用pop(),这可能会将已经访问的节点新填入到链表中。图7.1中,线程C添加节点Y到to\_be\_deleted链表中,即使线程B仍将其引用作为old\_head,之后会尝试访问其next指针。线程A删除节点时,会造成线程B发生未定义行为。

![](../../images/chapter7/7-1.png)

图7.1 三个线程同时调用pop(),要在try\_reclaim()对声明节点进行删除前对threads\_in\_pop进行检查。

为了将等待删除的节点加入删除链表,需要复用节点的next指针将等待删除节点链接在一起。将已存在的链表链接到删除链表后面,通过遍历的方式找到链表的末尾⑨,将最后一个节点的next指针替换为当前to\_be\_deleted指针⑩,并且将链表的第一个节点作为新的to\_be\_deleted指针进行存储⑪。这需要在循环中使用compare\_exchange\_weak来保证通过线程添加进来的节点,不会发生内存泄露。链表发生改变时,更新next指针会很方便。添加单个节点是一种特殊情况,因为需要将这个节点作为第一个节点进行添加(同时作为最后一个节点)⑫。

栈处于低负荷状态时,这种方式没有问题,因为在没有线程访问pop()。不过,这只是瞬时的状态,也就是在回收前,需要检查threads\_in\_pop计数为0③的原因,也是删除节点⑦前进行对计数器检查的原因。删除节点是一项非常耗时的工作,并且希望其他线程对链表做的修改越少越好。从第一次发现threads\_in\_pop是1,到尝试删除节点,会用耗费很长的时间,就会让线程有机会调用pop(),让threads\_in\_pop不为0,阻止节点的删除操作。

栈处于高负荷状态时,因为其他线程在初始化后都能使用pop(),所以to\_ne\_deleted链表将会无限增加,会再次泄露。不存在任何静态情况时,就得为回收节点寻找替代机制。关键是要确定无线程访问给定节点,这样给定节点就能回收,所以最简单的替换机制就是使用\textit{风险指针}(hazard pointer)。

\mySubsubsection{7.2.3}{使用风险指针检测不可回收的节点}

“风险指针”这个术语引用于Maged Michael的研究\footnote[1]{“Safe Memory Reclamation for Dynamic Lock-Free Objects Using Atomic Reads and Writes,” Maged M.Michael, *in PODC ’02: Proceedings of the Twenty-first Annual Symposium on Principles of Distributed Computing* (2002), ISBN 1-58113-485-1.},之所以这样叫是因为删除一个节点可能会让其他引用线程处于危险状态。其他线程持有已删除节点的指针对其进行解引用操作时,会出现未定义行为。基本观点就是,当有线程去访问(其他线程)删除的对象时,会先对这个对象设置风险指针,而后通知其他线程——使用这个指针是危险的行为。当这个对象不再需要,就可以清除风险指针。如果了解牛津/剑桥的龙舟比赛,其实这里使用的机制和龙舟比赛开赛时差不多:每个船上的舵手都举起手来,以表示他们还没有准备好。只要有舵手举着手,裁判就不能让比赛开始。当所有舵手的手都放下后,比赛才能开始。比赛未开始或感觉自己船队的情况有变时,舵手可以再次举手。

当线程想要删除一个对象,就必须检查系统中其他线程是否持有风险指针。当没有风险指针时,就可以安全删除对象。否则,就必须等待风险指针消失。这样,线程就需要周期性的检查要删除的对象是否能安全删除。

看起来很简单,在C++中应该怎么做呢?

首先,需要能存储指向访问对象的指针,也就是风险指针。指针必须能让所有线程看到,需要线程能够对数据结构进行访问。正确且高效的分配这些线程的确是一个挑战,所以这个问题放在后面解决。假设有一个get\_hazard\_pointer\_for\_current\_thread()函数,可以返回风险指针的引用。当读取一个指针,并且想要解引用它的时候,就需要这个函数——这种情况下head数值源于下面的列表:

\begin{cpp}
std::shared_ptr<T> pop()
{
  std::atomic<void*>& hp=get_hazard_pointer_for_current_thread();
  node* old_head=head.load();  // 1
  node* temp;
  do
  {
    temp=old_head;
    hp.store(old_head);  // 2
    old_head=head.load();
  } while(old_head!=temp); // 3
  // ...
}
\end{cpp}

while循环能保证node不会在读取旧head指针①时,以及设置风险指针②时被删除。这种模式下,其他线程不知道有线程对这个节点进行了访问。幸运的是,旧head节点要删除时,head本身会发生变化,所以需要对head进行检查并持续循环,直到head指针中的值与风险指针中的值相同③。使用默认的new和delete操作对风险指针进行操作时,会出现未定义行为,所以需要确定实现是否支持这样的操作,或使用自定义内存分配器来保证用法的正确性。

现在已经设置了风险指针,就可以对pop()进行处理了,基于了解到的安全知识,不会有其他线程来删除节点。每一次重新加载old\_head后,解引用读取到指针时,就需要更新风险指针。从链表中提取节点时,就可以清除风险指针。如果没有其他风险指针引用节点,就可以安全的删除节点了;否则,就需要将其添加到链表中,之后再进行删除。下面的代码就是对该方案的完整实现。

代码7.6 使用风险指针实现的pop()

\begin{cpp}
std::shared_ptr<T> pop()
{
  std::atomic<void*>& hp=get_hazard_pointer_for_current_thread();
  node* old_head=head.load();
  do
  {
    node* temp;
    do  // 1 直到将风险指针设为head指针
    {
      temp=old_head;
      hp.store(old_head);
      old_head=head.load();
    } while(old_head!=temp);
  }
  while(old_head &&
    !head.compare_exchange_strong(old_head,old_head->next));
  hp.store(nullptr);  // 2 当声明完成,清除风险指针
  std::shared_ptr<T> res;
  if(old_head)
  {
    res.swap(old_head->data);
    if(outstanding_hazard_pointers_for(old_head))  // 3 在删除之前对风险指针引用的节点进行检查
    {
      reclaim_later(old_head);  // 4
    }
    else
    {
      delete old_head;  // 5
    }
    delete_nodes_with_no_hazards();  // 6
  }
  return res;
}
\end{cpp}

首先,循环内部会对风险指针进行设置。“比较/交换”操作失败时会重载old\_head,再次进行设置①。因为需要在循环内部做一些实际的工作,所以要使用compare\_exchange\_strong():当compare\_exchange\_weak()伪失败后,风险指针将重置(没有必要)。过程能保证风险指针在解引用(old\_head)之前被正确的设置。已声明了一个风险指针时,就可以将其清除了②。如果想要获取一个节点,就需要检查其他线程上的风险指针,检查是否有其他指针引用该节点③。如果有,就不能删除节点,需要将其放回链表中,之后再进行回收④;如果没有,就能直接将这个节点删除⑤。最后需要对任意节点进行检查,可以调用reclaim\_later()。如果链表上没有任何风险指针引用节点,就可以安全的删除这些节点⑥。当有节点持有风险指针,就只能等待下一个调用pop()的线程退出。

当然,这些函数——get\_hazard\_pointer\_for\_current\_thread(), reclaim\_later(), outstanding\_hazard\_pointers\_for(), 和delete\_nodes\_with\_no\_hazards()——的实现细节我们还没有看到,先来看看它们是如何工作的。

为线程分配风险指针实例的具体方案:使用get\_hazard\_pointer\_for\_current\_thread()与程序逻辑没什么关系(不过会影响效率,接下会看到具体的情况)。可以使用一个简单的结构体:固定长度的“线程ID-指针”数组。get\_hazard\_pointer\_for\_curent\_thread()可以通过这个数据来找到第一个释放槽,并将当前线程的ID放入到这个槽中。线程退出时,槽就再次置空,可以通过默认构造\texttt{std::thread::id()}将线程ID放入槽中。实现如下:

代码7.7 get\_hazard\_pointer\_for\_current\_thread()函数的简单实现

\begin{cpp}
unsigned const max_hazard_pointers=100;
struct hazard_pointer
{
  std::atomic<std::thread::id> id;
  std::atomic<void*> pointer;
};
hazard_pointer hazard_pointers[max_hazard_pointers];

class hp_owner
{
  hazard_pointer* hp;

public:
  hp_owner(hp_owner const&)=delete;
  hp_owner operator=(hp_owner const&)=delete;
  hp_owner():
    hp(nullptr)
  {
    for(unsigned i=0;i<max_hazard_pointers;++i)
    {
      std::thread::id old_id;
      if(hazard_pointers[i].id.compare_exchange_strong(  // 6 尝试声明风险指针的所有权
         old_id,std::this_thread::get_id()))
      {
        hp=&hazard_pointers[i];
        break;  // 7
      }
    }
    if(!hp)  // 1
    {
      throw std::runtime_error("No hazard pointers available");
    }
  }

  std::atomic<void*>& get_pointer()
  {
    return hp->pointer;
  }

  ~hp_owner()  // 2
  {
    hp->pointer.store(nullptr);  // 8
    hp->id.store(std::thread::id());  // 9
  }
};

std::atomic<void*>& get_hazard_pointer_for_current_thread()  // 3
{
  thread_local static hp_owner hazard;  // 4 每个线程都有自己的风险指针
  return hazard.get_pointer();  // 5
}
\end{cpp}

get\_hazard\_pointer\_for\_current\_thread()的实现看起来很简单③:一个hp\_owner④类型的thread\_local(本线程所有)变量,用来存储当前线程的风险指针,返回这个变量所持有的指针⑤。之后的工作:有线程第一次调用这个函数时,新hp\_owner实例就被创建。这个实例的构造函数⑥会通过查询“所有者/指针”表,用compare\_exchange\_strong()来检查某个记录是否有所有者,并进行析构②。compare\_exchange\_strong()失败时,其他线程也可拥有这个记录,所以可以继续执行下去。当交换成功,当前线程就拥有了这些记录,而后进行存储并停止搜索⑦。遍历了列表也没有找到物所有权的记录①时,就说明有很多线程在使用风险指针,所以会抛出一个异常。

当创建hp\_owner后的访问会很快,因为指针在缓存中,所以表不需要再次遍历。

当线程退出时,hp\_owner的实例将会销毁。析构函数会在\texttt{std::thread::id()}设置拥有者ID前,将指针重置为nullptr,这样就允许其他线程对这条记录进行复用⑧⑨。

实现get\_hazard\_pointer\_for\_current\_thread()后,outstanding\_hazard\_pointer\_for()实现就简单了:只需要对风险指针表进行搜索,就可以找到对应的记录。

\begin{cpp}
bool outstanding_hazard_pointers_for(void* p)
{
  for(unsigned i=0;i<max_hazard_pointers;++i)
  {
    if(hazard_pointers[i].pointer.load()==p)
    {
      return true;
    }
  }
  return false;
}
\end{cpp}

实现不需要对记录的所有者进行验证:没有所有者的记录会是空指针,所以比较代码将总返回false。

reclaim\_later()和delete\_nodes\_with\_no\_hazards()可以对简单的链表进行操作,reclaim\_later()只是将节点添加到列表中,delete\_nodes\_with\_no\_hazards()就是搜索整个列表,并将无风险指针的记录进行删除。

代码7.8 回收函数的简单实现

\begin{cpp}
template<typename T>
void do_delete(void* p)
{
  delete static_cast<T*>(p);
}

struct data_to_reclaim
{
  void* data;
  std::function<void(void*)> deleter;
  data_to_reclaim* next;

  template<typename T>
  data_to_reclaim(T* p):  // 1
    data(p),
    deleter(&do_delete<T>),
    next(0)
  {}

  ~data_to_reclaim()
  {
    deleter(data);  // 2
  }
};

std::atomic<data_to_reclaim*> nodes_to_reclaim;

void add_to_reclaim_list(data_to_reclaim* node)  // 3
{
  node->next=nodes_to_reclaim.load();
  while(!nodes_to_reclaim.compare_exchange_weak(node->next,node));
}

template<typename T>
void reclaim_later(T* data)  // 4
{
  add_to_reclaim_list(new data_to_reclaim(data));  // 5
}

void delete_nodes_with_no_hazards()
{
  data_to_reclaim* current=nodes_to_reclaim.exchange(nullptr);  // 6
  while(current)
  {
    data_to_reclaim* const next=current->next;
    if(!outstanding_hazard_pointers_for(current->data))  // 7
    {
      delete current;  // 8
    }
    else
    {
      add_to_reclaim_list(current);  // 9
    }
    current=next;
  }
}
\end{cpp}

首先,reclaim\_later()是一个函数模板④。风险指针是一个通用解决方案,不能将栈节点的类型写死。使用\texttt{std::atomic<void*>}对风险指针进行存储,需要对任意类型的指针进行处理。但不能使用\texttt{void*}形式,因为当要删除数据项时,delete操作只能对实际类型指针进行操作。data\_to\_reclaim的构造函数就很优雅:reclaim\_later()只创建一个data\_to\_reclaim的实例,并且将实例添加到回收链表中⑤。add\_to\_reclaim\_list()③就是使用compare\_exchange\_weak()循环来访问链头。

回到data\_to\_reclaim的构造函数①:构造函数也是模板函数。会删除的成员数据类型为\texttt{void*},并为do\_deltete()函数提供一个合适的指针实例——将\texttt{void*}类型转换成要删除的类型,然后删除指针所指向的对象。\texttt{std::function<>}可以安全的产生函数指针,所以data\_to\_reclaim的析构函数可以通过调用存储的函数对数据进行删除②。

将节点添加入链表时,不会调用data\_to\_reclaim的析构函数。析构函数会在没有风险指针指向节点的时候调用,这也就是delete\_nodes\_with\_no\_hazards()的作用。

delete\_nodes\_with\_no\_hazards()将已声明的链表节点进行回收,使用的是exchange()函数⑥(这个步骤简单且关键,是为了保证只有一个线程进行回收操作)。其他线程就能自由将节点添加到链表中,或在不影响回收指定节点线程的情况下对节点进行回收。

只要有节点存在于链表中,就需要检查每个节点,查看风险指针是否指向节点⑦。如果没有风险指针,就可以安全的将记录删除(并且清除存储的数据)⑧。否则,只能将这个节点添加到链表的后面,再进行回收⑨。

实现虽然很简单,也的确安全的回收了删除的节点,不过增加了很多开销。遍历风险指针数组需要检查max\_hazard\_pointers原子变量,并且每次pop()调用时,都需要再检查一遍。原子操作很耗时,所以pop()成为了性能瓶颈,不仅需要遍历节点的风险指针链表,还要遍历等待链表上的每一个节点。有max\_hazard\_pointers在链表中时,就需要检查max\_hazard\_pointers个已存储的风险指针。

还有更好一点的方法吗?

\textbf{对风险指针(较好)的回收策略}

当然有,这里展示一个风险指针的简单实现,从而来解释技术问题。首先,考虑内存性能。比起对回收链表上的每个节点进行检查都要调用pop(),除非有超过max\_hazard\_pointer数量的节点存在于链表之上,否则就不需要尝试回收任何节点。这样就能保证至少有一个节点能够回收,如果等待链表中的节点数量达到max\_hazard\_pointers+1,那和之前的方案差不多。当获取了max\_hazard\_pointers数量的节点时,可以调用pop()对节点进行回收,所以这样也不好。不过,当有2×max\_hazard\_pointers个节点在列表中时,就能保证至少有max\_hazard\_pointers个节点可以回收。再次尝试回收任意节点前,至少会对pop()有max\_hazard\_pointers次调用,这就很不错了。比起检查max\_hazard\_pointers个节点就调用max\_hazard\_pointers次pop()(而且还不一定能回收节点),检查2×max\_hazard\_pointers个节点时,每max\_hazard\_pointers次对pop()的调用,就会有max\_hazard\_pointers个节点可回收。就意味着,对两个节点检查调用pop(),就有一个节点能回收。

这个方法的缺点(有增加内存使用的情况):需要对回收链表上的节点进行原子计数,并且还有很多线程争相对回收的链表进行访问。如果内存盈余,可以使用更多内存的来实现更好的回收策略:作为线程的本地变量,每个线程中的都拥有其自己的回收链表,这样就不需要原子计数了。这样的话,只需要分配max\_hazard\_pointers x max\_hazard\_pointers个节点。所有节点被回收完毕前时有线程退出,其本地链表可以像之前一样保存在全局中,并且添加到下一个线程的回收链表中,让下一个线程对这些节点进行回收。

风险指针另一个缺点:与IBM申请的专利所冲突\footnote[2]{Maged M. Michael, U.S. Patent and Trademark Office application number 20040107227, “Method for efficient implementation of dynamic lock-free data structures with safe memory reclamation.”}。要让写软件能够被他人使用,就必须拥有合法的知识产权,所以需要拥有合适的许可证。这对所有无锁内存回收技术都适用(这是一个活跃的研究领域),很多大公司都会有自己的专利。你可能会问,“为什么用了这么大的篇幅来介绍一个大多数人都没办法的技术呢?”。首先,使用这种技术可能不需要买一个许可证。比如,当使用GPL下的免费软件许可来进行软件开发,软件将不会包含到IBM的专利。其次,设计无锁代码时,还需要从使用者的角度进行思考,比如:高消耗的原子操作。最后,有一个建议将风险指针纳入到C++标准的未来修订中,这种指针的确很好用,希望将来能够使用到编译器供应商的实现。

所以,是否有非专利的内存回收技术能使用呢?很幸运,的确有。引用计数就是这样的机制。

\mySubsubsection{7.2.4}{使用引用计数}

回到7.2.2节的问题,“想要删除还能被其他读者线程访问的节点,该怎么办?"。当能安全并准确的了解节点是否还被引用,以及没有线程访问这些节点的具体时间,即可判断是够能将对应节点进行删除。风险指针是通过将使用中的节点存放到链表中,而引用计数是通过对每个节点上访问的线程数量进行统计。

看起来简单粗暴……不,优雅,实际上管理起来却是很困难:首先,由\texttt{std::shared\_ptr<>}来完成这个任务,其有内置引用计数的指针。不幸的是,虽然\texttt{std::shared\_ptr<>}上的一些操作是原子的,不过其也不能保证是无锁的。智能指针上的原子操作和对其他原子类型的操作并没有什么不同,但是\texttt{std::shared\_ptr<>}旨在用于有多个上下文的情况下,并且在无锁结构中使用原子操作,无异于对该类增加了很多性能开销。如果平台支持\texttt{std::atomic\_is\_lock\_free(\&some\_shared\_ptr)}实现返回true,那么所有内存回收问题就都迎刃而解了。使用\texttt{std::shared\_ptr<node>}构成的链表实现,如代码7.9所示。需要注意的是,next指针是从已弹出的结构中获取,为了避免指针让所有node陷入深度自旋中,所以\texttt{std::shared\_ptr}引用的最后一个node会被销毁。

代码7.9 无锁栈——使用无锁\texttt{std::shared\_ptr<>}的实现

\begin{cpp}
template<typename T>
class lock_free_stack
{
private:
  struct node
  {
    std::shared_ptr<T> data;
    std::shared_ptr<node> next;
    node(T const& data_):
      data(std::make_shared<T>(data_))
    {}
  };

  std::shared_ptr<node> head;
public:
  void push(T const& data)
  {
    std::shared_ptr<node> const new_node=std::make_shared<node>(data);
    new_node->next=head.load();
    while(!std::atomic_compare_exchange_weak(&head,
        &new_node->next,new_node));
  }
  std::shared_ptr<T> pop()
  {
    std::shared_ptr<node> old_head=std::atomic_load(&head);
    while(old_head && !std::atomic_compare_exchange_weak(&head,
        &old_head,old_head->next));
    if (old_head){
      std::atomic_store(&old_head->next, std::shared_ptr<node>());
      return old_head->data;
    }
    return std::shared_ptr<T>();
  }
  ~lock_free_stack(){
    while(pop());
  }
};
\end{cpp}

对\texttt{std::shared\_ptr<>}使用无锁原子操作的实现不仅很少见,而且能为其使用一致性的原子操作也很难。并发技术规范扩展提供了工具,可以来解决这个问题,扩展中在头文件\texttt{<experimental/atomic>}中提供了\texttt{std::experimental::atomic\_shared\_ptr<T>}。多数情况下这与\texttt{std::atomic<std::shared\_ptr<T>>}等价(除非是\texttt{std::atomic<>}不使用\texttt{std::shared\_ptr<T>}),因为其具有特殊的复制语义,可以正确的处理引用计数。也就是\texttt{std::experimental::atomic\_shared\_ptr<T>}能在确保原子操作的同事,正确的处理引用计数。与第5章中的其他原子类型一样,其实现也不确定是否无锁。代码7.10是对代码7.9的重写,不需要记住atomic\_load和atomic\_store调用顺序。

代码7.10 使用\texttt{std::experimental::atomic\_shared\_ptr<>}实现的栈结构

\begin{cpp}
template<typename T>
class lock_free_stack
{
private:
  struct node
  {
    std::shared_ptr<T> data;
    std::experimental::atomic_shared_ptr<node> next;
    node(T const& data_):
      data(std::make_shared<T>(data_))
    {}
  };
  std::experimental::atomic_shared_ptr<node> head;
public:
  void push(T const& data)
  {
    std::shared_ptr<node> const new_node=std::make_shared<node>(data);
    new_node->next=head.load();
    while(!head.compare_exchange_weak(new_node->next,new_node));
  }
  std::shared_ptr<T> pop()
  {
    std::shared_ptr<node> old_head=head.load();
    while(old_head && !head.compare_exchange_weak(
      old_head,old_head->next.load()));
    if(old_head) {
      old_head->next=std::shared_ptr<node>();
      return old_head->data;
    }
    return std::shared_ptr<T>();
  }
  ~lock_free_stack(){
    while(pop());
  }
};
\end{cpp}

一些情况下,使用\texttt{std::shared\_ptr<>}实现的结构并非无锁,需要手动管理引用计数。

一种方式是对每个节点使用两个引用计数:内部计数和外部计数。两个值的总和就是对这个节点的引用数。外部计数记录有多少指针指向节点,即在指针每次进行读取的时候,外部计数加1。当线程结束对节点的访问时,内部计数减1。指针在读取时,外部计数加1;读取结束时,内部计数减1。

不需要“外部计数-指针”时(该节点就不能被多线程所访问了),外部计数减1和在被弃用的时候,内部计数将会增加。当内部计数等于0,就没有指针对该节点进行引用,就可以将该节点安全的删除。使用原子操作来更新共享数据也很重要。来看一下使用这种技术实现的无锁栈,只有确定节点能安全删除的情况下才能进行节点回收。

下面程序清单中就展示了内部数据结构,以及对push()简单优雅的实现。

代码7.11 使用分离引用计数的方式推送一个节点到无锁栈中

\begin{cpp}
template<typename T>
class lock_free_stack
{
private:
  struct node;

  struct counted_node_ptr  // 1
  {
    int external_count;
    node* ptr;
  };

  struct node
  {
    std::shared_ptr<T> data;
    std::atomic<int> internal_count;  // 2
    counted_node_ptr next;  // 3

    node(T const& data_):
      data(std::make_shared<T>(data_)),
      internal_count(0)
    {}
  };

  std::atomic<counted_node_ptr> head;  // 4

public:
  ~lock_free_stack()
  {
    while(pop());
  }

  void push(T const& data)  // 5
  {
    counted_node_ptr new_node;
    new_node.ptr=new node(data);
    new_node.external_count=1;
    new_node.ptr->next=head.load();
    while(!head.compare_exchange_weak(new_node.ptr->next,new_node));
  }
};
\end{cpp}

外部计数包含在counted\_node\_ptr的指针中①,这个结构体会被node中的next指针③和内部计数②用到。counted\_node\_ptr是一个简单的结构体,可以特化\texttt{std::atomic<>}模板来对链表的头指针进行声明④。

counted\_node\_ptr体积够小,能够让\texttt{std::atomic<counted\_node\_ptr>}无锁。一些平台上支持双字比较和交换操作,可以直接对结构体进行操作。平台不支持这样的操作时,最好使用\texttt{std::shared\_ptr<>}变量,当类型的体积过大,超出了平台支持指令,那么原子\texttt{std::atomic<>}将使用锁来保证其操作的原子性(从而会让“无锁”算法“基于锁”来完成)。另外,如果想要限制计数器的大小,需要已知平台上指针所占的空间(比如,地址空间只剩下48位,而一个指针就要占64位),可以将计数存在一个指针空间内,为了适应平台也可以存在一个机器字当中。这样的技巧需要对特定系统有足够的了解,不过这些讨论超出本书的范畴。

push()相对简单⑤,可构造一个counted\_node\_ptr实例,去引用新分配出来的(带有相关数据的)node,并且将node中的next指针设置为当前head,之后使用compare\_exchange\_weak()对head的值进行设置。因为internal\_count刚被设置其值为0,并且external\_count是1,所以这个新节点只有一个外部引用(head指针)。

通常,pop()都有一个从繁到简的过程,实现代码如下。

代码7.12 使用分离引用计数从无锁栈中弹出一个节点

\begin{cpp}
template<typename T>
class lock_free_stack
{
private:
  void increase_head_count(counted_node_ptr& old_counter)
  {
    counted_node_ptr new_counter;

    do
    {
      new_counter=old_counter;
      ++new_counter.external_count;
    }
    while(!head.compare_exchange_strong(old_counter,new_counter));  // 1

    old_counter.external_count=new_counter.external_count;
  }
public:
  std::shared_ptr<T> pop()
  {
    counted_node_ptr old_head=head.load();
    for(;;)
    {
      increase_head_count(old_head);
      node* const ptr=old_head.ptr;  // 2
      if(!ptr)
      {
        return std::shared_ptr<T>();
      }
      if(head.compare_exchange_strong(old_head,ptr->next))  // 3
      {
        std::shared_ptr<T> res;
        res.swap(ptr->data);  // 4

        int const count_increase=old_head.external_count-2;  // 5

        if(ptr->internal_count.fetch_add(count_increase)==  // 6
           -count_increase)
        {
          delete ptr;
        }

        return res;  // 7
      }
      else if(ptr->internal_count.fetch_sub(1)==1)
      {
        delete ptr;  // 8
      }
    }
  }
};
\end{cpp}

当加载head值后就必须将外部引用加1,表明这个节点正在引用,可保证解引用时的安全性。在引用计数增加前解引用指针,线程就能够访问这个节点,从而当前引用指针就成为了一个悬空指针,这就是将引用计数分离的主要原因:通过增加外部引用计数,保证指针在访问期间的合法性。compare\_exchange\_strong()的循环中①完成增加,通过比较和设置整个结构体来保证,指针不会在同一时间内受其他线程修改。

计数增加就能安全的解引用ptr,并读取head指针的值,访问指向的节点②。当访问到链表的末尾,指针就是空指针。当指针不为空时,就尝试对head调用compare\_exchange\_strong()来删除这个节点③。

compare\_exchange\_strong()成功时就拥有对应节点的所有权,并且可以和data进行交换④后返回。这样数据就不会持续保存,因为其他线程也会对栈进行访问,所以会有其他指针指向这个节点。而后,可以使用原子操作fetch\_add⑥,将外部计数加到内部计数中去。如果引用计数为0,那么之前的值(fetch\_add返回的值),在相加之前肯定是负数,这种情况下就可以将节点删除。这里需要注意的是,相加的值要比外部引用计数少2⑤。当节点已经从链表中删除,就要减少一次计数,并且这个线程无法再次访问指定节点,所以还要再减1。无论节点是否被删除,都能完成操作,所以可以将获取的数据进行返回⑦。

“比较/交换”③失败时,就说明其他线程已经把对应节点删除了,或者其他线程添加了一个新的节点到栈中。无论是哪种原因,需要通过“比较/交换”的调用,对具有新值的head重新进行操作。不过,首先需要减少节点(要删除的节点)上的引用计数。这个线程将再也没有办法访问这个节点了。如果当前线程持有最后一个引用(因为其他线程已经将这个节点从栈上删除了),那么内部引用计数将会为1,所以减1的操作将会让计数器为0。这样,就能在循环⑧之前将对应节点删除。

目前,使用默认\texttt{std::memory\_order\_seq\_cst}内存序来规定原子操作的执行顺序。大多数系统中,这种操作方式很耗时,且同步操作的开销要高于内存序。现在,可以考虑对数据结构的逻辑进行修改,放宽内存序要求,没有必要在栈上增加过度的开销。现在让我们来检查一下栈的操作,并且思考,能对一些操作使用更加宽松的内存序么?如果使用了,能确保安全性么?

\mySubsubsection{7.2.5}{无锁栈上的内存模型}

修改内存序之前,需要检查一下操作间的依赖关系,再去确定适合这种关系的最佳内存序。为了保证这种方式能够工作,需要从线程的视角进行观察。其中最简单的视角就是向栈中推入一个数据项,之后让其他线程从栈中弹出这个数据项。

这里需要三个重要数据参与。

1. counted\_node\_ptr转移的数据head。
2. head引用的node。
3. 节点所指向的数据项。

做push()的线程,会先构造数据项,并设置head。做pop()的线程,会先加载head,再做“比较/交换”操作,并增加引用计数,读取对应的node节点,获取next的指向值。next的值是非原子对象,所以为了保证读取安全,必须确定存储(推送线程)和加载(弹出线程)的先行关系。因为原子操作就是push()函数中的compare\_exchange\_weak(),所以需要获取两个线程间的先行关系。compare\_exchange\_weak()必须是\texttt{std::memory\_order\_release}或更严格的内存序。不过,compare\_exchange\_weak()调用失败时,什么都不会改变,并且可以持续循环下去,所以使用\texttt{std::memory\_order\_relaxed}就足够了。

\begin{cpp}
void push(T const& data)
{
  counted_node_ptr new_node;
  new_node.ptr=new node(data);
  new_node.external_count=1;
  new_node.ptr->next=head.load(std::memory_order_relaxed)
  while(!head.compare_exchange_weak(new_node.ptr->next,new_node,
    std::memory_order_release,std::memory_order_relaxed));
}
\end{cpp}

pop()的实现呢?为了确定先行关系,必须在访问next值之前使用\texttt{std::memory\_order\_acquire}或更严格的内存序操作。因为,increase\_head\_count()中使用compare\_exchange\_strong(),会获取next指针指向的旧值,所以要其获取成功就需要确定内存序。如同调用push()那样,当交换失败,循环会继续,所以在失败时可使用自由序:

\begin{cpp}
void increase_head_count(counted_node_ptr& old_counter)
{
  counted_node_ptr new_counter;

  do
  {
    new_counter=old_counter;
    ++new_counter.external_count;
  }
  while(!head.compare_exchange_strong(old_counter,new_counter,
        std::memory_order_acquire,std::memory_order_relaxed));

  old_counter.external_count=new_counter.external_count;
}
\end{cpp}

compare\_exchange\_strong()调用成功时,ptr中的值就被存到old\_counter中。存储操作是push()中的一个释放操作,compare\_exchange\_strong()操作是一个获取操作,现在存储同步于加载,并且能够获取先行关系。因此,push()中存储ptr的值要先行于在pop()中对ptr->next的访问,目前的操作完全安全。

内存序对head.load()的初始化并不妨碍分析,现在就可以使用\texttt{std::memory\_order\_relaxed}。

接下来compare\_exchange\_strong()将old\_head.ptr->next设置为head。是否需要做什么来保证操作线程中的数据完整性呢?交换成功就能访问ptr->data,所以需要保证在push()线程中对ptr->data进行存储(在加载之前)。increase\_head\_count()中的获取操作,保证与push()线程中的存储和“比较/交换”操作同步。在push()线程中存储数据,先行于存储head指针;调用increase\_head\_count()先行于对ptr->data的加载。即使,pop()中的“比较/交换”操作使用\texttt{std::memory\_order\_relaxed},这些操作还是能正常运行。唯一不同的地方就是,调用swap()让ptr->data有所变化,且没有其他线程可以对同一节点进行操作(这就是“比较/交换”操作的作用)。

compare\_exchange\_strong()失败时,新值不会更新old\_head,并继续循环。因为确定了\texttt{std::memory\_order\_acquire}内存序在increase\_head\_count()中使用的可行性,所以使用\texttt{std::memory\_order\_relaxed}也可以。

其他线程呢?是否需要设置一些更为严格的内存序来保证其他线程的安全呢?回答是“不用”。因为,head只会因“比较/交换”操作有所改变,对于“读-改-写”操作来说,push()中的“比较/交换”操作是构成释放序列的一部分。因此,即使有很多线程在同一时间对head进行修改,push()中的compare\_exchange\_weak()与increase\_head\_count()(读取已存储的值)中的compare\_exchange\_strong()也是同步的。

剩余的就可以用来处理fetch\_add()操作(用来改变引用计数的操作),因为已知其他线程不可能对该节点的数据进行修改,所以从节点中返回数据的线程可以继续执行。不过,当线程获取修改后的值时,就代表操作失败(swap()是用来提取数据项的引用)。为了避免数据竞争,要保证swap()先行于delete操作。一种简单的解决办法:在“成功返回”分支中对fetch\_add()使用\texttt{std::memory\_order\_release}内存序,在“再次循环”分支中对fetch\_add()使用\texttt{std::memory\_order\_acquire}内存序。不过,这·有点矫枉过正:只有一个线程做delete操作(将引用计数设置为0的线程),所以只有这个线程需要获取操作。因为fetch\_add()是一个“读-改-写”操作,是释放序列的一部分,所以可以使用一个额外的load()做获取。当“再次循环”分支将引用计数减为0时,fetch\_add()可以重载引用计数,使用\texttt{std::memory\_order\_acquire}为了保持需求的同步关系。并且,fetch\_add()本身可以使用\texttt{std::memory\_order\_relaxed}。使用新pop()的栈实现如下。

代码7.13 基于引用计数和自由原子操作的无锁栈

\begin{cpp}
template<typename T>
class lock_free_stack
{
private:
  struct node;
  struct counted_node_ptr
  {
    int external_count;
    node* ptr;
  };

  struct node
  {
    std::shared_ptr<T> data;
    std::atomic<int> internal_count;
    counted_node_ptr next;

    node(T const& data_):
      data(std::make_shared<T>(data_)),
      internal_count(0)
    {}
  };

  std::atomic<counted_node_ptr> head;

  void increase_head_count(counted_node_ptr& old_counter)
  {
    counted_node_ptr new_counter;

    do
    {
      new_counter=old_counter;
      ++new_counter.external_count;
    }
    while(!head.compare_exchange_strong(old_counter,new_counter,
                                        std::memory_order_acquire,
                                        std::memory_order_relaxed));
    old_counter.external_count=new_counter.external_count;
  }
public:
  ~lock_free_stack()
  {
    while(pop());
  }

  void push(T const& data)
  {
    counted_node_ptr new_node;
    new_node.ptr=new node(data);
    new_node.external_count=1;
    new_node.ptr->next=head.load(std::memory_order_relaxed)
    while(!head.compare_exchange_weak(new_node.ptr->next,new_node,
                                      std::memory_order_release,
                                      std::memory_order_relaxed));
  }
  std::shared_ptr<T> pop()
  {
    counted_node_ptr old_head=
       head.load(std::memory_order_relaxed);
    for(;;)
    {
      increase_head_count(old_head);
      node* const ptr=old_head.ptr;
      if(!ptr)
      {
        return std::shared_ptr<T>();
      }
      if(head.compare_exchange_strong(old_head,ptr->next,
                                      std::memory_order_relaxed))
      {
        std::shared_ptr<T> res;
        res.swap(ptr->data);

        int const count_increase=old_head.external_count-2;

        if(ptr->internal_count.fetch_add(count_increase,
              std::memory_order_release)==-count_increase)
        {
          delete ptr;
        }

        return res;
      }
      else if(ptr->internal_count.fetch_add(-1,
                   std::memory_order_relaxed)==1)
      {
        ptr->internal_count.load(std::memory_order_acquire);
        delete ptr;
      }
    }
  }
};
\end{cpp}

这段历练就要告一段落了,我们已经获得比之前好很多的栈实现。深思熟虑后,通过使用更多的自由操作,在不影响并发性的同时提高性能。实现中的pop()有37行,而功能等同于代码6.1中基于锁的栈实现和代码7.2中无内存管理的无锁栈实现。接下来要设计的无锁队列,将看到类似的情况:无锁结构的复杂性(主要在于内存的管理)。

\mySubsubsection{7.2.6}{实现一个无锁的线程安全队列}

队列的挑战与栈的有些不同,因为push()和pop()在队列中,操作的不是同一个地方。因此,同步需求就不一样。需要保证对一端的修改是正确的,且对另一端是可见的。不过,代码6.6中队列有一个try\_pop()成员函数,其作用和代码7.2中简单的无锁栈的pop()功能差不多,就可以合理的假设无锁代码都很相似。这是为什么呢?

如果将代码6.6中的代码作为基础,就需要两个node指针:head和tail。可以让多线程对它们进行访问,所以这两个节点最好是原子的(就不用考虑互斥问题了)。让我们对代码6.6做一些修改,并看下应该从哪里开始设计。

代码7.14 单生产者/单消费者模型下的无锁队列

\begin{cpp}
template<typename T>
class lock_free_queue
{
private:
  struct node
  {
    std::shared_ptr<T> data;
    node* next;

    node():
       next(nullptr)
    {}
  };

  std::atomic<node*> head;
  std::atomic<node*> tail;

  node* pop_head()
  {
    node* const old_head=head.load();
    if(old_head==tail.load())  // 1
    {
      return nullptr;
    }
    head.store(old_head->next);
    return old_head;
  }
public:
  lock_free_queue():
      head(new node),tail(head.load())
  {}

  lock_free_queue(const lock_free_queue& other)=delete;
  lock_free_queue& operator=(const lock_free_queue& other)=delete;

  ~lock_free_queue()
  {
    while(node* const old_head=head.load())
    {
      head.store(old_head->next);
      delete old_head;
    }
  }
  std::shared_ptr<T> pop()
  {
    node* old_head=pop_head();
    if(!old_head)
    {
      return std::shared_ptr<T>();
    }

    std::shared_ptr<T> const res(old_head->data);  // 2
    delete old_head;
    return res;
  }

  void push(T new_value)
  {
    std::shared_ptr<T> new_data(std::make_shared<T>(new_value));
    node* p=new node;  // 3
    node* const old_tail=tail.load();  // 4
    old_tail->data.swap(new_data);  // 5
    old_tail->next=p;  // 6
    tail.store(p);  // 7
  }
};
\end{cpp}

一眼望去,这个实现没什么毛病,当只有一个线程调用push()和pop()时,这种情况下队列一点毛病没有。这里push()和pop()之间的先行关系就很重要了,这直接关系到获取到的data。对tail的存储⑦同步于对tail的加载①,存储之前节点的data指针⑤先行于存储tail。并且,加载tail先行于加载data指针②,所以对data的存储要先行于加载,一切都没问题。因此,这是一个完美的\textit{单生产者,单消费者}(SPSC, single-producer, single-consume)队列。

问题在于当多线程对push()或pop()并发调用。先看一下push():如果有两个线程并发调用push(),会新分配两个节点作为虚拟节点③,也会读取到相同的tail值④,因此也会同时修改同一个节点,同时设置data和next指针⑤⑥,明显的数据竞争!

pop\_head()函数也有类似的问题。当有两个线程并发的调用这个函数时,这两个线程就会读取到同一个head,并且会通过next指针去修改旧值。两个线程都能索引到同一个节点——真是一场灾难!不仅要保证只有一个pop()线程可以访问给定项,还要保证其他线程在读取head时,可以安全的访问节点中的next。这就和无锁栈中pop()的问题一样了。

pop()的问题解决了,那么push()呢?问题在于为了获取push()和pop()间的先行关系,就需要在为虚拟节点设置数据项前,更新tail指针。并发访问push()时,因为每个线程所读取到的是同一个tail,所以线程会进行竞争。

\textbf{多线程下的push()}

第一个选择是,在两个真实节点中添加一个虚拟节点。这种方法,需要当前tail节点更新next指针,这样让节点看起来像一个原子变量。当一个线程成功将next指针指向一个新节点,就说明其成功的添加了一个指针。否则,就需要再次读取tail,并重新对指针进行添加。这里就需要对pop()进行简单的修改,为了消除持有空指针的节点再次进行循环。这个方法的缺点:每次pop()函数的调用,通常都要删除两个节点,每次添加一个节点,都需要分配双份内存。

第二个选择是,让data指针原子化,通过“比较/交换”操作对其进行设置。如果“比较/交换”成功,就说明能获取tail,并能够安全的对其next指针进行设置,也就是更新tail。因为有其他线程对数据进行了存储,所以会导致“比较/交换”操作的失败,这时就要重新读取tail,重新循环。当原子操作对于\texttt{std::shared\_ptr<>}是无锁的,就可以轻松一下了。如果不是,就需要一个替代方案:一种可能是让pop()函数返回\texttt{std::unique\_ptr<>}(毕竟,这个指针只能引用指定对象),并且将数据作为普通指针存储在队列中的方案。这就需要队列支持存储\texttt{std::atomic<T*>}类型,对于compare\_exchange\_strong()的调用就很有必要了。使用类似于代码7.11中的引用计数模式,来解决多线程对pop()和push()的访问。

代码7.15 push()的第一次修订(不正确的)

\begin{cpp}
void push(T new_value)
{
  std::unique_ptr<T> new_data(new T(new_value));
  counted_node_ptr new_next;
  new_next.ptr=new node;
  new_next.external_count=1;
  for(;;)
  {
    node* const old_tail=tail.load();  // 1
    T* old_data=nullptr;
    if(old_tail->data.compare_exchange_strong(
      old_data,new_data.get()))  // 2
    {
      old_tail->next=new_next;
      tail.store(new_next.ptr);  // 3
      new_data.release();
      break;
    }
  }
}
\end{cpp}

使用引用计数方案可以避免竞争,不过竞争不只在push()中。可以再看一下7.14中的修订版push(),与栈中模式相同:加载原子指针①,并且对该指针解引用②。同时,另一个线程可以对指针进行更新③,最终回收该节点(在pop()中)。当节点回收后,再对指针进行解引用,就会导致未定义行为。有个诱人的方案,就是给tail也添加计数器,就像给head做的那样,不过队列中的节点的next指针中都已经拥有了一个外部计数。同一个节点上有两个外部计数,为了避免过早的删除节点,需要对之前引用计数方案进行修改。通过对node结构中外部计数器数量的统计,解决这个问题。外部计数器销毁时,统计值减1(将对应的外部计数添加到内部)。当内部计数是0,且没有外部计数器时,对应节点就可以安全的删除了。这个技术是我查阅Joe Seigh的“原子指针+”项目\footnote[5]{Atomic Ptr Plus Project, \url{http://atomic-ptr-plus.sourceforge.net/.}}的时候看到的。下面push()的实现,使用的就是这种方案。

代码7.16 使用带有引用计数tail,实现的无锁队列中的push()

\begin{cpp}
template<typename T>
class lock_free_queue
{
private:
  struct node;
  struct counted_node_ptr
  {
    int external_count;
    node* ptr;
  };

  std::atomic<counted_node_ptr> head;
  std::atomic<counted_node_ptr> tail;  // 1

  struct node_counter
  {
    unsigned internal_count:30;
    unsigned external_counters:2;  // 2
  };

  struct node
  {
    std::atomic<T*> data;
    std::atomic<node_counter> count;  // 3
    counted_node_ptr next;

    node()
    {
      node_counter new_count;
      new_count.internal_count=0;
      new_count.external_counters=2;  // 4
      count.store(new_count);

      next.ptr=nullptr;
      next.external_count=0;
     }
  };
public:
  void push(T new_value)
  {
    std::unique_ptr<T> new_data(new T(new_value));
    counted_node_ptr new_next;
    new_next.ptr=new node;
    new_next.external_count=1;
    counted_node_ptr old_tail=tail.load();

    for(;;)
    {
      increase_external_count(tail,old_tail);  // 5

      T* old_data=nullptr;
      if(old_tail.ptr->data.compare_exchange_strong(  // 6
           old_data,new_data.get()))
      {
        old_tail.ptr->next=new_next;
        old_tail=tail.exchange(new_next);
        free_external_counter(old_tail);  // 7
        new_data.release();
        break;
      }
      old_tail.ptr->release_ref();
    }
  }
};
\end{cpp}

代码7.16中,tail和head一样都是\texttt{atomic<counted\_node\_ptr>}类型①,并且node结构体中用count成员变量替换了之前的internal\_count③。count成员变量包括了internal\_count和外部external\_counters成员②。注意,这里需要2bit的external\_counters,因为最多就有两个计数器。因为使用了位域,所以就将internal\_count指定为30bit的值,就能保证计数器的总体大小是32bit。内部计数值就有充足的空间来保证这个结构体能放在一个机器字中(包括32位和64位平台)。为的就是避免条件竞争,将结构体作为单独的实体来更新。让结构体的大小保持在一个机器字内,对其的操作就如同原子操作一样,还可以在多个平台上使用。

node初始化时,internal\_count设置为0,external\_counter设置为2④,因为当新节点加入队列中时,都会被tail和上一个节点的next指针所指向。push()与代码7.14中的实现很相似,除了为了对tail中的值进行解引用,需要调用节点data成员变量的compare\_exchange\_strong()成员函数⑥保证值的正确性。在这之前还要调用increase\_external\_count()增加计数器的计数⑤,而后在对尾部的旧值调用free\_external\_counter()⑦。

push()处理完毕再来看一下pop()。下面的实现,将代码7.12中的引用计数pop()与7.14中队列pop()混合的版本。

代码7.17 使用尾部引用计数,将节点从无锁队列中弹出

\begin{cpp}
template<typename T>
class lock_free_queue
{
private:
  struct node
  {
    void release_ref();
  };
public:
  std::unique_ptr<T> pop()
  {
    counted_node_ptr old_head=head.load(std::memory_order_relaxed);  // 1
    for(;;)
    {
      increase_external_count(head,old_head);  // 2
      node* const ptr=old_head.ptr;
      if(ptr==tail.load().ptr)
      {
        ptr->release_ref();  // 3
        return std::unique_ptr<T>();
      }
      if(head.compare_exchange_strong(old_head,ptr->next))  // 4
      {
        T* const res=ptr->data.exchange(nullptr);
        free_external_counter(old_head);  // 5
        return std::unique_ptr<T>(res);
      }
      ptr->release_ref();  // 6
    }
  }
};
\end{cpp}

进入循环时,在将加载值的外部计数增加②之前,需要加载old\_head值作为启动①。当head与tail节点相同的时候,就能对引用进行释放③,因为队列中没有数据,所以返回的是空指针。如果队列中还有数据,可以使用compare\_exchange\_strong()来做声明④。与7.11中的栈一样,将外部计数和指针做为一个整体进行比较。当外部计数或指针有所变化时,需要将引用释放后,再次进行循环⑥。当交换成功时,已声明的数据就归线程所有。为已弹出节点释放外部计数后⑤,就能把对应的指针返回给调用函数了。当两个外部引用计数都被释放,且内部计数降为0时,节点就可以删除了。对应的引用计数函数将会在代码7.18,7.19和7.20中展示。

代码7.18 在无锁队列中释放一个节点引用

\begin{cpp}
template<typename T>
class lock_free_queue
{
private:
  struct node
  {
    void release_ref()
    {
      node_counter old_counter=
        count.load(std::memory_order_relaxed);
      node_counter new_counter;
      do
      {
        new_counter=old_counter;
        --new_counter.internal_count;  // 1
      }
      while(!count.compare_exchange_strong(  // 2
            old_counter,new_counter,
            std::memory_order_acquire,std::memory_order_relaxed));
      if(!new_counter.internal_count &&
         !new_counter.external_counters)
      {
        delete this;  // 3
      }
    }
  };
};
\end{cpp}

node::release\_ref()的实现,只是对7.12中lock\_free\_stack::pop()进行了小幅度修改。不过,代码7.12中的代码仅是处理单个外部计数的情况,所以要修改internal\_count①,只需要使用fetch\_sub就能让count结构体自动更新。因此,需要一个“比较/交换”循环②。当内外部计数都为0时,就代表这是最后一次引用,之后就可以将这个节点删除③。

代码7.19 从无锁队列中获取一个节点的引用

\begin{cpp}
template<typename T>
class lock_free_queue
{
private:
  static void increase_external_count(
    std::atomic<counted_node_ptr>& counter,
    counted_node_ptr& old_counter)
  {
    counted_node_ptr new_counter;
    do
    {
      new_counter=old_counter;
      ++new_counter.external_count;
    }
    while(!counter.compare_exchange_strong(
      old_counter,new_counter,
      std::memory_order_acquire,std::memory_order_relaxed));

    old_counter.external_count=new_counter.external_count;
  }
};
\end{cpp}

代码7.19展示的是,并不是所有引用的释放,都会得到一个新引用,并增加外部计数的值。increase\_external\_count()和代码7.13中的increase\_head\_count()很相似,不同的是increase\_external\_count()作为静态成员函数,通过将外部计数器作为第一个参数传入函数进行更新,而非只操作一个固定的计数器。

代码7.19 无锁队列中释放节点外部计数器

\begin{cpp}
template<typename T>
class lock_free_queue
{
private:
  static void free_external_counter(counted_node_ptr &old_node_ptr)
  {
    node* const ptr=old_node_ptr.ptr;
    int const count_increase=old_node_ptr.external_count-2;

    node_counter old_counter=
      ptr->count.load(std::memory_order_relaxed);
    node_counter new_counter;
    do
    {
      new_counter=old_counter;
      --new_counter.external_counters;  // 1
      new_counter.internal_count+=count_increase;  // 2
    }
    while(!ptr->count.compare_exchange_strong(  // 3
           old_counter,new_counter,
           std::memory_order_acquire,std::memory_order_relaxed));

    if(!new_counter.internal_count &&
       !new_counter.external_counters)
    {
      delete ptr;  // 4
    }
  }
};
\end{cpp}

与increase\_external\_count()对应的是free\_external\_counter()。这里的代码和代码7.11中的lock\_free\_stack::pop()类似,不过做了一些修改用来处理external\_counters计数。使用单个compare\_exchange\_strong()对计数结构体中的两个计数器进行更新③。和代码7.11一样,internal\_count会进行更新②,并且external\_counters将会减1①。当内外计数值都为0,就没有更多的节点可以引用,所以节点可以安全的删除④,这个操作需要作为独立的操作来完成(因此需要“比较/交换”循环)来避免条件竞争。如果将两个计数器分开来更新,两个线程的情况下,可能都会认为自己是最后一个引用者,从而将节点删除,从而导致未定义行为。

现在的队列工作正常且无竞争,但是有一个性能问题。当一个线程对old\_tail.ptr->data成功的完成compare\_exchange\_strong()(7.15中的⑥),就可以执行push()操作。并且,能确定没有其他线程在同时执行push()操作。让其他线程看到有新值的加入,因此在compare\_exchange\_strong()调用失败的时候,线程会继续循环。忙等待会消耗CPU的运算周期,因此这就是一个锁。push()的首次调用是要在其他线程完成后,将阻塞去除后才能完成,所以实现只是*半无锁*(no longer lock-free)结构。不仅如此,当线程阻塞时,操作系统会给不同的线程不同的优先级,用于获取互斥锁。当前情况下优先级相同,所以阻塞线程将会浪费CPU的运算周期,直到第一个线程完成其操作。处理的技巧出自于“无锁技巧包”:等待线程可以帮助push()线程完成操作。

\textbf{无锁队列中的线程互助}

为了恢复代码的无锁属性,需要让等待线程做一些事情,也就是帮进展缓慢的线程完成工作。

这种情况下,可以知道线程应该去做什么:尾节点的next指针需要指向一个新的虚拟节点,且tail之后也要更新,虚拟节是谁创建的并不重要。将next指针放入一个原子节点中时,就可以使用compare\_exchange\_strong()来设置next指针。next指针已经被设置,就可以使用compare\_exchange\_weak()循环对tail进行设置,能保证next指针始终引用的是同一个原始节点。如果引用的不是同一个原始节点,其他部分就已经更新,可以停止尝试再次循环。只需要对pop()进行微小的改动,其目的就是为了加载next指针。

代码7.21 修改pop()帮助push()完成工作

\begin{cpp}
template<typename T>
class lock_free_queue
{
private:
  struct node
  {
    std::atomic<T*> data;
    std::atomic<node_counter> count;
    std::atomic<counted_node_ptr> next;  // 1
  };
public:
  std::unique_ptr<T> pop()
  {
    counted_node_ptr old_head=head.load(std::memory_order_relaxed);
    for(;;)
    {
      increase_external_count(head,old_head);
      node* const ptr=old_head.ptr;
      if(ptr==tail.load().ptr)
      {
        return std::unique_ptr<T>();
      }
      counted_node_ptr next=ptr->next.load();  // 2
      if(head.compare_exchange_strong(old_head,next))
      {
        T* const res=ptr->data.exchange(nullptr);
        free_external_counter(old_head);
        return std::unique_ptr<T>(res);
      }
      ptr->release_ref();
    }
  }
};
\end{cpp}

修改很简单:next指针线程就是原子的①,所以load②也是原子的。在这个例子中,可以使用默认memory\_order\_seq\_cst内存序,所以这里可以忽略对load()的显式调用,并且依赖于加载对象隐式转换成counted\_node\_ptr,不过这里的显式调用就可以用来提醒:哪里需要显式添加内存序。

以下代码对push()有更多的展示。

代码7.22 无锁队列中简单的帮助性push()的实现

\begin{cpp}
template<typename T>
class lock_free_queue
{
private:
  void set_new_tail(counted_node_ptr &old_tail,  // 1
                    counted_node_ptr const &new_tail)
  {
    node* const current_tail_ptr=old_tail.ptr;
    while(!tail.compare_exchange_weak(old_tail,new_tail) &&  // 2
          old_tail.ptr==current_tail_ptr);
    if(old_tail.ptr==current_tail_ptr)  // 3
      free_external_counter(old_tail);  // 4
    else
      current_tail_ptr->release_ref();  // 5
  }
public:
  void push(T new_value)
  {
    std::unique_ptr<T> new_data(new T(new_value));
    counted_node_ptr new_next;
    new_next.ptr=new node;
    new_next.external_count=1;
    counted_node_ptr old_tail=tail.load();

    for(;;)
    {
      increase_external_count(tail,old_tail);

      T* old_data=nullptr;
      if(old_tail.ptr->data.compare_exchange_strong(  // 6
         old_data,new_data.get()))
      {
        counted_node_ptr old_next={0};
        if(!old_tail.ptr->next.compare_exchange_strong(  // 7
           old_next,new_next))
        {
          delete new_next.ptr;  // 8
          new_next=old_next;  // 9
        }
        set_new_tail(old_tail, new_next);
        new_data.release();
        break;
      }
      else  // 10
      {
        counted_node_ptr old_next={0};
        if(old_tail.ptr->next.compare_exchange_strong(  // 11
           old_next,new_next))
        {
          old_next=new_next;  // 12
          new_next.ptr=new node;  // 13
        }
        set_new_tail(old_tail, old_next);  // 14
      }
    }
  }
};
\end{cpp}

与代码7.15中的push()相似,对data进行设置⑥,就需要对另一线程帮忙的情况进行处理,else分支就是具体的帮助内容⑩。

对节点中的data指针进行设置⑥时,新版push()对next指针的更新使用的是compare\_exchange\_strong()⑦(使用compare\_exchange\_strong()来避免循环),当交换失败就能知道另有线程对next指针进行设置,所以就可以删除开始分配的那个新节点⑧。还需要获取next指向的值——其他线程对tail指针设置的值。

对tail的更新是在set\_new\_tail()中完成①。使用一个compare\_exchange\_weak()循环②来更新tail,如果其他线程尝试push()一个节点时,external\_count部分将会改变。不过,当其他线程成功的修改了tail时,就不能对其值进行替换。否则,队列中的循环将会结束,这是一个相当糟糕的主意。因此,“比较/交换”操作失败时,就需要保证ptr加载值要与tail指向的值相同。当新旧ptr相同时,循环退出③就代表对tail的设置已经完成,所以需要释放旧外部计数器④。当ptr值不一样时另一线程可能已经将计数器释放了,所以只需要对该线程持有的单次引用进行释放即可⑤。

当线程调用push()时,未能在循环阶段对data指针进行设置,这个线程可以帮助成功的线程完成更新。首先,会尝试更新next指针,让其指向该线程分配出来的新节点⑪。指针更新成功时,就可以将这个新节点作为新的tail节点⑫,且需要分配另一个新节点,用来管理队列中新推送的数据项⑬。再进入循环之前,可以通过调用set\_new\_tail来设置tail节点⑭。

读者们可能已经意识到,比起大量的new和delete操作,这样的代码更加短小精悍,因为新节点实在push()中分配,在pop()中销毁。因此,内存分配器的效率也需要考虑到,糟糕的分配器可能会让无锁容器的扩展特性消失。选择和实现高效的分配器,超出了本书的范畴,不过需要牢记:测试以及衡量分配器效率最好的办法,就是对使用前和使用后进行比较。为优化内存分配,包括每个线程有自己的分配器,以及使用回收列表对节点进行回收,而非将这些节点返回给分配器。

例子已经足够多了,让我们从这些例子中提取出一些指导建议吧。

-------

[1] “Safe Memory Reclamation for Dynamic Lock-Free Objects Using Atomic Reads and Writes,” Maged M.Michael, *in PODC ’02: Proceedings of the Twenty-first Annual Symposium on Principles of Distributed Computing* (2002), ISBN 1-58113-485-1.

[2] Maged M. Michael, U.S. Patent and Trademark Office application number 20040107227, “Method for efficient implementation of dynamic lock-free data structures with safe memory reclamation.”

[3] GNU General Public License http://www.gnu.org/licenses/gpl.html.

[4] IBM Statement of Non-Assertion of Named Patents Against OSS, http://www.ibm.com/ibm/licensing/patents/pledgedpatents.pdf.

[5] Atomic Ptr Plus Project, http://atomic-ptr-plus.sourceforge.net/.