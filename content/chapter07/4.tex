% # 7.4 本章总结

从第6章中的基于锁的数据结构起,本章简要的描述了一些无锁数据结构的实现(通过实现栈和队列)。这个过程中,需要小心使用原子操作的内存序,为了保证无数据竞争,以及让每个线程看到一个数据结构实例。并且,在无锁结构中对内存的管理是越来越难。还有,如何通过帮助线程的方式,来避免忙等待循环。

设计无锁数据结构是一项很困难的任务,并且很容易犯错。不过,这样的数据结构在某些重要情况下可对其性能进行增强。但愿通过本章的的一些例子,以及一些指导意见,可以帮助你设计出自己的无锁数据结构,或实现一份研究报告中的数据结构,或用以发现离职同事代码中的bug。

不管在线程间共享怎么样的数据,需要考虑数据结构应该如何使用,并且如何在线程间同步数据。通过设计并发访问的数据结构,就能对数据结构进行功能性封装,其他部分的代码就着重于对数据的执行,而非数据的同步。在第8章中会看到类似的行为:将并发数据结构转为一般的并发代码。并行算法是使用多线程的方式提高性能,因为算法需要工作线程共享它们的数据,所以对并发数据结构的选择就很关键了。