% # 9.1 线程池

很多公司里,雇员通常会在办公室度过他们的办公时光(偶尔也会外出访问客户或供应商),或是参加贸易展会。虽然外出可能很有必要,并且可能需要很多人一起去,不过对于一些特别的雇员来说,一趟可能就是几个月,甚至是几年。公司不可能给每个雇员都配一辆车,不过公司可以提供一些共用车辆。这样就会有一定数量车,来让所有雇员使用。当一个员工要去异地旅游时,他就可以从共用车辆中预定一辆,并在返回公司的时候将车交还。如果某天没有闲置的共用车辆,雇员就得不延后其旅程了。

线程池就是类似的调度机制。大多数系统中,将每个任务指定给某个线程是不切实际的,不过可以利用并发性,进行并发执行。线程池提供了这样的功能,将提交到线程池中的任务并发执行,提交的任务将会挂在任务队列上。工作线程会从队列中的获取任务,当任务执行完成后,再从任务队列中获取下一个任务。

创建一个线程池时,会遇到几个关键性的设计问题,比如:可使用的线程数量,高效的任务分配方式,以及是否需要等待一个任务完成。

本节将看到线程池是如何解决这些问题的,从最简单的线程池开始吧!

\mySubsubsection{9.1.1}{简单的线程池}

作为简单的线程池,其拥有固定数量的工作线程(通常工作线程数量与\texttt{std::thread::hardware\_concurrency()}相同)。工作需要完成时,可以调用函数将任务挂在任务队列中。每个工作线程都会从任务队列上获取任务,然后执行这个任务,执行完成后再回来获取新的任务。线程池中线程就不需要等待其他线程完成对应任务了。如果需要等待,就需要对同步进行管理。

下面清单中的代码就展示了一个最简单的线程池实现。

代码9.1 简单的线程池

\begin{cpp}
class thread_pool
{
  std::atomic_bool done;
  thread_safe_queue<std::function<void()> > work_queue;  // 1
  std::vector<std::thread> threads;  // 2
  join_threads joiner;  // 3

  void worker_thread()
  {
    while(!done)  // 4
    {
      std::function<void()> task;
      if(work_queue.try_pop(task))  // 5
      {
        task();  // 6
      }
      else
      {
        std::this_thread::yield();  // 7
      }
    }
  }

public:
  thread_pool():
    done(false),joiner(threads)
  {
    unsigned const thread_count=std::thread::hardware_concurrency();  // 8

    try
    {
      for(unsigned i=0;i<thread_count;++i)
      {
        threads.push_back(
          std::thread(&thread_pool::worker_thread,this));  // 9
      }
    }
    catch(...)
    {
      done=true;  // 10
      throw;
    }
  }

  ~thread_pool()
  {
    done=true;  // 11
  }

  template<typename FunctionType>
  void submit(FunctionType f)
  {
    work_queue.push(std::function<void()>(f));  // 12
  }
};
\end{cpp}

实现中有一组工作线程②,并且使用线程安全队列(见第6章)①来管理任务队列。这种情况下,用户不用等待任务,并且任务不需要返回任何值,所以可以使用\texttt{std::function<void()>}对任务进行封装。submit()会将函数或可调用对象包装成一个\texttt{std::function<void()>}实例,将其推入队列中⑫。

线程始于构造函数:使用\texttt{std::thread::hardware\_concurrency()}来获取硬件支持多少个并发线程⑧,这些线程会在worker\_thread()成员函数中执行⑨。

当有异常抛出时,线程启动就会失败,所以需要保证任何已启动的线程都能停止。有异常抛出时,使用*try-catch*来设置done标志⑩,还有join\_threads类的实例(来自于第8章)③来汇聚所有线程。当然也需要析构函数:仅设置done标志⑪,并且join\_threads确保所有线程在线程池销毁前全部执行完成。注意成员声明的顺序很重要:done标志和worker\_queue必须在threads数组之前声明,而数据必须在joiner前声明,这样就能确保成员以正确的顺序销毁。

worker\_thread函数很简单:从任务队列上获取任务⑤,以及同时执行这些任务⑥,执行一个循环直到设置done标志④。如果任务队列上没有任务,函数会调用\texttt{std::this\_thread::yield()}让线程休息⑦,并且给予其他线程向任务队列推送任务的机会。

这样简单的线程池就完成了,特别是任务没有返回值,或需要执行阻塞操作的任务。很多情况下,这样的线程池是不够用的,其他情况使用这样简单的线程池可能会出现问题,比如:死锁。同样,在简单例子中使用\texttt{std::async}能提供更好的功能(如第8章中的例子)。

本章将了解一下更加复杂的线程池实现,通过添加特性满足用户需求,或减少问题发生的几率。

\mySubsubsection{9.1.2}{等待线程池中的任务}

第8章中的例子中,线程间的任务划分完成后,代码会显式生成新线程,主线程通常是等待新线程在返回调用之后结束,确保所有任务都完成。使用线程池就需要等待任务提交到线程池中,而非直接提交给单个线程。与基于\texttt{std::async}的方法类似,使用代码9.1中的简单线程池,使用第4章中提到的工具:条件变量和future。虽然会增加代码的复杂度,不过要比直接对任务进行等待好很多。

通过增加线程池的复杂度,可以直接等待任务完成。使用submit()函数返回对任务描述的句柄,可用来等待任务的完成。任务句柄会用条件变量或future进行包装,从而简化线程池的实现。

一种特殊的情况是,执行任务的线程需要返回结果到主线程上进行处理。本这种情况下,需要用future对最终的结果进行转移。代码9.2展示了对简单线程池的修改,通过修改就能等待任务完成,以及在工作线程完成后,返回一个结果到等待线程中去,不过\texttt{std::packaged\_task<>}实例是不可拷贝的,仅可移动,所以不能再使用\texttt{std::function<>}来实现任务队列,因为\texttt{std::function<>}需要存储可复制构造的函数对象。包装一个自定义函数,用来处理可移动的类型,就是一个带有函数操作符的类型擦除类。只需要处理没有入参的函数和无返回的函数即可,所以这只是一个简单的虚函数调用。

代码9.2 可等待任务的线程池

\begin{cpp}
class function_wrapper
{
  struct impl_base {
    virtual void call()=0;
    virtual ~impl_base() {}
  };

  std::unique_ptr<impl_base> impl;
  template<typename F>
  struct impl_type: impl_base
  {
    F f;
    impl_type(F&& f_): f(std::move(f_)) {}
    void call() { f(); }
  };
public:
  template<typename F>
  function_wrapper(F&& f):
    impl(new impl_type<F>(std::move(f)))
  {}

  void operator()() { impl->call(); }

  function_wrapper() = default;

  function_wrapper(function_wrapper&& other):
    impl(std::move(other.impl))
  {}

  function_wrapper& operator=(function_wrapper&& other)
  {
    impl=std::move(other.impl);
    return *this;
  }

  function_wrapper(const function_wrapper&)=delete;
  function_wrapper(function_wrapper&)=delete;
  function_wrapper& operator=(const function_wrapper&)=delete;
};

class thread_pool
{
  thread_safe_queue<function_wrapper> work_queue;  // 使用function_wrapper,而非使用std::function

  void worker_thread()
  {
    while(!done)
    {
      function_wrapper task;
      if(work_queue.try_pop(task))
      {
        task();
      }
      else
      {
        std::this_thread::yield();
      }
    }
  }
public:
  template<typename FunctionType>
  std::future<typename std::result_of<FunctionType()>::type>  // 1
    submit(FunctionType f)
  {
    typedef typename std::result_of<FunctionType()>::type
      result_type;  // 2

    std::packaged_task<result_type()> task(std::move(f));  // 3
    std::future<result_type> res(task.get_future());  // 4
    work_queue.push(std::move(task));  // 5
    return res;  // 6
  }
  // 和之前一样
};
\end{cpp}

首先,修改的是submit()函数①,返回\texttt{std::future<>}保存任务的返回值,并且允许调用者等待任务完全结束。因为需要提供函数f的返回类型,所以使用\texttt{std::result\_of<>}:\texttt{std::result\_of<FunctionType()>::type}是FunctionType类型的引用实例(如f),并且没有参数。同样,函数中可以对result\_type typedef②使用\texttt{std::result\_of<>}。

然后,将f包装入\texttt{std::packaged\_task<result\_type()>}③,因为f是一个无参数的函数或是可调用对象,能够返回result\_type类型的实例。向任务队列推送任务⑤和返回future⑥前,就可以从\texttt{std::packaged\_task<>}中获取future④。注意,要将任务推送到任务队列中时,只能使用\texttt{std::move()},因为\texttt{std::packaged\_task<>}不可拷贝。为了对任务进行处理,队列里面存的就是function\_wrapper对象,而非\texttt{std::function<void()>}对象。

现在线程池允许等待任务,并且返回任务后的结果。下面的代码就展示了,如何让parallel\_accumuate函数使用线程池。

代码9.3 parallel\_accumulate使用可等待任务的线程池

\begin{cpp}
template<typename Iterator,typename T>
T parallel_accumulate(Iterator first,Iterator last,T init)
{
  unsigned long const length=std::distance(first,last);

  if(!length)
    return init;

  unsigned long const block_size=25;
  unsigned long const num_blocks=(length+block_size-1)/block_size;  // 1

  std::vector<std::future<T> > futures(num_blocks-1);
  thread_pool pool;

  Iterator block_start=first;
  for(unsigned long i=0;i<(num_blocks-1);++i)
  {
    Iterator block_end=block_start;
    std::advance(block_end,block_size);
    futures[i]=pool.submit([=]{
      accumulate_block<Iterator,T>()(block_start,block_end);
    }); // 2
    block_start=block_end;
  }
  T last_result=accumulate_block<Iterator,T>()(block_start,last);
  T result=init;
  for(unsigned long i=0;i<(num_blocks-1);++i)
  {
    result+=futures[i].get();
  }
  result += last_result;
  return result;
}
\end{cpp}

与代码8.4相比,有几个点需要注意。首先,工作量是依据使用的块数(num\_blocks①),而不是线程的数量。为了利用线程池的最大化可扩展性,需要将工作块划分为最小工作块。当线程池中线程不多时,每个线程将会处理多个工作块,不过随着硬件可用线程数量的增长,会有越来越多的工作块并发执行。

当有“测试并发执行最小工作块”的想法时,就需要谨慎了。向线程池提交任务有一定的开销,让工作线程执行这个任务,并且将返回值保存在\texttt{std::future<>}中,对于太小的任务,这样的开销不划算。如果任务块太小,使用线程池的速度可能都不及单线程。

如果任务块的大小合理,就不用担心了:打包任务、获取future或存储之后要汇入的\texttt{std::thread}对象。使用线程池的时候,这些都要注意。之后,就是调用submit()来提交任务②。

线程池也需要注意异常安全。任何异常都会通过submit()返回给future,并在获取future结果时抛出异常。如果函数因为异常退出,线程池的析构函数会丢掉那些没有完成的任务,等待线程池中的工作线程完成工作。

这个线程池还不错,因为任务都是相互独立的。不过,当任务队列中的任务有依赖关系时,就会遇到麻烦了。

\mySubsubsection{9.1.3}{等待依赖任务}

以快速排序算法为例:数据与中轴数据项比较,在中轴项两侧分为大于和小于的两个序列,然后再对这两组序列进行排序。这两组序列会递归排序,最后会整合成一个全排序序列。要将这个算法写成并发模式,需要保证递归调用能够使用硬件的并发能力。

回到第4章,第一次接触这个例子,使用\texttt{std::async}来执行每一层的调用,让标准库来选择,是在新线程上执行这个任务,还是当对应get()调用时进行同步执行。运行起来很不错,因为每一个任务都在其自己的线程上执行,或需要的时候进行调用。

回顾第8章,使用一个固定线程数量(根据硬件可用并发线程数)的结构体。这样的情况下,使用栈来挂起需要排序的数据块。每个线程在数据块排序前,会向数据栈上添加一组要排序的数据,然后对当前数据块排序结束后,接着对另一块进行排序。这会消耗有限的线程,所以等待其他线程完成排序可能会造成死锁。一种情况很可能会出现,就是所有线程都在等某一个数据块进行排序,不过没有线程在做这块数据的排序。可以通过拉取栈上数据块的线程,对数据块进行排序,来解决这个问题。

如果只用简单的线程池进行替换,例如:第4章替换\texttt{std::async}的线程池。只有固定数量的线程,因为线程池中没有空闲的线程,线程会等待任务。因此,需要有和第8章中类似的解决方案:当等待某个数据块完成时,去处理未完成的数据块。如果使用线程池来管理任务列表和相关线程,就不用再去访问任务列表了。可以对线程池做一些改动,自动完成这些事情。

最简单的方法就是在thread\_pool中添加一个新函数,来执行任务队列上的任务,并对线程池进行管理。高级线程池的实现可能会在等待函数中添加逻辑,或等待其他函数来处理这个任务,优先的任务会让其他的任务进行等待。下面代码中的实现,就展示了一个新run\_pending\_task()函数,对于快速排序的修改将会在代码9.5中展示。

代码9.4 run\_pending\_task()函数实现

\begin{cpp}
void thread_pool::run_pending_task()
{
  function_wrapper task;
  if(work_queue.try_pop(task))
  {
    task();
  }
  else
  {
    std::this_thread::yield();
  }
}
\end{cpp}

run\_pending\_task()的实现去掉了在worker\_thread()函数的主循环。函数任务队列中有任务的时候执行任务,没有的话就会让操作系统对线程进行重新分配。下面快速排序算法的实现要比代码8.1中版本简单许多,因为所有线程管理逻辑都移到线程池中了。

代码9.5 基于线程池的快速排序实现

\begin{cpp}
template<typename T>
struct sorter  // 1
{
  thread_pool pool;  // 2

  std::list<T> do_sort(std::list<T>& chunk_data)
  {
    if(chunk_data.empty())
    {
      return chunk_data;
    }

    std::list<T> result;
    result.splice(result.begin(),chunk_data,chunk_data.begin());
    T const& partition_val=*result.begin();

    typename std::list<T>::iterator divide_point=
      std::partition(chunk_data.begin(),chunk_data.end(),
                     [&](T const& val){return val<partition_val;});

    std::list<T> new_lower_chunk;
    new_lower_chunk.splice(new_lower_chunk.end(),
                           chunk_data,chunk_data.begin(),
                           divide_point);

    std::future<std::list<T> > new_lower=  // 3
      pool.submit(std::bind(&sorter::do_sort,this,
                            std::move(new_lower_chunk)));

    std::list<T> new_higher(do_sort(chunk_data));

    result.splice(result.end(),new_higher);
    while(!new_lower.wait_for(std::chrono::seconds(0)) ==
      std::future_status::timeout)
    {
      pool.run_pending_task();  // 4
    }

    result.splice(result.begin(),new_lower.get());
    return result;
  }
};

template<typename T>
std::list<T> parallel_quick_sort(std::list<T> input)
{
  if(input.empty())
  {
    return input;
  }
  sorter<T> s;

  return s.do_sort(input);
}
\end{cpp}

与代码8.1相比,这里将实际工作放在sorter类模板的do\_sort()成员函数中执行①,即使例子中仅对thread\_pool实例进行包装②。

线程和任务管理在线程等待的时候,就会少向线程池中提交一个任务③,执行任务队列上未完成的任务④,需要显式的管理线程和栈上要排序的数据块。当有任务提交到线程池中,可以使用\texttt{std::bind()}绑定this指针到do\_sort()上,绑定是为了让数据块进行排序。这种情况下,需要对new\_lower\_chunk使用\texttt{std::move()}将其传入函数,数据移动要比拷贝的开销少。

虽然,使用等待其他任务的方式解决了死锁,但这个线程池距离理想的线程池还差很远。

首先,每次对submit()的调用和对run\_pending\_task()的调用,访问的都是同一个队列。第8章中,当多线程去修改一组数据,就会对性能有所影响,所以这也是个问题。

\mySubsubsection{9.1.4}{避免队列中的任务竞争}

线程每次调用线程池的submit(),都会推送一个任务到工作队列中。就像工作线程为了执行任务,从任务队列中获取任务一样。随着处理器的增加,任务队列上就会有很多的竞争,会让性能下降。使用无锁队列会让任务没有明显的等待,但乒乓缓存会消耗大量的时间。

为了避免乒乓缓存,每个线程建立独立的任务队列。这样,每个线程就会将新任务放在自己的任务队列上,并且当线程上的任务队列没有任务时,去全局的任务列表中取任务。下面列表中的实现,使用了一个thread\_local变量,来保证每个线程都拥有自己的任务列表(如全局列表那样)。

代码9.6 线程池——线程具有本地任务队列

\begin{cpp}
class thread_pool
{
  thread_safe_queue<function_wrapper> pool_work_queue;

  typedef std::queue<function_wrapper> local_queue_type;  // 1
  static thread_local std::unique_ptr<local_queue_type>
    local_work_queue;  // 2

  void worker_thread()
  {
    local_work_queue.reset(new local_queue_type);  // 3
    while(!done)
    {
      run_pending_task();
    }
  }

public:
  template<typename FunctionType>
  std::future<typename std::result_of<FunctionType()>::type>
    submit(FunctionType f)
  {
    typedef typename std::result_of<FunctionType()>::type result_type;

    std::packaged_task<result_type()> task(f);
    std::future<result_type> res(task.get_future());
    if(local_work_queue)  // 4
    {
      local_work_queue->push(std::move(task));
    }
    else
    {
      pool_work_queue.push(std::move(task));  // 5
    }
    return res;
  }

  void run_pending_task()
  {
    function_wrapper task;
    if(local_work_queue && !local_work_queue->empty())  // 6
    {
      task=std::move(local_work_queue->front());
      local_work_queue->pop();
      task();
    }
    else if(pool_work_queue.try_pop(task))  // 7
    {
      task();
    }
    else
    {
      std::this_thread::yield();
    }
  }
// rest as before
};
\end{cpp}

因为不希望非线程池中的线程也拥有一个任务队列,使用\texttt{std::unique\_ptr<>}指向线程本地的工作队列②,这个指针在worker\_thread()中进行初始化③。\texttt{std:unique\_ptr<>}的析构函数会保证在线程退出时销毁队列。

submit()会检查当前线程是否具有一个工作队列④。如果有,就是线程池中的线程,可以将任务放入线程的本地队列中;否则,将这个任务放在线程池中的全局队列中⑤。

run\_pending\_task()⑥中的检查和之前类似,可以对是否存在本地任务队列进行检查。如果存在,就会从队列中的第一个任务开始处理。注意本地任务队列可以是一个普通的\texttt{std::queue<>}①,因为这个队列只能被一个线程所访问,就不存在竞争。如果本地线程上没有任务,就会从全局工作列表上获取任务⑦。

这样就能有效的避免竞争,不过当任务分配不均时,造成的结果就是:某个线程本地队列中有很多任务的同时,其他线程无所事事。例如:举一个快速排序的例子,一开始的数据块能在线程池上被处理,因为剩余部分会放在工作线程的本地队列上进行处理,这样的使用方式也违背使用线程池的初衷。

幸好这个问题有解:本地工作队列和全局工作队列上没有任务时,可从别的线程队列中窃取任务。

\mySubsubsection{9.1.5}{窃取任务}

为了让没有任务的线程从其他线程的任务队列中获取任务,就需要本地任务列表可以被其他线程访问,这样才能让run\_pending\_tasks()窃取任务。需要每个线程在线程池队列上进行注册,或由线程池指定一个线程。同样,还需要保证数据队列中的任务适当的被同步和保护,这样队列的不变量就不会被破坏。

实现一个无锁队列,让其线程在其他线程上窃取任务时,有推送和弹出一个任务的可能。不过,这个队列的实现超出了本书的讨论范围。为了证明这种方法的可行性,将使用一个互斥量来保护队列中的数据。我们希望任务窃取是不常见的现象,这样就会减少对互斥量的竞争,并且使得简单队列的开销最小。下面,实现了一个简单的基于锁的任务窃取队列。

代码9.7 基于锁的任务窃取队列

\begin{cpp}
class work_stealing_queue
{
private:
  typedef function_wrapper data_type;
  std::deque<data_type> the_queue;  // 1
  mutable std::mutex the_mutex;

public:
  work_stealing_queue()
  {}

  work_stealing_queue(const work_stealing_queue& other)=delete;
  work_stealing_queue& operator=(
    const work_stealing_queue& other)=delete;

  void push(data_type data)  // 2
  {
    std::lock_guard<std::mutex> lock(the_mutex);
    the_queue.push_front(std::move(data));
  }

  bool empty() const
  {
    std::lock_guard<std::mutex> lock(the_mutex);
    return the_queue.empty();
  }

  bool try_pop(data_type& res)  // 3
  {
    std::lock_guard<std::mutex> lock(the_mutex);
    if(the_queue.empty())
    {
      return false;
    }

    res=std::move(the_queue.front());
    the_queue.pop_front();
    return true;
  }

  bool try_steal(data_type& res)  // 4
  {
    std::lock_guard<std::mutex> lock(the_mutex);
    if(the_queue.empty())
    {
      return false;
    }

    res=std::move(the_queue.back());
    the_queue.pop_back();
    return true;
  }
};
\end{cpp}

这个队列对\texttt{std::deque<fuction\_wrapper>}进行了简单的包装①,能通过一个互斥锁来对所有访问进行控制了。push()②和try\_pop()③对队列的前端进行操作,try\_steal()④对队列的后端进行操作。

这就说明每个线程中的“队列”是一个后进先出的栈,最新推入的任务将会第一个执行。从缓存角度来看,这将对性能有所提升,因为任务相关的数据一直存于缓存中,要比提前将任务相关数据推送到栈上好。同样,这种方式很好的映射到某个算法上,例如:快速排序。之前的实现中,每次调用do\_sort()都会推送一个任务到栈上,并且等待这个任务执行完毕。通过对最新推入任务的处理,就可以保证在将当前所需数据块处理完成前,其他任务是否需要这些数据块,从而可以减少活动任务的数量和栈的使用次数。try\_steal()从队列末尾获取任务,为了减少与try\_pop()之间的竞争。使用在第6、7章中的所讨论的技术来让try\_pop()和try\_steal()并发执行。

现在拥有了一个很不错的任务队列,并且支持窃取。那如何在线程池中使用这个队列呢?这里简单的展示一下。

代码9.8 使用任务窃取的线程池

\begin{cpp}
class thread_pool
{
  typedef function_wrapper task_type;

  std::atomic_bool done;
  thread_safe_queue<task_type> pool_work_queue;
  std::vector<std::unique_ptr<work_stealing_queue> > queues;  // 1
  std::vector<std::thread> threads;
  join_threads joiner;

  static thread_local work_stealing_queue* local_work_queue;  // 2
  static thread_local unsigned my_index;

  void worker_thread(unsigned my_index_)
  {
    my_index=my_index_;
    local_work_queue=queues[my_index].get();  // 3
    while(!done)
    {
      run_pending_task();
    }
  }

  bool pop_task_from_local_queue(task_type& task)
  {
    return local_work_queue && local_work_queue->try_pop(task);
  }

  bool pop_task_from_pool_queue(task_type& task)
  {
    return pool_work_queue.try_pop(task);
  }

  bool pop_task_from_other_thread_queue(task_type& task)  // 4
  {
    for(unsigned i=0;i<queues.size();++i)
    {
      unsigned const index=(my_index+i+1)%queues.size();  // 5
      if(queues[index]->try_steal(task))
      {
        return true;
      }
    }
    return false;
  }

public:
  thread_pool():
    done(false),joiner(threads)
  {
    unsigned const thread_count=std::thread::hardware_concurrency();

    try
    {
      for(unsigned i=0;i<thread_count;++i)
      {
        queues.push_back(std::unique_ptr<work_stealing_queue>(  // 6
                         new work_stealing_queue));
        threads.push_back(
          std::thread(&thread_pool::worker_thread,this,i));
      }
    }
    catch(...)
    {
      done=true;
      throw;
    }
  }

  ~thread_pool()
  {
    done=true;
  }

  template<typename FunctionType>
  std::future<typename std::result_of<FunctionType()>::type> submit(
    FunctionType f)
  {
    typedef typename std::result_of<FunctionType()>::type result_type;
    std::packaged_task<result_type()> task(f);
    std::future<result_type> res(task.get_future());
    if(local_work_queue)
    {
      local_work_queue->push(std::move(task));
    }
    else
    {
      pool_work_queue.push(std::move(task));
    }
    return res;
  }

  void run_pending_task()
  {
    task_type task;
    if(pop_task_from_local_queue(task) ||  // 7
       pop_task_from_pool_queue(task) ||  // 8
       pop_task_from_other_thread_queue(task))  // 9
    {
      task();
    }
    else
    {
      std::this_thread::yield();
    }
  }
};
\end{cpp}

这段代码与代码9.6很相似。第一个不同在于,每个线程都有一个work\_stealing\_queue,而非只是普通的\texttt{std::queue<>}②。每个线程有一个属于自己的工作队列⑥,每个线程自己的工作队列将存储在线程池的全局工作队列中①。列表中队列的序号,会传递给线程函数,然后使用序号来索引队列③。为了能让闲置的线程窃取任务,线程池可以访问任意线程中的队列。run\_pending\_task()将会从线程的任务队列中取出一个任务来执行⑦,或从线程池队列中获取一个任务⑧,亦或从其他线程的队列中获取一个任务⑨。

pop\_task\_from\_other\_thread\_queue()④会遍历池中所有线程的任务队列,然后尝试窃取任务。为了避免每个线程都尝试从列表中的第一个线程上窃取任务,每一个线程都会从下一个线程开始遍历,通过自身的线程序号来确定开始遍历的线程序号。

使用线程池有很多好处,还有很多的方式能为某些特殊用法提升性能,不过这就当做留给读者的作业吧。特别是还没有探究动态变换大小的线程池,即使线程阻塞时(例如:I/O或互斥锁),程序都能保证CPU最优的使用率。

下面,我们来了解一下线程管理的高级用法——中断线程。