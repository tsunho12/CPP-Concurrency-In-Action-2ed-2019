% # 9.2 中断线程

很多情况下,使用信号来终止长时间运行的线程是合理的。这种线程的存在,可能是因为工作线程所在的线程池销毁,或是用户显式的取消了这个任务。不管是什么原因,原理都一样:需要使用信号来让未结束线程停止运行。这需要一种合适的方式让线程主动的停下来,而非戛然而止。

可能会给每种情况制定一个机制,但这样的意义不大。不仅因为用统一的机制会更容易在之后的场景中实现,而且写出来的中断代码不用担心在哪里使用。C++11标准没有提供这样的机制(草案上有积极的建议,说不定中断线程会在以后的C++标准中添加\footnote[1]{\url{P0660: A Cooperatively Interruptible Joining Thread, Rev 3, Nicolai Josuttis, Herb Sutter, Anthony Williams http://www.open-std.org/jtc1/sc22/wg21/docs/papers/2018/p0660r3.pdf.}}),不过实现这样的机制也并不困难。

了解一下应该如何实现这种机制前,先来了解一下启动和中断线程的接口。

\mySubsubsection{9.2.1}{启动和中断线程}

先看一下外部接口,需要从可中断线程上获取些什么?最起码需要和\texttt{std::thread}相同的接口,还要多加一个interrupt()函数:

\begin{cpp}
class interruptible_thread
{
public:
  template<typename FunctionType>
  interruptible_thread(FunctionType f);
  void join();
  void detach();
  bool joinable() const;
  void interrupt();
};
\end{cpp}

类内部可以使用\texttt{std::thread}来管理线程,并且使用一些自定义数据结构来处理中断。现在,从线程的角度能看到什么呢?“能用这个类来中断线程”——需要一个断点(*interruption point*)。在不添加多余的数据的前提下,为了使断点能够正常使用,就需要使用一个没有参数的函数:interruption\_point()。这意味着中断数据结构可以访问thread\_local变量,并在线程运行时,对变量进行设置,因此当线程调用interruption\_point()时,就会去检查当前运行线程的数据结构。

thread\_local标志是不能使用普通的\texttt{std::thread}管理线程的主要原因,需要使用一种方法分配出一个可访问的interruptible\_thread实例,就像新启动一个线程一样。使用已提供函数来做这件事情前,需要将interruptible\_thread实例传递给\texttt{std::thread}的构造函数,创建一个能够执行的线程,就像下面的代码所实现。

代码9.9 interruptible\_thread的基本实现

\begin{cpp}
class interrupt_flag
{
public:
  void set();
  bool is_set() const;
};
thread_local interrupt_flag this_thread_interrupt_flag;  // 1

class interruptible_thread
{
  std::thread internal_thread;
  interrupt_flag* flag;
public:
  template<typename FunctionType>
  interruptible_thread(FunctionType f)
  {
    std::promise<interrupt_flag*> p;  // 2
    internal_thread=std::thread([f,&p]{  // 3
      p.set_value(&this_thread_interrupt_flag);
      f();  // 4
    });
    flag=p.get_future().get();  // 5
  }
  void interrupt()
  {
    if(flag)
    {
      flag->set();  // 6
    }
  }
};
\end{cpp}

提供函数f是包装了一个Lambda函数③,线程将会持有f副本和本地promise变量(p)的引用②。新线程中,Lambda函数设置承诺值变量的值到this\_thread\_interrupt\_flag(在thread\_local①中声明)的地址中,为的是让线程能够调用提供函数的副本④。调用线程会等待与其future相关的promise处于就绪态,并且将结果存入到flag成员变量中⑤。

注意,即使Lambda函数在新线程上执行,对本地变量p进行悬空引用都没有问题,在新线程返回之前,interruptible\_thread构造函数会等待变量p,直到变量p不被引用。实现没有考虑汇入或分离线程,所以需要flag变量在线程退出或分离前声明,避免悬空。

interrupt()函数相对简单:线程去做中断时,需要合法指针作为中断标志,所以可以对标志进行设置⑥。

\mySubsubsection{9.2.2}{检查线程是否中断}

现在就可以设置中断标志了,当不检查线程是否中断时,意义就不大了。如果标志已设置,就可以抛出thread\_interrupted异常:

\begin{cpp}
void interruption_point()
{
  if(this_thread_interrupt_flag.is_set())
  {
    throw thread_interrupted();
  }
}
\end{cpp}

代码中可以在适当的地方使用该函数:

\begin{cpp}
void foo()
{
  while(!done)
  {
    interruption_point();
    process_next_item();
  }
}
\end{cpp}

虽然也能工作,但不理想。最好是在线程等待或阻塞的时候中断线程,因为这时的线程不能运行,也就不能调用interruption\_point()函数!线程等待时,什么方式才能去中断线程呢?

\mySubsubsection{9.2.3}{中断等待——条件变量}

仔细选择中断的位置,并通过显式调用interruption\_point()进行中断,不过线程阻塞等待时,这种办法就显得苍白无力了,例如:等待条件变量的通知。就需要一个新函数——interruptible\_wait()——就可以运行各种需要等待的任务,并且可以知道如何中断等待。之前提到,可能会等待一个条件变量,所以就从它开始:如何做才能中断一个等待的条件变量呢?最简单的方式,当设置中断标志时需要提醒条件变量,并在等待后立即设置断点。为了让其工作,需要提醒所有等待对应条件变量的线程,就能确保相应的线程能够唤醒。interrupt\_flag需要存储一个指针指向条件变量,所以用set()对其进行提醒。为条件变量实现的interruptible\_wait()可能会看起来会像下面代码中所示。

代码9.10 为\texttt{std::condition\_variable}实现的interruptible\_wait(有问题版)

\begin{cpp}
void interruptible_wait(std::condition_variable& cv,
std::unique_lock<std::mutex>& lk)
{
  interruption_point();
  this_thread_interrupt_flag.set_condition_variable(cv);  // 1
  cv.wait(lk);  // 2
  this_thread_interrupt_flag.clear_condition_variable();  // 3
  interruption_point();
}
\end{cpp}

假设函数能够设置和清除相关条件变量上的中断标志,代码会检查中断,通过interrupt\_flag为当前线程关联条件变量①,等待条件变量②,清理相关条件变量③,并且再次检查中断。如果线程在等待期间因条件变量所中断,中断线程将广播条件变量,并唤醒等待该条件变量的线程检查中断。不幸的是,代码有两个问题。第一个问题比较明显,如果想要线程安全:\texttt{std::condition\_variable::wait()}可以抛出异常,所以这里会直接退出,而没有通过条件变量删除相关的中断标志。这个问题很容易修复,就是在析构函数中添加删除操作即可。

第二个问题是,这段代码存在条件竞争。虽然,线程可以通过调用interruption\_point()中断,不过在调用wait()后,条件变量和相关中断标志就没有什么关系了。因为线程不是等待状态,所以不能通过条件变量的方式唤醒。就需要确保线程不会在最后一次中断检查和调用wait()间唤醒。这里不对\texttt{std::condition\_variable}的内部结构进行研究,但可通过一种方法来解决这个问题:使用lk上的互斥量对线程进行保护,这就需要将lk传递到set\_condition\_variable()函数中去。不幸的是,这将产生两个新问题:需要传递一个互斥量的引用到一个不知道生命周期的线程中(这个线程做中断操作),为该线程上锁(调用interrupt()的时候)。这可能会产生死锁,并且可能访问到已经销毁的互斥量,所以这种方法不可取。

当不能完全确定能中断条件变量等待——没有interruptible\_wait()情况下也可以时(可能有些严格),有没有其他选择呢?一个选择就是使用超时等待,使用wait\_for()并带有一个简单的超时量(比如,1ms)。线程被中断前,算是给了线程一个等待的上限(以时钟刻度为基准)。超时也不能帮助到我们,如果这样做了,等待线程将会看到更多因为超时而“伪”唤醒的线程。与interrupt\_flag相关的实现的一个实现放在下面的代码中展示。

代码9.11 为\texttt{std::condition\_variable}在interruptible\_wait中使用超时

\begin{cpp}
class interrupt_flag
{
  std::atomic<bool> flag;
  std::condition_variable* thread_cond;
  std::mutex set_clear_mutex;

public:
  interrupt_flag():
    thread_cond(0)
  {}

  void set()
  {
    flag.store(true,std::memory_order_relaxed);
    std::lock_guard<std::mutex> lk(set_clear_mutex);
    if(thread_cond)
    {
      thread_cond->notify_all();
    }
  }

  bool is_set() const
  {
    return flag.load(std::memory_order_relaxed);
  }

  void set_condition_variable(std::condition_variable& cv)
  {
    std::lock_guard<std::mutex> lk(set_clear_mutex);
    thread_cond=&cv;
  }

  void clear_condition_variable()
  {
    std::lock_guard<std::mutex> lk(set_clear_mutex);
    thread_cond=0;
  }

  struct clear_cv_on_destruct
  {
    ~clear_cv_on_destruct()
    {
      this_thread_interrupt_flag.clear_condition_variable();
    }
  };
};

void interruptible_wait(std::condition_variable& cv,
  std::unique_lock<std::mutex>& lk)
{
  interruption_point();
  this_thread_interrupt_flag.set_condition_variable(cv);
  interrupt_flag::clear_cv_on_destruct guard;
  interruption_point();
  cv.wait_for(lk,std::chrono::milliseconds(1));
  interruption_point();
}
\end{cpp}

如果有谓词(相关函数)进行等待,1ms的超时将会在谓词循环中完全隐藏:

\begin{cpp}
template<typename Predicate>
void interruptible_wait(std::condition_variable& cv,
                        std::unique_lock<std::mutex>& lk,
                        Predicate pred)
{
  interruption_point();
  this_thread_interrupt_flag.set_condition_variable(cv);
  interrupt_flag::clear_cv_on_destruct guard;
  while(!this_thread_interrupt_flag.is_set() && !pred())
  {
    cv.wait_for(lk,std::chrono::milliseconds(1));
  }
  interruption_point();
}
\end{cpp}

这会让谓词检查的次数增加许多,不过对于简单的wait()实现还是很好用的。超时变量很容易实现:通过指定时间,比如:1ms或更短。对于\texttt{std::condition\_variable}的等待,就需要小心应对了;\texttt{std::condition\_variable\_any}呢?还是能做的更好吗?

\mySubsubsection{9.2.4}{使用\texttt{std::condition\_variable\_any}中断等待}

\texttt{std::condition\_variable\_any}与\texttt{std::condition\_variable}的不同在于,\texttt{std::condition\_variable\_any}可以使用任意类型的锁,而不仅有\texttt{std::unique\_lock<std::mutex>}。可以让事情做起来更加简单,并且\texttt{std::condition\_variable\_any}可以比\texttt{std::condition\_variable}做的更好。因为能与任意类型的锁一起工作,就可以设计自己的锁,上锁/解锁interrupt\_flag的内部互斥量set\_clear\_mutex,并且锁也支持等待调用,就像下面的代码。

代码9.12 为\texttt{std::condition\_variable\_any}设计的interruptible\_wait

\begin{cpp}
class interrupt_flag
{
  std::atomic<bool> flag;
  std::condition_variable* thread_cond;
  std::condition_variable_any* thread_cond_any;
  std::mutex set_clear_mutex;

public:
  interrupt_flag():
    thread_cond(0),thread_cond_any(0)
  {}

  void set()
  {
    flag.store(true,std::memory_order_relaxed);
    std::lock_guard<std::mutex> lk(set_clear_mutex);
    if(thread_cond)
    {
      thread_cond->notify_all();
    }
    else if(thread_cond_any)
    {
      thread_cond_any->notify_all();
    }
  }

  template<typename Lockable>
  void wait(std::condition_variable_any& cv,Lockable& lk)
  {
    struct custom_lock
    {
      interrupt_flag* self;
      Lockable& lk;

      custom_lock(interrupt_flag* self_,
                  std::condition_variable_any& cond,
                  Lockable& lk_):
        self(self_),lk(lk_)
      {
        self->set_clear_mutex.lock();  // 1
        self->thread_cond_any=&cond;  // 2
      }

      void unlock()  // 3
      {
        lk.unlock();
        self->set_clear_mutex.unlock();
      }

      void lock()
      {
        std::lock(self->set_clear_mutex,lk);  // 4
      }

      ~custom_lock()
      {
        self->thread_cond_any=0;  // 5
        self->set_clear_mutex.unlock();
      }
    };
    custom_lock cl(this,cv,lk);
    interruption_point();
    cv.wait(cl);
    interruption_point();
  }
  // rest as before
};

template<typename Lockable>
void interruptible_wait(std::condition_variable_any& cv,
                        Lockable& lk)
{
  this_thread_interrupt_flag.wait(cv,lk);
}
\end{cpp}

自定义的锁类型在构造时,需要所锁住内部set\_clear\_mutex①,对thread\_cond\_any指针进行设置,并引用\texttt{std::condition\_variable\_any}传入锁的构造函数中②。锁的引用将会在之后进行存储,其变量必须锁住。现在可以安心的检查中断,不用担心竞争了。如果中断标志已经设置,标志应该是在锁住set\_clear\_mutex时设置的。当条件变量调用自定义锁的unlock(),就会对可锁对象和set\_clear\_mutex进行解锁③,这就允许线程可以尝试中断其他线程获取set\_clear\_mutex锁。以及在内部wait()调用之后,检查thread\_cond\_any指针。这就是在替换\texttt{std::condition\_variable}后,所拥有的功能(不包括管理)。当wait()结束等待(因为等待,或因为伪唤醒),因为线程会调用lock()函数,所以依旧要求锁住内部set\_clear\_mutex④。wait()调用时,custom\_lock的析构函数中⑤清理thread\_cond\_any指针(同样会解锁set\_clear\_mutex)之前,可以再次对中断进行检查。

\mySubsubsection{9.2.5}{中断其他阻塞调用}

这次轮到中断条件变量的等待了,不过其他阻塞情况,比如:互斥锁,等待future等等,该怎么处理呢?通常情况下,可以使用\texttt{std::condition\_variable}的超时选项,因为实际运行中不可能很快的将条件变量的等待终止(不访问内部互斥量或future的话)。不过,某些情况下知道在等待什么,就可以让循环在interruptible\_wait()函数中运行。作为一个例子,\texttt{std::future<>}重载了interruptible\_wait()的实现:

\begin{cpp}
template<typename T>
void interruptible_wait(std::future<T>& uf)
{
  while(!this_thread_interrupt_flag.is_set())
  {
    if(uf.wait_for(lk,std::chrono::milliseconds(1))==
       std::future_status::ready)
      break;
  }
  interruption_point();
}
\end{cpp}

等待会在中断标志设置好的时候,或future准备就绪时停止,不过实现中每次等待future的时间只有1ms。中断请求确定前,平均等待的时间为0.5ms(这里假设存在一个高精度的时钟)。通常wait\_for至少会等待一个时钟周期,如果时钟周期为15ms,结束等待的时间将会是15ms,而不是1ms。接受与不接受这种情况,得视情况而定。如果时钟支持的话,可以削减超时时间。这种方式将会将线程唤醒很多次来检查标志,增加线程切换的开销。

我们已经了解如何使用interruption\_point()和interruptible\_wait()函数检查中断。当中断被检查出来了,要如何处理它呢?

\mySubsubsection{9.2.6}{处理中断}

从中断线程的角度看,中断就是thread\_interrupted异常,因此能像处理其他异常那样进行处理。特别是使用标准catch块对其进行捕获:

\begin{cpp}
try
{
  do_something();
}
catch(thread_interrupted&)
{
  handle_interruption();
}
\end{cpp}

捕获中断进行处理,其他线程再次调用interrupt()时,线程将会再次被中断,这就被称为*断点*(interruption point)。如果线程执行的是一系列独立的任务,就会需要断点。中断一个任务,就意味着丢弃这个任务,并且该线程会执行任务列表中的其他任务。

因为thread\_interrupted是一个异常,在能够被中断的代码中,之前线程安全的注意事项都是适用的,为了确保资源不会泄露,并在数据结构中留下对应的退出状态。通常,线程中断是可行的,所以只需要让异常传播即可。不过,当异常传入\texttt{std::thread}的析构函数时,将会调用\texttt{std::terminate()},并且整个程序将会终止。为了避免这种情况,需要将interruptible\_thread变量作为参数传入的函数中放置catch(thread\_interrupted)处理块,可以将catch块包装进interrupt\_flag的初始化过程中。因为异常将会终止独立进程,这样就能保证未处理的中断是异常安全的。interruptible\_thread构造函数中对线程的初始化,实现如下:

\begin{cpp}
internal_thread=std::thread([f,&p]{
        p.set_value(&this_thread_interrupt_flag);

        try
        {
          f();
        }
        catch(thread_interrupted const&)
        {}
      });
\end{cpp}

下面,我们来看个更加复杂的例子。

\mySubsubsection{9.2.7}{退出时中断后台任务}

试想在桌面上查找一个应用。这就需要与用户互动,应用的状态需要能在显示器上显示,就能看出应用有什么改变。为了避免影响GUI的响应时间,通常会将处理线程放在后台运行。后台进程需要一直执行,直到应用退出。后台线程会作为应用启动的一部分被启动,并且在应用终止的时候停止运行。通常这样的应用只有在机器关闭时才会退出,因为应用需要更新应用最新的状态,就需要全时间运行。在某些情况下,当应用关闭,需要使用有序的方式将后台线程关闭,其中一种方式就是中断。

下面代码中为一个系统实现了简单的线程管理部分。

代码9.13 后台监视文件系统

\begin{cpp}
std::mutex config_mutex;
std::vector<interruptible_thread> background_threads;

void background_thread(int disk_id)
{
  while(true)
  {
    interruption_point();  // 1
    fs_change fsc=get_fs_changes(disk_id);  // 2
    if(fsc.has_changes())
    {
      update_index(fsc);  // 3
    }
  }
}

void start_background_processing()
{
  background_threads.push_back(
    interruptible_thread(background_thread,disk_1));
  background_threads.push_back(
    interruptible_thread(background_thread,disk_2));
}

int main()
{
  start_background_processing();  // 4
  process_gui_until_exit();  // 5
  std::unique_lock<std::mutex> lk(config_mutex);
  for(unsigned i=0;i<background_threads.size();++i)
  {
    background_threads[i].interrupt();  // 6
  }
  for(unsigned i=0;i<background_threads.size();++i)
  {
    background_threads[i].join(); // 7
  }
}
\end{cpp}

启动时,后台线程就已经启动④。之后,对应线程将会处理GUI⑤。用户要求进程退出时,后台进程将会被中断⑥,并且主线程会等待每一个后台线程结束后才退出⑦。后台线程运行在一个循环中,并时刻检查磁盘的变化②,对其序号进行更新③。调用interruption\_point()函数①,可以在循环中对中断进行检查。

为什么中断线程前,会对线程进行等待?为什么不中断每个线程,让它们执行下一个任务?答案就是“并发”。线程中断后,不会马上结束,因为需要对下一个断点进行处理,并且在退出前执行析构函数和异常处理部分。因为需要汇入每个线程,所以就会让中断线程等待,即使线程还在做着有用的工作——中断其他线程。只有当没有工作时(所有线程都被中断)不需要等待,这就允许中断线程并行的处理自己的中断,并更快的完成中断。

中断机制很容易扩展到更深层次的中断调用,或在特定的代码块中禁用中断,这些内容就当做留给读者的作业吧。

% -----

% [1] P0660: A Cooperatively Interruptible Joining Thread, Rev 3, Nicolai Josuttis, Herb Sutter, Anthony Williams http://www.open-std.org/jtc1/sc22/wg21/docs/papers/2018/p0660r3.pdf.