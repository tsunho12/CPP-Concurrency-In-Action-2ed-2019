% # 11.1 与并发相关的错误类型

在并发代码中可以发现各式各样的错误,这些错误不会集中于某个方面。不过,有一些错误与使用并发直接相关,本章重点关注这些错误。通常,并发相关的错误通常有两大类:

\begin{itemize}
\item 不必要的阻塞
\item 条件竞争
\end{itemize}

这两大类的颗粒度很大,让我们将其细分为较小的问题。

\mySubsubsection{11.1.1}{不必要的阻塞}

“不必要的阻塞”是什么意思?线程阻塞的时候,不能处理任何任务,因为它在等待其他“条件”的达成。通常这些“条件”就是一个互斥量、一个条件变量或一个future,也可能是一个I/O操作。这是多线程代码的先天特性,不过也不是任何时候都会衍生成“不必要的阻塞”。

为什么阻塞是不必要的?通常,因为其他线程在等待该阻塞线程上的某些操作完成,如果该线程阻塞了,那些线程必然会阻塞。

这个主题可以分成以下几个种类:

\begin{itemize}
\item 死锁——如在第3章所见,死锁的情况下,两个线程会互相等待。当线程产生死锁,应该完成的任务就会搁置。举个例子来说,一些线程是负责对用户界面操作的线程,在死锁的情况下,用户界面就会无响应。或者,界面接口会保持响应,不过有些任务就无法完成,比如:查询无结果返回或文档未打印。
\item 活锁——与死锁的情况类似。不同的地方在于线程不是阻塞等待,而是在循环中持续检查,例如:自旋锁。比较严重的情况下,其表现和死锁一样(应用不会做任何处理,停止响应),因为线程还在循环中被检查,而不是阻塞等待,所以CPU的使用率还居高不下。不太严重的情况下,使用随机调度,活锁的问题还可以解决。
\item I/O阻塞或外部输入——当线程被外部输入所阻塞,线程也就不能做其他事情了(即使,等待输入的情况永远不会发生)。当为外部输入所阻塞,就会让人不太高兴,因为可能有其他线程正在等待这个线程完成某些任务。
\end{itemize}

简单的介绍了“不必要阻塞”的组成。那么,条件竞争呢?

\mySubsubsection{11.1.2}{条件竞争}

条件竞争在多线程代码中很常见——很多条件竞争表现为死锁与活锁。而且,并非所有条件竞争都是恶性的——对独立线程相关操作的调度,决定了条件竞争发生的时间。很多条件竞争是良性的,比如:哪一个线程去处理任务队列中的下一个任务。不过,很多并发错误的引起也是因为条件竞争。

条件竞争常会产生以下几种类型的错误:

\begin{itemize}
\item 数据竞争——因为未同步访问一块共享内存,将会导致代码产生未定义行为。第5章已经介绍了数据竞争,也了解了C++的内存模型。数据竞争通常发生在错误的使用原子操作上,做同步线程的时候,或没使用互斥量保护共享数据时。
\item 破坏不变量——主要表现为悬空指针(因为其他线程已经将要访问的数据删除了),随机存储错误(因为局部更新,导致线程读取了不一样的数据),以及双重释放(比如:当两个线程对同一个队列同时执行pop操作,想要删除同一个关联数据)等等。破坏不变量可以看作为“基于数据”的问题。当独立线程需要以一定顺序执行某些操作时,错误的同步会导致条件竞争,比如:破坏访问顺序。
\item 生命周期问题——虽然这类问题也能归结为破坏了不变量,不过这里将其作为一个单独的类别给出。这里的问题是线程会访问不存在的数据,这可能是因为删除或销毁了数据,或者转移到其他对象中去了。生命周期问题,通常是在一个线程引用了局部变量,在线程还没有完成前,局部变量的“死期”就到了。不过这个问题并不只存在这种情况下。当手动调用join()等待线程完成工作,需要保证异常抛出时,join()还会等待其他未完成工作的线程。这是线程中基本异常安全的应用。
\end{itemize}

恶性条件竞争就如同一个杀手。死锁和活锁会表现为:应用挂起和反应迟钝,或超长时间完成任务。当一个线程产生死锁或活锁,可以用调试器附着到该线程上进行调试。条件竞争,破坏不变量,以及生命周期问题,其表现代码都是可见的(比如,随机崩溃或错误输出)——可能重写了系统部分的内存使用方式(不会改太多)。其中,可能是因为执行时间,导致问题无法定位到具体的位置。这就是共享内存系统的诅咒——需要通过线程尝试限制可访问的数据,并且还要正确的使用同步,应用中的任何线程都可以复写(可被其他线程访问的)数据。

现在已经了解了这两大类中都有哪些具体问题了。下面就让我们来了解,如何在代码中定位和修复这些问题。