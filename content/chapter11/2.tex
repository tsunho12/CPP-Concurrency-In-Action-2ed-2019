% # 11.2 定位并发Bug的技巧

了解了与并发相关的错误类型,以及如何在代码中的体现后。这些信息可以帮助我们来判断,代码中是否存在隐藏的错误。

最简单的就是看代码。虽然看起来比较明显,但是要彻底的修复问题,却是很难的。读刚写完的代码,要比读已经存在很久的代码容易的多。同理,当在评审别人写好的代码时,很容易给出一个通读结果,比如:与自己的代码标准作对比,以及高亮标出显而易见的问题。为什么要花时间来仔细梳理代码?想想之前提到的并发相关的问题——也要考虑非并发问题(也可以在很久以后做这件事。不过,最后bug依旧存在)。我们可以在评审代码的时候,考虑一些具体的事情,并且发现问题。

即使已经很对代码进行了很详细的评审,依旧会错过一些bug,这就需要确定一下代码是否做了对应的工作。因此,在测试多线程代码时,需要一些代码评审的技巧。

\mySubsubsection{11.2.1}{代码评审——发现潜在的错误}

评审多线程代码时,重点要检查与并发相关的错误。如果可能,可以让同事/同伴来评审。因为不是他们写的代码,他们将会考虑这段代码是怎么工作的,就可能会覆盖到一些你没有想到的情况,从而找出一些潜在的错误。评审人员需要花时间去做评审——并非在休闲时间简单的扫一眼。大多数并发问题需要的不仅仅是一次快速浏览——通常需要在找到问题上花费很多时间。

如果让你的同事来评审代码,他/她肯定对你的代码不是很熟悉。因此,他/她会从不同的角度来看你的代码,然后指出你没有注意的事情。如果你的同事都没有空,你可以叫朋友,或传到网络上,让网友评审(注意,别传一些机密代码上去)。实在没有人评审,不要着急。对于初学者,可以将代码放置一段时间——先去做应用的另外的部分,或是阅读一本书籍,亦或出去溜达溜达。休息之后,当再集中注意力做某些事情(潜意识会考虑很多问题)。同样,当你做完其他事情,回头再看这段代码,就会有些陌生——你可能会从另一个角度来看你自己以前写的代码。

另一种方式就是自我评审。可以向别人详细的介绍你所写的功能,可能并不是一个真正的人——可能要对玩具熊或橡皮鸡来进行解释,并且我个人觉得写一些比较详细的注释是非常有益的。在解释过程中,会考虑每一行过后,会发生什么事情,有哪些数据被访问了,等等。问自己关于代码的问题,并且向自己解释这些问题。我觉得这是种非常有效的技巧——通过自问自答,对每个问题认真考虑,这些问题往往都会揭示一些问题,也会有益于任何形式的代码评审。

\textbf{评审多线程代码需要考虑的问题}

评审代码的时候考虑和代码相关的问题,以及有利于找出代码中的问题。对于问题,评审者需要在代码中找到相应的回答或错误。我认为下面这些问题是必须的(当然,不是一个综合性的列表),你也可以找一些其他问题来帮助你找到代码的问题。

这里,列一下我的清单:

\begin{itemize}
    \item 并发访问时,哪些数据需要保护?
    \item 如何确定访问数据受到了保护?
    \item 是否会有多个线程同时访问这段代码?
    \item 这个线程获取了哪个互斥量?
    \item 其他线程可能获取哪些互斥量?
    \item 两个线程间的操作是否有依赖关系?如何满足这种关系?
    \item 这个线程加载的数据是合法数据吗?数据是否被其他线程修改过?
    \item 当假设其他线程可以对数据进行修改,这将意味着什么?怎么确保这样的事情不会发生?
\end{itemize}

我最喜欢最后一个问题,因为它让我去考虑线程之间的关系。通过假设一个bug和一行代码相关联,你就可以扮演侦探来追踪bug出现的原因。为了让你自己确定代码里面没有bug,需要考虑代码运行的各种情况。数据被多个互斥量所保护时,这种方式尤其有用,比如:使用线程安全队列(第6章),可以对队头和队尾使用独立的互斥量:就是为了确保在持有一个互斥量时,访问是安全的,必须保持有其他互斥量的线程不能同时访问同一元素。需要特别关注的是,对公共数据的显式处理,使用一个指针或引用的方式来获取数据。

倒数第二个问题也很重要,这里很容易产生错误:先释放再获取一个互斥量的前提是,其他线程可能会修改共享数据。虽然很明显,但当互斥锁不是立即可见——可能因为是内部对象——就会不知不觉的掉入陷阱中。第6章已经了解到这种情况是怎么引起条件竞争的,以及如何给细粒度线程安全数据结构带来麻烦的。不过,非线程安全栈将top()和pop()操作分开是有意义的,当多线程并发的访问这个栈,问题会马上出现,因为在两个操作的调用间,内部互斥锁已经释放,并且另一个线程对栈进行了修改。解决方案就是将两个操作合并,就能用同一个锁来对操作的执行进行保护,也就消除了条件竞争的问题。

OK,你已经评审过代码了(或者让别人看过)。现在,确定代码没有问题?

就像需要用味觉来证明,你现在吃的东西——怎么测试才能确认你的代码没有bug呢?

\mySubsubsection{11.2.2}{定位并发相关的错误}

写单线程应用时,测试起来相对简单。原则上,设置各种可能的输入(或设置成感兴趣的情况),然后执行。如果应用行为和期望输出一致,就能判断其能对给定输入集给出正确的答案。检查错误状态(比如:处理磁盘满载错误)就会比处理可输入测试复杂的多,不过原理是一样的——设置初始条件,然后让程序执行。

因为不确定线程的调度情况,所以测试多线程代码的难度就要比单线程大好几个数量级。因此,即使使用测试单线程的输入数据,如果有条件变量潜藏在代码中,那么代码的结果可能会时对时错。只是因为条件变量可能会在有些时候,等待其他事情,从而导致结果错误或正确。

因为与并发相关的bug难以判断,所以设计并发代码时需要格外谨慎。设计时,每段代码都需要进行测试,保证没有问题,这样才能在测试出现问题的时候,剔除并发相关的bug——例如,对队列的push和pop,分别进行并发的测试,就要好于直接使用队列测试其中全部功能。这种思想能帮你在设计代码的时候,考虑什么样的代码是可以用来测试正在设计的这个结构——本章后续章节中会看到与设计测试代码相关的内容。

测试的目的就是为了消除与并发相关的问题。如果在单线程测试时遇到了问题,那这个问题就是普通的bug,而非并发相关的bug。当问题发生在*未测试区域*(in the wild),也就是没有在测试范围之内,这样的情况就要特别注意。bug出现在应用的多线程部分,并不意味着该问题是多线程相关的bug。使用线程池管理某一级并发的时候,通常会有一个可配置的参数,用来指定工作线程的数量。当手动管理线程时,就需要将代码改成单线程的方式进行测试。不管哪种方式,将多线程简化为单线程后,就能将与多线程相关的bug排除掉。反过来说,当问题在单芯系统中消失(即使还是以多线程方式),在多芯系统或多核系统中出现,就能确定你是否被多线程相关的bug坑了。可能是条件变量的问题,还有可能是同步或内存序的问题。

测试并发的代码很多,不过通过测试的代码结构就没那么多了。对结构的测试也很重要,就像对环境的测试一样。

如果你依旧将测试并发队列当做一个测试例,就需要考虑这些情况:

\begin{itemize}
    \item 使用单线程调用push()或pop(),来确定在一般情况下队列是否正常
    \item 其他线程调用pop()时,使用另一线程在空队列上调用push()
    \item 空队列上,以多线程的方式调用push()
    \item 满载队列上,以多线程的方式调用push()
    \item 空队列上,以多线程的方式调用pop()
    \item 满载队列上,以多线程的方式调用pop()
    \item 非满载队列上(任务数量小于线程数量),以多线程的方式调用pop()
    \item 当一线程在空队列上调用pop()的同时,以多线程的方式调用push()
    \item 当一线程在满载队列上调用pop()的同时,以多线程的方式调用push()
    \item 当多线程在空队列上调用pop()的同时,以多线程方式调用push()
    \item 当多线程在满载队列上调用pop()的同时,以多线程方式调用push()
\end{itemize}

这是我所能想到的场景,可能还有更多,之后需要考虑测试环境的因素:

\begin{itemize}
    \item “多线程”是有多少个线程(3个,4个,还是1024个?)
    \item 系统中是否有足够的处理器,能让每个线程运行在属于自己的处理器上
    \item 测试需要运行在哪种处理器架构上
    \item 测试中如何对“同时”进行合理的安排
\end{itemize}

这些因素的考虑会具体到一些特殊情况。四个因素都需要考虑,第一个和最后一个会影响测试结构本身(在11.2.5节中会介绍),另外两个就和实际的物理测试环境相关了。使用线程数量相关的测试代码需要独立测试,可通过很多结构化测试获得最合适的调度方式。了解这些技巧前,先来了解一下如何让应用更容易测试。

\mySubsubsection{11.2.3}{可测试性设计}

测试多线程代码很困难,所以需要将其变得简单一些。很重要的一件事就是设计代码时,考虑其的可测试性。可测试的单线程代码设计已经说烂了,而且其中许多建议现在依旧适用。通常,如果代码满足一下几点,就很容易进行测试:

\begin{itemize}
    \item 每个函数和类的关系都很清楚。
    \item 函数短小精悍。
    \item 测试用例可以完全控制测试代码周边的环境。
    \item 执行特定操作的代码应该集中测试,而非分布式测试。
    \item 需要在完成编写后,考虑如何进行测试。
\end{itemize}

以上这些在多线程代码中依旧适用。实际上,我会认为对多线程代码的可测试性要比单线程的更为重要,因为多线程的情况更加复杂。最后一个因素尤为重要:即使不在写完代码后,去写测试用例,这也是一个很好的建议,能让你在写代码之前,想想应该怎么去测试它——用什么作为输入,什么情况看起来会让结果变得糟糕,以及如何激发代码中潜在的问题等等。

并发代码测试的一种最好的方式:去并发化测试。如果代码在线程间的通讯路径上出现问,就可以让一个已通讯的单线程进行执行,这样会减小问题的难度。在对数据进行访问的应用进行测试时,可以使用单线程的方式进行。这样线程通讯和对特定数据块进行访问时只有一个线程,更容易进行测试。

例如:当应用设计为一个多线程状态机时,可以将其分为若干块。将每个逻辑状态分开,就能保证对于每个可能的输入事件、转换或其他操作结果的正确性。这就是单线程测试的技巧,测试用例提供的输入事件将来自于其他线程。之后,核心状态机和消息路由的代码,就能保证时间能以正确的顺序传递给单独测试的线程,不过对于多并发线程,需要为测试专门设计简单的逻辑状态。

或者将代码分割成多个块(比如:读共享数据/变换数据/更新共享数据),就能使用单线程来测试变换数据的部分。麻烦的多线程测试问题,转换成单线程测试读和更新共享数据,就会简单许多。

某些库会用其内部变量存储状态时需要小心,当多线程使用同一库中的函数,这个状态就会共享。这是一个问题,并且问题不会马上出现在访问共享数据的代码中。不过,随着你对这个库的熟悉,就会清楚这样的情况会在什么时候出现。之后,可以适当的加一些保护和同步或使用B计划——让多线程安全并发访问的功能。

将并发代码设计的有更好的测试性,要比以代码分块的方式处理并发相关的问题好很多。当然,还要注意对非线程安全库的调用。11.2.1节中那些问题,也需要在评审自己代码的时候格外注意。虽然,这些问题和测试(可测试性)没有直接的关系,但带上“测试帽子”时,就要考虑这些问题了,并且还要考虑如何测试已写好的代码,这就会影响设计方向的选择,也会让测试更加容易一些。

我们已经了解了如何能让测试变得更加简单,以及将代码分成一些“并发”块(比如,线程安全容器或事件逻辑状态机)以“单线程”的形式(可能还通过并发块和其他线程进行互动)进行测试。

下面就让我们了解一下测试多线程代码的技术。

\mySubsubsection{11.2.4}{多线程测试技术}

想通过一些技巧写一些较短的代码,来对函数进行测试,比如:如何处理调度序列上的bug?

这里的确有几个方法能进行测试,让我们从蛮力测试(或称压力测试)开始。

\textbf{蛮力测试}

代码有问题的时候,要求蛮力测试一定能看到这个错误。这意味着代码要运行很多遍,可能会有很多线程在同一时间运行。只能在线程出现特殊调度时,增加代码运行的次数,从而提升bug出现的几率。当有几次代码测试通过,你可能会对代码的正确性有一些信心。如果连续运行10次都通过,你就会更有信心。如果你运行十亿次都通过了,那么你就会认为这段代码没有问题了。

自信的来源是每次测试的结果。如果你的测试粒度很细,就像测试之前的线程安全队列,那么蛮力测试会让你对这段代码持有高度的自信。另一方面,当测试对象体积较大的时候,调度序列将会很长,即使运行了十亿次测试用例,也不让你对这段代码产生什么信心。

蛮力测试的缺点是,可能会误导你。如果写出来的测试用例就为了不让有问题的情况发生,那么怎么运行,测试都不会失败,可能会因环境的原因,出现几次失败的情况。最糟糕的情况就是,问题不会出现在你的测试系统中,因为在某些特殊的系统中,这段代码就会出现问题。除非代码运行在与测试机系统相同的系统中,不过特殊的硬件和操作系统的因素结合起来,可能就会让运行环境与测试环境有所不同,问题可能就会随之出现。

这里有一个经典的案例,在单处理器系统上测试多线程应用。因为每个线程都在同一个处理器上运行,任何事情都是串行的,并且还有很多条件竞争和乒乓缓存,这些问题可能在真正的多处理器系统中根本不会出现。还有其他变数:不同处理器架构提供不同的的同步和内存序机制。比如,在x86和x86-64架构上,原子加载操作通常是相同的,无论是使用memory\_order\_relaxed,还是memory\_order\_seq\_cst(详见5.3.3节)。这就意味着在x86架构上使用自由内存序没有问题,但在有更精细的内存序指令集的架构(比如:SPARC)下,这样使用就可能产生错误。

如果你希望应用能跨平台使用,就要在相关的平台上进行测试,这就是我把处理器架构也列在测试需要考虑的清单中的原因(详见11.2.2)。

要避免误导的产生,关键点在于成功的蛮力测试。这就需要进行仔细考虑和设计,不仅仅是选择相关的单元测试,还要遵守测试系统的设计准则,以及选定测试环境。保证尽可能的测试到代码的各个分支,尽可能多的测试线程间的互相作用。还有,需要知道哪部分有测试覆盖,哪些没有覆盖。

虽然,蛮力测试能够给你一些信心,不过不保证能找到所有的问题。如果有时间将下面的技术应用到你的代码或软件中,就能保证找到所有的问题。

\textbf{仿真测试}

名字比较口语化,我需要解释一下这个测试是什么意思:使用一种特殊的软件,用来模拟代码运行的真实情况。你应该知道这种软件,能让一台物理机上运行多个虚拟环境或系统环境,而硬件环境则由监控软件来完成。除了环境是模拟的以外,模拟软件会记录对数据序列访问,上锁,以及对每个线程的原子操作。然后使用C++内存模型的规则,重复的运行,从而识别条件竞争和死锁。

虽然,这种组合测试可以保证所有与系统相关的问题都会被找到,不过过于零碎的程序将会在这种测试中耗费太长时间,因为组合数目和执行的操作数量将会随线程的增多呈指数增长态势。这个测试最好留给需要细粒度测试的代码段,而非整个应用。另一个缺点就是,代码对操作的处理,往往会依赖与模拟软件的可用性。

所以,测试需要在正常情况下,运行很多次,不过这样可能会错过一些问题。也可以在一些特殊情况下运行多次,不过这样更像是为了验证某些问题。

还有其他的测试选项吗?

第三个选项就是使用专用库,在运行测试的时候,检查代码中的问题。

\textbf{使用专用库}

虽然,这个选择不会像仿真方式提供彻底的检查,不过可以通过特别实现的库(使用同步原语)来发现一些问题,比如:互斥量,锁和条件变量。例如,访问某块公共数据的时候,就要将指定的互斥量上锁。数据被访问后,发现一些互斥量已经上锁,就需要确定相关的互斥量是否被访问线程锁住。如果没有,测试库将报告这个错误。当需要测试库对某块代码进行检查时,可以对相应的共享数据进行标记。

一个线程同时持多个互斥量时,测试库也会对锁的序列进行记录。如果其他线程以不同的顺序进行上锁,即使在运行的时候测试用例没有发生死锁,测试库都会将这个行为记录为有“潜在死锁”的可能。

测试多线程代码时,另一种库可能会用到,以线程原语实现的库,比如:互斥量和条件变量。当多线程代码在等待,或是条件变量通过notify\_one()提醒的某个线程,测试者可以通过线程获取到锁,就可以让你来安排一些特殊的情况,以验证代码是否会在这些特定的环境下产生期望的结果。

C++标准库实现中,某些测试工具已经存在于标准库中,没有实现的测试工具,可以基于标准库进行实现。

了解完各种运行测试代码的方式,将让我们来了解一下,如何以想要的调度方式来构建代码。

\mySubsubsection{11.2.5}{构建多线程测试代码}

11.2.2节中提过,需要找一种合适的调度方式来处理测试中“同时”的部分,现在就是解决这个问题的时候。

在特定时间内,需要安排一系列线程,同时执行指定的代码段。两个线程的情况,就很容易扩展到多个线程。

首先,需要知道每个测试的不同之处:

\begin{itemize}
    \item 环境布置代码,必须首先执行
    \item 线程设置代码,需要在每个线程上执行
    \item 线程上执行的代码,需要有并发性
    \item 并发执行结束后,后续代码需要对代码的状态进行断言检查
\end{itemize}

这几条后面再解释,先考虑一下11.2.2节中的一个特殊的情况:一个线程在空队列上调用push(),同时让其他线程调用pop()。

通常,搭建环境的代码比较简单:创建队列即可。线程在执行pop()的时候,没有线程设置代码。线程设置代码是在执行push()操作的线程上进行的,其依赖与队列的接口和对象的存储类型。如果存储的对象需要很大的开销才能构建,或必须在堆上分配的对象,最好在线程设置代码中进行构建或分配,这样就不会影响到测试结果。另外,如果队列中只存简单的int类型对象,构建int对象时就不会有太多额外的开销。实际上,已测试代码相对简单——一个线程调用push(),另一个线程调用pop()——“完成后”的代码到底是什么样子呢?

这个例子中pop()具体做的事情,会直接影响“完成后”代码。如果有数据块,返回的肯定就是数据了,push()操作就成功的向队列中推送了一块数据,并在在数据返回后,队列依旧是空的。如果pop()没有返回数据块,也就是队列为空的情况下操作也能执行,这样就需要两个方向的测试:要不pop()返回push()推送到队列中的数据块,之后队列依旧为空;要不pop()会示意队列中没有元素,但同时push()向队列推送了一个数据块。这两种情况都是真实存在的,需要避免的情况是:pop()队列时,队列为空,或pop()返回数据块的同时,队列中还有数据块。为了简化测试,可以假设pop()可阻塞。在最终代码中,需要用断言判断弹出的数据与推入的数据正确性,还要判断队列为空。

了解了各个代码块,就需要保证所有事情按计划进行。一种方式是使用一组\texttt{std::promise}来表示就绪状态。每个线程使用promise来表示是否准备好,然后让\texttt{std::promise}等待(复制)一个\texttt{std::shared\_future}。主线程会等待每个线程上的promise设置后才开始。这样每个线程能够同时开始,并且在准备代码执行完成后,并发代码就可以开始执行了。任何线程的特定设置都需要在设置promise前完成。最终,主线程会等待所有线程完成,并且检查最终状态。还需要格外关心异常,所有线程在准备好的情况下,再按下“开始”键。否则,未准备好的线程就不会运行。

下面的代码,构建了这样的测试。

代码11.1 对一个队列并发调用push()和pop()的测试用例
\begin{cpp}
void test_concurrent_push_and_pop_on_empty_queue()
{
  threadsafe_queue<int> q;  // 1

  std::promise<void> go,push_ready,pop_ready;  // 2
  std::shared_future<void> ready(go.get_future());  // 3

  std::future<void> push_done;  // 4
  std::future<int> pop_done;

  try
  {
    push_done=std::async(std::launch::async,  // 5
                         [&q,ready,&push_ready]()
                         {
                           push_ready.set_value();
                           ready.wait();
                           q.push(42);
                         }
      );
    pop_done=std::async(std::launch::async,  // 6
                        [&q,ready,&pop_ready]()
                        {
                          pop_ready.set_value();
                          ready.wait();
                          return q.pop();  // 7
                        }
      );
    push_ready.get_future().wait();  // 8
    pop_ready.get_future().wait();
    go.set_value();  // 9

    push_done.get();  // 10
    assert(pop_done.get()==42);  // 11
    assert(q.empty());
  }
  catch(...)
  {
    go.set_value();  // 12
    throw;
  }
}
\end{cpp}

首先,环境设置代码中创建了空队列①。然后,为准备状态创建promise对象②,并且为go信号获取一个\texttt{std::shared\_future}对象③。再后,创建了future用来表示线程是否结束④。这些都需要放在try块外面,再设置go信号时抛出异常,就不需要等待其他线程完成任务了(这会产生死锁——如果测试代码产生死锁,测试代码就是不理想的代码)。

try块中可以启动线程⑤⑥——使用\texttt{std::launch::async}保证每个任务在自己的线程上完成。注意,使用\texttt{std::async}会让任务更容易成为线程安全的任务,因为析构函数会对future进行线程汇入,所以这里不用普通\texttt{std::thread}。Lambda函数会捕捉指定的任务(在队列中引用),并且为promise准备相关的信号,同时对从go中获取的ready做一份拷贝。

如之前所说,每个任务集都有ready信号,并且会在执行测试代码前,等待所有的ready信号。主线程不同——等待所有线程的信号前⑧,提示所有线程可以开始进行测试了⑨。

最终,异步调用等待线程完成后⑩⑪,主线程会从中获取future,再调用get()成员函数获取结果,最后对结果进行检查。注意这里pop操作通过future返回检索值⑦,所以能获取最终的结果⑪。

有异常抛出时,需要通过对go信号的设置来避免悬空指针的产生,再重新抛出异常⑫。future与之后声明的任务相对应④,所以future会首先销毁。如果future没有就绪,析构函数将会等待相关任务完成后执行操作。

虽然使用测试模板对两个调用进行测试,这便于测试的进行。例如,启动线程就是很耗时的过程,如果没有线程在等待go信号,推送线程可能会在弹出线程开始之前就已经完成了,这样就失去了测试的作用。以这种方式使用future,就是为了保证线程都在运行,并且阻塞在同一个future上。future解除阻塞后,将会让所有线程运行起来。熟悉了这个结构后,就能以同样的模式创建新的测试用例。这种模式很容易进行扩展,可以轻松的测试两个以上的线程。

目前,我们已经了解了多线程代码的正确性测试。虽然这是最最重要的问题,但是不是我们做测试的唯一原因:多线程性能的测试同样重要。

下面就让我们来了解一下性能测试。

\mySubsubsection{11.2.6}{测试多线程代码性能}

选择以并发的方式开发应用,就是为了能够使用日益增长的处理器数量,通过处理器数量的增加,来提升应用的执行效率。因此,确定性能是否有真正的提高就很重要了(就像其他优化一样)。

并发效率中有个特别的问题——可扩展性——你希望代码能很快的运行24次,或在24芯的机器上对数据进行24(或更多)次处理,或其他等价情况。如8.4.2节中所述,当有重要的代码以单线程方式运行时,就会限制性能的提高。因此,在做测试之前,回顾一下代码的设计结构是很有必要的。通过分析就能判断,代码在24芯的机器上时,性能会不会提高24倍,或是因为有串行部分的存在,最大的加速比只有3。

对数据访问时,处理器之间会有竞争,会对性能有很大的影响。需要合理的权衡性能和处理器的数量。处理器数量太少,就会等待很久。而处理器过多,又会因为竞争的原因等待很久。

因此,在对应的系统上通过不同的配置,检查多线程的性能就很有必要,这样可以得到一张性能图。最起码(如果条件允许)需要在一个单处理器的系统上和一个多处理核芯的系统上进行测试。