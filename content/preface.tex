% \subsection*{前言}
% \addcontentsline{toc}{subsection}{前言}

与多线程的邂逅是在毕业后的第一份工作中。那时,正在写一个填充数据库的程序,需要处理的数据量很大,每条记录都是独立的,并且需要在插入数据库之前,对数据量进行合理分配。为了充分利用10核UltraSPARC CPU(Ultra Scalable Processor ARChitecture,究极可扩处理器架构(大端)),我们使用了多线程,每个线程会处理所要记录的数据。我们使用C++和POSIX线程库完成编码,也遇到了一些问题——当时,多线程对于我们来说是一个新事物——最后还是完成了。也是做这个项目的时候,我开始注意C++标准委员会和刚刚发布的C++标准。

我对多线程和并发有着浓厚的兴趣。虽然,别人觉得多线程和并发难用、复杂,还会让代码出现各种各样的问题,不过,在我看来这是一种强有力的工具,能充分使用硬件资源,让程序运行的更快。

从那以后,我开始使用并发在单核机器上对应用性能和响应时长进行改善,多线程可以帮助你隐藏一些耗时的操作,比如I/O操作。同时,我也学着在操作系统上使用多线程,并且了解Intel CPU如何处理任务切换。

同时,对C++的兴趣让我与\href{http://accu.org/}{ACCU}有了联系,之后是\href{http://www.bsigroup.com/en-GB/standards/}{BSI}(英国标准委员会)中的C++标准委员会,还有Boost。也是因为兴趣的原因,我参与了Boost线程库的初期开发(虽然初期版本已被放弃)。我曾是Boost线程库的主要开发者和维护者,现在这项任务已经交给了其他人。

作为C++标准委员会的一员,希望改善现有标准的缺陷和不足,并为新标准提出建议(新标准命名为C++0x是希望它能在2009年发布,不过最后因为2011年才发布,所以官方命名为C++11)。我也参与很多BSI的工作,并且也为自己的建议起草建议书。因为我起草及合著的多线程和并发相关的草案,将会成为新标准的一部分,所以当委员会将多线程提上C++标准的日程时,我高兴得差点飞起来。我也会继续关注并参与C++17标准并行部分的修订、并发技术标准(的扩展),以及给未来并发标准的一些建议。新标准将我(计算机相关)的两大兴趣爱好——C++和多线程——结合起来,想想还有点小激动。

本书旨在教会其他C++开发者如何安全、高效地使用C++17线程库和并发技术标准(的扩展)。希望我对C++和多线程的热情,也能感染读者们。